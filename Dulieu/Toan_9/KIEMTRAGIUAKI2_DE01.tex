\def\x{210}
\setcounter{bt}{0}
\setcounter{ex}{0}
%%%Tùy chọn 1: Kì thi(có thể bỏ trống)
%%%Tùy chọn 3: lớp (có thể bỏ trống)
%%%Tùy chọn 4: Sở/Phòng (có thể bỏ trống)
%%%Tùy chọn 5: Ngày thi (không được bỏ trống)
\begin{name}[Kiểm tra giữa kì II][Toán][9][Sở Giáo Dục và Đào Tạo][Ngày 31 tháng 12 năm 2023]{Trường THCS}{2023 - 2024}
\end{name}
\section{Kiểm~tra~giữa~kì~II huyện Hóc Môn}
%%%==========Phần tự luận============%%%
\nhanmanh{BÀI TẬP TỰ LUẬN}
\Opensolutionfile{ansbt}[LOIGIAITL/LGTL-1]
\luuloigiaibt
%%%==============BT_1==============%%%
\begin{bt}[3,0 điểm]
	Giải hệ phương trình và phương trình sau:
	\begin{enumerate}
		\item $\left\{\begin{array}{l}x-y=3\\ 2x+3y=16\end{array}\right.$
		\item $\left\{\begin{array}{l}3x-4y=17\\ 2x+5y=-4\end{array}\right.$
		\item $x^2-4x+3=0$
	\end{enumerate}
	\loigiai{}
\end{bt}
%%%==============BT_2==============%%%
\begin{bt}[1,5 điểm]
	Cho parabol $(P)\colon y=x^2$ và đưởng thẳng $(d)\colon y=2x+3$
	\begin{enumerate}
		\item Vẽ $(P)$ và $(d)$ trên cùng mặt phẳng tọa độ.
		\item Tìm tọa độ giao điểm của $(P)$ và $(d)$ bằng phép tính.
	\end{enumerate}
	\loigiai{}
\end{bt}

%%%==============BT_3==============%%%
\begin{bt}[1,0 điểm]
	Một chiếc xe đạp có đường kính bánh xe là $0,55\mathrm{m}$ chạy hết đoạn đường $200\mathrm{m}$ thi bánh xe lăn được mấy vòng? (Kết quả làm tròn đến hàng đơn vị)
	\loigiai{}
\end{bt}

%%%==============BT_4==============%%%
\begin{bt}[1,0 điểm][][Nguồn:]
	Số tiền y (đồng) bạn Lan để dành được sau $x$ (ngày) được liên hệ với ahau theo công thức hàm số $y=a x+b(a, b$ là hằng số, $a \times 0$) có đồ thị được biểu diễn như hình bên.
	\begin{enumerate}
		\item Xác định $a$ và $b$.
		\item Bạn Lan muốn có 3200000 đồng để mua một đôi giày thể thao yêu thích thi phài để dành tiền trong thời gian bao lâu?
	\end{enumerate}
	\loigiai{}
\end{bt}
%%%==============BT_1==============%%%
\begin{bt}[1,5 điểm]
	Trong một buổi lao động trồng cây, mỗi bạn nam lớp 9A được phân công trồng 3 cây, mỗi bạn nữ trồng 2 cây. Tổng cộng lớp $9\mathrm{~A}$ trồng được 113 cây. Tính số bạn nam và số bạn nữ của lớp $9\mathrm{~A}$ biết tổng số học sinh của lớp $9\mathrm{~A}$ là 45 bạn.
	\loigiai{}
\end{bt}
%%%==============BT_5==============%%%
\begin{bt}[2,5 điểm]
	Cho $\triangle ABC$ nhọn $(AB < AC)$. Đường tròn tâm $O$ đường kính $BC$ cắt $AB, AC$ lần lượt tại $I$ và $K$. $H$ là giao điểm của $BK$ và $CI$.
	\begin{enumerate}
		\item Tính $\widehat{BIC}$ và chứng minh tứ giác $AIHK$ nội tiếp.
		\item Gọi $\mathscr{D}$ là trung điểm $AH$. Chứng minh $DI$ là tiếp tuyến của $(O)$.
		\item Tiếp tuyến tại $B$ của $(O)$ cắt tia $KD$ ở điểm $E$. Tia $AE$ cắt tia $CB$ tại $F$. $FH$ cắt $BE$ tại $N$.
	\end{enumerate}
	Chứng minh  $ON \; \parallel$ IC.
	\loigiai{}
\end{bt}
\Closesolutionfile{ansbt}

\label{\x}