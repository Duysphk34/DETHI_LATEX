\def\x{210}
\setcounter{bt}{0}
\setcounter{ex}{0}
%%%Tùy chọn 1: Kì thi
%%%Tùy chọn 2: Môn
%%%Tùy chọn 3: lớp
%%%Tùy chọn 4: Sở/Phòng
%%%Tùy chọn 5: Ngày thi
\begin{tcolorbox}
	\begin{name}[Kiểm tra giữa kì II][Hóa học][10][Sở Giáo dục và Đào tạo]{Trường THCS }{2023 - 2024}
	\end{name}
\end{tcolorbox}
%%%==========Phần trắc nghiệm 1 phương án============%%%
\tieumuc{Bài Tập Trắc Nghiệm}-- \textit{Mỗi câu chỉ chọn một phương án.}
\Opensolutionfile{ans}[Ans/DATN-1]
\luuloigiaiex
\Opensolutionfile{ansex}[LOIGIAITN/LGTN-1]
%%%=============EX_1=============%%%
\begin{ex}
	Quy ước về dấu của nhiệt phản ứng $\left(\Delta_r H_{298}^0\right)$ nào sau đây là đúng?
	\choice
	{Phản ứng tỏa nhiệt có $\Delta_r H_{298}^0> 0$}
	{Phản ứng thu nhiệt có $\Delta_r H_{298}^0< 0$}
	{Phản ứng tỏa nhiệt có $\Delta_r H_{298}^0< 0$}
	{Phản ứng thu nhiệt có $\Delta_r H_{298}^0=0$}
	\loigiai{}
\end{ex}
%%%=============EX_2=============%%%
\begin{ex}
	Nung nóng 2 ống nghiệm chứa $\mathrm{NaHCO}_3$ và $P$, xảy ra các phản ứng sau:
	\[
	\begin{aligned}
		& 2 \mathrm{NaHCO}_3(\mathrm{s}) \to \mathrm{Na}_2 CO_3(\mathrm{s})+CO_2(\mathrm{g})+H_2 O(g)(1) \\
		& 4 P(s)+5 O_2(\mathrm{g}) \to 2 P_2 O_5(\mathrm{s})(2)
	\end{aligned}
	\]
	Khi ngừng đun nóng, phản ứng (1) ngừng lại còn phản ứng (2) tiếp tục xảy ra, chứng tỏ
	\choice
	{phản ứng (1) tỏa nhiệt, phản ứng (2) thu nhiệt}
	{phản ứng (1) thu nhiệt, phản ứng (2) tỏa nhiệt}
	{cả hai phản ứng đều tỏa nhiệt}
	{cả hai phản ứng đều thu nhiệt}
	\loigiai{}
\end{ex}
%%%=============EX_3=============%%%
\begin{ex}
	Phát biểu nào sau đây đúng?
	\choice
	{Phản ứng có biến thiên enthalpy càng âm phản ứng càng thu nhiệt}
	{Phản ứng có biến thiên enthalpy càng âm phản ứng càng toả nhiệt}
	{Với phản ứng toả nhiệt, năng lượng của hệ chất phản ứng thấp hơn năng lượng của hệ sản phẩm}
	{Các phản ứng toả nhiệt đều phản ứng ở điều kiện chuẩn}
	\loigiai{}
\end{ex}
%%%=============EX_4=============%%%
\begin{ex}
	Cho phản ứng: $\mathrm{Cl}_2+\mathrm{Ca}(OH)_2 \to \mathrm{CaOCl}_2+H_2O$
	Cho biết $\mathrm{Cl}_2$ đóng vai trò là
	\choice
	{chất khừ}
	{chất oxi hóa}
	{chất oxi hóa và môi trường}
	{vừa là chất khử, vừa là chất oxi hóa}
	\loigiai{}
\end{ex}
%%%=============EX_5=============%%%
\begin{ex}
	Một nhóm học sinh nghiên cứu xác định biến thiên enthalpy chuẩn của phản ứng giữa magnesium oxide và dung dịch hydrochloric acid $\left(\mathrm{HCl}, M=36,5\mathrm{~g} \cdot \mathrm{mol}^{-1}\right)$ bằng phản ứng sau:
	$$
	\mathrm{MgO}(s)+2 \mathrm{HCl}(a q) \to \mathrm{MgCl}_2(a q)+H_2 O(l)
	$$
	
	Nhiệt độ ban đầu ghi nhận trong thiết bị đo (bom nhiệt lượng kế) là $29,8^{\circ} C$. Sau khi phản ứng xảy ra hoàn toàn, nhiệt độ đo được là $38,1^{\circ} C$. Phát biểu nào sau đây sai?
	\choice
	{Phản ứng trên là phản ứng thu nhiệt}
	{Phản ứng trên là phản ứng toả nhiệt}
	{Phản ứng trên có thể thuận lợi xảy ra ở điều kiện thường}
	{Giả sử nếu nước sinh ra ở thể hơi thì biến thiên enthalpy của phản ứng trên sẽ thay đồi}
	\loigiai{}
\end{ex}
%%%=============EX_6=============%%%
\begin{ex}
	Cho phương trình nhiệt hoá học sau:
	$$
	\mathrm{CH}_4(\mathrm{~g})+\mathrm{H}_2\mathrm{O}(g) \to 3 \mathrm{H}_2(\mathrm{~g})+\mathrm{CO}(g) \Delta_r \mathrm{H}_{298}^{\circ}=+209,8 \mathrm{~kJ} \cdot \mathrm{mol}^{-1}
	$$
	
	Cho biết các giá trị enthalpy tạo thành chuẩn $\left(\mathrm{kJ} \cdot \mathrm{mol}^{-1}\right)$ của: $H_2O(g)$ và $CO(g)$ lần lượt là $-241,85$ và 110,50. Giá trị enthalpy tạo thành của $CH_4(\mathrm{~g})$ là
	\choice
	{$-78,45\mathrm{~kJ} \cdot \mathrm{mol}^{-1}$}
	{$-74,53\mathrm{~kJ} \cdot \mathrm{mol}^{-1}$}
	{$+78,45\mathrm{~kJ} \cdot \mathrm{mol}^{-1}$}
	{$+74,53\mathrm{~kJ} \cdot \mathrm{mol}^{-1}$}
	\loigiai{}
\end{ex}
%%%=============EX_7=============%%%
\begin{ex}[1 điểm]
	\immini{%
		Đồ thị biểu diễn đường cong động học của phản ứng $O_2(\mathrm{~g})+2H_2(\mathrm{~g}) \to 2H_2O(g)$ như hình bên :
		Đường cong nào của hydrogen?
		\choice
		{Đường cong số (1)}
		{Đường cong số (2)}
		{Đường cong số (3)}
		{Đường cong số (2) hoặc (3) đều đúng}
	}{%
	\begin{tikzpicture}[declare function={r=6;},font=\scriptsize]
	  \draw (0,0)--({r+0.5},0) node[below,pos=0.7]{thời gian};
	  \draw (0,0)--(0,{r-1.5}) node[above,pos=0.7,sloped]{nồng độ};
	  \draw[violet] (0,0) .. controls ++(80:3cm) and ++(-180:3cm) ..(r,3)node[right]{(1)};
	  \draw[cyan] (0,3.2) .. controls ++(-70:2.8cm) and ++(-180:2.8cm) ..(r,1)node[right]{(2)};
	  \draw[red] (0,3.2) .. controls ++(-80:3cm) and ++(-180:3cm) ..(r,0.25) node[right]{(3)};
	\end{tikzpicture}
	}
	\loigiai{}
\end{ex}
%%%=============EX_8=============%%%
\begin{ex}
	Dựa vào phương trình nhiệt hóa học của phản ứng sau:
	$3\mathrm{Fe}(s)+4H_2O(l) \to \mathrm{Fe}_3O_4(\mathrm{~s})+4H_2(\mathrm{~g}) \Delta_r H_{298}^0=+26,32\mathrm{~kJ}$
	Giá trị $\Delta_r H_{298}^{\circ}$ của phản ứng: $\mathrm{Fe}_3O_4(\mathrm{~s})+4H_2(\mathrm{~g}) \to 3\mathrm{Fe}(s)+4H_2O(l)$ là
	\choice
	{$-26,32\mathrm{~kJ}$}
	{$+13,16\mathrm{~kJ}$}
	{$+19,74\mathrm{~kJ}$}
	{$-10,28\mathrm{~kJ}$}
	\loigiai{}
\end{ex}
%%%=============EX_9=============%%%
\begin{ex}
	Tốc độ phản ứng là
	\choice
	{độ biến thiên nồng độ của một trong các chất phản ứng hoặc sản phẩm trong một đơn vị thể tích}
	{độ biến thiên nồng độ của một trong các chất phản ứng hoặc sản phẩm trong một đơn vị thời gian}
	{độ biến thiên số mol của một trong các chất phản ứng hoặc sản phẩm trong một đơn vị thể tích}
	{độ biến thiên thể tích của một trong các chất phản ứng hoặc sản phẩm trong một đơn vị thời gian}
	\loigiai{}
\end{ex}
%%%=============EX_10=============%%%
\begin{ex}
	Cho phản ứng: $2\mathrm{KClO}_3(\mathrm{s}) \xrightarrow{\mathrm{MnO}_2, t^{\circ}} 2\mathrm{KCl}(s)+3O_2(\mathrm{g})$. Yếu tố không ảnh hường đến tốc độ của phản ứng trên là:
	\choice
	{Kích thước các tinh thể $\mathrm{KClO}_3$}
	{Áp suất}
	{Chất xúc tác}
	{Nhiệt độ}
	\loigiai{}
\end{ex}
%%%=============EX_11=============%%%
\begin{ex}
	Cho một mẩu đá vôi nặng 10 gam vào $200\mathrm{ml}$ dung dịch $\mathrm{HCl} 2M$. Tốc độ phản ứng ban đầu sẽ giảm khi
	\choice
	{nghiền nhỏ đá vôi trước khi cho vào}
	{thêm $100\mathrm{ml}$ dung dịch $\mathrm{HCl} 4M$}
	{giảm nhiệt độ của phản ứng}
	{cho thêm $500\mathrm{ml}$ dung dịch $\mathrm{HCl} 1M$ vào hệ ban đầu}
	\loigiai{}
\end{ex}
%%%=============EX_12=============%%%
\begin{ex}
	Trong phản ứng điều chế khí oxygen trong phòng thí nghiệm bằng cách nhiệt phân muối potassium chlorate $\left(\mathrm{KClO}_3\right)$:
	\begin{enumerate}[(a)]
	\item Dùng chất xúc tác manganese dioxide $\left(\mathrm{MnO}_2\right)$.
	\item Nung hỗn hợp potassium chlorate và manganese dioxide ở nhiệt độ cao.
	\item Dùng phương pháp dời nước để thu khí oxygen.
	\end{enumerate}
	Những biện pháp nào dưới đây được sử dụng để làm tăng tốc độ phản ứng là:
	\choice
	{a, c}
	{a, b}
	{b, c}
	{a, b, c}
	\loigiai{}
\end{ex}
%%%=============EX_13=============%%%
\begin{ex}
	Để hoà tan hết một mẫu $\mathrm{Al}$ trong dung dịch $\mathrm{HCl}$ ở $25^{\circ} C$ cần 36 phút. Cũng mẫu $\mathrm{Al}$ đó tan hết trong dung dịch acid nói trên ở $45^{\circ} C$ trong 4 phút. Hỏi để hoà tan hết mẫu $\mathrm{Al}$ đó trong dung dịch acid nói trên ở $60^{\circ} C$ thì cần thời gian bao nhiêu giây?
	\choice
	{45,465 giây}
	{56,342 giây}
	{46,188 giây}
	{38,541 giây}
	\loigiai{}
\end{ex}
%%%=============EX_14=============%%%
\begin{ex}
	Cho phản ứng $\mathrm{PbO}_2+\mathrm{HCl} \to \mathrm{PbCl}_2+\mathrm{Cl}_2+H_2O$. (Hệ số cân bằng của phương trình là các số nguyên tối giản). Hệ số cân bằng của $\mathrm{HCl}$ là:
	\choice
	{$16$}
	{$8$}
	{$6$}
	{$4$}
	\loigiai{}
\end{ex}
%%%=============EX_15=============%%%
\begin{ex}
	Cho phản ứng: $2H_{2(\mathrm{~g})}+O_{2(\mathrm{~g})} \to 2H_2O_{(g)} \Delta_r H_{298}^{\circ}=-483,64\mathrm{~kJ}$
	Cứ $2\mathrm{~mol} H_2O_{(g)}$ tạo thành theo phương trình trên sẽ:
	\choice
	{tỏa ra nhiệt lượng là $483,64\mathrm{~kJ}$}
	{tỏa ra nhiệt lượng là $967,28\mathrm{~kJ}$}
	{thu vào nhiệt lượng là $483,64\mathrm{~kJ}$}
	{thu vào nhiệt lượng là $967,28\mathrm{~kJ}$}
	\loigiai{}
\end{ex}
%%%=============EX_16=============%%%
\begin{ex}
	Nung nóng hai ống nghiệm chứa $C$ và $CH_4$, xảy ra các phản ứng sau:
	$$
	\begin{aligned}
		& C_{(s)}+O_{2(\mathrm{g})} \to CO_{2(\mathrm{g})}(1) \\
		& CH_{4(\mathrm{g})}+O_{2(\mathrm{g})} \to CO_{2(\mathrm{g})}+H_2 O(g)(2)
	\end{aligned}
	$$
	
	Khi ngừng đun nóng, cả 2 phản ứng tiếp tục xảy ra, chứng tỏ
	\choice
	{phản ứng (1) toả nhiệt, phản ứng (2) thu nhiệt}
	{phản ứng (1) thu nhiệt, phản ứng (2) toả nhiệt}
	{cả 2 phản ứng đều toả nhiệt}
	{cả 2 phản ứng đều thu nhiệt}
	\loigiai{}
\end{ex}
%%%=============EX_17=============%%%
\begin{ex}
	Phản ứng $2NO(g)+O_2(g) \to 2NO_2(g)$ có biểu thức tốc độ tức thời: $v=k \cdot C_{NO}^2\cdot C_{O_2}$. Nếu nồng độ của $NO$ giảm 2 lần, giữ nguyên nồng độ oxygen, thì tốc độ sẽ
	\choice
	{giảm 4 lần}
	{giảm 2 lần}
	{giảm 3 lần}
	{giữ nguyên}
	\loigiai{}
\end{ex}
%%%=============EX_18=============%%%
\begin{ex}
	Cho các phản ứng sau:
	\begin{enumerate}[(a)]
	\item $\mathrm{Ca}(OH)_2+\mathrm{Cl}_2\to \mathrm{CaOCl}_2+H_2O$
	\item $2NO_2+2\mathrm{NaOH} \to \mathrm{NaNO}_3+\mathrm{NaNO}_2+H_2O$
	\item $O_3+2\mathrm{Ag} \to \mathrm{Ag}_2O+O_2$
	\item $2H_2\mathrm{~S}+SO_2\to 3\mathrm{~S}+2H_2O$
	\item $4\mathrm{KClO}_3\to \mathrm{KCl}+3\mathrm{KClO}_4$
	\end{enumerate}
	Số phản ứng oxi hóa-khử là
	\choice
	{$4$}
	{$5$}
	{$2$}
	{$3$}
	\loigiai{}
\end{ex}
%%%=============EX_19=============%%%
\begin{ex}
	Số oxi hóa cùa $\mathrm{Cl}$ trong $\mathrm{NaCl}$ là:
	\choice
	{$-2$}
	{$-1$}
	{$+1$}
	{$+2$}
	\loigiai{}
\end{ex}
%%%=============EX_20=============%%%
\begin{ex}
	Cho biến thiên enthalpy của phản ứng sau ở điều kiện chuẩn:
	$$
	\mathrm{CO}(g)+1 / 2\mathrm{O}_2(\mathrm{g}) \to \mathrm{CO}_2(\mathrm{g})\ \Delta_r \mathrm{H}_{298}^{\circ}=-283,0 \mathrm{~kJ}
	$$
	
	Biết nhiệt tạo thành chuẩn của $CO_2: \Delta_f H_{298\left(CO_{2, g}\right)}^o=-393,5\mathrm{~kJ} / \mathrm{mol}$. Nhiệt tạo thành chuẩn của $CO$ là
	\choice
	{$+110,5\mathrm{~kJ}$}
	{$-110,5\mathrm{~kJ}$}
	{$-141,5\mathrm{~kJ}$}
	{$-221,0\mathrm{~kJ}$}
	\loigiai{}
\end{ex}
%%%=============EX_21=============%%%
\begin{ex}
	Cho phản ứng $SO_2+\mathrm{KMnO}_4+H_2O\to K_2SO_4+\mathrm{MnSO}_4+H_2SO_4$. (Hệ số cân bằng của phương trình là các số nguyên tối giản). Hệ số cân bằng của $K_2SO_4$ là:
	\choice
	{$2$}
	{$1$}
	{$4$}
	{$5$}
	\loigiai{}
\end{ex}
%%%=============EX_22=============%%%
\begin{ex}
	Số oxi hóa của $P$ trong $P_2O_5$ là:
	\choice
	{$+3$}
	{$+5$}
	{$-5$}
	{$-3$}
	\loigiai{}
\end{ex}
%%%=============EX_23=============%%%
\begin{ex}
	Yếu tố nào dưới đây được sử dụng để làm tăng tốc độ phản ứng khi rắc men vào tinh bột đã được nấu chín để ủ ancol (rượu)?
	\choice
	{áp suất}
	{Nồng độ}
	{Nhiệt độ}
	{Chất xúc tác}
	\loigiai{}
\end{ex}
%%%=============EX_24=============%%%
\begin{ex}
	Biểu đồ nào sau đây không biểu diễn sự phụ thuộc nồng độ chất tham gia với thời gian?
	\choice
	{\canhdongH{%
		\begin{tikzpicture}[declare function={r=6;},font=\scriptsize,scale=0.5]
			\draw (0,0)--({r+0.5},0) node[below,pos=0.7]{thời gian};
			\draw (0,0)--(0,{r-1.5}) node[above,pos=0.7,sloped]{nồng độ};
			\draw[red] (0,3.2) .. controls ++(-80:3cm) and ++(-180:3cm) ..(r,0.25);
		\end{tikzpicture}
		}
	}
	{%
		\canhdongH{\begin{tikzpicture}[declare function={r=6;},font=\scriptsize,scale=0.5]
			\draw (0,0)--({r+0.5},0) node[below,pos=0.7]{thời gian};
			\draw (0,0)--(0,{r-1.5}) node[above,pos=0.7,sloped]{nồng độ};
			\draw[violet] (0,0) .. controls ++(80:3cm) and ++(-180:3cm) ..(r,3);
		\end{tikzpicture}}
	}
	{%
		\canhdongH{\begin{tikzpicture}[declare function={r=6;},font=\scriptsize,scale=0.5]
			\draw (0,0)--({r+0.5},0) node[below,pos=0.7]{thời gian};
			\draw (0,0)--(0,{r-1.5}) node[above,pos=0.7,sloped]{nồng độ};
			\draw (0,3)--(5,0.25);
		\end{tikzpicture}}
	}
	{%
		\canhdongH{\begin{tikzpicture}[declare function={r=6;},font=\scriptsize,scale=0.5]
			\draw (0,0)--({r+0.5},0) node[below,pos=0.7]{thời gian};
			\draw (0,0)--(0,{r-1.5}) node[above,pos=0.7,sloped]{nồng độ};
			\draw[red] (0,4) .. controls ++(-80:4cm) and ++(-180:4cm) ..(r,0.25);
		\end{tikzpicture}}
	}
	\loigiai{}
\end{ex}
\Closesolutionfile{ansex}
\Closesolutionfile{ans}

%%%==========Phần trắc nghiệm đúng sai============%%%
\tieumuc{Bài Tập Trắc Nghiệm Đúng Sai}-- \textit{Trong mỗi câu có 4 ý tương ứng A, B, C, D; Học sinh chọn đúng hoặc sai.}
\Opensolutionfile{ans}[Ans/DATAM1]
\Opensolutionfile{ansbook}[Ans/DATNTF-1]
\luulgEXTF
\Opensolutionfile{ansex}[LOIGIAITN/LGTNTF-1]
%%%============EX_1==============%%%
\begin{ex}
	Cho các phát biểu sau:
	\choiceTF
	{Khi nhiệt độ tăng thì tốc độ phản ứng tăng}
	{Quá trình nước đóng băng là quá trình thu nhiệt}
	{Nhiệt tạo thành của các đơn chất bền bằng 0}
	{Nhiệt độ của hệ phản ứng sẽ tăng lên nếu phản ứng thu nhiệt}
	\loigiai{}
\end{ex}
%%%============EX_2==============%%%
\begin{ex}
	Cho 2 gam kẽm viên vào cốc đựng $50\mathrm{ml}$ dung dịch $H_2SO_44M$ ở nhiệt độ thường (25). Tốc độ của phản ứng giảm khi
	\choiceTF
	{thay 2 gam kẽm viên bằng 2 gam kẽm bột}
	{thêm $50\mathrm{ml}$ dung dịch $H_2SO_44M$ nữa}
	{thay $50\mathrm{ml}$ dung dịch $H_2SO_44M$ bằng $100\mathrm{ml}$ dung dịch $H_2SO_42M$}
	{đun nóng dung dịch}
	\loigiai{}
\end{ex}
\Closesolutionfile{ansex}
\Closesolutionfile{ansbook}
\Closesolutionfile{ans}	

%%%==========Phần tự luận============%%%
\tieumuc{Bài tập tự luận}
\Opensolutionfile{ansbt}[LOIGIAITL/LGTL-1]
\luuloigiaibt
%%%=============BT_1=============%%%
\begin{bt}
	Khí $H_2$ có thể được điều chế bằng cách cho miếng sắt vào dung dịch $\mathrm{HCl}$. Hãy đề xuất thêm một biện pháp khác để điều chế khí $H_2$ theo cách này.
\end{bt}
%%%=============BT_2=============%%%
\begin{bt}[0,5 điểm]
	Cho phản ứng đốt cháy methane sau: $CH_{4(\mathrm{~g})}+2O_{2(\mathrm{~g})} \to CO_{2(\mathrm{~g})}+2H_2O_{(g)}$ Tính nhiệt toả ra khi đốt cháy hoàn toàn $12\mathrm{~kg}$ khí methane $\left(CH_4\right)$, biết năng lượng liên kết trung bình cho trong bảng sau:
	
	\begin{tabular}{|C{0.2\textwidth}|*{3}{C{0.2\textwidth}|}}
		\hline \rowcolor{\mycolor!25}
		\textbf { Liên kết } & \textbf { Eb ($kJ/mol$) } & \textbf { Liên kết } & \textbf { Eb ($kJ/mol$) } \\
		\hline 
		$\mathrm{C}-\mathrm{C}$ & $346 $& $\mathrm{C}=\mathrm{O}$ & $799$ \\
		\hline 
		$\mathrm{C}-\mathrm{H}$ & $418$ & $\mathrm{O}-\mathrm{H}$ & $467$ \\
		\hline 
		$\mathrm{O}_2$ & $495$ & & \\
		\hline
	\end{tabular}
	\loigiai{}
\end{bt}
%%%=============BT_3=============%%%
\begin{bt}[1,0 điểm]
	Dữ liệu thí nghiệm của phản ứng: $SO_2\mathrm{Cl}_2(\mathrm{~g}) \to SO_2(\mathrm{~g})+\mathrm{Cl}_2(\mathrm{~g})$ được trình bày ở bàng sau:
	
	\begin{tabular}{|*{4}{C{0.2\textwidth}|}}
		\hline\rowcolor{\mycolor!25}
		\begin{tabular}{l} \textbf{Nồng độ (M)} \\
			\textbf{Thời gian (phút)}
		\end{tabular}
		& $\mathbf{SO}_{\mathbf{2}} \mathbf{Cl}_{\mathbf{2}}$ & $\mathbf{SO}_{\mathbf{2}}$ & $\mathbf{Cl}_{\mathbf{2}}$ \\
		\hline
		0 & 1,00 & 0 & 0 \\
		\hline
		100 & 0,87 & 0,13 & 0,13 \\
		\hline
		200 & 0,78 & $?$ & $?$ \\
		\hline
	\end{tabular}
	\begin{enumerate}[a)]
		\item Tính tốc độ trung bình của phản ứng theo $SO_2\mathrm{Cl}_2$ trong thời gian 100 phút.
		\item Sau 200 phút, nồng độ của $SO_2$ và $\mathrm{Cl}_2$ thu được là bao nhiêu?
	\end{enumerate}
	\loigiai{}
\end{bt}
\Closesolutionfile{ansbt}

\label{\x}