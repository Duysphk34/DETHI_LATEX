\section{Đề ôn thi giữa kỳ 2 toán 10}
\subsection{Phần trắc nghiệm}
Câu trắc nghiệm nhiều phương án lựa chọn. Học sinh trả lời từ
câu 1 đến câu 12. Mỗi câu hỏi học sinh \textit{chỉ chọn một} phương án.
\setcounter{ex}{0}% Reset lại số đếm câu hỏi
\Opensolutionfile{ans}[Ans/Dapan]
\hienthiloigiaiex
\begin{ex}%[0D3N1-2]%[Dự án đề kiểm tra Toán 10 GHKII-NH23-24-Dot2-Tổng Nguyễn]%[Deso10-KNTT]
	Tập xác định $\mathscr{D}$ của hàm số $y=\sqrt{x-2}$ là
	\choice
	{ $\mathscr{D}=(-\infty;2)$}
	{$\mathscr{D}=(-\infty;2]$}
	{$\mathscr{D}=(2;+\infty)$}
	{\True $\mathscr{D}=[2;+\infty)$}
	\loigiai{Hàm hàm số $y=\sqrt{x-2}$ xác định khi $x-2\geq 0\Leftrightarrow x\geq 2$.\\
		Vậy tập xác định của hàm số $y=\sqrt{x-2}$ là $\mathscr{D}=[2;+\infty)$.
		
	}
\end{ex}

\begin{ex}%[0D3N1-1]%[Dự án đề kiểm tra Toán 10 GHKII-NH23-24-Dot2-Tổng Nguyễn]%[Deso10-KNTT]
	Biết đồ thị hàm số $y=x^2+bx+1$ đi  qua điểm $A(-1;3)$. Tính $b$.
	\choice
	{\True $b=-1$}
	{$b=1$}
	{$b=3$}
	{$b=-2$}
	\loigiai{Biết đồ thị hàm số $y=x^2+bx+1$ đi  qua điểm $A(-1;3)$ nên $$(-1)^2-b+1=3\Leftrightarrow b=-1.$$
	}
\end{ex}
\begin{ex}%[0D3N2-2]%[Dự án đề kiểm tra Toán 10 GHKII-NH23-24-Dot2-Tổng Nguyễn]%[Deso10-KNTT]
	Đỉnh  $I$ của parabol $y=x^2-4x+4$ có tọa độ  là
	\choice
	{$I(1;1)$}
	{$I(-1;1)$}
	{\True $I(2;0)$}
	{$I(-1;2)$}
	\loigiai{Tọa độ đỉnh của pararbol
		\[\heva{&x=-\dfrac{(-4)}{2\cdot 1}=2\\&y=2^2-4\cdot 2+4=0} \Rightarrow I(2;0).\]
	}
\end{ex}
\begin{ex}%[0D7V1-2]%[Dự án đề kiểm tra Toán 10 GHKII-NH23-24-Dot2-Tổng Nguyễn]%[Deso10-KNTT]
	Với giá trị nào của tham số $m$ thì tam thức bậc hai $y=x^2-(m+2)x+5m+1$ không âm với mọi $x \in \mathbb{R}$?
	\choice
	{$m >16$}
	{\True $0\leq m \leq 16$}
	{$m<16$}
	{$0<m<16$}
	\loigiai{Tam thức bậc hai $y=x^2-(m+2)x+5m+1$ (có $a=1>0$) không âm với mọi $x \in \mathbb{R}$ khi 
		\[\Delta\leq 0\Leftrightarrow (m+2)^2-4(5m+1)\leq 0 \Leftrightarrow m^2-16m\leq 0\Leftrightarrow 0\leq m\leq 16.\]
	}
\end{ex}
\begin{ex}%[0D7H3-1]%[Dự án đề kiểm tra Toán 10 GHKII-NH23-24-Dot2-Tổng Nguyễn]%[Deso10-KNTT]
	Tập nghiệm của phương trình $\sqrt{x^2-3x+1}=\sqrt{x-2}$ là 
	\choice
	{$S=\left\{3;1\right\}$}
	{\True $S=\left\{3\right\}$}
	{$S=\left\{1\right\}$}
	{$S=\left\{3;6\right\}$}
	\loigiai{ $\sqrt{x^2-3x+1}=\sqrt{x-2}$  \quad (*)\\
		Bình phương hai vế của phương trình ta được
		\[x^2-3x+1=x-2\Leftrightarrow x^2-4x+3=0\Leftrightarrow x=1 \ \text{hoặc} \ x=3.\]
		Thay $x=1$, $x=3$ vào (*)  ta thấy  chỉ có $x=3$ thỏa mãn.\\
		Vậy phương trình đã cho có tập nghiệm $S=\left\{3\right\}$.
	}
\end{ex}

\begin{ex}%[0D7H3-1]%[Dự án đề kiểm tra Toán 10 GHKII-NH23-24-Dot2-Tổng Nguyễn]%[Deso10-KNTT]
	Tập nghiệm của phương trình $\sqrt{x^2-x-2}=\sqrt{2x^2+x-1}$ là
	\choice
	{$S=\left\{3\right\}$}
	{ $S=\left\{-1;2\right\}$}
	{$S=\left\{1\right\}$}
	{\True $S=\left\{-1\right\}$}
	\loigiai{ $\sqrt{x^2-x-2}=\sqrt{2x^2+x-1}$  \quad (*)\\
		Bình phương hai vế của phương trình ta được
		\[x^2-x-2=2x^2+x-1\Leftrightarrow x^2+2x+1=0\Leftrightarrow x=-1.\]
		Thay $x=-1$ vào (*)  ta thấy  $x=-1$ thỏa mãn.\\
		Vậy phương trình đã cho có tập nghiệm $S=\left\{-1\right\}$.
	}
\end{ex}
\begin{ex}%[0H9N3-1]%[Dự án đề kiểm tra Toán 10 GHKII-NH23-24-Dot2-Tổng Nguyễn]%[Deso10-KNTT]
	Trong mặt phẳng $Oxy$, cho đường thẳng $d \colon 2x+3y-4=0$. Véc-tơ nào dưới đây là véc-tơ pháp tuyến của đường thẳng $d$?
	\choice
	{$\overrightarrow{n}_1=(3;2)$}
	{\True $\overrightarrow{n}_2=(-4;-6)$}
	{$\overrightarrow{n}_3=(2;-3)$}
	{$\overrightarrow{n}_4=(-2;3)$}
	\loigiai{Đường thẳng $d \colon 2x+3y-4=0$ có véc-tơ pháp tuyến $\overrightarrow{n}=(2;3)$.\\
		Ta thấy $\overrightarrow{n}_2=-2\overrightarrow{n}=(-4;-6)$.\\
		Vậy $\overrightarrow{n}_2=(-4;-6)$ là véc-tơ pháp tuyến của $d$.
	}
\end{ex}
\begin{ex}%[0H9H3-2]%[Dự án đề kiểm tra Toán 10 GHKII-NH23-24-Dot2-Tổng Nguyễn]%[Deso10-KNTT]
	Trong mặt phẳng $Oxy$, đường thẳng đi qua hai điểm $A(-2;4)$, $B(-6;1)$ có phương trình là
	\choice
	{$3x+4y-10=0$}
	{\True $3x-4y+22=0$}
	{$3x-4y+8=0$}
	{$3x-4y-22=0$}
	\loigiai{Đường thẳng $AB$ đi qua điểm $A(-2;4)$ và có véc-tơ chỉ phương $\overrightarrow{AB}=(-4;-3)$, suy ra véc-tơ pháp tuyến là $\overrightarrow{n}=(3;-4)$. Do đó $AB$ có phương trình là
		\[3(x+2)-4(y-4)=0\Leftrightarrow 3x-4y+22=0.\]
	}
\end{ex}
\begin{ex}%[0H9H3-2]%[Dự án đề kiểm tra Toán 10 GHKII-NH23-24-Dot2-Tổng Nguyễn]%[Deso10-KNTT]
	Trong mặt phẳng $Oxy$, đường thẳng $\Delta$ đi qua điểm $M(1;-1)$ và song song với đường thẳng $d \colon x-2y+1=0$ có phương trình 
	\choice
	{$x-2y+3=0$}
	{\True $x-2y-3=0$}
	{$x-2y+5=0$}
	{$x+2y+1=0$}
	\loigiai{Vì $\Delta \parallel d$ nên có dạng $x-2y+C=0$ với $C \neq 1$.\\
		$\Delta$ đi qua điểm $M(1;-1)$ nên $1-2\cdot (-1)+C=0\Leftrightarrow C= -3$ thỏa $C \neq 1$.\\
		Vậy $\Delta \colon x-2y-3=0$.	
	}
\end{ex} 
\begin{ex}%[0H9N3-5]%[Dự án đề kiểm tra Toán 10 GHKII-NH23-24-Dot2-Tổng Nguyễn]%[Deso10-KNTT]
	Trong mặt phẳng $Oxy$, khoảng cách từ điểm $M(1;3)$ đến đường thẳng $\Delta\colon 3x+4y-5=0$ bằng 
	\choice
	{$1$}
	{\True $2$}
	{$3$}
	{$4$}
	\loigiai{
		Khoảng cách từ điểm $M(1;3)$ đến đường thẳng $\Delta\colon 3x+4y-5=0$ 
		\[\mathrm{d}(M,\Delta)=\dfrac{|3\cdot 1+4\cdot 3-5|}{\sqrt{3^2+4^2}}=2.\]
	}
\end{ex}
\begin{ex}%[0H9V4-2]%[Dự án đề kiểm tra Toán 10 GHKII-NH23-24-Dot2-Tổng Nguyễn]%[Deso10-KNTT]
	Trong mặt phẳng $Oxy$, cho tam giác $ABC$ có $A(1;-2)$, $B(1;2)$ và $C(5;2)$. Phương trình đường tròn ngoại tiếp tam giác $ABC$ là
	\choice
	{$x^2+y^2-3x+2y+1=0$}
	{$x^2+y^2-3x+1=0$}
	{$x^2+y^2-6x-1=0$}
	{\True $x^2+y^2-6x+1=0$}
	\loigiai{Ta có $\overrightarrow{BA}=(0;4)$, $\overrightarrow{BC}=(3;0)$.\\
		Ta thấy $\overrightarrow{BA}\cdot \overrightarrow{BC}=0\cdot 3+4\cdot 0=0$. Do đó $\overrightarrow{BA} \perp \overrightarrow{BC}$ hay  tam giác $ABC$ vuông tại $B$.\\
		Do đó tâm của đường tròn ngoại tiếp tam giác $ABC$ là trung điểm $I$ của cạnh $AC$. Suy ra $I(3;0)$.\\
		Bán kính của đường tròn ngoại tiếp tam giác $ABC$ là $R=\dfrac{1}{2}AC=\dfrac{1}{2}\sqrt{(5-1)^2+(2+2)^2}=2\sqrt{2}$.\\
		Vậy phương trình đường tròn tâm $I(3;0)$, bán kính $R=2\sqrt{2}$ là
		\[(x-3)^2+y^2=8\Leftrightarrow x^2+y^2-6x+1=0.\]
	}
\end{ex}
\begin{ex}%[0H9H4-3]%[Dự án đề kiểm tra Toán 10 GHKII-NH23-24-Dot2-Tổng Nguyễn]%[Deso10-KNTT]
	Trong mặt phẳng $Oxy$, tiếp tuyến với đường tròn $x^2+y^2-4x+8y-5=0$ tại điểm $A(-1;0)$ có phương trình là 
	\choice
	{$4x+3y+4=0$}
	{$3x+4y+3=0$}
	{\True $3x-4y+3=0$}
	{$-3x+y-22=0$}
	\loigiai{Đường tròn $x^2+y^2-4x+8y-5=0$ có tâm $I(2;-4)$.\\
		Tiếp tuyến với đường tròn trên tại $A(-1;0)$, véc-tơ pháp tuyến $\overrightarrow{IA}=(-3;4)$ có phương trình 
		\[-3(x+1)+4(y-0)=0\Leftrightarrow -3x+4y-3=0\Leftrightarrow 3x-4y+3=0.\]
	}
\end{ex}
\Closesolutionfile{ans}
\bangdapan{Dapan}

\subsection{Câu trắc nghiệm đúng sai}
Học sinh trả lời từ câu 1 đến câu 4.
Trong mỗi ý \circlenum{A}, \circlenum{B}, \circlenum{C} và \circlenum{D} ở mỗi câu, học sinh chọn đúng hoặc sai.
\setcounter{ex}{0}
\LGexTF
\Opensolutionfile{ansbook}[ansbook/DapanDS]
\Opensolutionfile{ans}[Ans/DapanT]
%%%============EX_1==============%%%
%%%============EX_1==============%%%
\begin{ex}%[0D3H1-1]%[Dự án đề kiểm tra khối 10, 11 GKII NH2324 - Đợt 2 - Trung Anh]%[Đề số 10 - KNTT]
	Xác định tính đúng sai của các khẳng định sau
	\choiceTF
	{\True Hàm số $y=(2m-1)x^2+x+3$ là hàm số bậc nhất khi $m\neq\dfrac{1}{2}$}
	{Hàm số $y=(4m^2-1)x^2-2x$ là hàm số bậc nhất khi $m\neq\dfrac{1}{2}$}
	{\True Hàm số $y=(m^2-2m)x^3+x^2-1$ là hàm số bậc hai khi $m=0$ hoặc $m=2$}
	{\True Hàm số $y=mx^2+(3-x)(m^2x+2)$ là hàm số bậc nhất khi $m\neq0$ và $m\neq1$}
	\loigiai{
		\begin{enumerate}
			\item Hàm số $y=(2m-1)x^2+x+3$ là hàm số bậc hai khi $2m-1\neq0\Leftrightarrow m\neq\dfrac{1}{2}$;
			\item Hàm số $y=(4m^2-1)x^2-2x$ là hàm số bậc hai khi $4m^2-1\neq0\Leftrightarrow m\neq\pm\dfrac{1}{2}$;
			\item Hàm số $y=(m^2-2m)x^3+x^2-1$ là hàm số bậc hai khi $m^2-2m=0\Leftrightarrow m=0$ hoặc $m=2$;
			\item Hàm số $y=mx^2+(3-x)(mx^2+2)=(-m^2+m)x^2+(3m^2-2)x+6$ là hàm số bậc nhất khi $-m^2+m\neq0\Leftrightarrow m\neq0$ và $m\neq1$.
		\end{enumerate}
	}
\end{ex}

%%%============EX_2==============%%%
\begin{ex}%[0D7H3-1]%[Dự án đề kiểm tra khối 10, 11 GKII NH2324 - Đợt 2 - Trung Anh]%[Đề số 10 - KNTT]
	Cho các phương trình sau $\sqrt{x^2-x-2}=\sqrt{-x^2+2x+3}\,\, (1)$ và $\sqrt{x+2}=\sqrt{3x^2-x+1}\,\, (2)$. Khi đó
	\choiceTF
	{\True Phương trình $(1)$ có hai nghiệm phân biệt}
	{Phương trình $(2)$ có một nghiệm}
	{\True Tổng các nghiệm của phương trình $(1)$ bằng $\dfrac{3}{2}$}
	{\True Tổng các nghiệm của phương trình $(2)$ bằng $\dfrac{2}{3}$}
	\loigiai{
		Bình phương hai vế của phương trình $(1)$, ta được
		\[x^2-x-2=-x^2+2x+3\Leftrightarrow 2x^2-3x-5=0\Leftrightarrow\hoac{&x=-1\\&x=\dfrac{5}{2}.}\]
		Thay các giá trị $x=-1$ và $x=\dfrac{5}{2}$ vào phương trình đã cho, ta thấy chúng đều thỏa mãn.\\
		Vậy tập nghiệm của phương trình $(1)$: $S_1=\left\{-1;\dfrac{5}{2}\right\}$.\\
		Bình phương hai vế của phương trình $(2)$, ta được
		\[3x^2-x+1=x+2\Leftrightarrow 3x^2-2x-1=0\Leftrightarrow\hoac{&x=1\\&x=-\dfrac{1}{3}.}\]
		Thay các giá trị $x=1$ và $x=-\dfrac{1}{3}$ vào phương trình đã cho, ta thấy chúng đều thỏa mãn.\\
		Vậy tập nghiệm của phương trình $(2)$ là $S_2=\left\{-\dfrac{1}{3};1\right\}$.
	}
\end{ex}

%%%============EX_3==============%%%
\begin{ex}%[0H9V3-2]%[Dự án đề kiểm tra khối 10, 11 GKII NH2324 - Đợt 2 - Trung Anh]%[Đề số 10 - KNTT]
	Trong mặt phẳng tọa độ $Oxy$, cho hình chữ nhật $ABCD$ có tâm $I(6;2)$ và các điểm $M(1;5)$, $N(3;4)$ lần lượt thuộc các đường thẳng $AB$, $BC$. Biết rằng trung điểm $E$ của cạnh $CD$ thuộc đường thẳng $\triangle: x+y-5=0$ và hoành độ của điểm $E$ nhỏ hơn $7$. Khi đó
	\choiceTF
	{\True Phương trình đường thẳng $BC$ là $x-3=0$}
	{Phương trình đường thẳng $AB$ là $x+y-6=0$}
	{\True Tọa độ điểm $A$ là $(9;5)$}
	{Tọa độ điểm $B$ là $(3;3)$}
	\loigiai{
		\immini{Gọi $P$ là điểm đối xứng với $M(1;5)$ qua $I(6;2)$ suy ra $P=(11;-1)$ và $P$ thuộc đường thẳng $CD$.\\
			Ta có $E$ thuộc $\triangle$ nên giả sử $E(t;5-t)$. Khi đó $\overrightarrow{IE}=(t-6;3-t)$ và $\overrightarrow{PE}=(t-11;6-t)$.\\
			Vì $E$ là trung điểm của $CD$ nên $IE\perp PE$. Do đó ta có
			\[\overrightarrow{IE}\cdot\overrightarrow{PE}=0\Leftrightarrow(t-6)(t-11)+(3-t)(6-t)=0\Leftrightarrow t^2-13t+42=0.\]
			Suy ra $t=6$ hoặc $t=7$. Vì hoành độ của điểm $E$ nhỏ hơn $7$ nên $E(6;-1)$.\\
			Đường thẳng $BC$ đi qua $N(3;4)$ và vuông góc với $CD$ nên phương trình $BC$ là $x-3=0$.\\
			Đường thẳng $AB$ đi qua $M(1;5)$ và song song với $CD$ nên phương trình $AB$ là $y-5=0$.\\
			Từ phương trình các cạnh, ta tìm được $A(9;5)$, $B(3;5)$, $C(3;-1)$ và $D(9;-1)$.
			
		}
		{
			\begin{tikzpicture}[line join=round, line cap=round,thick]
				\coordinate (A) at (0,3);
				\coordinate (B) at (5,3);
				\coordinate (D) at (0,0);
				\coordinate (C) at ($(B)+(D)-(A)$);
				\coordinate (I) at (intersection of A--C and B--D);
				\coordinate (E) at ($(C)!1/2!(D)$);
				\coordinate (F) at ($(A)!1/2!(B)$);
				\coordinate (P) at ($(E)!1/3!(D)$);
				\coordinate (M) at ($(F)!1/3!(B)$);
				\draw(A)--(B)--(C)--(D)--cycle;
				\draw (A)--(C) (B)--(D) (M)--(P);
				\foreach \i/\g in {A/90,B/90,C/-90,D/-90,I/-90,E/-90,M/90,P/-90}{\draw[fill=white](\i) circle (1.5pt) ($(\i)+(\g:3mm)$) node[scale=1]{$\i$};}
			\end{tikzpicture}
		}
	}
\end{ex}

%%%============EX_4==============%%%
\begin{ex}%[0H9V4-5]%[Dự án đề kiểm tra khối 10, 11 GKII NH2324 - Đợt 2 - Trung Anh]%[Đề số 10 - KNTT]
	Cho đường tròn $(C)$ có phương trình $x^2+y^2-6x+2y+6=0$ và hai điểm $A(1;-1)$, $B(1;3)$. Khi đó
	\choiceTF
	{\True Điểm $A$ thuộc đường tròn}
	{Điểm $B$ nằm trong đường tròn}
	{\True $x=1$ là phương trình tiếp tuyến của $(C)$ tại $A$}
	{Qua $B$ kẻ được hai tiếp tuyến với $(C)$ có phương trình là $x=1$ và $3x+4y-12=0$}
	\loigiai{
		Đường tròn $(C)$ có tâm $I(3;-1)$ và bán kính $R=\sqrt{9+1-6}=2$.
		\begin{itemize}
			\item Ta có $IA=2=R$, $IB=2\sqrt{5}>R$ suy ra điểm $A$ thuộc đường tròn và điểm $B$ nằm ngoài đường tròn;
			\item Tiếp tuyến của $(C)$ tại điểm $A$ nhận $\overrightarrow{AI}=(2;0)$ là một VTPT có phương trình là \[2(x-(-1))+0(y+1)=0\,\,\text{hay}\,\, x=1.\]
			\item Phương trình đường thẳng $\triangle$ đi qua $B$ có dạng $a(x-1)+b(y-3)=0$ (với $a^2+b^2\neq0$) hay $ax+by-a-3b=0$.\\
			Đường thẳng $\triangle$ là tiếp tuyến của đường tròn nên suy ra $\mathrm{d}(I,\triangle)=R$. Điều này tương đương \[\dfrac{|3a-b-a-3b|}{\sqrt{a^2+b^2}}=2\Leftrightarrow(a-2b)^2=a^2+b^2\Leftrightarrow 3b^2-4ab=0\Leftrightarrow\hoac{&b=0\\&3b=4a.}\]
			Với $b=0$, chọn $a=1$. Khi đó phương trình tiếp tuyến là $x=1$.\\
			Với $3b=4a$, chọn $a=3\Rightarrow b=4$. Khi đó phương trình tiếp tuyến là $3x+4y-15=0$.\\
			Vậy qua $B$ kẻ được hai tiếp tuyến với $(C)$ có phương trình là $x=1$ và $3x+4y-15=0$.
		\end{itemize}
	}
\end{ex}
\Closesolutionfile{ans}
\Closesolutionfile{ansbook}

\begin{center}
	\textbf{\textsf{BẢNG ĐÁP ÁN ĐÚNG SAI}}
\end{center}
\input{Ansbook/DapanDS}

\subsection{Phần tự luận}

\hienthiloigiaibt
%%Câu 1.
\begin{bt}%[0D3V1-7]
	Một ngân hàng A thông báo phí dịch vụ SMS Bank hàng tháng như sau $9\,000$ đồng với $0-15$ tin nhắn; $30\,000$ đồng với $16-50$ tin nhắn; $55\,000$ đồng với $51-100$ tin nhắn và $7\,000$ đồng với mỗi tin nhắn từ tin nhắn thứ $101$ trở lên. Khách hàng B phải trả $125\,000$ đồng tiền SMS Banking trong tháng. Số lượng tin nhắn của khách hàng B trong tháng là bao nhiêu?
	\loigiai{
		Gọi $x \in \mathbb{N}$ là số tin nhắn được dùng, $f(x)$ là giá tiền khi dùng $x$ tin nhắn.\\
		Ta có
		$$f(x)= \heva{&9000 & \text { khi } x \in[0;15], \\ &30\,000 & \text { khi } x \in[16;50], \\ &55\,000 & \text { khi } x \in[51;100], \\ &55\,000+(x-100) \cdot 7\,000 & \text { khi } x \geq 101.}
		$$
		Do khách hàng $B$ dùng hết $125\,000$ nên khách hàng đã sử dụng tới mức thứ tư của hàm giá, tức là
		$$55\,000+(x-100) \cdot 7\,000=125\,000\Leftrightarrow x=110.$$
		Vậy khách hàng B xử dụng $110$ tin nhắn.
	}
\end{bt}
%%Câu 2.
\begin{bt}%[0D3V2-2]
	Khi nuôi cá thí nghiệm trong hồ, một nhà sinh học tìm được quy luật rằng: Nếu trên mỗi đơn vị diện tích của mặt hồ có $n$ con cá thì trung bình mỗi con cá sau một vụ cân nặng $P(n)=360-10n$ (đơn vị khối lượng). Hỏi người nuôi phải thả bao nhiêu con cá trên một đơn vị diện tích để trọng lượng cá sau mỗi vụ thu được là nhiều nhất?
	\loigiai{
		Tổng trọng lượng cá thu được sau một vụ là $$T(n)=n(360-10n)=-10n^2+360n.$$
		Đây là hàm số bậc hai (theo $n$) có $a=-10<0, b=360 \Rightarrow-\dfrac{b}{2a}=18, T(18)=3\,240$.\\
		Vậy, người nuôi cần thả $18$ con cá trên một đơn vị diện tích để đạt tổng trọng lượng cá lớn nhất là $3\,240$ (đơn vị khối lượng).
	}
\end{bt}
%%Câu 3.
\begin{bt}%[0D7V3-6]
	Một công ty muốn làm một đường ống dẫn từ một điểm $A$ trên bờ đến một điểm $B$ trên một hòn đảo. Hòn đảo cách bờ biển $6$ km. Giá để xây đường ống trên bờ là $50\,000$ USD mỗi km, giá để xây đường ống dưới nước là $130\,000$ USD mỗi km; $B$ là điểm trên bờ biển sao cho $BB'$ vuông góc với bờ biển. Khoảng cách từ $A$ đến $B'$ là $9$ km. Biết rằng chi phí làm đường ống này là $1\,170\,000$ USD. Hỏi vị trí $C$ cách vị trí $A$ bao nhiêu km?	
	\loigiai{
		\begin{center}
			\begin{tikzpicture}[scale=1,font=\footnotesize,line join=round,line cap=round,>=stealth]
				\path
				(0,0) coordinate (B')
				(6,0) coordinate (A)
				(0,3) coordinate (B)
				++(130:1) coordinate (I)
				(3.5,0)coordinate (C)
				;
				\draw (A)--(B)--(B')--cycle (B)--(C);
				\draw (I) let \p1=($(I)-(B)$) in circle ({veclen(\x1,\y1)});
				\foreach \p/\q in {A/0,B/140,C/-90,B'/-90}
				\fill[black] (\p) circle (1.0pt) ($(\p)+(\q:2.5mm)$) node{$\p$};
				\fill (I) circle (1.0pt)node[above]{Đảo};	
				\node at ($(B')!0.5!(C)$)[below]{$x$ km};
				\node at ($(A)!0.5!(C)$)[below]{$(9-x)$ km};	
				\node at ($(B)!0.5!(B')$)[left]{$6$ km};	
				\node at ($(A)+(130:3.5)$)[below]{biển};
				\node at ($(A)!0.5!(B')$)[below=0.5cm]{Bờ biển};
			\end{tikzpicture}
		\end{center}
		Gọi $x=B'C(0 \leq x \leq 9)$, khi đó $BC=\sqrt{x^2+36}$.\\
		Số tiền xây đường ống trên bờ $(9-x) \times 50000$; số tiền xây đường ống dưới biển:
		$130\,000\times \sqrt{x^2+36}$.\\ 	
		Tổng chi phí bỏ ra để làm đường ống là $(9-x) \times 50000+130000 \times \sqrt{x^2+36}$.\\
		Theo giả thiết 
		\begin{eqnarray*}
			&&(9-x) \cdot 50\,000+130\,000 \sqrt{x^2+36}=170000\\
			&\Leftrightarrow& 5(9-x)+13 \sqrt{x^2+36}=17 \\
			&\Leftrightarrow& 13 \sqrt{x^2+36}=5x+72\\
			&\Leftrightarrow&\heva{& 5x+72 \geq 0\\& 169\left(x^2+36\right)=25x^2+720x+5184}\\
			&\Leftrightarrow&\heva{
				& x \geq-\dfrac{72}{5} \\
				& 144x^2-720x+900=0} \Leftrightarrow x=\dfrac{5}{2}.
		\end{eqnarray*}
		Ta có $B'C=2{,}5 \mathrm{~km} \Rightarrow AC=9-2{,}5=6{,}5 \mathrm{~km}$.\\
		Vậy, ví trí $C$ cách vị trí $A$ một khoảng bằng $6{,}5$ km.		
	}
\end{bt}
%%Câu 4.
\begin{bt}%[0H9V3-6]
	Cho ba điểm $A(-1;4)$, $B(1;1)$, $C(3;-1)$. Tìm điểm $N$ thuộc trục hoành sao cho $|NA-NC|$ bé nhất.
	\loigiai{
		Ta thấy $y_A \cdot y_C=4 \cdot(-1)<0$ nên $A,C$ nằm khác phía so với trục $Ox$.\\
		Lấy điểm $C'$ đối xứng với $C$ qua $Ox$. Suy ra $C'(3;1)$ và $C$, $A$ cùng phía so với $Ox$.\\
		Ta có $N \in Ox \Rightarrow N C=N C'$. Vì vậy $|NA-NC|=\left|NA-NC'\right| \leq AC'$.\\
		Suy ra $|NA-NC|_{\max}=AC'$; giá trị lớn nhất này đạt được khi $A,C',N$ thẳng hàng ($N$ nằm ngoài $A$, $C'$).\\
		Gọi $N(a;0) \in Ox \Rightarrow \overrightarrow{AN}=(a+1;-4), \overrightarrow{AC}=(4;-3)$.\\
		Vì $\overrightarrow{AN}, \overrightarrow{AC}$ cùng phương nên $\dfrac{a+1}{4}=\dfrac{-4}{-3} \Leftrightarrow-3a-3=-16 \Leftrightarrow a=\dfrac{13}{3}$.\\
		Vậy $N\left(\dfrac{13}{3};0\right)$ thỏa mãn đề bài.		
	}
\end{bt}
%%Câu 5.
\begin{bt}%[0H9V3-6]
	Cho $A(1;6)$, $B(-3;4)$, $\Delta\colon \heva{& x=1+t \\	& y=1+2t} (t \in \mathbb{R})$. Tìm $N \in \Delta$ sao cho khoảng cách từ gốc tọa độ $O$ đến $N$ nhỏ nhất.
	\loigiai{
		$N \in \Delta$ để $ON$ nhỏ nhất thì $ON \perp \Delta$
		Ta có $N \in \Delta \Rightarrow N(1+t;1+2t), t \in \mathbb{R}$.
		Suy ra $\overrightarrow{ON}=(1+t;1+2t)$.\\
		Véc-tơ chỉ phương của $\Delta$ là $\overrightarrow{u}_{\Delta}=(1;2)$.
		Vì $ON \perp \Delta$ nên  
		$$\overrightarrow{ON} \perp \overrightarrow{u}_{\Delta}
			\Leftrightarrow \overrightarrow{ON} \cdot \overrightarrow{u_{\Delta}}=0 \Leftrightarrow 1(1+t)+2(1+2t)=0 \Leftrightarrow t=-\dfrac{3}{5}.
		$$
		Vậy $N\left(\dfrac{2}{5};-\dfrac{1}{5}\right)$.
	}
\end{bt}
%%Câu 6.
\begin{bt}%[0H9C3-8]
	Trên màn hình rađa của đài kiểm soát không lưu của sân bay $A$ có hệ trục toạ độ $Oxy$, trong đó đơn vị trên mỗi trục tính theo kilômét và đài kiểm soát coi là gốc toạ độ $O$. Nếu máy bay bay trong phạm vi cách đài kiểm soát $200$ km thì sẽ hiện trên màn hình rađa. Một máy bay khởi hành từ sân bay $B$ lúc $7$ giờ $30$ phút. Sau thời gian $t$ (giờ), vị trí của máy bay được xác định bởi điểm $M$ có toạ độ như sau $\Delta\colon\heva{& x=410-460t \\ & y=1200-460t.}$. Hỏi lúc mấy giờ máy bay bay gần đài kiểm soát không lưu của sân bay $A$ nhất?
	\begin{center}
		\begin{tikzpicture}[scale=1, font=\footnotesize, line join=round, line cap=round,>=stealth]
			\def\xmin{-3} \def\xmax{3}
			\def\ymin{-3} \def\ymax{3}			
			\draw[->] (\xmin,0)--(\xmax,0) node [below]{$x$};
			\draw[->] (0,\ymin)--(0,\ymax) node [left]{$y$};
			\node at (0,0) [below left]{$O$};
			\node at (0,0) [above left]{$A$};
			\clip (\xmin+0.1,\ymin+0.1) rectangle (\xmax-0.1,\ymax-0.1);
			\draw (0,0) circle(2cm);
			\fill (0,2) circle (1.0pt) node[above right]{$200$}
			(2,0) circle (1.0pt) node[below right]{$200$}
			(0,-2) circle (1.0pt) node[below right]{$-200$}
			(-2,0) circle (1.0pt) node[below left]{$-200$};
		\end{tikzpicture}
	\end{center}
	\loigiai{
	Giả sử máy bay di chuyển theo đường thẳng $\Delta\colon \heva{& x=410-460t \\ & y=1200-460t.}$\\
	Gọi $d$ là đường thẳng đi qua $O$ và vuông góc với $\Delta$. Véc-tơ pháp tuyến của $d$ là $\overrightarrow{n}=(-460;-460)$.\\
	Phương trình của $d$ là $-460(x-0)-460(y-0)=0 \Rightarrow x+y=0$.\\
	Giả sử $d$ vuông góc với $\Delta$ tại $H$.\\
	Suy ra $H$ là vị trí máy bay gần đài kiểm soát nhất, ta có $410-460t+1200-460t=0 \Rightarrow t=1{,}75$.\\
	Vậy lúc $9$ giờ $15$ phút máy bay gần trạm kiểm soát nhất.
	}
\end{bt}