%%%=============Bắt Đầu_BT_1=============%%%
\begin{bt}
	 Cho $1.68$ gam Mg tác dụng vừa đủ với $600$ ml dung dịch $HCl$.Sau khi phản ứng xảy ra hoàn toàn thu được V (lít) $H_2$ (đktc)
	\begin{enumerate}
		\item Viết phương trình hóa học xảy ra?
		\item Tính V
		\item Tính nồng độ mol của dung dịch axit đã dùng?
	\end{enumerate}
	\loigiai{
	\begin{enumerate}
	\item Phương trình phản ứng\\
		$\begin{matrix}
		 Mg&+&2HCl&\xrightarrow&MgCl_2&+&H_2\downarrow[1.65][thick]\\
		0.07&&\rightarrow 0.14&&\rightarrow 0.07&&\rightarrow 0.07
		\end{matrix}$
	\item Ta có $n_{Mg}=\dfrac{1.68}{24}=0.07$ (mol)
	\\
	 Theo phương trình hóa học, ta có $n_{H_2}=n_{Mg}=0.07$ (mol)
	\\
	 $\Rightarrow V_{H_2}=0.07\cdot 22{,}4=1.568$ (lít)
	\item Ta có $n_{HCl}=2n_{Mg}=2\cdot0.07=0.14$ (mol)
	\\$\Rightarrow {C_{M}}_{\left(HCl\right)}=\dfrac{0.14}{0.6}=0.233$ (M)
	\end{enumerate}
	}
\end{bt}
%%%=============Kết Thúc_BT_1=============%%%
%%%=============Bắt Đầu_BT_2=============%%%
\begin{bt}
	 Cho $0.48$ gam Mg tác dụng vừa đủ với $500$ ml dung dịch $HCl$.Sau khi phản ứng xảy ra hoàn toàn thu được V (lít) $H_2$ (đktc)
	\begin{enumerate}
		\item Viết phương trình hóa học xảy ra?
		\item Tính V
		\item Tính nồng độ mol của dung dịch axit đã dùng?
	\end{enumerate}
	\loigiai{
	\begin{enumerate}
	\item Phương trình phản ứng\\
		$\begin{matrix}
		 Mg&+&2HCl&\xrightarrow&MgCl_2&+&H_2\downarrow[1.65][thick]\\
		0.02&&\rightarrow 0.04&&\rightarrow 0.02&&\rightarrow 0.02
		\end{matrix}$
	\item Ta có $n_{Mg}=\dfrac{0.48}{24}=0.02$ (mol)
	\\
	 Theo phương trình hóa học, ta có $n_{H_2}=n_{Mg}=0.02$ (mol)
	\\
	 $\Rightarrow V_{H_2}=0.02\cdot 22{,}4=0.448$ (lít)
	\item Ta có $n_{HCl}=2n_{Mg}=2\cdot0.02=0.04$ (mol)
	\\$\Rightarrow {C_{M}}_{\left(HCl\right)}=\dfrac{0.04}{0.5}=0.08$ (M)
	\end{enumerate}
	}
\end{bt}
%%%=============Kết Thúc_BT_2=============%%%
%%%=============Bắt Đầu_BT_3=============%%%
\begin{bt}
	 Cho $0.96$ gam Mg tác dụng vừa đủ với $300$ ml dung dịch $HCl$.Sau khi phản ứng xảy ra hoàn toàn thu được V (lít) $H_2$ (đktc)
	\begin{enumerate}
		\item Viết phương trình hóa học xảy ra?
		\item Tính V
		\item Tính nồng độ mol của dung dịch axit đã dùng?
	\end{enumerate}
	\loigiai{
	\begin{enumerate}
	\item Phương trình phản ứng\\
		$\begin{matrix}
		 Mg&+&2HCl&\xrightarrow&MgCl_2&+&H_2\downarrow[1.65][thick]\\
		0.04&&\rightarrow 0.08&&\rightarrow 0.04&&\rightarrow 0.04
		\end{matrix}$
	\item Ta có $n_{Mg}=\dfrac{0.96}{24}=0.04$ (mol)
	\\
	 Theo phương trình hóa học, ta có $n_{H_2}=n_{Mg}=0.04$ (mol)
	\\
	 $\Rightarrow V_{H_2}=0.04\cdot 22{,}4=0.896$ (lít)
	\item Ta có $n_{HCl}=2n_{Mg}=2\cdot0.04=0.08$ (mol)
	\\$\Rightarrow {C_{M}}_{\left(HCl\right)}=\dfrac{0.08}{0.3}=0.267$ (M)
	\end{enumerate}
	}
\end{bt}
%%%=============Kết Thúc_BT_3=============%%%
%%%=============Bắt Đầu_BT_4=============%%%
\begin{bt}
	 Cho $2.16$ gam Mg tác dụng vừa đủ với $500$ ml dung dịch $HCl$.Sau khi phản ứng xảy ra hoàn toàn thu được V (lít) $H_2$ (đktc)
	\begin{enumerate}
		\item Viết phương trình hóa học xảy ra?
		\item Tính V
		\item Tính nồng độ mol của dung dịch axit đã dùng?
	\end{enumerate}
	\loigiai{
	\begin{enumerate}
	\item Phương trình phản ứng\\
		$\begin{matrix}
		 Mg&+&2HCl&\xrightarrow&MgCl_2&+&H_2\downarrow[1.65][thick]\\
		0.09&&\rightarrow 0.18&&\rightarrow 0.09&&\rightarrow 0.09
		\end{matrix}$
	\item Ta có $n_{Mg}=\dfrac{2.16}{24}=0.09$ (mol)
	\\
	 Theo phương trình hóa học, ta có $n_{H_2}=n_{Mg}=0.09$ (mol)
	\\
	 $\Rightarrow V_{H_2}=0.09\cdot 22{,}4=2.016$ (lít)
	\item Ta có $n_{HCl}=2n_{Mg}=2\cdot0.09=0.18$ (mol)
	\\$\Rightarrow {C_{M}}_{\left(HCl\right)}=\dfrac{0.18}{0.5}=0.36$ (M)
	\end{enumerate}
	}
\end{bt}
%%%=============Kết Thúc_BT_4=============%%%
%%%=============Bắt Đầu_BT_5=============%%%
\begin{bt}
	 Cho $1.2$ gam Mg tác dụng vừa đủ với $400$ ml dung dịch $HCl$.Sau khi phản ứng xảy ra hoàn toàn thu được V (lít) $H_2$ (đktc)
	\begin{enumerate}
		\item Viết phương trình hóa học xảy ra?
		\item Tính V
		\item Tính nồng độ mol của dung dịch axit đã dùng?
	\end{enumerate}
	\loigiai{
	\begin{enumerate}
	\item Phương trình phản ứng\\
		$\begin{matrix}
		 Mg&+&2HCl&\xrightarrow&MgCl_2&+&H_2\downarrow[1.65][thick]\\
		0.05&&\rightarrow 0.1&&\rightarrow 0.05&&\rightarrow 0.05
		\end{matrix}$
	\item Ta có $n_{Mg}=\dfrac{1.2}{24}=0.05$ (mol)
	\\
	 Theo phương trình hóa học, ta có $n_{H_2}=n_{Mg}=0.05$ (mol)
	\\
	 $\Rightarrow V_{H_2}=0.05\cdot 22{,}4=1.12$ (lít)
	\item Ta có $n_{HCl}=2n_{Mg}=2\cdot0.05=0.1$ (mol)
	\\$\Rightarrow {C_{M}}_{\left(HCl\right)}=\dfrac{0.1}{0.4}=0.25$ (M)
	\end{enumerate}
	}
\end{bt}
%%%=============Kết Thúc_BT_5=============%%%
%%%=============Bắt Đầu_BT_6=============%%%
\begin{bt}
	 Cho $2.4$ gam Mg tác dụng vừa đủ với $500$ ml dung dịch $HCl$.Sau khi phản ứng xảy ra hoàn toàn thu được V (lít) $H_2$ (đktc)
	\begin{enumerate}
		\item Viết phương trình hóa học xảy ra?
		\item Tính V
		\item Tính nồng độ mol của dung dịch axit đã dùng?
	\end{enumerate}
	\loigiai{
	\begin{enumerate}
	\item Phương trình phản ứng\\
		$\begin{matrix}
		 Mg&+&2HCl&\xrightarrow&MgCl_2&+&H_2\downarrow[1.65][thick]\\
		0.1&&\rightarrow 0.2&&\rightarrow 0.1&&\rightarrow 0.1
		\end{matrix}$
	\item Ta có $n_{Mg}=\dfrac{2.4}{24}=0.1$ (mol)
	\\
	 Theo phương trình hóa học, ta có $n_{H_2}=n_{Mg}=0.1$ (mol)
	\\
	 $\Rightarrow V_{H_2}=0.1\cdot 22{,}4=2.24$ (lít)
	\item Ta có $n_{HCl}=2n_{Mg}=2\cdot0.1=0.2$ (mol)
	\\$\Rightarrow {C_{M}}_{\left(HCl\right)}=\dfrac{0.2}{0.5}=0.4$ (M)
	\end{enumerate}
	}
\end{bt}
%%%=============Kết Thúc_BT_6=============%%%
%%%=============Bắt Đầu_BT_7=============%%%
\begin{bt}
	 Cho $1.44$ gam Mg tác dụng vừa đủ với $500$ ml dung dịch $HCl$.Sau khi phản ứng xảy ra hoàn toàn thu được V (lít) $H_2$ (đktc)
	\begin{enumerate}
		\item Viết phương trình hóa học xảy ra?
		\item Tính V
		\item Tính nồng độ mol của dung dịch axit đã dùng?
	\end{enumerate}
	\loigiai{
	\begin{enumerate}
	\item Phương trình phản ứng\\
		$\begin{matrix}
		 Mg&+&2HCl&\xrightarrow&MgCl_2&+&H_2\downarrow[1.65][thick]\\
		0.06&&\rightarrow 0.12&&\rightarrow 0.06&&\rightarrow 0.06
		\end{matrix}$
	\item Ta có $n_{Mg}=\dfrac{1.44}{24}=0.06$ (mol)
	\\
	 Theo phương trình hóa học, ta có $n_{H_2}=n_{Mg}=0.06$ (mol)
	\\
	 $\Rightarrow V_{H_2}=0.06\cdot 22{,}4=1.344$ (lít)
	\item Ta có $n_{HCl}=2n_{Mg}=2\cdot0.06=0.12$ (mol)
	\\$\Rightarrow {C_{M}}_{\left(HCl\right)}=\dfrac{0.12}{0.5}=0.24$ (M)
	\end{enumerate}
	}
\end{bt}
%%%=============Kết Thúc_BT_7=============%%%
%%%=============Bắt Đầu_BT_8=============%%%
\begin{bt}
	 Cho $1.92$ gam Mg tác dụng vừa đủ với $500$ ml dung dịch $HCl$.Sau khi phản ứng xảy ra hoàn toàn thu được V (lít) $H_2$ (đktc)
	\begin{enumerate}
		\item Viết phương trình hóa học xảy ra?
		\item Tính V
		\item Tính nồng độ mol của dung dịch axit đã dùng?
	\end{enumerate}
	\loigiai{
	\begin{enumerate}
	\item Phương trình phản ứng\\
		$\begin{matrix}
		 Mg&+&2HCl&\xrightarrow&MgCl_2&+&H_2\downarrow[1.65][thick]\\
		0.08&&\rightarrow 0.16&&\rightarrow 0.08&&\rightarrow 0.08
		\end{matrix}$
	\item Ta có $n_{Mg}=\dfrac{1.92}{24}=0.08$ (mol)
	\\
	 Theo phương trình hóa học, ta có $n_{H_2}=n_{Mg}=0.08$ (mol)
	\\
	 $\Rightarrow V_{H_2}=0.08\cdot 22{,}4=1.792$ (lít)
	\item Ta có $n_{HCl}=2n_{Mg}=2\cdot0.08=0.16$ (mol)
	\\$\Rightarrow {C_{M}}_{\left(HCl\right)}=\dfrac{0.16}{0.5}=0.32$ (M)
	\end{enumerate}
	}
\end{bt}
%%%=============Kết Thúc_BT_8=============%%%
%%%=============Bắt Đầu_BT_9=============%%%
\begin{bt}
	 Cho $0.24$ gam Mg tác dụng vừa đủ với $600$ ml dung dịch $HCl$.Sau khi phản ứng xảy ra hoàn toàn thu được V (lít) $H_2$ (đktc)
	\begin{enumerate}
		\item Viết phương trình hóa học xảy ra?
		\item Tính V
		\item Tính nồng độ mol của dung dịch axit đã dùng?
	\end{enumerate}
	\loigiai{
	\begin{enumerate}
	\item Phương trình phản ứng\\
		$\begin{matrix}
		 Mg&+&2HCl&\xrightarrow&MgCl_2&+&H_2\downarrow[1.65][thick]\\
		0.01&&\rightarrow 0.02&&\rightarrow 0.01&&\rightarrow 0.01
		\end{matrix}$
	\item Ta có $n_{Mg}=\dfrac{0.24}{24}=0.01$ (mol)
	\\
	 Theo phương trình hóa học, ta có $n_{H_2}=n_{Mg}=0.01$ (mol)
	\\
	 $\Rightarrow V_{H_2}=0.01\cdot 22{,}4=0.224$ (lít)
	\item Ta có $n_{HCl}=2n_{Mg}=2\cdot0.01=0.02$ (mol)
	\\$\Rightarrow {C_{M}}_{\left(HCl\right)}=\dfrac{0.02}{0.6}=0.033$ (M)
	\end{enumerate}
	}
\end{bt}
%%%=============Kết Thúc_BT_9=============%%%
%%%=============Bắt Đầu_BT_10=============%%%
\begin{bt}
	 Cho $0.72$ gam Mg tác dụng vừa đủ với $400$ ml dung dịch $HCl$.Sau khi phản ứng xảy ra hoàn toàn thu được V (lít) $H_2$ (đktc)
	\begin{enumerate}
		\item Viết phương trình hóa học xảy ra?
		\item Tính V
		\item Tính nồng độ mol của dung dịch axit đã dùng?
	\end{enumerate}
	\loigiai{
	\begin{enumerate}
	\item Phương trình phản ứng\\
		$\begin{matrix}
		 Mg&+&2HCl&\xrightarrow&MgCl_2&+&H_2\downarrow[1.65][thick]\\
		0.03&&\rightarrow 0.06&&\rightarrow 0.03&&\rightarrow 0.03
		\end{matrix}$
	\item Ta có $n_{Mg}=\dfrac{0.72}{24}=0.03$ (mol)
	\\
	 Theo phương trình hóa học, ta có $n_{H_2}=n_{Mg}=0.03$ (mol)
	\\
	 $\Rightarrow V_{H_2}=0.03\cdot 22{,}4=0.672$ (lít)
	\item Ta có $n_{HCl}=2n_{Mg}=2\cdot0.03=0.06$ (mol)
	\\$\Rightarrow {C_{M}}_{\left(HCl\right)}=\dfrac{0.06}{0.4}=0.15$ (M)
	\end{enumerate}
	}
\end{bt}
%%%=============Kết Thúc_BT_10=============%%%