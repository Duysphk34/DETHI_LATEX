%%%Tùy chọn 1: Kì thi
%%%Tùy chọn 2: Môn
%%%Tùy chọn 3: lớp
%%%Tùy chọn 4: Sở/Phòng
%%%Tùy chọn 5: Ngày thi
%\pgfmathtruncatemacro{\x}{random(101,401)}
\begin{name}[Kiểm tra giữa kì II][Toán][7][Sở Giáo dục và Đào tạo]{Trường THCS  }{2023 - 2024}
\end{name}
\tieumuc{Trắc nghiệm (2 điểm)}
\Opensolutionfile{ans}[Ans/DATN-7CKII-De-04]
%\luuloigiaiex
%\hienthiloigiaiex
\taoNdongke[4]{ex}
\Opensolutionfile{ansex}[LOIGIAITN/LGTN-7CKII-De-04]
%%%=============EX_1=============%%%
\begin{ex}
	Điểm kiểm tra môn toán của học sinh trong tồ 1 và tô 2 của lớp 7A được liệt kê ở bảng sau:
	\begin{center}
		\begin{tabular}{|*{10}{C{0.075\linewidth}}|}
			\hline
			8 & 9 & 7 & 10 & 5 & 7 & 8 & 7 & 9 & 8 \\
			6 & 7 & 9 & 6 & 4 & 10 & 7 & 9 & 7 & 8 \\
			\hline
		\end{tabular}
	\end{center}
	Số học sinh của hai tổ đó là:
	\choice
	{20 học sinh}
	{15 học sinh}
	{10 học sinh}
	{4 học sinh}
	\loigiai{}
\end{ex}
%%%=============EX_2=============%%%
\begin{ex}
	Từ các số $1; 2; 4; 6; 8; 9$ ta lấy ngẫu nhiên một số. Xác suất để lấy được một số chẵn là:
	\choice
	{$\dfrac{1}{6}$}
	{$\dfrac{1}{3}$}
	{$\dfrac{1}{2}$}
	{$\dfrac{2}{3}$}
	\loigiai{}
\end{ex}
%%%=============EX_3=============%%%
\begin{ex}
	Biểu thức đại số nào sau đây biểu thị chu vi hình chữ nhật có chiều dài bằng $5(\mathrm{~cm})$ và chiều rộng bằng $x(\mathrm{~cm})$?
	\choice
	{$5+x$}
	{$(5-x) \cdot 2$}
	{$5x$}
	{$(5+x) \cdot 2$}
	\loigiai{}
\end{ex}
%%%=============EX_4=============%%%
\begin{ex}
	Giá trị của biểu thức $x^2-x$ tại $x=-3$ là:
	\choice
	{$9$}
	{$12$}
	{$-12$}
	{$-9$}
	\loigiai{}
\end{ex}
%%%=============EX_5=============%%%
\begin{ex}
	Bậc của đa thức $9x^5-x+x^2+x^6-11$ là:
	\choice
	{$11$}
	{$9$}
	{$6$}
	{$5$}
	\loigiai{}
\end{ex}
%%%=============EX_6=============%%%
\begin{ex}
	Trong các biểu thức sau, biểu thức nào là đơn thức một biến:
		\choice[4]
		{$1+x^2$}
		{$\dfrac{1}{3} x^2$}
		{$5x y^2z$}
		{$2x+3$}
	\loigiai{}
\end{ex}
%%%=============EX_7=============%%%
\begin{ex}
	\immini{Các tam giác cân trong hình vẽ bên là
		\choice[2]
		{$\triangle ABC; \triangle CAD$}
		{$\triangle ABC; \triangle ABD$}
		{$\triangle ACD; \triangle ABD$}
		{$\triangle ABD$}}{\begin{tikzpicture}[declare function={r=3;}]
			\path (0:0)	 coordinate (C)
			(120:r)  coordinate (A)
			(180:r)  coordinate (B)
			(0:r)  coordinate (D);
			\foreach \x/\y in {A/C,B/C,D/C}{
				\draw[blue!50!black] (\x)--(\y) node[midway,sloped]{\tikz{\draw (-90:2pt)--(90:2pt);}};
			}
			\draw[blue!50!black] (A)--(B)--(D)--cycle;
			\foreach \t/\g in {A/90,B/180,C/-90,D/0}{
				\draw[fill=red,draw=black] (\t) circle (1pt) node[shift={(\g:7pt)},font=\scriptsize]{$ \t $};
			}
	\end{tikzpicture}}
	\loigiai{}
\end{ex}
%%%=============EX_8=============%%%
\begin{ex}
	\immini{Cho $\triangle DEF$ có trung tuyến $DM$ và trọng tâm $G$ (hình vẽ). Khi đó:
		\choice
		{$\dfrac{DG}{DM}=\dfrac{1}{3}$}
		{$\dfrac{GM}{DG}=\dfrac{1}{2}$}
		{$\dfrac{GM}{DM}=\dfrac{1}{2}$}
		{$DM=3DG$}}{
		\begin{tikzpicture}[declare function={r=2;}]
			\path (0:0)	 coordinate (M)
			(110:2*r)  coordinate (D)
			(180:r)  coordinate (E)
			(0:r)  coordinate (F)
			($(D)!0.5!(F)$) coordinate (N)
			(intersection of D--M and E--N) coordinate (G)
			;
			\foreach \x/\y in {E/M,F/M}{
				\draw[blue!50!black] (\x)--(\y) node[midway,sloped]{\tikz{\draw (-90:2pt)--(90:2pt);}};
			}
			\foreach \x/\y in {D/N,N/F}{
				\draw[blue!50!black] (\x)--(\y) node[midway,sloped]{\tikz{\draw[double] (-90:2pt)--(90:2pt);}};
			}
			\draw[blue!50!black] (D)--(E)--(F)--cycle (D)--(M)(E)--(N);
			\foreach \t/\g in {D/90,E/180,M/-90,F/0,N/45,G/-25}{
				\draw[fill=red,draw=black] (\t) circle (1pt) node[shift={(\g:7pt)},font=\scriptsize]{$ \t $};
			}
		\end{tikzpicture}
		}
	\loigiai{}
\end{ex}
\Closesolutionfile{ansex}
\Closesolutionfile{ans}

%%%==========Phần trắc nghiệm đúng sai============%%%

%\tieumuc{Bài Tập Trắc Nghiệm Đúng Sai}-- \textit{Trong mỗi câu có 4 ý tương ứng A, B, C, D; Học sinh chọn đúng hoặc sai.}
%\Opensolutionfile{ans}[Ans/DATAM1]
%\Opensolutionfile{ansbook}[Ans/DATNTF-1]
%\luulgEXTF
%\Opensolutionfile{ansex}[LOIGIAITN/LGTNTF-1]
%
%
%\Closesolutionfile{ansex}
%\Closesolutionfile{ansbook}
%\Closesolutionfile{ans}	

%%%==========Phần tự luận============%%%
\tieumuc{Tự Luận (7 điểm)}
\Opensolutionfile{ansbt}[LOIGIAITL/LGTL-7CKII-De-04]
%\luuloigiaibt
\taoNdongke[10]{bt}

%%%==============BT_1==============%%%
\begin{bt}[1,5 điểm]
	Biểu đồ đoạn thẳng dưới đây biểu diễn tổng sản phẩm quốc nội (GDP) của nước ta trong giai đoạn từ năm 2014 đến năm 2019 (xem hình \ref{fig:GDP}).
	\begin{center}
		\begin{figure}[!htp]
			\begin{center}
				\begin{tikzpicture}[declare function={d=1cm;r=0.4;}]
					\foreach \i in {0,1,...,6}{
						\pgfmathtruncatemacro{\it}{\i*50}
						\pgfmathsetmacro{\ig}{\i+0.5}
						\draw[thin,gray!30,dashed] (0,\i)--+(12,0) ;
						\draw (-2pt,\i)node[left]{\it}--(2pt,\i);
						\draw[thin,gray!30,dashed] (0,\ig)--+(12,0);
					}
					\draw[blue,>=stealth,->] (0,0)--(0,7) node[left,text width 	=2.0cm,align=center] {GDP\\(tỉ đô la)} ;
					\draw[blue,>=stealth,->] (0,0)--+(12.5,0) node[below] {Năm};
					
					\foreach[count=\i] \t/\y/\year in{1/3.724/2014 	,3/3.864/2015,5/4.104/2016,7/4.476/2017,9/4.904/2018,11/5.22/2019}{%
						\pgfmathsetmacro{\it}{\y*50}
						\path (\t,\y) node[above] {\pgfmathprintnumber[fixed, 	precision=1]{\it}};
						\draw (\t,0)node[below]{\year}--(\t,\y) circle (1pt);
						
						\ifnum\i>1
							\draw[cyan!50!black](\prevt,\prevy)--(\t,\y);
						\fi
						\xdef\prevy{\y}
						\xdef\prevt{\t}
					}
				\end{tikzpicture}
			\end{center}
			\caption{\label{fig:GDP}}
		\end{figure}
	\end{center}
	\begin{enumerate}
		\item GDP năm 2016 là bao nhiêu?
		\item So với năm 2014, GDP năm 2019 đã tăng bao nhiều tỉ đô la.
		\item GDP năm 2017 đã tăng bao nhiêu phần trăm so với năm 2015.
	\end{enumerate}
	\loigiai{}
\end{bt}

%%%==============BT_2==============%%%
\begin{bt}[1,5 điểm]
	Một hộp có 100 chiếc thẻ cùng loại, mỗi thẻ được ghi một trong các số tự nhiên từ 1 đến 100, hai thẻ khác nhau thì ghi hai số khác nhau. Rút ngẫu nhiên một thẻ trong hộp. Tính xác suất của biến cố
	\begin{enumerate}
		\item A: "Số xuất hiện trên thẻ được rút ra là số có một chữ số".
		\item B: "Số xuất hiện trên thẻ được rút ra là số tròn chục".
		\item C: "Số xuất hiện trên thẻ được rút ra là số có tổng các chữ số bằng 10 ".
	\end{enumerate}
	\loigiai{}
\end{bt}
%%%==============BT_3==============%%%
\begin{bt}[$3\left(1,5\right.$ điểm]Cho hai đa thức $P(x)=2x^2+5x-1$ và $Q(x)=2x^2-5x-15$
\begin{enumerate}
	\item Tính $A(x)=P(x)+Q(x)$ và $B(x)=P(x)-Q(x)$.
	\item Tìm nghiệm của đa thức $A(x)$.
\end{enumerate}
\loigiai{}
\end{bt}
%%%==============BT_4==============%%%
\begin{bt}[3,0 điểm]
	Cho tam giác $ABC$ nhọn. Láy điểm $M$ là trung điểm của cạnh $AC$. Trên tia đối của tia $MB$ lấy điểm $\mathscr{D}$ sao cho $MB=MD$.
	\begin{enumerate}
		\item Chứng minh: $\triangle AMD=\triangle CMB$.
		\item Chứng minh $CD=AB$ và $CD \parallel AB$.
		\item Lấy điểm $N$ là trung điểm của cạnh $AB$ và điểm $E$ là trung điểm của cạnh $CD$. Chứng minh điểm $M$ là trung điểm của đoạn $NE$.
	\end{enumerate}
	\loigiai{}
\end{bt}

%%%==============BT_5==============%%%
\begin{bt}[0,5 điểm]
	Tìm tất cả các số nguyên dương $x, y, z$ thỏa mãn:
	\[
	\dfrac{2 z-4 x}{3}=\dfrac{3 x-2 y}{4}=\dfrac{4 y-3 z}{2} \quad \text{và}\quad 200 < y^2+z^2 < 450.
	\]
	\loigiai{}
\end{bt}

\Closesolutionfile{ansbt}
\fileend
\begin{center}
	\rule[4pt]{2cm}{1pt}\large \textbf{HẾT}\rule[4pt]{2cm}{1pt}
\end{center}
