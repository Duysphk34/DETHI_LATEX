\begin{dangntd}{PHẢN ỨNG ĐỐT CHÁY ESTE}
	\begin{ntdppg}{Phương pháp giải}
	\schemestart
	\chemfig{C_nH_{2n+2-2k}O_m}
	\+
	\chemfig{\dfrac{3n+1-k-m}{2}O_2}
	\arrow{->[][][]}[,.6,,,]
	\chemfig{nCO_2}
	\+
	$\left(n+1-k\right)$\chemfig{H_2O}; 
	\schemestop	\\
	Trong đó: $ k=\Sigma(\Pi +\text{vòng})$
	
	\tieumuc{Một số công thức cần nhớ:}
{\setlength{\columnsep}{-15pt}	
\begin{multicols}{2}
	\begin{itemize}
		\item $ n_C=n_{CO_2} ; n_H=2n_{H_2O}$
		\item $m_{\text{este}} = m_C + m_H + m_O$		
		\item $n_{\text{este}}*(k-1)=n_{CO_2} + n_{H_2O}$
		\columnbreak
		\item Bảo toàn O: $ n_{O(\text{Este})}+2n_{O_2} =2n_{CO_2}+n_{H_2O}$
		\item Chỉ số C = $ \dfrac{n_C}{n_{\text{este}}} $;Chỉ số H = $ \dfrac{n_H}{n_{\text{este}}} $
		\item Chỉ số O = $ \dfrac{n_{O(\text{este})}}{n_{\text{este}}}=\dfrac{2n_{CO_2}+n_{H_2O}-2n_{O_2}}{n_{\text{este}}} $	
	\end{itemize}
\end{multicols}}

\tieumuc{CTTQ một số este thường gặp}
\begin{itemize}
	\item Este no, mạch hở, đơn chức: $ C_nH_2nO_2 (n\geq2) $
	\item Este no, mạch hở, hai chức: $ C_nH_{2n-2}O_4 (n\geq4)$
	\item Este không no $(1C=C)$, mạch hở, đơn chức: $ C_nH_{2n-2}O_2 (n\geq3) $
\end{itemize}
	\end{ntdppg}
\end{dangntd}

\begin{vdm}{Ví dụ mẫu}
\end{vdm}
%%%%%%%%%%%%%%%%%%%%%%%%%%%%% Bắt đầu Câu 1%%%%%%%%%%%%%%%%%%%%%%%%%%%%%%%%%%
\begin{vdex}[1]
Đốt cháy hoàn toàn este no, đơn chức, mạch hở thu được 4,4 gam $\mathrm{CO}_2$ và y $\mathrm{mol} ~\mathrm{H_2O}$. Giá trị của y là:
\choice
{%
$ 0,5 $
}
{%
$ 0,2  $
}
{%
\True$ 0,1 $
}
{%
 $ 1,8 $
}
\huongdan{
	Khi đốt cháy este no đon chức, mạch hở  ta có: $ n_{H_2O}=n_{CO_2}=0,1~\mathrm{mol}$	
}
\end{vdex}
%%%%%%%%%%%%%%%%%%%%%%%%%%%%% Kết thúc Câu 1 %%%%%%%%%%%%%%%%%%%%%%%%%%%%%%%%%%
%%%%%%%%%%%%%%%%%%%%%%%%%%%%% Bắt đàu Câu 2 %%%%%%%%%%%%%%%%%%%%%%%%%%%%%%%%%%

\begin{vdex}[2]
	Hỗn hợp $\mathrm{X}$ gồm ba chất có công thức là $\mathrm{CH}_2 \mathrm{O}_2, \mathrm{C}_2 \mathrm{H}_4 \mathrm{O}_2, \mathrm{C}_4 \mathrm{H}_8 \mathrm{O}_2$. Đốt cháy hoàn toàn một lượng hỗn hợp $\mathrm{X}$, thu được $0,8 \mathrm{~mol}~\mathrm{H}_2 \mathrm{O}$ và $\mathrm{m}$ gam $\mathrm{CO}_2$. Giá trị của $\mathrm{m}$ là
	\choice
	{%
		$ 17,92 $
	}
	{%
		$ 70,40  $
	}
	{%
		\True $ 35,20 $
	}
	{%
		$ 17,60 $
	}
	\huongdan{
	Khi đốt cháy este no đon chức, mạch hở  ta có: $ n_{CO_2}=n_{H_2O}=0,8~\mathrm{mol} $
	$ \Rightarrow m_{CO_2} =0.8\cdot44=35,20~\mathrm{gam}$
	}
\end{vdex}
%%%%%%%%%%%%%%%%%%%%%%%%%%%%% Kết thúc Câu 2 %%%%%%%%%%%%%%%%%%%%%%%%%%%%%%%%%%
%%%%%%%%%%%%%%%%%%%%%%%%%%%%% Bắt đàu Câu 3  %%%%%%%%%%%%%%%%%%%%%%%%%%%%%%%%%%
\begin{vdex}[2]
Đốt cháy hoàn toàn 0,15 mol este thu được 19,8 gam $\mathrm{CO}_2$ và $0,45 \mathrm{~mol}~ \mathrm{H}_2 \mathrm{O}$. Công thức phân tử este là
	\choice
	{%
		$\mathrm{C}_5 \mathrm{H}_{10} \mathrm{O}_2$.
	}
	{%
	\True $\mathrm{C}_3 \mathrm{H}_6 \mathrm{O}_2$
	}
	{%
		$\mathrm{C}_4 \mathrm{H}_8 \mathrm{O}_2$
	}
	{%
		$\mathrm{C}_2 \mathrm{H}_4 \mathrm{O}_2$
	}
	\huongdan{
	Ta có: $ n_{CO_2}=n_{H_2O}=0,45~\mathrm{mol} \Rightarrow  $ este no, đơn chức, mạch hở có CTTQ là $ C_nH_{2n}O_2 $\\
	có số C $ =\dfrac{n_{CO_2}}{n_{\text{este}}}= \dfrac{0,45}{0,15}=3 $.
	Vậy công thức phân tử của este cần tìm là: $ C_3H_6O_2 $
	}
\end{vdex}

%%%%%%%%%%%%%%%%%%%%%%%%%%%%% Kết thúc Câu 3 %%%%%%%%%%%%%%%%%%%%%%%%%%%%%%%%%%
%%%%%%%%%%%%%%%%%%%%%%%%%%%%% Bắt đàu Câu 4  %%%%%%%%%%%%%%%%%%%%%%%%%%%%%%%%%%
\begin{vdex}[2][(Đề TSĐH B - 2007)]
	Hai este đơn chức $\mathrm{X}$ và $\mathrm{Y}$ là đồng phân của nhau. Khi hoá hơi 1,85 gam $\mathrm{X}$ thu được thề tích hơi đúng bằng thể tích của 0,7 gam $\mathrm{N}_2$ (đo ở cùng điều kiện). Công thức cấu tạo thu gọn của $X$ và $\mathrm{Y}$ là
	\choice
	{%
	\True $\mathrm{HCOOC}_2 \mathrm{H}_5$ và $\mathrm{CH}_3 \mathrm{COOCH}_3$
	}
	{%
		$\mathrm{C}_2 \mathrm{H}_3 \mathrm{COOC}_2 \mathrm{H}_5$ và $\mathrm{C}_2 \mathrm{H}_5 \mathrm{COOC}_2 \mathrm{H}_3$.
	}
	{%
		$\mathrm{C}_4 \mathrm{H}_8 \mathrm{O}_2$
	}
	{%
		$\mathrm{C}_2 \mathrm{H}_5 \mathrm{COOCH}_3$ và $\mathrm{HCOOCH}\left(\mathrm{CH}_3\right)_3$
	}
	\huongdan{
	
	}
\end{vdex}

%%%%%%%%%%%%%%%%%%%%%%%%%%%%% Kết thúc Câu 4 %%%%%%%%%%%%%%%%%%%%%%%%%%%%%%%%%%
%%%%%%%%%%%%%%%%%%%%%%%%%%%%% Bắt đầu Câu 5  %%%%%%%%%%%%%%%%%%%%%%%%%%%%%%%%%%

\begin{vdex}[2][(Đề TSĐH B - 2007)]
	Hai este đơn chức $\mathrm{X}$ và $\mathrm{Y}$ là đồng phân của nhau. Khi hoá hơi 1,85 gam $\mathrm{X}$ thu được thề tích hơi đúng bằng thể tích của 0,7 gam $\mathrm{N}_2$ (đo ở cùng điều kiện). Công thức cấu tạo thu gọn của $X$ và $\mathrm{Y}$ là
	\choice
	{%
		\True $\mathrm{HCOOC}_2 \mathrm{H}_5$ và $\mathrm{CH}_3 \mathrm{COOCH}_3$
	}
	{%
		$\mathrm{C}_2 \mathrm{H}_3 \mathrm{COOC}_2 \mathrm{H}_5$ và $\mathrm{C}_2 \mathrm{H}_5 \mathrm{COOC}_2 \mathrm{H}_3$.
	}
	{%
		$\mathrm{C}_4 \mathrm{H}_8 \mathrm{O}_2$
	}
	{%
		$\mathrm{C}_2 \mathrm{H}_5 \mathrm{COOCH}_3$ và $\mathrm{HCOOCH}\left(\mathrm{CH}_3\right)_3$
	}
	\huongdan{
		
	}
\end{vdex}

%%%%%%%%%%%%%%%%%%%%%%%%%%%%% Kết thúc Câu 5 %%%%%%%%%%%%%%%%%%%%%%%%%%%%%%%%%%
%%%%%%%%%%%%%%%%%%%%%%%%%%%%% Bắt đàu Câu 6  %%%%%%%%%%%%%%%%%%%%%%%%%%%%%%%%%%

\begin{vdex}[2][(Đề TSĐH A - 2011)]
Đốt cháy hoàn toàn 0,11 gam một este $X$ (tạo nên từ một axit cacboxylic đơn chức và một ancol đơn chức) thu được 0,22 gam $\mathrm{CO}_2$ và 0,09 gam $\mathrm{H}_2 \mathrm{O}$. Số este đồng phân của $X$ là
	\choice
	{%
		 $ 6 $
	}
	{%
		$ 2 $
	}
	{%
	\True	$ 4 $
	}
	{%
		$ 5 $
	}
	\huongdan{
		
	}
\end{vdex}

%%%%%%%%%%%%%%%%%%%%%%%%%%%%% Kết thúc Câu 6 %%%%%%%%%%%%%%%%%%%%%%%%%%%%%%%%%%
%%%%%%%%%%%%%%%%%%%%%%%%%%%%% Bắt đàu Câu 7  %%%%%%%%%%%%%%%%%%%%%%%%%%%%%%%%%%
\begin{vdex}[3][(Đề TSCĐ - 2010)]
	 Hỗn hợp $Z$ gồm hai este $X$ và $Y$ tạo bởi cùng một ancol và hai axit cacboxylic kế tiếp nhau trong dãy đồng đẳng $\left(\mathrm{M}_{\mathrm{X}}<\mathrm{M}_{\mathrm{Y}}\right)$. Đốt cháy hoàn toàn $\mathrm{m}$ gam $\mathrm{Z}$ cần dùng 6,16 lít khí $\mathrm{O}_2$ (đktc), thu được 5,6 lít khí $\mathrm{CO}_2$ (đktc) và 4,5 gam $\mathrm{H}_2 \mathrm{O}$. Công thức este $\mathrm{X}$ và giá trị của $\mathrm{m}$ tương ứng là
	\choice
	{%
		$\mathrm{CH}_3 \mathrm{COOCH}_3$ và $ 6,7 $ 
	}
	{%
		$\mathrm{HCOOC}_2 \mathrm{H}_5$ và $ 9,5 $
	}
	{%
		\True $\mathrm{HCOOCH}_3$ và $ 6,7 $
	}
	{%
		$(\mathrm{HCOO})_2 \mathrm{C}_2 \mathrm{H}_4$ và $ 6,6 $
	}
	\huongdan{
		\noindent Ta có: $ n_{O_2}=\dfrac{6,16}{22,4} =0,275~\mathrm{mol}$; $ n_{CO_2}=n_{H_2O}=0,25~\mathrm{mol}$\\
		Vì $ n_{CO_2}=n_{H_2O} $ nên este là no, đơn ,hở có CTTQ là $ C_{\overline{n}}H_{2\overline{n}}O_2 $\par\noindent
		Áp dụng định luật bảo toàn nguyên tố [O], ta có:\\ 
		$ \begin{aligned}
			 n_{O/\text{este}} &=2n_{CO_2} +n_{H_2O}-2n_{O_2}\\
			 &=2.0,25 +0,25-2.0,275 =0,2 (~\mathrm{mol})
		\end{aligned} $\\
	$ \Rightarrow n_{\text{este}} = \dfrac{1}{2}n_{O/\text{este}} =0{,}1~\mathrm{mol}$ 
	$ \Rightarrow \overline{C}=\dfrac{n_{CO_2}}{n_{\text{este}}}=\dfrac{0,25}{0,1}=2,5$\\
	$ \Rightarrow Z $  gồm :
$ \begin{cases}
	\chemfig{HCOOCH_3} \left(X\right)\\
	\chemfig{CH_3COOCH_3} \left(Y\right)
\end{cases} $\\
Áp dụng định luật bảo toàn khối lượng cho phản ứng, ta có:\\ 
$ \begin{aligned}
	m_{\text{este}} &=m_{CO_2} +m_{H_2O}-m_{O_2}\\
	&=0,25.44 +4,5-.0,275.32 =6,7 (~\mathrm{gam})
\end{aligned} $\\

	}
\end{vdex}

%%%%%%%%%%%%%%%%%%%%%%%%%%%%% Kết thúc Câu 7 %%%%%%%%%%%%%%%%%%%%%%%%%%%%%%%%%%
%%%%%%%%%%%%%%%%%%%%%%%%%%%%% Bắt đàu Câu 8  %%%%%%%%%%%%%%%%%%%%%%%%%%%%%%%%%%
\begin{bttl}{Bài tập tự luyện}
\end{bttl}
\Opensolutionfile{ans}[DAPAN/BTTL05]
%%%%%%%%%%%%%%%%%%%%%%%%%%%%%%%% Bắt đầu câu 1 %%%%%%%%%%%%%%%%%%%%%%%%%%%%%%%

\begin{ex}[2][(Đề MH - 2018)]
Đốt cháy hoàn toàn hỗn hợp metyl axetat và etyl axetat, thu được $\mathrm{CO} 2$ và $\mathrm{m}$ gam H2O. Hấp thụ toàn bộ sản phẩm cháy vào dung dịch $\mathrm{Ca}(\mathrm{OH}) 2$ dư, thu được 25 gam kết tủa. Giá trị của m là
	\choice
{%
	 $ 5,4 $
}
{%
\True	$ 4,5 $
}
{%
	$ 3,6 $
}
{%
	$ 6,3 $
}	
\sodongkeex[5]
\end{ex}
%%%%%%%%%%%%%%%%%%%%%%%%%%%%%%%% Kết thúc câu 1 %%%%%%%%%%%%%%%%%%%%%%%%%%%%%%%
%%%%%%%%%%%%%%%%%%%%%%%%%%%%%%%% Bắt đầu câu 2 %%%%%%%%%%%%%%%%%%%%%%%%%%%%%%%
\begin{ex}[2][(Đề TSĐH B - 2011)]
	Hỗn hợp X gồm vinyl axetat, metyl axetat và etyl fomat. Đốt cháy hoàn toàn $3,08 \mathrm{gam} \mathrm{X}$, thu được 2,16 gam $\mathrm{H}_2 \mathrm{O}$. Phần trăm số mol của vinyl axetat trong $\mathrm{X}$ là
	\choice
	{%
		$75 \%$
	}
	{%
		$72,08 \%$
	}
	{%
		$27,92 \%$
	}
	{%
		\True $25 \%$
	}	
	\sodongkeex[5]
\end{ex}
%%%%%%%%%%%%%%%%%%%%%%%%%%%%%%%% Kết thúc câu 2 %%%%%%%%%%%%%%%%%%%%%%%%%%%%%%%
%%%%%%%%%%%%%%%%%%%%%%%%%%%%%%%% Bắt đầu câu 3 %%%%%%%%%%%%%%%%%%%%%%%%%%%%%%%
\begin{ex}[2][(Đề TSĐH A - 2011)]
	Đốt cháy hoàn toàn 3,42 gam hỗn hợp gồm axit acrylic, vinyl axetat, metyl acrylat và axit oleic, rồi hấp thụ toàn bộ sản phẩm cháy vào dung dịch $\mathrm{Ca}(\mathrm{OH})_2(\mathrm{du})$. Sau phản ứng thu được $18 \mathrm{gam}$ kết tủa và dung dịch $\mathrm{X}$. Khối lượng $\mathrm{X}$ so với khối lượng dung dịch $\mathrm{Ca}(\mathrm{OH})_2$ ban đầu đã thay đổi như thế nào?
	\choice
	{%
		Giảm $ 7,74 $ gam
	}
	{%
	 Tăng $ 7,92 $ gam
	}
	{%
		Tăng $ 2,70 $ gam
	}
	{%
	\True 	Giảm $ 7,38 $ gam
	}	
	\sodongkeex[5]
\end{ex}
%%%%%%%%%%%%%%%%%%%%%%%%%%%%%%%% Kết thúc câu 3 %%%%%%%%%%%%%%%%%%%%%%%%%%%%%%%
%%%%%%%%%%%%%%%%%%%%%%%%%%%%%%%% Bắt đầu câu 4 %%%%%%%%%%%%%%%%%%%%%%%%%%%%%%%
\begin{ex}[2][(Đề THPT QG - 2016)]
	Đốt cháy hoàn toàn 0,33 mol hỗn hợp $X$ gồm metyl propionat, metyl axetat và 2 hiđrocacbon mạch hở cần vừa đủ $1,27 \mathrm{~mol} \mathrm{O} 2$, tạo ra 14,4 gam $\mathrm{H} 2 \mathrm{O}$. Nếu cho $0,33 \mathrm{~mol} \mathrm{X}$ vào dung dịch $\mathrm{Br} 2$ dư thì số $\mathrm{mol} \mathrm{Br} 2$ phản ứng tối đa là
	\choice
	{%
		$ 0,33 $
	}
	{%
		$ 0,26 $
	}
	{%
		$ 0,30 $
	}
	{%
		\True $ 0,40 $
	}	
	\sodongkeex[5]
\end{ex}
%%%%%%%%%%%%%%%%%%%%%%%%%%%%%%%% Kết thúc câu 4 %%%%%%%%%%%%%%%%%%%%%%%%%%%%%%%
%%%%%%%%%%%%%%%%%%%%%%%%%%%%%%%% Bắt đầu câu 5 %%%%%%%%%%%%%%%%%%%%%%%%%%%%%%%
\Closesolutionfile{ans}

\newpage
\subsection{Một số phương pháp trong giải toán este}
\subsubsection{Kỹ thuật đồng đẳng hóa este}
	\begin{ntdppg}{Phương pháp giải}
			\begin{enumerate}
					\item \textbf{Bản chất của phương pháp}\\
					- Đồng đẳng là những chất : giống nhau về công thức cấu tạo hơn kém nhau một hay nhiều nhóm $ CH_2 $ (metylen)$ \Rightarrow $ tính chất hóa học tương tự nhau.\\
					- Ví dụ: Xét hợp chất $ C_nH_{2n+1}COOCH_3 $\\
					\schemestart
					\chemfig{HCOOCH_3}
					\arrow{<=>[\scriptsize$ +CH_2 $][\scriptsize$ -CH_2 $][]}[,0.6,,,]
					\chemfig{CH_3COOCH_3}
					\arrow{<=>[\scriptsize$ +CH_2 $][\scriptsize$ -CH_2 $][]}[,0.6,,,]
					\chemfig{C_2H_5COOCH_3}
					\arrow{<=>[\scriptsize$ +CH_2 $][\scriptsize$ -CH_2 $][]}[,0.6,,,]
					\chemfig{C_nH_{2n+1}COOCH_3}
					\schemestop\\
					$ \Rightarrow $ \chemfig{C_nH_{2n+1}COOCH_3}  coi như 
					$ \begin{cases}
					HCOOCH_3\\
					CH_2	
					\end{cases} $
					\item \textbf{Dấu hiệu}\\
					- Bài toán cho rõ đặc điểm của chất.\\
					- Bài toán phức tạp nhiều giai đoạn, nhiều chất.\\
					- Có phản ứng đốt cháy.
					\item \textbf{Các bước giải toán}
					\begin{itemize}
						\item Tách nhóm $ -CH_2 $ : Quy đổi chất phức tạp $ \Rightarrow $ 
						$ \begin{cases}
							\text{Chất đầu dãy đồng đẳng}\\
							 - CH_2 
						\end{cases} $
					\item Dựa vào dữ kiện và số liệu đầu bài lập hệ phương trình và giải hệ phương trình.
					\item Ghép nhóm $ -CH_2 $  $ \Rightarrow $ chất tổng quát ban đầu
					\end{itemize}
				\item \textbf{Kết hợp với phương pháp bảo toàn khối lượng, bảo toàn nguyên tố (C,H,O)}
				\item \textbf{Cách quy đổi một số dãy đồng đẳng hay gặp}
				
				\begin{itemize}
					
					\item Axit no, đơn chức: $ C_nH_{2n+1}COOH $ \schemestart \arrow{->[\scriptsize\text{ĐĐH}][\scriptsize$ CH_2 $][3pt]}[,0.8,,,]\schemestop
					$ \begin{cases} 
						HCOOH\\
						CH_2
					\end{cases} $
				
				\item Axit không no, có $ 1 $ liên kết $ \pi $ , đơn chức : $ C_nH_{2n-1}COOH $\\
				
				- Không có đồng phân hình học  $ (n\geq 2) $:\schemestart
				\arrow{->[\scriptsize\text{ĐĐH}][\scriptsize$ CH_2 $][3pt]}[,0.8,,,]
				\schemestop
				$ \begin{cases} 
					C_2H_3COOH\\
					CH_2
				\end{cases} $
			
				- Có đồng phân hình học $ (n\geq 3) $ :\schemestart
				\arrow{->[\scriptsize\text{ĐĐH}][\scriptsize$ CH_2 $][3pt]}[,0.8,,,]
				\schemestop
				$ \begin{cases} 
					C_3H_5COOH\\
					CH_2
				\end{cases} $
			
			\item Ancol no, đơn chức: $ C_nH_{2n+1}OH $ \schemestart \arrow{->[\scriptsize\text{ĐĐH}][\scriptsize$ CH_2 $][3pt]}[,0.8,,,]\schemestop
			$ \begin{cases} 
				CH_3OH\\
				CH_2
			\end{cases} $
			
			\item Este no, đơn chức: $ C_nH_{2n}O_2 $ \schemestart \arrow{->[\scriptsize\text{ĐĐH}][\scriptsize$ CH_2 $][3pt]}[,0.8,,,]\schemestop
			$ \begin{cases} 
				HCOOCH_3\\
				CH_2
			\end{cases} $
			
				\end{itemize}
				\end{enumerate}
			
			\begin{notegsnd}
				\begin{itemize}
					\item Nhóm $ -CH_2 $ là thành phần khối lượng.$ \Rightarrow $ Có mặt ở phương trình liên quan đến khối lượng
					\item $ CH_2 $ không phải là chất $ \Rightarrow $ không được tính vào số mol của các chất.
				\end{itemize}
			\end{notegsnd}
		\end{ntdppg}
	
\begin{vdm}{Ví dụ mẫu}
\end{vdm}
%%%%%%%%%%%%%%%%%%%BẮT ĐẦU VÍ DỤ 1%%%%%%%%%%%%%%%%%%%%%%%%%%%%%%%%%%%%%%
\begin{vdex}[1][Kỹ thuật đồng đẳng hóa]
	$\mathrm{X}$ là một este no, đơn chức, mạch hở. Đốt cháy hoàn toàn $ 5,1 $ gam $\mathrm{X}$ thu được $4,5 \mathrm{~gam}~\mathrm{H_2O}$. Công thức phân tử của $\mathrm{X}$ là
	\choice
	{%
		$\mathrm{C}_2 \mathrm{H}_4 \mathrm{O}_2$
	}
	{%
		$\mathrm{C}_3 \mathrm{H}_6 \mathrm{O}_2$
	}
	{%
		$\mathrm{C}_4 \mathrm{H}_8 \mathrm{O}_2$
	}
	{%
		\True $\mathrm{C}_5 \mathrm{H}_{10} \mathrm{O}_2$
	}
	\huongdan
	{%
		Vì $ X $ là este no, đơn , hở nên $ X $ có công thức tổng quát là:$ C_nH_{2n}O_2 $
		\begin{itemize}
			\item \textbf{Cách 1:Tính theo phương trình phản ứng:} \\
			\begin{tabular}{ccccc}
				\chemfig{C_nH_{2n}O_2}	&  \schemestart \arrow{->[\scriptsize$ +O_2 $][][3pt]}[,.8,,,]\schemestop & n\chemfig{CO_2}  &  $ + $  & n\chemfig{H_2O} \\
				$ \dfrac{5,1}{14n+32} $	&   &   &      & $ 0,25 $\\
			\end{tabular}\\
			$ \Rightarrow  \dfrac{5,1}{14n+32} \cdot n =0,25 \Rightarrow n=5 $ \\
			Vậy công thức phân tử este cần tìm là $\mathrm{C}_5 \mathrm{H}_{10} \mathrm{O}_2$
			\item \textbf{Cách 2: Dùng kỹ thuật đồng đẳng hóa}\\
			Este no, đơn, hở \schemestart \arrow{->[\scriptsize ĐĐH][][3pt]}[,.8,,,]\schemestop$ \begin{cases}
				\chemfig{HCOOCH_3} :x~\mathrm{mol}\\
				\chemfig{CH_2} :y~\mathrm{mol}
			\end{cases} $
			\schemestart \arrow{->[\scriptsize $ +O_2 $][][3pt]}[,.8,,,]\schemestop 
			$ \left\{ \begin{array}{l}
				H_2O  \\
				0,25~\mathrm{mol}
			\end{array}\right.$\\
			Theo đề bài ta có:
			\begin{align} 
				m_{\text{este}}&=60x+14y=5,1\tag{I} \label{eq:pt1}\\
				n_{H_2O}&=4x+2y=0,5\tag{II} \label{eq:pt2}
			\end{align}
			Từ phương trình \ref{eq:pt1} và \ref{eq:pt2} ta có hệ phương trình:
			$\begin{cases}
				60x+14y=5,1\\
				4x+2y=0,5
			\end{cases}  $ $ \Leftrightarrow \begin{cases}
				x=0,05\\
				y=0,15
			\end{cases}  $ \\$ \Rightarrow  $ Hỗn hợp 
			$  \begin{cases}
				HCOOCH_3\\
				3CH_2
			\end{cases}  $ $ \Rightarrow C_5H_{10}O_2 $
			
		\end{itemize}
	}
\end{vdex}
%%%%%%%%%%%%%%%%%%% KẾT THÚC VÍ DỤ 1 %%%%%%%%%%%%%%%%%%%%%%%%%%%%%%%%%%%%%%
%%%%%%%%%%%%%%%%%%% BẮT ĐẦU VÍ DỤ 2 %%%%%%%%%%%%%%%%%%%%%%%%%%%%%%%%%%%%%%

\begin{vdex}[2][Kỹ thuật đồng đẳng hóa]
Đốt cháy hoàn toàn 7,6 gam hỗn hợp $\mathrm{X}$ gồm một axit cacboxylic no, đơn chức, mạch hở và một ancol đơn chức (có số nguyên tử cacbon trong phân tử khác nhau), thu được 6,72 lít khí $\mathrm{CO}_2$ (đktc) và 7,2 gam $\mathrm{H}_2 \mathrm{O}$. Thực hiện phản ứng este hóa 7,6 gam hỗn hợp trên có xúc tác $\mathrm{H}_2 \mathrm{SO}_4$ đặc với hiệu suất $60 \%$ thu được $\mathrm{m}$ gam este. Giá trị của $\mathrm{m}$ là
\choice
{%
	\True $ 3,06 $
}
{%
	$ 5,10 $
}
{%
	$ 3,60 $
}
{%
	$ 2,04 $
}
\huongdan{%
\begin{tikzpicture}[declare function={d=1.0cm;},>=stealth]
	\path 
	(0,0) coordinate (A)
	(.3,d) coordinate (B)
	(2,d) coordinate (C)
	(.3,-d) coordinate (Bt)
	(2,-d) coordinate (Ct)
	;
	\tikzset{%
		muitenhainhanh/.pic={%				 		
			\draw[->] (A)--(B)--(C) ;
			\draw[->](A)--(Bt)--(Ct) ;		
		}
	}
	\path 
	(B)--(C) node[sloped,pos=.5,above] {$ + O_2 $}
	(Bt)--(Ct) node[sloped,pos=.5,above] {\scriptsize $ H_2SO_4 $ đặc }
	(Bt)--(Ct) node[sloped,pos=.5,below] {\scriptsize $ t^\circ$  }
	;
\path pic [local bounding box=A1] at (0,0) {muitenhainhanh};
\path (A1.west) node (hhx)[anchor=east] {Hỗn hợp $ X $ \schemestart\arrow{->[ \scriptsize ĐĐH ][][3pt]}[,.8,,,] \schemestop$ \begin{cases}
		\chemfig{HCOOH}: x~\mathrm{mol}\\
		\chemfig{CH_3OH}:y~\mathrm{mol}\\
		\chemfig{CH_2}:z~\mathrm{mol}
	\end{cases} $};	
\path (C.east) node[anchor=west] {%
	$\begin{aligned}
			\underbrace{~~CO_2~~}_{\scriptsize 0,3~\mathrm{mol}}
			+ \underbrace{~~H_2O~~}_{\scriptsize 0,4~\mathrm{mol}}
		\end{aligned}$} ;
\path (Ct.east) node[anchor=west] {%
m(gam) 
$\begin{cases}
\chemfig{HCOOCH_3}\\
\chemfig{CH_2}
\end{cases}$
} 
;
\end{tikzpicture}\\
Theo đề bài:
$ \left\{\begin{array}{l*7{c}}
\text{Bảo toàn khối lượng:}  &46x&+&32y&+&14z&=&7,6 \\
\text{Bảo toàn C:}&\phantom{46}x&+&\phantom{32}y&+&\phantom{14}z&=&0.3\\
\text{Bảo toàn H:}&\phantom{4} 2x&+&\phantom{3}4y&+&\phantom{1}2z&=&0.8
\end{array}\right. $
$ \Leftrightarrow \begin{cases}
x=0,05\\y=0,1\\z=0,15
\end{cases} $\\
Ghép nhóm $ CH_2 $, ta có $\left[\begin{array}{l}
	z=x+y\\
	z=3x+0y
\end{array}\right.$ $ \Rightarrow  $ Xét hai trường hợp
\begin{itemize}
	\item z=x+y
	$ \Rightarrow \begin{cases}
		\schemestart \chemfig{HCOOH}\arrow{->[\scriptsize $+~1CH_2$ ][][]}[,0.8,,,] \chemfig{CH_3COOH}\schemestop\\
		\schemestart \chemfig{CH_3OH}\arrow{->[\scriptsize $+~1CH_2$ ][][]}[,0.8,,,] \chemfig{C_2H_5OH}\schemestop
	\end{cases} $\\ $ \Rightarrow $ Phản ứng este hóa :\\
\begin{tabular}{*7{c}}
\chemfig{CH_3COOH}&$ + $ & \chemfig{C_2H_5OH} &\schemestart\arrow{<=>[\scriptsize$ H_2SO_4 $][\scriptsize$ t^\circ $][3pt]}[,0.8,,,]\schemestop &\chemfig{CH_3COOC_2H_5} & $ + $& \chemfig{H_2O}\\
$ 0,05~\mathrm{mol} $& & $ 0,1~\mathrm{mol} $& & $ 0,05~\mathrm{mol} $& & \\
\end{tabular}
\\Vì ancol dư nên $ n_{CH_3COOC_2H_5\text{(lt)}} = n_{CH_3COOH}=0,05~\mathrm{mol}$\\ $ \Rightarrow m_{CH_3COOC_2H_5\text{(tt)}}=0,05.88.0,6 = 2,64~\mathrm{gam}$ (loại)
\item z=3x+0y
	$ \Rightarrow \begin{cases}
		\schemestart \chemfig{HCOOH}\arrow{->[\scriptsize $+~3CH_2$ ][][]}[,0.8,,,] \chemfig{C_3H_7COOH}\schemestop\\
		\schemestart \chemfig{CH_3OH}\arrow{->[\scriptsize $+~0CH_2$ ][][]}[,0.8,,,] \chemfig{CH_3OH}\schemestop
	\end{cases} $\\ $ \Rightarrow $ Phản ứng este hóa :\\
\begin{tabular}{*7{c}}
\chemfig{C_3H_7COOH}&$ + $ & \chemfig{CH_3OH} &\schemestart\arrow{<=>[\scriptsize$ H_2SO_4 $][\scriptsize$ t^\circ $][3pt]}[,0.8,,,]\schemestop &\chemfig{C_3H_7COOCH_3} & $ + $& \chemfig{H_2O}\\
$ 0,05~\mathrm{mol} $& & $ 0,1~\mathrm{mol} $& & $ 0,05~\mathrm{mol} $& & \\
\end{tabular}\\
Vì ancol dư nên $ n_{C_3H_7COOCH_3\text{(lt)}} = n_{C_3H_7COOH}=0,05~\mathrm{mol}$\\ $ \Rightarrow m_{C_3H_7COOCH_3\text{(tt)}}=0,05.102.0,6 = 3,06~\mathrm{gam}$
\end{itemize}
}			
\end{vdex}

%%%%%%%%%%%%%%%%%%% KẾT THÚC VÍ DỤ 2 %%%%%%%%%%%%%%%%%%%%%%%%%%%%%%%%%%%%%%
%%%%%%%%%%%%%%%%%%% BẮT ĐẦU VÍ DỤ 3 %%%%%%%%%%%%%%%%%%%%%%%%%%%%%%%%%%%%%%
\begin{vdex}[2][Kỹ thuật đồng đẳng hóa]
	Hỗn hợp X gồm một axit cacboxylic thuộc dãy đồng đẳng của axit fomic và một ancol đơn chức. Đốt cháy hoàn toàn 21,7 gam $\mathrm{X}$ thu được 20,16 lít khí $\mathrm{CO}_2$ (đktc) và 18,9 gam $\mathrm{H}_2 \mathrm{O}$. Mặt khác, thực hiện phản ứng este hóa 21,7 gam $\mathrm{X}$ với hiệu suất $80 \%$, thu được $\mathrm{m}$ gam este. Giá trị của $\mathrm{m}$ là
	\choice
	{%
		$ 9,18 $
	}
	{%
		$ 16,32 $
	}
	{%
	\True	$ 12,24 $
	}
	{%
		$ 15,30 $
	}
	\huongdan
	{%
		
		
		
	}			
\end{vdex}
%%%%%%%%%%%%%%%%%%% KẾT THÚC VÍ DỤ 3 %%%%%%%%%%%%%%%%%%%%%%%%%%%%%%%%%%%%%%
%%%%%%%%%%%%%%%%%%% BẮT ĐẦU VÍ DỤ 4 %%%%%%%%%%%%%%%%%%%%%%%%%%%%%%%%%%%%%%
\begin{vdex}[2][Kỹ thuật đồng đẳng hóa]
Đun nóng 26,5 gam hỗn hợp $\mathrm{X}$ chứa một axit không no (có 1 liên kết đôi $\mathrm{C}=\mathrm{C}$ trong phân tử) đơn chức, mạch hở và một ancol no đơn chức, mạch hở với $\mathrm{H}_2 \mathrm{SO}_4$ đặc làm xúc tác, thu được $\mathrm{m}$ gam hỗn hợp $\mathrm{Y}$ gồm este, axit và ancol. Đốt cháy hoàn toàn $\mathrm{m}$ gam $\mathrm{Y}$ cần dùng 36,96 lít khí $\mathrm{O}_2$ (đktc), thu được 55,0 gam $\mathrm{CO}_2$. Mặt khác, cho $\mathrm{m}$ gam $\mathrm{Y}$ tác dụng với dung dịch chứa $0,2 \mathrm{~mol}$ $\mathrm{NaOH}$, rồi cô cạn dung dịch thu được bao nhiêu gam chất rắn khan?
	\choice
	{%
		$ 16,1 $
	}
	{%
		$ 18,2 $
	}
	{%
	\True	$ 20,3 $
	}
	{%
		$ 18,5 $
	}
	\huongdan
	{%
		
		
		
	}			
\end{vdex}
%%%%%%%%%%%%%%%%%%% KẾT THÚC VÍ DỤ 4 %%%%%%%%%%%%%%%%%%%%%%%%%%%%%%%%%%%%%%
%%%%%%%%%%%%%%%%%%% BẮT ĐẦU VÍ DỤ 5 %%%%%%%%%%%%%%%%%%%%%%%%%%%%%%%%%%%%%%
\begin{vdex}[2][Kỹ thuật đồng đẳng hóa]
Hỗn hợp $\mathrm{X}$ gồm este $\mathrm{Y}$ (no, đơn chức, mạch hở); este $\mathrm{Z}$ (đơn chức, mạch hở, có 2 liên kết $\pi$, tạo bởi ancol no). Đốt cháy $0,25 \mathrm{~mol} \mathrm{X}$ thu được 15,68 lít $\mathrm{CO}_2$ ở đktc và 10,8 gam $\mathrm{H}_2 \mathrm{O}$. Phát biểu nào sau đây sai?
	\choice
	{%
		$Y$ là metyl fomat
	}
	{%
		$X$ có phản ứng tráng gương
	}
	{%
	\True	$\mathrm{Z}$ có 2 CTCT thỏa mãn
	}
	{%
		Tổng số nguyên tử trong $X$ bằng $ 20 $
	}
	\huongdan
	{%
		
		
		
	}			
\end{vdex}
%%%%%%%%%%%%%%%%%%% KẾT THÚC VÍ DỤ 5 %%%%%%%%%%%%%%%%%%%%%%%%%%%%%%%%%%%%%%
%%%%%%%%%%%%%%%%%%% BẮT ĐẦU VÍ DỤ 6 %%%%%%%%%%%%%%%%%%%%%%%%%%%%%%%%%%%%%%

\begin{vdex}[2][Kỹ thuật đồng đẳng hóa]
	Axit hữu cơ đơn chức $\mathrm{X}$ mạch hở phân tử có một liên kết đôi $\mathrm{C}=\mathrm{C}$ và có đồng phân hình học. Hai ancol $\mathrm{Y}, \mathrm{Z}$ đơn chức là đồng đẳng kế tiếp $\left(\mathrm{M}_{\mathrm{Y}}<\mathrm{M}_{\mathrm{Z}}\right)$. Đốt cháy hoàn toàn 0,26 mol hỗn hợp $\mathrm{E}$ gồm $\mathrm{X}, \mathrm{Y}, \mathrm{Z}$ cần $0,6 \mathrm{~mol} \mathrm{O}_2$ thu được $0,46 \mathrm{~mol} \mathrm{CO}_2$ và $0,6 \mathrm{~mol} \mathrm{H}_2 \mathrm{O}$. Phần trăm khối lượng của $\mathrm{Z}$ trong hỗn hợp $\mathrm{E}$ là
	\choice
	{%
		$32,08 \%$
	}
	{%
       \True  $7,77 \%$
	}
	{%
		$32,43 \%$
	}
	{%
		$48,65 \%$
	}
	\huongdan
	{%
		
		
		
	}			
\end{vdex}
%%%%%%%%%%%%%%%%%%% KẾT THÚC VÍ DỤ 6 %%%%%%%%%%%%%%%%%%%%%%%%%%%%%%%%%%%%%%
%%%%%%%%%%%%%%%%%%% BẮT ĐẦU VÍ DỤ 7 %%%%%%%%%%%%%%%%%%%%%%%%%%%%%%%%%%%%%%

\begin{vdex}[2][Kỹ thuật đồng đẳng hóa]
Hỗn hợp E gồm este $\mathrm{X}$ đơn chức và axit cacboxylic $\mathrm{Y}$ hai chức (đều mạch hở, không no có một liên kết đôi $\mathrm{C}=\mathrm{C}$ trong phân tử). Đốt cháy $\mathrm{m}$ gam $\mathrm{E}$ thu được $0,43 \mathrm{~mol}$ khí $\mathrm{CO}_2$ và $0,32 \mathrm{~mol}~\mathrm{H}_2 \mathrm{O}$.Mặt khác, 46,6 gam $\mathrm{E}$ phản ứng với $\mathrm{NaOH}$ vừa đủ được 55,2 gam muối khan và chất $\mathrm{Z}$ có tỉ khối hơi so với $\mathrm{H}_2$ là 16. Phần trăm khối lượng của $\mathrm{Y}$ trong hỗn hợp $\mathrm{E}$ có giá trị gần nhất với
	\choice
	{%
		$46,5 \%$
	}
	{%
		$48,0 \%$
	}
	{%
		$43,5 \%$
	}
	{%
	\True	$41,5 \%$
	}
	\huongdan
	{%
		
		
		
	}			
\end{vdex}
%%%%%%%%%%%%%%%%%%% KẾT THÚC VÍ DỤ 7 %%%%%%%%%%%%%%%%%%%%%%%%%%%%%%%%%%%%%%
%%%%%%%%%%%%%%%%%%% BẮT ĐẦU VÍ DỤ 8 %%%%%%%%%%%%%%%%%%%%%%%%%%%%%%%%%%%%%%
\begin{vdex}[2][Kỹ thuật đồng đẳng hóa]
	$\mathrm{X}$ là este no, đơn chức, $\mathrm{Y}$ là axit cacboxylic đơn chức, không no chứa một liên kết đôi $\mathrm{C}=\mathrm{C}$; $\mathrm{Z}$ là este hai chức tạo bởi axit $\mathrm{Y}$ và ancol no $\mathrm{T}(\mathrm{X}, \mathrm{Y}, \mathrm{Z}$ đều mạch hở,). Đốt cháy $\mathrm{m}$ gam hỗn hợp $\mathrm{E}$ chứa $\mathrm{X}, \mathrm{Y}, \mathrm{Z}$ (số $\mathrm{mol} \mathrm{Y}$ bằng số $\mathrm{mol} \mathrm{Z}$ ) cần dùng 7,504 lít $\mathrm{O}_2$ (đktc), thu được tổng khối lượng $\mathrm{CO}_2$ và $\mathrm{H}_2 \mathrm{O}$ là 19,74 gam. Mặt khác, $\mathrm{m}$ gam $\mathrm{E}$ làm mất màu tối đa dung dịch chứa $22,4 \mathrm{~g} \mathrm{Br}$. Biết $\mathrm{E}$ có khả năng tham gia phản ứng tráng bạc. Khối lượng của $\mathrm{X}$ trong $\mathrm{E}$ là
	\choice
	{%
	\True 	6,6
	}
	{%
		7,6
	}
	{%
		8,6
	}
	{%
		9,6
	}
	\huongdan
	{%
		
		
		
	}			
\end{vdex}
%%%%%%%%%%%%%%%%%%% KẾT THÚC VÍ DỤ 8 %%%%%%%%%%%%%%%%%%%%%%%%%%%%%%%%%%%%%%
%%%%%%%%%%%%%%%%%%% BẮT ĐẦU VÍ DỤ 9 %%%%%%%%%%%%%%%%%%%%%%%%%%%%%%%%%%%%%%

\begin{vdex}[2][Kỹ thuật đồng đẳng hóa]
	Hỗn hợp A gồm một axit no, hở, đơn chức và hai axit không no, hở, đơn chức (gốc hiđrocacbon chứa một liên kết đôi), kế tiếp nhau trong dãy đồng đẳng. Cho $\mathrm{A}$ tác dụng hoàn toàn với $150 \mathrm{ml}$ dung dịch $\mathrm{NaOH} 2,0 \mathrm{M}$. Để trung hòa vừa hết lượng $\mathrm{NaOH}$ dư cần thêm vào $100 \mathrm{ml}$ dung dịch $\mathrm{HCl}$ 1,0 M được dung dịch D. Cô cạn cẩn thận $\mathrm{D}$ thu được 22,89 gam chất rắn khan. Mặt khác đốt cháy hoàn toàn $\mathrm{A}$ rồi cho toàn bộ sản phẩm cháy hấp thụ hết vào bình đựng lượng dư dung dịch $\mathrm{NaOH}$ đặc, khối lượng bình tăng thêm $26,72 \mathrm{gam}$. Phần trăm khối lượng của axit không no có khối lượng phân tử nhỏ hơn trong hỗn hợp $\mathrm{A}$ là
	\choice
	{%
		$35,52 \%$
	}
	{%
		$40,82 \%$
	}
	{%
		$44,24 \%$
	}
	{%
	\True	$22,78 \%$
	}
	\huongdan
	{%
		
		
		
	}			
\end{vdex}
%%%%%%%%%%%%%%%%%%% KẾT THÚC VÍ DỤ 9 %%%%%%%%%%%%%%%%%%%%%%%%%%%%%%%%%%%%%%
%%%%%%%%%%%%%%%%%%% BẮT ĐẦU VÍ DỤ 10 %%%%%%%%%%%%%%%%%%%%%%%%%%%%%%%%%%%%%%

\begin{vdex}[2][Kỹ thuật đồng đẳng hóa]
$X$ là este no, 2 chức, $Y$ là este tạo bởi glixerol và một axit cacboxylic đơn chức, không no chứa một liên kết $\mathrm{C}=\mathrm{C}$ ( $\mathrm{X}, \mathrm{Y}$ đều mạch hở). Đốt cháy hoàn toàn 17,02 gam hỗn hợp $\mathrm{E}$ chứa $\mathrm{X}, \mathrm{Y}$ thu được 18,144 lít $\mathrm{CO}_2$ (đktc). Mặt khác đun nóng $0,12 \mathrm{~mol} \mathrm{E}$ cần dùng $570 \mathrm{ml}$ dung dịch $\mathrm{NaOH}$ 0,5M; cô cạn dung dịch sau phản ứng thu được hỗn hợp chứa 3 muối có khối lượng $\mathrm{m}$ gam và hỗn hợp 2 ancol có cùng số nguyên tử cacbon. Giá trị của $\mathrm{m}$ là
	\choice
	{%
	\True	27,09 gam
	}
	{%
		27,24 gam
	}
	{%
		19,63 gam
	}
	{%
		28,14 gam
	}
	\huongdan
	{%
		
		
		
	}			
\end{vdex}
%%%%%%%%%%%%%%%%%%% KẾT THÚC VÍ DỤ 10 %%%%%%%%%%%%%%%%%%%%%%%%%%%%%%%%%%%%%%

\subsubsection{Phương pháp quy đổi}
\begin{ntdppg}{Phương pháp giải}
	\begin{enumerate}
		\item \textbf{Phương pháp thủy phân hóa}
		\begin{itemize}
		\item Dấu hiệu: Đầu bài cho các chất: axit, ancol,este tạo bởi axit và ancol đó
		\item Bản chất: Khi thủy phân ta quy este về axit và ancol\\
		\schemestart
		este
		\+
		\chemfig{H_2O}
		\arrow{<=>}[,0.6,,,]
		axit
		\+
		ancol
		\schemestop\\
		$\Rightarrow $ este $ =  $ axit $ + $ ancol $ - H_2O $ 
		
		\begin{notegsnd}
			- Số mol $ COOH $ trong $ X' $  $ = $ số mol $ NaOH/KOH $ phản ứng với $ X $\\
			- Lượng axit/ancol trong $ X' $  $ = $ lượng axit/ancol ban đầu $ + $ lượng axit/ancol sinh ra từ phản ứng thủy phân $ X $\\
			- $ n_{H_2O}< 0 $ ( giá trị âm)
		\end{notegsnd}
		\end{itemize}
		\item \textbf{Phương pháp thủy phân hóa kết hợp với đồng đẳng hóa}\\
		
		Hỗn hợp $ E $ 
		$ \begin{cases}
			 \text{-X: axit no, đơn chức} \\
			 \text{-Y: ancol no, đơn chức}\\
			 \text{-Z: Este tạo bởi X,Y}
			\end{cases} 
		$
			 \schemestart
			 \arrow{->[\scriptsize\text{ĐĐH} ][(1)][3pt]}[,0.8,,,]
			 \schemestop$ \begin{cases}
			\chemfig{HCOOH} \\
			\chemfig{CH_3OH}\\
			\chemfig{HCOOCH_3}\\
			\chemfig{CH_2}
		\end{cases} 
		$\schemestart
		\arrow{->[\scriptsize\text{TPH} ][(2)][3pt]}[,0.8,,,]
		\schemestop$ \begin{cases}
			\chemfig{HCOOH} \\
			\chemfig{CH_3OH}\\
			\chemfig{CH_2}\\
			\chemfig{H_2O}
		\end{cases} 
		$	 	 
		
		\item \textbf{Thủy phân hóa và quy đổi chất thành các nguyên tố}
		
		\begin{tikzpicture}[declare function={d=1.0cm;}]
			\path 
			(0,0) coordinate (A)
			(.3,d) coordinate (B)
			(3,d) coordinate (C)
			(.3,-d) coordinate (Bt)
			(3,-d) coordinate (Ct)
			;
			\tikzset{%
				muitenhainhanh/.pic={%				 		
					\draw[->] (A)--(B)--(C) ;
					\draw[->](A)--(Bt)--(Ct) ;		
				}
			}
			\path 
			(B)--(C) node[sloped,pos=.5,above] {tp hóa}
			(Bt)--(Ct) node[sloped,pos=.5,above] {quy đổi}
			;
			\path pic [local bounding box=A1] at (0,0) {muitenhainhanh};
			\path (A1.west) node[anchor=east] {Hỗn hợp $ E $ ($ m~\mathrm{gam}$)}
			(C.east) node[anchor=west] {axit $ + $ ancol $ -H_2O $}
			(Ct.east) node[anchor=west] {$ C  + H + O $}
			;
		\end{tikzpicture}
		
	\end{enumerate}
\end{ntdppg}
\begin{vdm}{Bài tập vận dụng}
\end{vdm}
%%%%%%%%%%%%%%%%%%%%% Bắt dầu ví dụ 1 %%%%%%%%%%%%%%%%%%%%%%%%%
\begin{vdex}[1][Phương pháp quy đổi]
Đốt cháy hoàn toàn 2,76 gam hỗn hợp $\mathrm{E}$ gồm axit cacboxylic đơn chức $\mathrm{X}$, ancol metylic $\mathrm{Y}$ và ứng vừa đủ với $15 \mathrm{ml}$ dung dịch $\mathrm{NaOH} 2 \mathrm{M}$, thu được $0,03 \mathrm{~mol} \mathrm{CH}_3 \mathrm{OH}$. Công thức của $\mathrm{X}$ là	
	\choice
{%
$\mathrm{C}_2 \mathrm{H}_5 \mathrm{COOH}$
}
{%
$\mathrm{CH}_3 \mathrm{COOH}$	
}
{%
\True $\mathrm{C}_2 \mathrm{H}_3 \mathrm{COOH}$	
}
{%
$\mathrm{C}_3 \mathrm{H}_5 \mathrm{COOH}$	
}
\huongdan
{%


}
\end{vdex}
%%%%%%%%%%%%%%%%%%%%% Kết thúc ví dụ 1 %%%%%%%%%%%%%%%%%%%%%%%%%
%%%%%%%%%%%%%%%%%%%%% Bắt dầu ví dụ 2 %%%%%%%%%%%%%%%%%%%%%%%%%
\begin{vdex}[2][Phương pháp quy đổi]
	Xà phòng hóa hoàn toàn 1,40 gam hỗn hợp $\mathrm{X}$ gồm $\mathrm{RCOOH}, \mathrm{RCOOC}_2 \mathrm{H}_5, \mathrm{C}_2 \mathrm{H}_5 \mathrm{OH}$ với $200 \mathrm{ml}$ dung dịch $\mathrm{KOH} 0,1 \mathrm{M}$, thu được 0,46 gam $\mathrm{C}_2 \mathrm{H}_5 \mathrm{OH}$. Mặt khác, đốt cháy hoàn toàn 1,40 gam hỗn hợp $\mathrm{X}$ thu được 1,12 lít $\mathrm{CO}_2$ (đktc) và 4,32 gam $\mathrm{H}_2 \mathrm{O}$. Công thức của este trong $\mathrm{X}$ là
	\choice
	{%
		$\mathrm{CH}_3 \mathrm{COOCH}_3$
	}
	{%
		$\mathrm{CH}_3 \mathrm{COOC}_2 \mathrm{H}_5$	
	}
	{%
		$\mathrm{HCOOC}_2 \mathrm{H}_5$	
	}
	{%
	\True	$\mathrm{C}_2 \mathrm{H}_5 \mathrm{COOC}_2 \mathrm{H}_5$	
	}
	\huongdan
	{%
		
		
	}
\end{vdex}
%%%%%%%%%%%%%%%%%%%%% Kết thúc ví dụ 2 %%%%%%%%%%%%%%%%%%%%%%%%%
%%%%%%%%%%%%%%%%%%%%% Bắt dầu ví dụ 3 %%%%%%%%%%%%%%%%%%%%%%%%%

\begin{vdex}[2][Phương pháp quy đổi]
	Cho hỗn hợp $\mathrm{X}$ gồm các chất: $\left(\mathrm{CH}_3 \mathrm{COO}\right)_3 \mathrm{C}_3 \mathrm{H}_5, \mathrm{CH}_3 \mathrm{COOCH}_2 \mathrm{CH}\left(\mathrm{OOCCH}_3\right) \mathrm{CH}_2 \mathrm{OH}$, $\mathrm{CH}_3 \mathrm{COOH}, \mathrm{CH}_3 \mathrm{COOCH}_2 \mathrm{CHOHCH} \mathrm{C}_2 \mathrm{OH}$ và $\mathrm{CH}_2 \mathrm{OHCHOHCH} \mathrm{CH}_2 \mathrm{OH}$, trong đó $\mathrm{CH}_3 \mathrm{COOH}$ chiếm $10 \%$ tổng số mol hỗn hợp. Đun nóng $\mathrm{m}$ gam hỗn hợp $\mathrm{X}$ với dung dịch $\mathrm{NaOH}$ vừa đủ, thu được dung dịch chứa 20,5 gam natri axetat và $0,604 \mathrm{~m}$ gam glixerol. Mặt khác, để đốt cháy $\mathrm{m}$ gam hổn hợp $\mathrm{X}$ cần $\mathrm{V}$ lit khí $\mathrm{O}_2$ (đktc). Giá trị của $\mathrm{V}$ gần nhất là
	\choice
	{%
	\True	$25,3$
	}
	{%
		$24,6$	
	}
	{%
		$24,9$	
	}
	{%
		$25,5$	
	}
	\huongdan
	{%
		
		
	}
\end{vdex}
%%%%%%%%%%%%%%%%%%%%% Kết thúc ví dụ 3 %%%%%%%%%%%%%%%%%%%%%%%%%
%%%%%%%%%%%%%%%%%%%%% Bắt dầu ví dụ 4 %%%%%%%%%%%%%%%%%%%%%%%%%
\begin{vdex}[3][Phương pháp quy đổi]
	Cho hỗn hợp $\mathrm{X}$ gồm một axit cacboxylic đơn chức, một ancol đơn chức và este tạo bởi axit và ancol đó. Cho 1,55 gam hỗn hợp $\mathrm{X}$ tác dụng vừa hết với $125 \mathrm{ml}$ dung dịch $\mathrm{KOH} 0,1 \mathrm{M}$ thu được $\mathrm{m}$ gam muối và $0,74 \mathrm{gam}$ ancol có số mol tương ứng là $0,01 \mathrm{~mol}$. Mặt khác, đốt cháy hoàn toàn 1,55 gam $\mathrm{X}$ thu được $0,0775 \mathrm{~mol} \mathrm{CO}_2$ và $0,07 \mathrm{~mol}$ nước. Giá trị của $\mathrm{m}$ là?	
	\choice
	{%
		1,3875 gam
	}
	{%
	\True	1,375 gam	
	}
	{%
		1,175 gam	
	}
	{%
		1,275 gam	
	}
	\huongdan
	{%
		
		
	}
\end{vdex}
%%%%%%%%%%%%%%%%%%%%% Kết thúc ví dụ 4 %%%%%%%%%%%%%%%%%%%%%%%%%
%%%%%%%%%%%%%%%%%%%%% Bắt dầu ví dụ 5 %%%%%%%%%%%%%%%%%%%%%%%%%
\begin{vdex}[3][Phương pháp quy đổi]
	Cho $\mathrm{X}, \mathrm{Y}\left(\mathrm{M}_{\mathrm{X}}<\mathrm{M}_{\mathrm{Y}}\right)$ là hai axit kế tiếp thuộc cùng dãy đồng đẳng axit fomic; $\mathrm{Z}$ là este hai chức tạo bởi $\mathrm{X}, \mathrm{Y}$ và ancol T. Đốt cháy 12,52 gam hỗn hợp $\mathrm{E}$ chứa $\mathrm{X}, \mathrm{Y}, \mathrm{Z}, \mathrm{T}$ (đều mạch hở) cần dùng 8,288 lít $\mathrm{O}_2$ (đktc), thu được 7,2 gam nước. Mặt khác, đun nóng 12,52 gam $\mathrm{E}$ cần dùng $380 \mathrm{ml}$ dung dịch $\mathrm{NaOH} 0,5 \mathrm{M}$. Biết rằng ở điều kiện thường, ancol $\mathrm{T}$ không tác dụng được với $\mathrm{Cu}(\mathrm{OH})_2$. Phần trăm số $\mathrm{mol}$ của $\mathrm{X}$ có trong hỗn hợp $\mathrm{E}$ là
	\choice
	{%
		$50 \%$
	}
	{%
	\True	$60 \%$	
	}
	{%
		$75 \%$	
	}
	{%
		$70 \%$	
	}
	\huongdan
	{%
		
		
	}
\end{vdex}
%%%%%%%%%%%%%%%%%%%%% Kết thúc ví dụ 5 %%%%%%%%%%%%%%%%%%%%%%%%%
%%%%%%%%%%%%%%%%%%%%% Bắt dầu ví dụ 6 %%%%%%%%%%%%%%%%%%%%%%%%%

\begin{vdex}[4][Phương pháp quy đổi]
	Cho $\mathrm{X}, \mathrm{Y}$ là hai chất thuộc dãy đồng đẳng của axit acrylic và có $\mathrm{M}_{\mathrm{X}}<\mathrm{M}_{\mathrm{Y}} ; \mathrm{Z}$ là một ancol có cùng số nguyên tử $\mathrm{C}$ với $\mathrm{X}$; T là este hai chức tạo bởi $\mathrm{X}, \mathrm{Y}, \mathrm{Z}$. Đốt cháy hoàn toàn 11,16g hỗn hợp $\mathrm{E}$ gồm $\mathrm{X}, \mathrm{Y}, \mathrm{Z}, \mathrm{T}$ cần vừa đủ 13,216 lít khí $\mathrm{O}_2$ ( đktc), thu được khí $\mathrm{CO}_2$ và $9,36 \mathrm{~g}$ nước. Mặt khác $11,16 \mathrm{~g}$ E tác dụng tối đa với dung dịch chứa $0,04 \mathrm{~mol} \mathrm{Br}_2$. Khối lượng muối thu được khi cho cùng lượng $\mathrm{E}$ trên tác dụng hết với dung dịch $\mathrm{NaOH}$ dư là:
	\choice
	{%
		$4,40 \mathrm{~g}$
	}
	{%
	\True	$4,04 \mathrm{~g}$	
	}
	{%
		$4,68 \mathrm{~g}$	
	}
	{%
		$3,16 \mathrm{~g}$	
	}
	\huongdan
	{%
		
		
	}
\end{vdex}
%%%%%%%%%%%%%%%%%%%%% Kết thúc ví dụ 6 %%%%%%%%%%%%%%%%%%%%%%%%%

\begin{dangntd}{Bài toán tổng hợp}
	\begin{ntdppg}{Phương pháp giải}
		\begin{itemize}
			\item Vận dụng linh hoạt định luật bảo toàn nguyên tố, định luật bảo toàn khối lượng.
			\item Dùng phương pháp quy đổi: Quy về chất nhỏ nhất và nhóm $ -CH_2- $
			\item Dùng kĩ thuật đồng đẳng hóa
		\end{itemize}
	\end{ntdppg}
\end{dangntd}


