\def\writeANS{\TLdung{A}\TLdung{B}\TLdung{C}\TLdung{D}}
\begin{loigiaiex}{13}
  \begin {itemize} \item Số cách xếp $8$ học sinh theo một hàng dọc là $8!=40320$ (cách) \item Số cách xếp học sinh nam đứng cạnh nhau là $5!=120$ (cách).\\ Số cách xếp học sinh nữ đứng cạnh nhau là $3!=6$ (cách).\\ Số cách xếp chỗ hai nhóm học sinh nam và học sinh nữ là $2!=2$ (cách).\\ Như vậy, số cách xếp học sinh cùng giới đứng cạnh nhau là $120\cdot 6\cdot 2=1440$ (cách). \item Số cách xếp học sinh nữ đứng cạnh nhau là $3!=6$ (cách).\\ Giả sử nhóm học sinh nữ là một đối tượng, ta cần xếp chỗ nhóm nữ và $5$ học sinh nam (nghĩa là $6$ đối tượng).\\ Số cách sắp xếp nhóm học sinh nữ và $5$ học sinh nam là $6!=720$ (cách).\\ Như vậy, số cách xếp học sinh nữ luôn đứng cạnh nhau là $6\cdot 720=4320$ (cách). \item Số cách xếp $5$ học sinh nam là $5!=120$ (cách).\\ Khi xếp $5$ học sinh nam, ta có $6$ khoảng trống được tạo ra (tính cả hai đầu hàng). Để sắp xếp $3$ bạn nữ không đứng cạnh nhau, ta cần chọn ra $3$ khoảng trống từ $6$ khoảng trống trên. Số cách xếp các bạn nữ khi đó là $C^3_6=20$ (cách).\\ Như vậy, số cách xếp không có em nữ nào đứng cạnh nhau là $120\cdot 20=2400$ (cách). \end {itemize}  \phantom {a}\hfill { \faKey ~\writeANS } 
\end{loigiaiex}
\def\writeANS{\TLdung{A}\TLsai{B}\TLdung{C}\TLsai{D}}
\begin{loigiaiex}{14}
  Ta có khai triển $$(1-x)^6=C^0_6 x^6 -C^1_6 x^5+ C^2_6 x^4 - C^3_6 x^3 +C^4_6 x^2- C^5_6 x +C^6_6.$$ \begin {itemize} \item Hệ số của $x^2$ trong khai triển là $C_6^2=C^4_6$. \item Hệ số của $x^3$ trong khai triển là $-C_6^3$. \item Hệ số của $x^5$ trong khai triển là $-C_6^1=-C^5_6$. \item Thay $x=1$ vào hai vế của khai triển ta có $$0=C_6^0-C_6^1+C_6^2-C_6^3+C_6^4-C_6^5+C_6^6.$$ \end {itemize}  \phantom {a}\hfill { \faKey ~\writeANS } 
\end{loigiaiex}
\def\writeANS{\TLdung{A}\TLdung{B}\TLsai{C}\TLdung{D}}
\begin{loigiaiex}{15}
  \begin {itemize} \item Ta có $2\overrightarrow {a}=(4;-4)$, $3\overrightarrow {c}=(0;-3)$.\\ Suy ra $2\overrightarrow {a}-\overrightarrow {b}-3\overrightarrow {c}=(4-4-0;-4-1+3)=(0;-2)$. \item Ta có $\overrightarrow {e}=(1;-1)$ và $\overrightarrow {a}=(2;-2)$. Suy ra $\overrightarrow {e}=\dfrac {1}{2}\overrightarrow {a}$.\\ Do đó, véctơ $\overrightarrow {e}$ cùng phương, cùng hướng với $\overrightarrow {a}$. \item Ta có $\overrightarrow {f}=\left (-1 ;-\dfrac {1}{4}\right )$ và $\overrightarrow {b}=(4;1)$. Suy ra $\overrightarrow {f}=-\dfrac {1}{4}\overrightarrow {b}$.\\ Do đó, véctơ $\overrightarrow {f}$ cùng phương, ngược hướng với $\overrightarrow {a}$. \item Ta có $\dfrac {1}{2} \overrightarrow {b}+\dfrac {5}{2} \overrightarrow {c}=\left (\dfrac {1}{2}\cdot 4+\dfrac {5}{2}\cdot 0; \dfrac {1}{2}\cdot 1+\dfrac {5}{2}\cdot (-1) \right )=(2;-2)$.\\ Suy ra $\dfrac {1}{2} \overrightarrow {b}+\dfrac {5}{2} \overrightarrow {c}=\overrightarrow {a}$. \end {itemize}  \phantom {a}\hfill { \faKey ~\writeANS } 
\end{loigiaiex}
\def\writeANS{\TLsai{A}\TLsai{B}\TLdung{C}\TLsai{D}}
\begin{loigiaiex}{16}
  \begin {center} \begin {tikzpicture}[scale=1,>=stealth, font=\footnotesize , line join=round, line cap=round] \draw [fill=black] (2,4) coordinate (A) node[above]{$A$} circle (1pt); \draw [fill=black] (0,0) coordinate (B) node[left]{$B$} circle (1pt); \draw [fill=black] (6,0) coordinate (C) node[right]{$C$} circle (1pt); \draw [fill=black] (3,0) coordinate (M) node[below]{$M$} circle (1pt); \draw [fill=black] ($(A)!0.667!(M)$) coordinate (G) node[above right]{$G$} circle (1pt); \coordinate (N) at ($(B)!1.5!(G)$); \coordinate (P) at ($(C)!1.5!(G)$); \node at (1,1) []{$d_1$}; \node at (4,1) []{$d_2$}; \draw (B)--(A)--(C)--(B) (A)--(M) (C)--(P) (B)--(N); \end {tikzpicture} \end {center} Gọi $d_1\colon 2x-y+1=0$ và $d_2\colon x+3y-3=0$. Do điểm $A(1;2)$ không thuộc đường thẳng $d_1$ và $d_2$, nên ta giả sử $d_1$ và $d_2$ lần lượt là đường trung tuyến từ đỉnh $B$ và đỉnh $C$.\\ Gọi $G$ là trọng tâm của tam giác $ABC$ và $M$ là trung điểm của $BC$.\\ Tọa độ của $G$ thỏa mãn hệ phương trình $$\heva {&2x-y+1=0\\ &x+3y-3=0}\Leftrightarrow \heva {&x=0\\&y=1}.$$ Như vậy $G(0;1)$. Khi đó $\overrightarrow {AG}=(-1;-1)$. Suy ra $\overrightarrow {AM}=\dfrac {3}{2}\overrightarrow {AG}=\left (-\dfrac {3}{2};-\dfrac {3}{2}\right )$. \\ Mà $A(1;2)$ nên $M\left (-\dfrac {1}{2};\dfrac {1}{2}\right )$.\\ Gọi $B(a;2a+1)$ và $C(-3b+3;b)$ (do $B\in d_1$ và $C\in d_2$).\\ Do $M$ là trung điểm của $BC$ nên ta có hệ phương trình $$\heva {&x_B+x_C=2x_M\\&y_B+y_C=2y_M}\Leftrightarrow \heva {&a-3b+3=-1\\&2a+1+b=1}\Leftrightarrow \heva {&a-3b=-4\\&2a+b=0} \Leftrightarrow \heva {&a=-\dfrac {4}{7}\\&b=\dfrac {8}{7}}.$$ Khi đó điểm $B\left (-\dfrac {4}{7};-\dfrac {1}{7}\right )$ và điểm $C\left (-\dfrac {3}{7};\dfrac {8}{7}\right )$. \\ Do vai trò của $B$, $C$ như nhau nên ngoài kết quả trên ta còn có kết quả điểm $C\left (-\dfrac {4}{7};-\dfrac {1}{7}\right )$ và điểm $B\left (-\dfrac {3}{7};\dfrac {8}{7}\right )$. \begin {itemize} \item Phương án A còn thiếu một trường hợp $C\left (-\dfrac {4}{7};-\dfrac {1}{7}\right )$. \item Phương án B còn thiếu một trường hợp $B\left (-\dfrac {3}{7};\dfrac {8}{7}\right )$. \item Ta có $\overrightarrow {BC}=\left (\dfrac {1}{7};\dfrac {9}{7}\right )$ là véctơ chỉ phương của đường thẳng $BC$.\\ Đường thẳng $BC$ đi qua điểm $B\left (-\dfrac {4}{7};-\dfrac {1}{7}\right )$ và nhận $\overrightarrow {n}=(9;-1)$ làm véctơ pháp tuyến có phương trình là $$9\left (x+\dfrac {4}{7}\right )-1\left (y+\dfrac {1}{7}\right )=0 \text { hay } 9x-y+5=0.$$ \item Phương án C còn thiếu một trường hợp vì có hai điểm $C$ thỏa mãn nên cũng có tương ứng hai phương trình đường thẳng $AC$ thỏa mãn đề. \end {itemize}  \phantom {a}\hfill { \faKey ~\writeANS } 
\end{loigiaiex}
