\section{Đề ôn thi giữa kỳ 2 toán 10}
\subsection{Phần trắc nghiệm}
Câu trắc nghiệm nhiều phương án lựa chọn. Học sinh trả lời từ
câu 1 đến câu 12. Mỗi câu hỏi học sinh \textit{chỉ chọn một} phương án.

\Opensolutionfile{ans}[Ans/Dapan]

\hienthiloigiaiex
	%%%=============EX_1=============%%%
	\begin{ex}
		Trên giá sách có $10$ cuốn sách Toán khác nhau, $7$ cuốn sách Ngữ văn khác nhau và có $5$ cuốn truyện khác nhau. Số cách để Nam chọn một quyển sách để đọc là
		\choice
		{$350$ cách}
		{$75$ cách}
		{$10$ cách}
		{\True $22$ cách}
		\loigiai{
			Theo quy tắc cộng số cách chọn một quyển sách là $10+7+5=22$ (cách).
		}
	\end{ex}
	%%%=============EX_2=============%%%
	\begin{ex}
		Với $k, n$ là các số tự nhiên và $1 \leq k \leq n$, công thức nào sau đây là đúng?
		\choice
		{$A_n^k=\dfrac{n !}{k !}$}
		{\True $A_n^k=\dfrac{n !}{(n-k) !}$}
		{$A_n^k=\dfrac{k !}{n !}$}
		{$A_n^k=\dfrac{(n-k) !}{k !}$}
		\loigiai{
			Công thức đúng là $A_n^k=\dfrac{n !}{(n-k) !}$.
		}
	\end{ex}
	%%%=============EX_3=============%%%
	\begin{ex}
		Cho $k, n$ là các số nguyên dương thoả mãn $n \geq k$. Trong các phát biểu sau, phát biểu nào đúng?
		\choice
		{\True $A_n^k=n(n-1) \cdots(n-k+1)$}
		{$A_n^k=n(n-1) \ldots k$}
		{$A_n^k=\dfrac{n !}{(n-k) ! k !}$}
		{$A_n^k=\dfrac{n !}{k !}$}
		\loigiai{
			Phát biểu đúng là $A_n^k=n(n-1) \cdots(n-k+1)$.
		}
	\end{ex}
	%%%=============EX_4=============%%%
	\begin{ex}
		Một đề thi trắc nghiệm có 10 câu hỏi, mỗi câu có $1$ đáp án đúng trong $4$ đáp án. Giả sử các đáp án được chọn ngẫu nhiên. Số khả năng làm đúng $4$ câu trên $10$ câu của đề thi đó là
		\choice
		{$C_{10}^{10}$}
		{\True $C_{10}^4$}
		{$3^6 C_{10}^4$}
		{$3^6 A_{10}^4$}
		\loigiai{
			Số khả năng làm đúng $4$ câu trên $10$ câu của đề thi đó là $C_{10}^4$.
		}
	\end{ex}
	%%%=============EX_5=============%%%
	\begin{ex}
		Có bao nhiêu số tự nhiên có $2020$ chữ số sao cho tổng các chữ số trong mỗi số bằng $3$?
		\choice
		{$2041209$}
		{$2037172$}
		{\True $2041210$}
		{$4039$}
		\loigiai{
			Gọi số tự nhiên có 3 chữ số là $\overline{a_1a_2\ldots a_{2020}}$. \\
			Theo giả thiết $a_1+a_2+\cdots+a_{2020}=3$ ($a_1,a_2,...,a_{2020} \in \{0;1;2;\ldots;9\}$;$a_1 \ne 0$). \\
			Ta có các trường hợp
			\begin{itemize}
				\item $a_1=1$; trong các chữ số còn lại có 2 chữ số 1:
				Có $C_{2019}^2$ số.
				\item $a_1=1$; trong các chữ số còn lại có 1 chữ số 2: Có $C_{2019}^1$ số.
				\item $a_1=2$; trong các chữ số còn lại có 1 chữ số 1: Có $C_{2019}^1$ số.
				\item $a_1=3$: Có đúng 1 số.
			\end{itemize}
			Vậy có $C_{2019}^2+2C_{2019}^1+1=2041210$ số.
		}
	\end{ex}
	%%%=============EX_6=============%%%
	\begin{ex}
		Hệ số của $x^2$ trong khai triển biểu thức $(2-3 x)^4$ là
		\choice
		{\True $216$}
		{$-216$}
		{$72$}
		{$-72$}
		\loigiai{
			Ta có $(2-3 x)^4=2^4+C_4^1 2^3(-3x)^1+C_4^2 2^2(-3x)^2+C_4^3 2^1(-3x)^3+(-3x)^4$. \\
			Suy ra hệ số của $x^2$ là $C_4^2 2^2(-3)^2=216$.
		}
	\end{ex}

%%%=============EX_1=============%%%
Thí sinh trả lời tù câu $1$ đến câu $12$. Mỗi câu hỏi thi sinh chỉ chọn một phuơng án đủng nhất.
\begin{ex}%[0D8H1-1]%[Dự án đề kiểm tra toán khối 11 GHKII-NH23-24- Đợt 2-Nguyễn Quang Hiệp]%[Đề số 3 - CD]
	Trên giá sách có $10$ cuốn sách Toán khác nhau, $7$ cuốn sách Ngữ văn khác nhau và có $5$ cuốn truyện khác nhau. Số cách để Nam chọn một quyển sách để đọc là
	\choice
	{$350$ cách}
	{$75$ cách}
	{$10$ cách}
	{\True $22$ cách}
	\loigiai{
		Số cách chọn một quyển sách là $10+7+5$ cách.
	}
\end{ex}
\begin{ex}%[0D8N2-1]%[Dự án đề kiểm tra toán khối 11 GHKII-NH23-24- Đợt 2-Nguyễn Quang Hiệp]%[Đề số 3 - CD]
	Với $k, n$ là các số tự nhiên và $1 \leq k \leq n$, công thức nào sau đây là đúng?
	\choice
	{$\mathrm{A}_n^k=\dfrac{n !}{k !}$}
	{\True $\mathrm{A}_n^k=\dfrac{n !}{(n-k) !}$}
	{$\mathrm{A}_n^k=\dfrac{k !}{n !}$}
	{$\mathrm{A}_n^k=\dfrac{(n-k) !}{k !}$}
	\loigiai{
		Theo SGK công thức đúng là $\mathrm{A}_n^k=\dfrac{n !}{(n-k) !}$.
	}
\end{ex}
\begin{ex}%[0D8N2-1]%[Dự án đề kiểm tra toán khối 11 GHKII-NH23-24- Đợt 2-Nguyễn Quang Hiệp]%[Đề số 3 - CD]
	Cho $k, n$ là các số nguyên dương thoả mãn $n \geq k$. Trong các phát biểu sau, phát biểu nào đúng?
	\choice
	{\True $\mathrm{A}_n=n(n-1) \ldots(n-k+1)$}
	{$\mathrm{A}_n^k=n(n-1) \ldots k$}
	{$\mathrm{A}_n^k=\dfrac{n !}{(n-k) ! k !}$}
	{$\mathrm{A}_n^k=\dfrac{n !}{k !}$}
	\loigiai{
		Cho $k, n$ là các số nguyên dương thoả mãn $n \geq k$, phát biểu nào đúng là $\mathrm{A}_n=n(n-1) \ldots(n-k+1)$.
	}
\end{ex}
\begin{ex}%[0D8H2-5]%[Dự án đề kiểm tra toán khối 11 GHKII-NH23-24- Đợt 2-Nguyễn Quang Hiệp]%[Đề số 3 - CD]
	Một đề thi trắc nghiệm có $10$ câu hỏi, mỗi câu có $1$ đáp án đúng trong $4$ đáp án. Giả sử các đáp án được chọn ngẫu nhiên. Số khả năng làm đúng $4$ câu trên $10$ câu của đề thi đó là
	\choice
	{$\mathrm{C}_{10}^{10}$}
	{$\mathrm{C}_{10}^4$}
	{\True $3^6 \mathrm{C}_{10}^4$}
	{$3^6 \mathrm{A}_{10}^4$}
	\loigiai{
		Mỗi cách chọn $4$ câu làm đúng trong $10$ câu là một tổ hợp chập $4$ của $10$ phần tử nên số cách chọn là $\mathrm{C}_{10}^4$.\\
		Vì $6$ câu còn lại làm sai mà có $3$ đáp án sai mỗi câu nên số khả năng làm đúng $4$ câu trên $10$ câu của đề thi đó là $3 \cdot 3 \cdot 3 \cdot 3 \cdot 3 \cdot 3 \cdot \mathrm{C}_{10}^4=3^6 \mathrm{C}_{10}^4$.}
\end{ex}
\begin{ex}%[0D8V2-3]%[Dự án đề kiểm tra toán khối 11 GHKII-NH23-24- Đợt 2-Nguyễn Quang Hiệp]%[Đề số 3 - CD]
	Có bao nhiêu số tự nhiên có $2020$ chữ số sao cho tổng các chữ số trong mỗi số bằng $3$?
	\choice
	{$2041209$}
	{$2037172$}
	{\True $2041210$}
	{$4039$}
	\loigiai{
		Do tổng các chứ số trong mỗi số là $3$ nên ta xét các trương hợp sau:
		\begin{itemize}
			\item Trường hợp 1: có một số duy nhất là số $300 \ldots 0$ (có tất cå $2019$ số $0$).
			\item Trường hợp 2: có $3$ chữ số $1$ trong số cần tìm.	Vị trí đầu khác $0$ nên có $1$ cách xếp. Hai chữ số $1$ còn lại có $\mathrm{C}_{2019}^{2}$ cách xếp nên trường hợp này có $\mathrm{C}_{2019}^{2}$ số.
			\item Truờng hợp 3: chỉ có hai chữ số khác $0$ và chữ số $1$ và chữ số $2$ còn lại đều là chữ số $0$. Vị trí đầu có $2$ cách xếp. Có $\mathrm{C}_{2019}^{1}$ cách xếp chữ số còn lại nên trường hợp này có $2 \cdot \mathrm{C}_{2019}^{1}$ số. Vậy có tất cả $2041210$ số.  
		\end{itemize}
	}
\end{ex}
\begin{ex}%[0D8H3-2]%[Dự án đề kiểm tra toán khối 11 GHKII-NH23-24- Đợt 2-Nguyễn Quang Hiệp]%[Đề số 3 - CD]
	Hệ số của $x^2$ trong khai triển biểu thức $(2-3 x)^4$ là
	\choice
	{\True $216$}
	{$-216$}
	{$72$}
	{$-72$}
	\loigiai{
		Ta có $(2-3 x)^4=(3 x-2)^4$.\\
		Số hạng chứa $x^2$ trong khai triển biểu thức $(2-3 x)^4=(3 x-2)^4$ là $6 \cdot(3 x)^2 \cdot(-2)^2=216 x^2$.\\
		Vậy hệ số của $x^2$ là $216$.}
\end{ex}
\begin{ex}%[0H9H1-1]%[Dự án đề kiểm tra toán khối 11 GHKII-NH23-24- Đợt 2-Nguyễn Quang Hiệp]%[Đề số 3 - CD]
	Trong mặt phẳng tọa độ $O x y$, cho tam giác $A B C$ có $A(-1;-5), B(5; 2)$ và trọng tâm là gốc tọa độ. Tọa độ điểm $C$ là
	\choice
	{$(4;-3)$}
	{$(-4;-3)$}
	{\True $(-4; 3)$}
	{$(4; 3)$}
	\loigiai{
		Giả sử $C(x; y)$ trọng tâm tam giác $A B C$ là gốc tọa độ, tức là $O(0; 0)$ nên ta có
		$\heva{&\dfrac{-1+5+x}{3}=0\\&\dfrac{-5+2+y}{3}=0} \Leftrightarrow \heva{&x=-4\\&y=3.}$\\
		Vậy $C(-4;3)$.}
\end{ex} 
\begin{ex}%[0H9H1-1]%[Dự án đề kiểm tra toán khối 11 GHKII-NH23-24- Đợt 2-Nguyễn Quang Hiệp]%[Đề số 3 - CD]
	Trong mặt phẳng toạ độ $O x y$, cho tam giác $A B C$ và $M(4;-1), N(0; 2), P(5; 3)$ lần lượt là trung điểm của các cạnh $B C, C A, A B$. Toạ độ điểm $B$ là
	\choice
	{$(1; 6)$}
	{\True $(9; 0)$}
	{$(-1;-2)$}
	{$(0; 9)$}
	\loigiai{
		Giả sử $B(x; y)$. Ta có $\overrightarrow{P B}=(x-5; y-3), \overrightarrow{N M}=(4;-3)$.\\
		Vì $M N$ là đường trung bình ứng với cạnh $A B$, mà $P$ là trung điểm $A B$ nên
		$\overrightarrow{P B}=\overrightarrow{N M} \Leftrightarrow\heva{&x-5=4\\ &y-3=-3} \Leftrightarrow \heva{&x=9\\ &y=0.}$\\
		Vậy $B(9; 0)$.}
\end{ex}
\begin{ex}%[0H9H1-1]%[Dự án đề kiểm tra toán khối 11 GHKII-NH23-24- Đợt 2-Nguyễn Quang Hiệp]%[Đề số 3 - CD]
	Trong mặt phẳng toạ độ $O x y$, cho hai điểm $A(-3; 4)$ và $B(6;-2)$. Điểm $M$ thuộc trục tung sao cho ba điểm $A, B, M$ thẳng hàng. Toạ độ điểm $M$ là
	\choice
	{$(0; 3)$}
	{$(0;-3)$}
	{$(0;-2)$}
	{\True $(0; 2)$}
	\loigiai{
		Do $M \in O y$ nên giả sử $M(0; m)$. Ta có $\overrightarrow{A M}=(3; m-4), \overrightarrow{A B}=(9;-6)$.\\
		Vì $A, B, M$ thẳng hàng nên $\dfrac{3}{9}=\dfrac{m-4}{-6} \Leftrightarrow m=2$. Vậy $M(0; 2)$.}
\end{ex}
\begin{ex}%[0H9H1-1]%[Dự án đề kiểm tra toán khối 11 GHKII-NH23-24- Đợt 2-Nguyễn Quang Hiệp]%[Đề số 3 - CD]
	Trong mặt phẳng toạ độ $O x y$, cho hai điểm $A(-4; 5)$ và $B(8;-1)$. Điểm $P$ thuộc trục hoành sao cho ba điểm $A, B, P$ thẳng hàng. Toạ độ điểm $P$ là
	\choice
	{$(0; 3)$}
	{$(0;-3)$}
	{$(-6; 0)$}
	{$(6; 0)$}
	\loigiai{
		Do $P \in O x$ nên giả sử $P(p; 0)$.\\
		Ta có $\overrightarrow{A P}=(p+4;-5), \overrightarrow{A B}=(12;-6)$.\\
		Vì $A, B, P$ thẳng hàng nên $\dfrac{p+4}{12}=\dfrac{-5}{-6} \Leftrightarrow p=6$. Vậy $P(6; 0)$.\\}
\end{ex}
\begin{ex}%[0H9H3-2]%[Dự án đề kiểm tra toán khối 11 GHKII-NH23-24- Đợt 2-Nguyễn Quang Hiệp]%[Đề số 3 - CD]
	Trong mặt phẳng tọa độ $O x y$, cho ba điểm $A(2; 4), B(0;-2), C(5; 3)$. Đường thẳng đi qua điểm $A$ và song song với đường thẳng $B C$ có phương trình là
	\choice
	{$x-y+5=0$}
	{$x+y-5=0$}
	{\True $x-y+2=0$}
	{$x+y=0$}
	\loigiai{
		Ta có $\overrightarrow{BC}=(5;5)$.\\
		Gọi $(d)$ là đường thẳng đi qua $A$ và song song với $BC$.\\
		Khi đó vec-tơ pháp tuyến của $(d)$ là $\overrightarrow{n}_{d}=(1;-1)$.\\
		Khi đó phương trình của đường thẳng $(d)$ là 
		\[1\cdot (x-2)-1\cdot (y-4)=0\Leftrightarrow x-y+2=0.\]
	}
\end{ex}
\begin{ex}%[0H9H3-5]%[Dự án đề kiểm tra toán khối 11 GHKII-NH23-24- Đợt 2-Nguyễn Quang Hiệp]%[Đề số 3 - CD]
	Trong mặt phẳng tọa độ $O x y$, cho điểm $M(2; 4)$ và đường thẳng $\Delta\colon \heva{&x=5+3 t\\ &y=-5-4 t}$. Khoảng cách từ $M$ đến đường thẳng $\Delta$ là
	\choice
	{$\dfrac{5}{2}$}
	{\True $3$}
	{$5$}
	{$\dfrac{9}{5}$}
	\loigiai{
		Đường thẳng $\Delta$ có vec-tơ chỉ phương là $\overrightarrow{u}_{\Delta}=(3;-4)\Rightarrow \overrightarrow{n}_{\Delta}=(4;3)$ và $\Delta$ đi qua điểm $A(5;-5)$.\\
		Phương trình của đường thẳng $\Delta$ là 
		\[4(x-5)+3(y+5)=0 \Leftrightarrow 4x+3y-5=0.\]
		Khi đó $\mathrm{d}(M,\Delta)=\dfrac{|4\cdot 2+3\cdot 4-5|}{\sqrt{4^2+3^2}}=3$.
	}
\end{ex}

\Closesolutionfile{ans}
\bangdapan{Dapan}

\subsection{Câu trắc nghiệm đúng sai}
Học sinh trả lời từ câu 1 đến câu 4.
Trong mỗi ý \circlenum{A}, \circlenum{B}, \circlenum{C} và \circlenum{D} ở mỗi câu, học sinh chọn đúng hoặc sai.
\setcounter{ex}{0}
\LGexTF
\Opensolutionfile{ansbook}[ansbook/DapanDS]
\Opensolutionfile{ans}[Ans/DapanT]
%%%============EX_1==============%%%
%%%============EX_1==============%%%
% Câu 1
\begin{ex}%[0D8H2-7]%[Dự án đề kiểm tra toán khối 11 GHKII-NH23-24- Đợt 2-Nguyễn Quang Hiệp]%[Đề số 3 - CD]
	Một trường trung học phổ thông có $20$ bạn học sinh tham dự tọa đàm về tháng Thanh niên do Quận Đoàn tổ chức. Vị trí ngồi của trường là khu vực gồm $4$ hàng ghế, mỗi hàng có $6$ ghế, khi đó
	\choiceTF
	{Có $\mathrm{C}_{20}^{6}$ cách sắp xếp $6$ bạn ngồi vào hàng ghế đầu tiên}
	{\True Sau khi sắp xếp xong hàng ghế đầu tiên, có $\mathrm{A}_{14}^{6}$ cách sắp xếp 6 bạn ngồi vào hàng ghế thứ hai}
	{\True Sau khi sắp xếp xong hàng ghế thứ hai, có $\mathrm{A}_{8}^{6}$ cách sắp xếp 6 bạn ngồi vào hàng ghế thứ ba}
	{Sau khi sắp xếp xong hàng ghế thứ ba, có $\mathrm{C}_{6}^{2}$ cách sắp xếp các bạn còn lại ngồi vào hàng ghế cuối cùng}
	\loigiai{		
		\begin{enumerate}
			\item Mỗi cách chọn $6$ bạn trong $20$ bạn để ngồi vào hàng ghế đầu tiên là một chinh hợp chập $6$ của $20$. Vậy có $\mathrm{A}_{20}^{6}$ cách xếp $6$ bạn ngồi vào hàng ghế đầu tiên.
			\item Mỗi cách chọn $6$ bạn trong $14$ bạn để ngồi vào hàng ghế thứ hai là một chỉnh hợp chập $6$ của $14$. Vậy có $\mathrm{A}_{14}^{6}$ cách xếp $6$ bạn ngồi vào hàng ghế thứ hai sau khi sắp xếp xong hàng ghế đầu tiên.
			\item Mỗi cách chọn $6$ bạn trong $8$ bạn để ngồi vào hàng ghế thứ ba là một chỉnh hợp chập $6$ của $8$. Vậy có $\mathrm{A}_{8}^{6}$ cách xếp $6$ bạn ngồi vào hàng ghế thứ ba sau khi sắp xếp xong hai hàng ghế đầu.
			\item Còn lại $2$ bạn ngồi vào hàng ghế cuối cùng. Một cách chọn $2$ ghế trong $6$ ghế để xếp chỗ ngồi cho $2$ bạn là một chỉnh hợp chập $2$ của $6$. Vậy có $\mathrm{A}_{6}^{2}$ cách xếp 2 bạn còn lại ngồi vào hàng ghế cuối cùng. 		
		\end{enumerate}	
	}
\end{ex}
%-------------------------------------- 

% Câu 2
\begin{ex}%[0D8V3-5]%[Dự án đề kiểm tra toán khối 11 GHKII-NH23-24- Đợt 2-Nguyễn Quang Hiệp]%[Đề số 3 - CD]
	Khai triển $(x+1)^{5}$. Khi đó	
	\choiceTF
	{\True Hệ số của $x^{4}$ là $5$}
	{\True Số hạng không chứa $x$ là $1$}
	{$\mathrm{C}_{5}^{0}+\mathrm{C}_{5}^{1}+\mathrm{C}_{5}^{2}+\mathrm{C}_{5}^{3}+\mathrm{C}_{5}^{4}+\mathrm{C}_{5}^{5}=3^{5}$}
	{\True $32 \mathrm{C}_{5}^{0}+16 \mathrm{C}_{5}^{1}+8 \mathrm{C}_{5}^{2}+4 \mathrm{C}_{5}^{3}+2 \mathrm{C}_{5}^{4}+\mathrm{C}_{5}^{5}=3^{5}$}
	\loigiai{
		\begin{enumerate}
			\item Ta có \[(x+1)^{5}=\mathrm{C}_{5}^{0} x^{5}+\mathrm{C}_{5}^{1} x^{4}+\mathrm{C}_{5}^{2} x^{3}+\mathrm{C}_{5}^{3} x^{2}+\mathrm{C}_{5}^{4} x+\mathrm{C}_{5}^{5}=1+5 x+10 x^{2}+10 x^{3}+5 x^{4}+x^{5}.\quad \left(*\right)\]
			\item Từ khai triển $\left(*\right)$, thay $x=1$, ta được\allowdisplaybreaks
			\begin{eqnarray*}
				&&(1+1)^{5}=\mathrm{C}_{5}^{0} \cdot 1^{5}+\mathrm{C}_{5}^{1} \cdot 1^{4}+\mathrm{C}_{5}^{2} \cdot 1^{3}+\mathrm{C}_{5}^{3} \cdot  1^{2}+\mathrm{C}_{5}^{4} \cdot 1+\mathrm{C}_{5}^{5}\\ &\Leftrightarrow&2^5=\mathrm{C}_{5}^{0}+\mathrm{C}_{5}^{1}+\mathrm{C}_{5}^{2}+\mathrm{C}_{5}^{3}+\mathrm{C}_{5}^{4}+\mathrm{C}_{5}^{5}.
			\end{eqnarray*}
			\item Từ khai triển $\left(*\right)$, thay $x=2$, ta được\allowdisplaybreaks
			\begin{eqnarray*}
				&&(2+1)^{5}=\mathrm{C}_{5}^{0} \cdot 2^{5}+\mathrm{C}_{5}^{1} \cdot 2^{4}+\mathrm{C}_{5}^{2} \cdot 2^{3}+\mathrm{C}_{5}^{3} \cdot 2^{2}+\mathrm{C}_{5}^{4} \cdot 2+\mathrm{C}_{5}^{5}\\
				&\Leftrightarrow &3^{5}=32 \mathrm{C}_{5}^{0}+16 \mathrm{C}_{5}^{1}+8 \mathrm{C}_{5}^{2}+4 \mathrm{C}_{5}^{3}+2 \mathrm{C}_{5}^{4}+\mathrm{C}_{5}^{5}.
			\end{eqnarray*}
		\end{enumerate}	
	}
\end{ex}

%--------------------------------------
% Câu 3
\begin{ex}%[0H9N1-1]%[Dự án đề kiểm tra toán khối 11 GHKII-NH23-24- Đợt 2-Nguyễn Quang Hiệp]%[Đề số 3 - CD]
	Cho $\vec{a}=3 \vec{i}+\vec{j}, \vec{b}=-2 \vec{j}$.	
	\choiceTF
	{$\vec{a}=(-3;1)$}
	{\True $\vec{b}=(0 ;-2)$}
	{$\vec{a}+\vec{b}=(3 ; 1)$}
	{$\vec{a}-\vec{b}=(3 ;-3)$}
	\loigiai{
		Ta có $\vec{a}=(3 ; 1), \vec{b}=(0 ;-2) \Rightarrow \vec{a}+\vec{b}=(3 ;-1), \vec{a}-\vec{b}=(3 ; 3)$.
	}
\end{ex}
%--------------------------------------
% Câu 4
\begin{ex}%[0H9V3-2]%[Dự án đề kiểm tra toán khối 11 GHKII-NH23-24- Đợt 2-Nguyễn Quang Hiệp]%[Đề số 3 - CD]
	Trong mặt phẳng tọa độ $O x y$, cho tam giác $D E F$ có $D(1 ;-1), E(2 ; 1), F(3 ; 5)$.	
	\choiceTF
	{Đường thẳng vuông góc với đường thẳng $E F$ nhận $\overrightarrow{E F}$ là một vec-tơ chỉ phương}
	{Phương trình đường cao kẻ từ $D$ là $x+y=0$}
	{\True Gọi $I$ là trung điểm của $D F$. Tọa độ của điểm $I$ là $(2 ; 2)$}
	{\True Đường trung tuyến kẻ từ $E$ có phương trình là $x-2=0$}
	\loigiai{
		Đường cao kẻ từ $D$ là đường thẳng vuông góc với đường thằng $E F$ nên nhận $\overrightarrow{E F}(1 ; 4)$ là một vec-tơ pháp tuyến.\\
		Do đó, đường cao kẻ từ $D$ có phương trình là $(x-1)+4(y+1)=0 \Leftrightarrow x+4 y+3=0$.\\
		Gọi $I$ là trung điểm cua $D F$. Tọa độ của điểm $I$ là $(2 ; 2)$.\\
		Đường trung tuyến kẻ từ $E$ có vec-tơ chi phương là $\overrightarrow{E I}(0 ; 1)$ nên nhận $\vec{n}(1 ; 0)$ là một vec-tơ pháp tuyến.\\
		Do đó, đường trung tuyến kẻ từ $E$ có phương trình là $x-2=0$. 	
	}
\end{ex}

\Closesolutionfile{ans}
\Closesolutionfile{ansbook}





\begin{center}
	\textbf{\textsf{BẢNG ĐÁP ÁN ĐÚNG SAI}}
\end{center}
\input{Ansbook/DapanDS}

%\subsection{Phần tự luận}
%
%\hienthiloigiaibt
%%%=============BT_1=============%%%
\subsection{Câu trả lời ngắn}
\noindent
\textit{Thí sinh trả lời đáp án từ câu 1 đến câu 6}.
\hienthiloigiaibt
%%%=============BT_1=============%%%%
\begin{bt}%[0D8H1-5]%[Dự án đề kiểm tra Toán khối 10 GHKII NH23-24-Đợt 1-Nguyễn Tiến]%[Đề số 3-CD]
	Cho tập hợp $A=\{0;1;2;3;4;5\}$. Có thể lập được bao nhiêu số tự nhiên chẵn có bốn chữ số khác nhau?
	\loigiai{
		Gọi số tự nhiên có bốn chữ số là $\overline{abcd}$.
		\begin{itemize}
			\item \textbf{Trường hợp 1:} $d=0$, suy ra $d$ có $1$ cách chọn.\\
			$a\in A\setminus\{0\}$, suy ra $a$ có $5$ cách chọn.\\
			$b\in A\setminus\{a; d\}$, suy ra $b$ có $4$ cách chọn.\\
			$c\in A\setminus\{a; b; d\}$, suy ra $c$ có $3$ cách chọn.\\
			Số các số tự nhiên trong trường hợp này là $1\cdot 5\cdot 4\cdot 3=60$.
			\item \textbf{Trường hợp 2:} $d\in\{2;4\}$, suy ra $d$ có $2$ cách chọn.\\
			$a\in A\setminus\{0; d\}$, suy ra $a$ có $4$ cách chọn.\\
			$b\in A\setminus\{a; d\}$, suy ra $b$ có $4$ cách chọn.\\
			$c\in A\setminus\{a; b; d\}$, suy ra $c$ có $3$ cách chọn.\\
			Số các số tự nhiên trong trường hợp này là $2\cdot 4\cdot 4\cdot 3=96$.
		\end{itemize}
		Vậy số các số tự nhiên thỏa mãn đề bài là $60+96=156$.
	}
\end{bt}
%%%=============BT_2=============%%%
\begin{bt}%[0D8H2-2]%[Dự án đề kiểm tra Toán khối 10 GHKII NH23-24-Đợt 1-Nguyễn Tiến]%[Đề số 3-CD]
	Giải bất phương trình $2\mathrm{C}_{n+1}^2+3\mathrm{A}_n^2-20<0$.
	\loigiai{
		Điều kiện: $n\in\mathbb{N}$, $n\geq 2$.\\
		Ta có
		\allowdisplaybreaks
		\begin{eqnarray*}
			& & 2\mathrm{C}_{n+1}^2+3\mathrm{A}_n^2-20<0\\
			&\Rightarrow & 2\cdot \dfrac{(n+1)!}{2!\cdot (n-1)!}+3\cdot \dfrac{n!}{(n-2)!}-20<0\\
			&\Rightarrow & n\cdot (n+1)+3n\cdot (n-1)-20<0\\
			&\Rightarrow & 2n^2-n-10<0 \Rightarrow -2<n<\dfrac{5}{2}.
		\end{eqnarray*}
		Vì $n\in\mathbb{N}$, $n\geq 2$ $\Rightarrow n=2$.\\
		Thử lại, ta thấy $n=2$ thỏa mãn bất phương trình đã cho.\\
		Vậy tập nghiệm cần tìm là $S=\{2\}$.
	}
\end{bt}
%%%=============BT_3=============%%%
\begin{bt}%[0D8H3-4]%[Dự án đề kiểm tra Toán khối 10 GHKII NH23-24-Đợt 1-Nguyễn Tiến]%[Đề số 3-CD]
	Tìm hệ số của $x^9$ trong khai triển $(1-x)^{25}$.
	\loigiai{
		Ta có
		\allowdisplaybreaks
		\begin{eqnarray*}
			(1-x)^{25} &= & \displaystyle\sum\limits_{k=0}^{25} \mathrm{C}_{25}^k\cdot 1^{25-k}\cdot (-x)^k\\
			&= & \displaystyle\sum\limits_{k=0}^{25} \mathrm{C}_{25}^k\cdot 1^{25-k}\cdot (-1)^k\cdot x^k.
		\end{eqnarray*}
		Để số mũ của $x$ bằng $9$ thì số $k=9$.\\
		Vậy hệ số của $x^9$ là $\mathrm{C}_{25}^9\cdot 1^{16}\cdot (-1)^9=-\mathrm{C}_{25}^9=-\mathrm{C}_{25}^{16}$.
	}
\end{bt}
%%%==============BT_4==============%%%
\begin{bt}%[0H9H2-1]%[Dự án đề kiểm tra Toán khối 10 GHKII NH23-24-Đợt 1-Nguyễn Trần Anh Tuấn]%[Đề số 3-CD]
	Cho các vectơ $\overrightarrow{a}=\dfrac{1}{2} \overrightarrow{i}-5\overrightarrow{j}$, $\overrightarrow{b}=x \overrightarrow{i}-4\overrightarrow{j}$. Tìm $x$ để $\overrightarrow{a} \perp \overrightarrow{b}$
	\loigiai{
		Ta có $\overrightarrow{a}=\left(\dfrac{1}{2};-5\right)$, $\overrightarrow{b}=(x;-4)$; $\overrightarrow{a} \perp \overrightarrow{b} \Leftrightarrow \dfrac{1}{2} x+(-5)(-4)=0\Leftrightarrow x=-40$.
	}
\end{bt}

%%%==============BT_5==============%%%
\begin{bt}%[0H9V3-4]%[Dự án đề kiểm tra Toán khối 10 GHKII NH23-24-Đợt 1-Nguyễn Trần Anh Tuấn]%[Đề số 3-CD]
	Tìm tham số $m$ để góc giữa hai đường thẳng $\Delta_1 \colon \heva{&x=-1+m t \\&y=9+t}$, $\Delta_2 \colon x+my-4=0$ bằng $60^{\circ}$.
	\loigiai{
		Hai đường thẳng đã cho có cặp vectơ pháp tuyến $\overrightarrow{n}_1=(1;-m), \overrightarrow{n}_2=(1; m)$.\\
		Ta có $\cos \left(\Delta_1, \Delta_2\right)=\dfrac{\left|\overrightarrow{n}_1 \cdot \overrightarrow{n}_2\right|}{\left|\overrightarrow{n}_1\right| \cdot\left|\overrightarrow{n}_2\right|}=\dfrac{\left|1-m^2\right|}{\sqrt{1+m^2} \cdot \sqrt{1+m^2}}=\cos 60^{\circ} \Rightarrow \dfrac{\left|1-m^2\right|}{1+m^2}=\dfrac{1}{2}$.\\
		$\Rightarrow 2\left|1-m^2\right|=1+m^2\Rightarrow\hoac{&2\left(1-m^2\right)=1+m^2\\
			&2\left(1-m^2\right)=-1-m^2} \Rightarrow\hoac{&3m^2=1\\
			&m^2=3} \Rightarrow \hoac{&m=\pm \sqrt{3} \\&m=\pm \sqrt{\dfrac{1}{3}}.}$\\
		Vậy $\hoac{&m=\pm \sqrt{3} \\&m=\pm \sqrt{\dfrac{1}{3}}}$ thỏa mãn.
	}
\end{bt}
%%%==============BT_6==============%%%
\begin{bt}%[0H9V3-7]%[Dự án đề kiểm tra Toán khối 10 GHKII NH23-24-Đợt 1-Nguyễn Trần Anh Tuấn]%[Đề số 3-CD]
	Trong mặt phẳng với hệ tọa độ $Oxy$, cho tam giác $ABC$. Gọi $H$, $K$ lần lượt là chân đường cao hạ từ các đỉnh $B$, $C$ của tam giác $ABC$. Tìm tọa độ các đỉnh của tam giác $ABC$ biết $H(5;-1)$, $K\left(\dfrac{1}{5}; \dfrac{3}{5}\right)$, phương trình đường thẳng $BC$ là $x+3y+4=0$ và điểm $B$ có hoành độ âm.
	\loigiai{
		\begin{center}
			% Author{Nguyễn Trần Anh Tuấn (Rito-Nguyen)}
			%% Hình tam giác nhọn
			\begin{tikzpicture}[line join = round, line cap = round,>=stealth,font=\footnotesize,scale=1]
				\path 
				(0,0)   coordinate (B)
				(60:4) coordinate (A)
				(0:5) coordinate (C)
				($(B)!0.5!(C)$) coordinate (I)				% Lấy trung điểm							
				;					
				\path[opacity=0,name path=duongtron] (I) let \p1=($(C)-(I)$) in circle({veclen(\x1,\y1)});	
				\path[name path=ba,overlay] (B) -- (A)--([turn]0:5cm);
				\path[name path=ca,overlay] (C) -- (A)--([turn]0:5cm);
				\path[name intersections={of= duongtron and ba,by={B,K}}];
				\path[name intersections={of= duongtron and ca,by={C,H}}];
				\draw (A)--(B)--(C)--cycle (B)--(H)--(I)--(K)--(C)
				pic[draw,thin,angle radius=3mm] {right angle = B--K--C}
				pic[draw,thin,angle radius=3mm] {right angle = B--H--C}
				;
				\draw (I) circle (2.5);
				\foreach \i/\g in {A/90,B/-135,C/-45,K/135,H/45,I/-90}{
					\draw[fill=black](\i) circle (1pt) ($(\i)+(\g:3mm)$) node[scale=1]{$\i$};}
			\end{tikzpicture}
		\end{center}
		Gọi $I$ là trung điểm của $BC$. Có $I \in BC$ nên $I(-3t-4; t)$.\\
		Do $\widehat{BHC}=\widehat{BKC}=90^{\circ} \Rightarrow B, K, H, C$ cùng thuộc đường tròn tâm $I$, đường kính $BC$. \\
		Khi đó $IH=IK$ hay $ IH^2=IK^2$\\
		Suy ra $(9+3t)^2+(-1-t)^2=\left(\dfrac{21}{5}+3t\right)^2+\left(\dfrac{3}{5}-t\right)^2$.\\
		Giải phương trình trên ta được $t=-2$. \\
		Suy ra $I(2;-2) \Rightarrow IH=\sqrt{(5-2)^2+(-1+2)^2}=\sqrt{10}$.\\
		Do $B \in BC \Rightarrow B(-3 b-4; b)$ với $x_B< 0\Leftrightarrow-3b-4< 0\Leftrightarrow b >-\dfrac{4}{3}$.\\
		Có $IB=IH\Leftrightarrow IB^2=IH^2\Rightarrow(-3b-6)^2+(b+2)^2=10\Leftrightarrow 10(b+2)^2=10\Leftrightarrow(b+2)^2=1$
		\[\Leftrightarrow \hoac{&b+2=1 \\&b+2=-1} \Leftrightarrow \hoac{&b=-1 \text{ (thỏa mãn)}\\	&b=-3 \text{ (loại)} } \Rightarrow B(-1;-1).\]
		Lại có $I$ là trung điểm $BC$ nên $\heva{&x_C=2x_I-x_B=2\cdot 2-(-1)=5\\&y_C=2y_I-y_B=2\cdot(-2)-(-1)=-3} \Rightarrow C(5;-3)$.\\
		Đường thẳng $AB$ đi qua hai điểm $K\left(\dfrac{1}{5}; \dfrac{3}{5}\right)$ và $B(-1;-1)$ nên có phương trình là
		\[\dfrac{x-\dfrac{1}{5}}{-1-\dfrac{1}{5}}=\dfrac{y-\dfrac{3}{5}}{-1-\dfrac{3}{5}} \Leftrightarrow-\dfrac{8}{5}\left(x-\dfrac{1}{5}\right)=-\dfrac{6}{5}\left(y-\dfrac{3}{5}\right) \Leftrightarrow 4 x-3 y+1=0.\]
		Đường thẳng $AC$ đi qua hai điểm $H(5;-1)$ và $C(5;-3)$ nên có phương trình là $x=5$.\\
		Do $A=AB \cap AC$, suy ra tọa độ điểm $A$ là nghiệm của hệ phương trình
		\[\heva{&4x-3y+1=0\\&x=5} \Leftrightarrow\heva{&x=5\\&y=7&} \Rightarrow A(5; 7).\]
		Vậy tọa độ của điểm $A$ là $A(5; 7)$, tọa độ điểm $B$ là $B(-1;-1)$, tọa độ điểm $C$ là $C(5;-3)$.
	}
\end{bt}