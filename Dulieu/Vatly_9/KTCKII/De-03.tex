%%%Tùy chọn 1: Kì thi
%%%Tùy chọn 2: Môn
%%%Tùy chọn 3: lớp
%%%Tùy chọn 4: Sở/Phòng
%%%Tùy chọn 5: Ngày thi
\begin{name}[Kiểm tra cuối kì II][Vật lý][9][Sở Giáo dục và Đào tạo]{Trường THCS}{2023 - 2024}
\end{name}
\tieumuc{Trắc nghiệm (4 điểm)}
\Opensolutionfile{ans}[Ans/DATN-VL9CKII-De-03]
%\luuloigiaiex
\hienthiloigiaiex
%\taoNdongke[4]{ex}
%\tatloigiaiex
\Opensolutionfile{ansex}[LOIGIAITN/LGTN-VL9CKII-De-03]
%%%=============EX_1=============%%%
\begin{ex}
	Trên hình vẽ (xem hình \ref{fig:hinh03}),hình nào vẽ đúng đường đi của tia khúc xạ khi chiếu chùm tia tới SI từ không khí đến mặt nước.
	\begin{figure}[!thb]
		\begin{center}
			\subcaptionbox{\label{subfig:at}}{
				\begin{tikzpicture}[declare function={r=3.2;h=1.3cm;i=50;}]
					\path(0,0) coordinate (P)
					(r,0) coordinate (Q)	
					($(r,0)+(0,-h)$) coordinate (C)
					($(P)!0.5!(Q)$) coordinate (I)
					($(I)+(0,h)$) coordinate (N)
					($(I)+(0,-h)$) coordinate (M)
					($(I)!1.3!i:(N)$) coordinate (S)
					($(I)!1.1!{i-25}:(M)$) coordinate (R)
					;
					\draw[red,line width=1pt] (S)--(I);
					\draw[red,line width=1pt] (I)--(R);
					\foreach \x/\y in {S/I,I/R}{
						\path (\x)--(\y) node[pos=0.5,sloped,midway] {\tikz{
								\path[fill=red] (0,-2pt)--(0,2pt)--(4pt,0)--cycle;
						}};
					}
					\draw[gray,line width=0.5pt] ([yshift=-h]I)--([yshift=h]I);
					\path[fill=cyan!30,opacity=0.3] (P) rectangle (C);
					\draw[cyan,line width=1pt] (P)--(Q);
					\foreach \t/\g in {P/200,Q/0,N/90,M/-90,I/45,S/120}{
						\path (\t)  node[shift={(\g:5pt)},font=\scriptsize]{$ \t $};
					}
				\end{tikzpicture}
			}\hspace{0.25cm}
			\subcaptionbox{\label{subfig:bt}}{
				\begin{tikzpicture}[declare function={r=3.2;h=1.3cm;i=50;}]
					\path(0,0) coordinate (P)
					(r,0) coordinate (Q)	
					($(r,0)+(0,-h)$) coordinate (C)
					($(P)!0.5!(Q)$) coordinate (I)
					($(I)+(0,h)$) coordinate (N)
					($(I)+(0,-h)$) coordinate (M)
					($(I)!1.3!i:(N)$) coordinate (S)
					($(I)!1.53!{i-0}:(M)$) coordinate (R)
					;
					\draw[red,line width=1pt] (S)--(I);
					\draw[red,line width=1pt] (I)--(R);
					\foreach \x/\y in {S/I,I/R}{
						\path (\x)--(\y) node[pos=0.5,sloped,midway] {\tikz{
								\path[fill=red] (0,-2pt)--(0,2pt)--(4pt,0)--cycle;
						}};
					}
					\draw[gray,line width=0.5pt] ([yshift=-h]I)--([yshift=h]I);
					\path[fill=cyan!30,opacity=0.3] (P) rectangle (C);
					\draw[cyan,line width=1pt] (P)--(Q);
					\foreach \t/\g in {P/200,Q/0,N/90,M/-90,I/45,S/120}{
						\path (\t)  node[shift={(\g:5pt)},font=\scriptsize]{$ \t $};
					}
				\end{tikzpicture}
			}\hspace{0.25cm}
			\subcaptionbox{\label{subfig:ct}}{
				\begin{tikzpicture}[declare function={r=3.2;h=1.3cm;i=48;}]
					\path(0,0) coordinate (P)
					(r,0) coordinate (Q)	
					($(r,0)+(0,-h)$) coordinate (C)
					($(P)!0.5!(Q)$) coordinate (I)
					($(I)+(0,h)$) coordinate (N)
					($(I)+(0,-h)$) coordinate (M)
					($(I)!1.3!i:(N)$) coordinate (S)
					($(I)!1.36!{i+15}:(M)$) coordinate (R)
					;
					\draw[red,line width=1pt] (S)--(I);
					\draw[red,line width=1pt] (I)--(R);
					\foreach \x/\y in {S/I,I/R}{
						\path (\x)--(\y) node[pos=0.5,sloped,midway] {\tikz{
								\path[fill=red] (0,-2pt)--(0,2pt)--(4pt,0)--cycle;
						}};
					}
					\draw[gray,line width=0.5pt] ([yshift=-h]I)--([yshift=h]I);
					\path[fill=cyan!30,opacity=0.3] (P) rectangle (C);
					\draw[cyan,line width=1pt] (P)--(Q);
					\foreach \t/\g in {P/200,Q/0,N/90,M/-90,I/45,S/120}{
						\path (\t)  node[shift={(\g:5pt)},font=\scriptsize]{$ \t $};
					}
				\end{tikzpicture}
			}\hspace{0.25cm}
			\subcaptionbox{\label{subfig:dt}}{
				\begin{tikzpicture}[declare function={r=3.2;h=1.3cm;i=50;}]
					\path(0,0) coordinate (P)
					(r,0) coordinate (Q)	
					($(r,0)+(0,-h)$) coordinate (C)
					($(P)!0.5!(Q)$) coordinate (I)
					($(I)+(0,h)$) coordinate (N)
					($(I)+(0,-h)$) coordinate (M)
					($(I)!1.3!i:(N)$) coordinate (S)
					($(I)!1.56!{-i}:(M)$) coordinate (R)
					;
					\draw[red,line width=1pt] (S)--(I);
					\draw[red,line width=1pt] (I)--(R);
					\foreach \x/\y in {S/I}{
						\path (\x)--(\y) node[pos=0.5,sloped,midway] {\tikz{
								\path[fill=red] (0,-2pt)--(0,2pt)--(4pt,0)--cycle;
						}};
					}
					\foreach \x/\y in {I/R}{
						\path (\x)--(\y) node[pos=0.5,sloped,midway] {\tikz{
								\path[fill=red] (0,-2pt)--(0,2pt)--(-4pt,0)--cycle;
						}};
					}
					\draw[gray,line width=0.5pt] ([yshift=-h]I)--([yshift=h]I);
					\path[fill=cyan!30,opacity=0.3] (P) rectangle (C);
					\draw[cyan,line width=1pt] (P)--(Q);
					\foreach \t/\g in {P/200,Q/0,N/90,M/-90,I/45,S/120}{
						\path (\t)  node[shift={(\g:5pt)},font=\scriptsize]{$ \t $};
					}
				\end{tikzpicture}
			}
		\end{center}
		\caption{\label{fig:hinh03}}
	\end{figure}
	\choice
	{Hình \ref{subfig:ct} }
	{\True Hình \ref{subfig:at}}
	{Hình \ref{subfig:bt}}
	{Hình \ref{subfig:dt}}
	\loigiai{}
\end{ex}

%%%=============EX_2=============%%%
\begin{ex}
	Để truyền đi cùng một công suất điện, nếu đường dây tải điện dài gấp đôi thì công suất hao phí vì tỏa nhiệt sẽ
	\choice
	{\True tăng 2 lần}
	{tăng 4 lần}
	{giảm 2 lần}
	{không tăng, không giảm}
	\loigiai{}
\end{ex}

%%%=============EX_3=============%%%
\begin{ex}
	Có một tia sáng chiếu từ không khí xiên góc vào mặt nước thì
	\choice
	{góc khúc xạ sẽ lớn hơn góc tới}
	{góc khúc xạ sẽ bằng góc tới}
	{\True góc khúc xạ sẽ nhỏ hơn góc tới}
	{cả ba trường hợp A, B, C đều có thể xảy ra}
	\loigiai{}
\end{ex}

%%%=============EX_4=============%%%
\begin{ex}
	Đặt vào hai đầu cuộn sơ cấp của máy biến áp một hiệu điện thế 50 V thì hiệu điện thế giữa hai đầu cuộn thứ cấp là 100 V. Biết cuộn thứ cấp có 500 vòng dây. Số vòng dây của cuộn sơ cấp là
	\choice
	{100 vòng}
	{150 vòng}
	{200 vòng}
	{\True 250 vòng}
	\loigiai{}
\end{ex}

%%%=============EX_5=============%%%
\begin{ex}
	Trường hợp nào dưới đây sẽ không có hiện tượng khúc xạ ánh sáng?
	\choice
	{Tia sáng đi từ không khí vào nước}
	{Tia sáng đi từ không khí vào rượu}
	{Tia sáng đi từ không khí vào dầu ăn}
	{\True Tia sáng truyền trong nước}
	\loigiai{}
\end{ex}

%%%=============EX_6=============%%%
\begin{ex}
	Một vật sáng AB đặt vuông góc với trục chính của một thấu kính hội tụ và nằm ngoài khoảng tiêu cự sẽ cho ảnh
	\choice
	{ảo, cùng chiều, lớn hơn vật}
	{ảo, ngược chiều, lớn hơn vật}
	{\True thật, ngược chiều, có thể lớn hơn hoặc nhỏ hơn vật}
	{thật, ngược chiều, lớn hơn vật}
	\loigiai{}
\end{ex}

%%%=============EX_7=============%%%
\begin{ex}
	Cách làm nào dưới đây có thể tạo ra dòng điện cảm ứng trong một cuộn dây dẫn kín?
	\choice
	{Mắc xen vào cuộn dây dẫn một chiếc pin}
	{Dùng một nam châm đặt gần đầu cuộn dây}
	{Cho một cực của nam châm chạm vào cuộn dây dẫn}
	{\True Đưa một cực của thanh nam châm từ ngoài vào trong cuộn dây}
	\loigiai{}
\end{ex}

%%%=============EX_8=============%%%
\begin{ex}
	Ảnh của một ngọn nến qua một thấu kính phân kì
	\choice
	{có thể là ảnh thật, có thể là ảnh ảo}
	{\True chỉ có thể là ảnh ảo, nhỏ hơn ngọn nến}
	{chỉ có thể là ảnh ảo, lớn hơn ngọn nến}
	{chỉ có thể là ảnh thật, nhỏ hơn ngọn nến}
	\loigiai{}
\end{ex}

%%%=============EX_9=============%%%
\begin{ex}
	Dùng máy ảnh để chụp ảnh một vật đặt cách máy 2 m.Ảnh trên màn hứng ảnh cách vật kính 2 cm có chiều cao là 2 cm. Hãy tính chiều cao của vật?
	\choice
	{\True 2 m}
	{5 m}
	{4 m}
	{8 m}
	\loigiai{}
\end{ex}

%%%=============EX_10=============%%%
\begin{ex}
	Trong trường hợp nào dưới đây, mắt không phải điều tiết
	\choice
	{\True Nhìn vật ở điểm cực viễn}
	{Nhìn vật ở điểm cực cận}
	{Nhìn vật nằm trong khoảng từ cực cận đến cực viễn}
	{Nhìn vật đặt gần mắt hơn điểm cực cận}
	\loigiai{}
\end{ex}

%%%=============EX_11=============%%%
\begin{ex}
	Trên giá đỡ của một kính lúp có ghi 5x. Đó là
	\choice
	{một thấu kính hội tụ có tiêu cự 2,5 cm}
	{một thấu kính phân kì có tiêu cự 2,5 cm}
	{\True một thấu kính hội tụ có tiêu cự 5 cm}
	{một thấu kính phân kì có tiêu cự 5 cm}
	\loigiai{}
\end{ex}

%%%=============EX_12=============%%%
\begin{ex}
	Thấu kính hội tụ có tiêu cự nào dưới đây có thể dùng làm kính lúp được?
	\choice
	{40 cm}
	{30 cm}
	{\True 20 cm}
	{25 cm}
	\loigiai{}
\end{ex}

%%%=============EX_13=============%%%
\begin{ex}
	Sự phân tích ánh sáng trắng được quan sát trong thí nghiệm nào sau đây?
	\choice
	{Chiếu một chùm sáng trắng vào một gương phẳng}
	{Chiếu một chùm sáng trắng qua một tấm thủy tinh mỏng}
	{\True Chiếu một chùm sáng trắng qua một lăng kính}
	{Chiếu một chùm sáng trắng qua một thấu kính phân kì}
	\loigiai{}
\end{ex}

%%%=============EX_14=============%%%
\begin{ex}
	Trong các công việc nào dưới đây, ta đã sử dụng tác dụng nhiệt của ánh sáng?
	\choice
	{Đưa một chậu cây ra ngoài sân phơi cho đỡ cớm}
	{Kê bàn học cạnh của sổ cho sáng}
	{\True Phơi thóc ngoài sân khi trời nắng}
	{Cho ánh sáng chiếu vào bộ pin Mặt Trời của máy tính để nó hoạt động}
	\loigiai{}
\end{ex}

%%%=============EX_15=============%%%
\begin{ex}
	Cách làm nào dưới đây tạo ra sự trộn các ánh sáng màu?
	\choice
	{Chiếu một chùm sáng vàng qua một kính lọc màu đỏ}
	{Chiếu một chùm sáng đỏ qua một kính lọc màu vàng}
	{Chiếu một chùm sáng trắng qua một kính lọc màu đỏ và sau đó qua kính lọc màu vàng}
	{\True Chiếu một chùm sáng đỏ và một chùm sáng vàng vào một tờ giấy trắng}
	\loigiai{}
\end{ex}

%%%=============EX_16=============%%%
\begin{ex}
	Phân tích một chùm sáng là
	\choice
	{\True tìm cách tách từ chùm sáng đó ra những chùm sáng màu khác nhau}
	{tìm cách tách ra một chùm sáng màu đỏ từ chùm sáng đó}
	{tìm cách tạo ra ánh sáng trắng từ chùm sáng đó}
	{tìm cách tách ra một chùm sáng có màu bất kì từ chùm sáng đó}
	\loigiai{}
\end{ex}
\Closesolutionfile{ansex}
\Closesolutionfile{ans}

%%%==========Phần trắc nghiệm đúng sai============%%%

\tieumuc{Trắc Nghiệm Đúng Sai (1 điểm)}(\textit{Trong mỗi câu có 4 ý tương ứng A, B, C, D; Học sinh chọn đúng hoặc sai.})
\Opensolutionfile{ans}[Ans/DATAM-VL9CKII-De-03]
\Opensolutionfile{ansbook}[Ans/DATNTF-VL9CKII-De-03]
\LGexTF
%\taoNdongke{ex}
\Opensolutionfile{ansex}[LOIGIAITN/LGTNTF-VL9CKII-De-03]
%%%=============EX_16=============%%%
\begin{ex}
Cho sơ đồ tạo ảnh $A'B'$ của vật $AB$ tạo bởi một thấu kính như hình vẽ (xem hình \ref{fig:TKHT03} ).
	\begin{figure}[thp]
	\begin{center}
		\begin{tikzpicture}[declare function={h=2.2;d=4;dp=4;f=2;},font=\scriptsize\sffamily\bfseries]
			\path[draw=\mycolor,line width =1.5pt] (0,-h) coordinate (M)--(0,h) coordinate (N);
			\path[draw=\mycolor](-d,0) coordinate (Xp)--(d,0) coordinate (X);
			\path (0,0) coordinate (O) 
			(-f,0) coordinate (F) 
			(f,0) coordinate (F')
			({-0.75*f},0) coordinate (A)
			({-0.75*f},{0.4*h}) coordinate (B)
			({-1.5*f},0) coordinate (A')
			({-1.5*f},{0.8*h}) coordinate (B');
			\draw
			(A)edge[-{Stealth[scale=2,length=2.5,width=1.5]}]node[pos=0.5,above,sloped,font=\scriptsize]{2cm}(B) (A') edge[-{Stealth[scale=2,length=2.5,width=1.5]}]node[pos=0.5,above,sloped,font=\scriptsize]{4cm} (B');
			
			\path (O)--(A)--([turn]90:0.35) coordinate (xt)
			($(O)-(A)+(xt)$) coordinate (yt);
			\draw[dash pattern=on 2pt off 2pt] (O)--(yt) (A)--(xt);
			\draw[>=stealth,|<->|] (xt)--(yt) node[fill=white,inner sep=0pt,font=\scriptsize,midway,sloped]{$30\ \mathrm{cm}$};
			
			\foreach \t/\g in {O/-135,A'/-90,A/45,B/90,B'/90}{
				\path[fill=red,draw=black] (\t)  node[shift={(\g:5pt)}]{\t};}
		\end{tikzpicture}
	\end{center}
	\caption{\label{fig:TKHT03}}
\end{figure}
	\choiceTF
	{ Ảnh $A'B'$ hứng được trên màn chắn}
	{\True Thấu kính này có phần rìa mỏng hơn phần giữa}
	{Để cho ảnh ngược chiều so với vật ta dịch chuyển vật lại gần thấu kính.}
	{\True Ảnh $A'B'$ cách thấu kính là $60$ cm}
	\loigiai{}
\end{ex}

\Closesolutionfile{ansex}
\Closesolutionfile{ansbook}
\Closesolutionfile{ans}	

%%%==========Phần tự luận============%%%
\tieumuc{Tự Luận (5 điểm)}
\Opensolutionfile{ansbt}[LOIGIAITL/LGTL-VL9CKII-De-03]
%\luuloigiaibt
%\hienthiloigiaibt
%\taoNdongke[15]{bt}
%%%=============BT_1=============%%%
\begin{bt}[$1{,}0$ điểm]Có những cách nào để làm giảm hao phí điện năng do tỏa nhiệt khi truyền tải điện năng đi xa? Cách nào làm có lợi nhất? Vì sao?
	\loigiai{}
\end{bt}
%%%=============BT_2=============%%%
\begin{bt}[$1{,}0$ điểm]Cuộn dây sơ cấp của một máy biến thế có $4400$ vòng.Khi đặt vào hai đầu cuộn sơ cấp một hiệu điện thế xoay chiều $220$ V thì giữa hai đầu cuộn thứ cấp có hiệu điện thế xoay chiều là $12$ V.
	\begin{enumerate}[a)]
		\item Máy này là máy tăng thế hay hạ thế?
		\item Tính số vòng của cuộn dây thứ cấp tương ứng?
	\end{enumerate}
	\loigiai{}
\end{bt}
%%%=============BT_3=============%%%
\begin{bt}[$3{,}0$ điểm]Đặt một vật trước một thấu kính hội tụ có tiêu cự $f= 8\mathrm{~cm}$.Vật $AB$ cách thấu kính một khoảng $24\mathrm{~cm}$.$A$ nằm trên trục chính.
	\begin{enumerate}[a)]
		\item Giả sử $AB=30\mathrm{~cm}$. Tính khoảng cách $d'$ từ ảnh đến thấu kính và chiều cao $A'B'$ của ảnh?
		\item Trong trường hợp thấu kính là phân kì, tính khoảng cách từ ảnh đến thấu kính.
	\end{enumerate}
	\loigiai{}
\end{bt}
\Closesolutionfile{ansbt}
\fileend
\begin{center}
	\rule[4pt]{2cm}{1pt}\large \textbf{HẾT}\rule[4pt]{2cm}{1pt}
\end{center}











