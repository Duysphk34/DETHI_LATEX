
\section{Đề ôn thi giữa kỳ 2 toán 10}
\subsection{Phần trắc nghiệm}
Câu trắc nghiệm nhiều phương án lựa chọn. Học sinh trả lời từ
câu 1 đến câu 12. Mỗi câu hỏi học sinh \textit{chỉ chọn một} phương án.
\Opensolutionfile{ans}[Ans/Dapan]

\hienthiloigiaiex
%%%=============EX_1=============%%%
\begin{ex}%[0D8N1-2]
	Một công việc được hoàn thành bởi hai hành động liên tiếp. Nếu hành động thứ nhất có $a$ cách thực hiện và ứng với mỗi cách thực hiện hành động thứ nhất có $b$ cách thực hiện hành động thứ hai thì số cách để hoàn thành công việc đó là:
	\choice{\True  $a b$}
	{$a+b$}
	{$a b+1$}
	{$a+b+1$}
	\loigiai{Theo Quy tắc nhân, ta có số cách hoàn thành công việc là $ab$ cách.
	}
\end{ex}
%%%=============EX_2=============%%%
\begin{ex}%[0D8N1-1]
	Bạn An đến thư viện trường để mượn một quyển sách Toán học hoặc Vật lí để đọc. Tại đó có 100 quyển sách Toán học và 120 quyển sách Vật lí. Bạn An có số cách chọn sách là:
	\choice{$100$}
	{$120$}
	{$12000$}
	{\True $220$}
	\loigiai{Theo quy tắc cộng, bạn An có $220$ cách chọn một quyển sách.
	}
\end{ex}
%%%=============EX_3=============%%%
\begin{ex}%[0D8N1-4]
	Có bao nhiêu số nguyên dương nhỏ hơn $40$ và nguyên tố cùng nhau với $33$ (hai số gọi là nguyên tố cùng nhau nếu chúng có ước chung lớn nhất là $1$)?
	\choice{$25$ số}
	{$26$ số}
	{\True$24$ số}
	{ $36$ số}
	\loigiai{
		\begin{enumerate}[$\bullet$]
			\item Do $33$ chỉ có ba ước dương khác $1$ là $3$, $11$ và $33$ nên tập hợp các số có ước dương khác $1$ là $3$ hoặc $11$ là $\{3; 6; 9; 11; 12; 15; 18; 21; 22; 24; 27; 30; 33; 36; 39\}$.
			\item Tập hợp này có $15$ phần tử.
		\end{enumerate}
		Vậy số các số nhỏ hơn $40$ và nguyên tố cùng nhau với $33$ là $39-15=24$ số.
	}
\end{ex}
%%%=============EX_4=============%%%
\begin{ex}%[0D8N1-3]
	Tổ $1$ có có $3$ nam và $7$ nữ. Hỏi có bao nhiêu cách chọn $2$ học sinh mà có cả nam và nữ?
	\choice{$21$}
	{$10$}
	{\True $\mathrm{A}_{10}^2$}
	{$\mathrm{C}_{10}^2$}
	\loigiai{
		\begin{enumerate}[$\bullet$]
			\item Chọn $1$ học sinh nam, ta có $\mathrm{C}_3^1=3$ cách chọn.
			\item Chọn một học sinh nữ, ta có $\mathrm{C}_7^1=7$.
			\item Theo quy tắc nhân, ta có $21$ cách chọn $1$ nam và $1$ nữ.
		\end{enumerate}
	}
\end{ex}

%%%=============EX_5=============%%%
\begin{ex}%[0D8N1-1]
	Giả sử có thể di chuyển từ tỉnh $A$ đến tỉnh $B$ bằng các phương tiện: ô tô, tàu hoả và máy bay. Mỗi ngày có $6$ chuyến ô tô, $3$ chuyến tàu hoả và $2$ chuyến bay. Số cách di chuyển từ $A$ đến $B$ là
	\choice{\True $11$}
	{$36$}
	{ $18$}
	{$6$}
	\loigiai{Đi từ tỉnh $A$ đến tỉnh $B$, có
		\begin{enumerate}[$\bullet$]
			\item $6$ cách đi bằng ô tô
			\item $3$ cách đi bằng tàu hỏa
			\item $2$ cách đi bằng máy bay
		\end{enumerate}
		Theo Quy tắc cộng, ta có $6+3+2=11$ cách.}
\end{ex}
%%%=============EX_6=============%%%
\begin{ex}%[0D8N3-2]
	Khai triển của $(4x-y)^5$ là
	\choice
	{$1024x^5-1280x^4y-640x^3y^2-160x^2y^3-20xy^4-y^5$}
	{\True $1024x^5-1280x^4y+640x^3y^2-160x^2y^3+20xy^4-y^5$}
	{$1024x^5+1280x^4y-640x^3y^2+160x^2y^3+20xy^4+y^5$}
	{$1024x^5+1280x^4y+640x^3y^2-160x^2y^3-20xy^4-y^5$}
	\loigiai{\\Ta có,
		\begin{align*}
			(4x-y)^5&=\mathrm{C}_5^0(4x)^5+\mathrm{C}_5^1(4x)^4(-y)+\mathrm{C}_5^2(4x)^3(-y)^2+\mathrm{C}_5^3(4x)^2(-y)^3+\mathrm{C}_5^4(4x)(-y)^4+\mathrm{C}_5^5(-y)^5\\
			&=1024x^5-1280x^4y+640x^3y^2-160x^2y^3+20xy^4-y^5.	
		\end{align*} 
	}
\end{ex}
%%%=============EX_7=============%%%
\begin{ex}%[0H9N1-2]
	Trong mặt phẳng tọa độ $Oxy$, cho vectơ $\vec{u}=(-2 ; 3)$. Đẳng thức nào sau đây là đúng?
	\choice
	{$\vec{u}=2 \vec{i}+3 \vec{j}$}
	{$\vec{u}=3 \vec{i}+2 \vec{j}$}
	{\True $\vec{u}=-2 \vec{i}+3 \vec{j}$}
	{$\vec{u}=-2 \vec{j}+3 \vec{i}$}
	\loigiai{}
\end{ex}

%%%=============EX_8=============%%%
\begin{ex}%[0H9N1-3]
	Trong mặt phẳng tọa độ $Oxy$, cho tam giác $ABC$ có $A(0;2), \, B(-1;1), \, C(a; b)$ và điểm $G(1;3)$ là trọng tâm của tam giác $ABC$. Khi đó $a+b$ là
	\choice
	{$2$}
	{$-2$}
	{\True $10$}
	{$-10$}
	\loigiai{Ta có $\heva{\dfrac{x_A+x_B+x_C}{3}=x_G\\ \dfrac{y_A+y_B+y_C}{3}=y_G}\Leftrightarrow \heva{\dfrac{0-1+a}{3}=1\\ \dfrac{2+1+b}{3}=3} \Leftrightarrow \heva{a=4\\ b=6} \Rightarrow a+b=10.$
	}
\end{ex}
%%%=============EX_9=============%%%
\begin{ex}%[0H9N1-1]
	Trong mặt phẳng tọa độ $Oxy$, cho vectơ $\vec{a}$ và $\vec{b}$ được thể hiện như hình bên. Nếu $\vec{c}=\vec{a}+\vec{b}$ thì độ dài của vectơ $\vec{c}$ là
	\begin{center}
		\begin{tikzpicture}[scale=0.8]
			\draw[->] (-3.5,0) -- (3.5,0);
			\foreach \x in {-3.,-2.,-1.,1.,2.,3.}
			\draw[shift={(\x,0)},color=black] (0pt,1pt) -- (0pt,-1pt) node[above] {\footnotesize $\x$};
			\draw(3.6,0) node [right]{$x$};
			\draw[->,color=black] (0.,-1.5) -- (0.,4.5);
			\foreach \y in {-1.,1.,2.,3.,4.}
			\draw[shift={(0,\y)},color=black] (2pt,0pt) -- (-2pt,0pt) node[right] {\footnotesize $\y$};
			\draw(0,4.6) node [above] {$y$};
			\draw[color=black] (0pt,-10pt) node[right] {\footnotesize $O$};
			\draw [->] (0.,0.) -- (-2.,4.);
			\draw [->] (0.,0.) -- (-2.,-1.);
			\draw[dashed] (0,-1) -- (-2,-1)--(-2,4)--(0,4);
			\begin{scriptsize}
				\draw[color=black] (-0.8,2.2) node {$\vec{a}$};
				\draw[color=black] (-0.77,-0.63) node {$\vec{b}$};
			\end{scriptsize}
		\end{tikzpicture}
	\end{center}
	\choice
	{$2$}
	{$3$}
	{$4$}
	{\True$5$}
	\loigiai{Ta có $\heva{&\vec{a}=(-2;4)\\ &\vec{b}=(-2;-1)} \Rightarrow \vec{c}=\vec{a}+\vec{b}=(-4;3)\Rightarrow \left| \vec{c} \right|=5$.}
\end{ex}

%%%=============EX_10=============%%%
\begin{ex}%[0H9N1-1]
	Trong mặt phẳng tọa độ $Oxy$, cho $\vec{a}=(2 ;-3), \vec{b}=(-1 ; 2)$. Toạ độ của vectơ $\vec{u}=2 \vec{a}-\vec{3 b}$ là
	\choice
	{\True  $(7 ;-12)$}
	{$(7 ; 12)$}
	{$(1 ;-12)$}
	{$(1 ; 0)$}
	\loigiai{Ta có, $\heva{&2\vec{a}=(4;-6)\\& 3\vec{b}=(-3;6)} \Rightarrow \vec{u}=2 \vec{a}-3\vec{b}=(7;-12)$.}
\end{ex}
%%%=============EX_11=============%%%
\begin{ex}%[0D2K3]
	Phương trình đường thẳng đi qua hai điểm $M(-1 ; 0), N(3 ; 1)$ là:
	\choice
	{\True $x-4 y+1=0$}
	{$x-4 y-1=0$}
	{$4 x+y+4=0$}
	{$4 x+y-4=0$}
	\loigiai{Đường thẳng $MN$ đi qua $M(-1;0)$, nhận $\vec{MN}=(4;1)$ làm vtcp, suy ra $MN$ nhận $\vec{n}=(1;-4)$ làm vtpt. Do đó $MN: (x+1)-4(y-0)=0 \\ \Leftrightarrow MN: x-4y+1=0.$}
\end{ex}
%%%=============EX_12=============%%%
\begin{ex}%[0H9N2-2]
	Phương trình tham số của đường thẳng đi qua $A(-2 ; 1)$, nhận $\vec{u}=(3 ;-1)$ làm vectơ chỉ phương là
	\choice
	{\True $\heva{&x=-2+3t\\&y=1-t}$}
	{$\heva{&x=3-2t\\&y=-1+t}$}
	{$3x-y+7=0$}
	{$-2x+y+7=0$}
	\loigiai{Đường thẳng đi qua $A(-2 ; 1)$, nhận $\vec{u}=(3 ;-1)$ làm vectơ chỉ phương có phương trình tham số là $\heva{&x=-2+3t\\&y=1-t} (t \in \mathbf{R})$.}
\end{ex}

\Closesolutionfile{ans}
\bangdapan{Dapan}

\setcounter{subsection}{1}
\subsection{Câu trắc nghiệm đúng sai}
Học sinh trả lời từ câu 1 đến câu 4.
Trong mỗi ý \circlenum{A}, \circlenum{B}, \circlenum{C} và \circlenum{D} ở mỗi câu, học sinh chọn đúng hoặc sai.
\setcounter{ex}{0}
\LGexTF
\Opensolutionfile{ansbook}[ansbook/DapanDS]
\Opensolutionfile{ans}[Ans/DapanT]

\begin{ex}%[0D8H2-2]%[Dự án đề kiểm tra Toán 10 GHKII NH23-24- Bùi Thanh Cương]
	Đội văn nghệ của nhà trường gồm $4$ học sinh lớp $12$A, $3$ học sinh lớp $12$B và $2$ học sinh lớp $12$C. Chọn ngẫu nhiên $5$ học sinh từ đội văn nghệ để biểu diễn trong lễ bế giảng, khi đó:
	\choiceTF
	{ Chọn $5$ học sinh tùy ý từ $9$ học sinh có: $120$ cách}
	{\True Chọn $5$ học sinh chỉ có lớp $12$A và $12$B có: $21$ cách}
	{ Chọn $5$ học sinh chỉ có lớp $12$B và $12$C có: $2$ cách}
	{Có $90$ cách chọn sao cho lớp nào cũng có học sinh được chọn} 
	\loigiai{
		\begin{itemize}
			\item Chọn $5$ học sinh tùy ý từ $9$ học sinh có: $\mathrm{C}_9^5=126$ cách.
			\item Chọn $5$ học sinh chỉ có lớp $12$A và $12$B có: $\mathrm{C}_7^5=21$ cách.
			\item Chọn $5$ học sinh chỉ có lớp $12$B và $12$C có: $\mathrm{C}_5^5=1$ cách.
			\item Số cách chọn $5$ học sinh sao cho lớp nào cũng có học sinh là: $\mathrm{C}_9^5-\left( \mathrm{C}_7^5+\mathrm{C}_6^5-\mathrm{C}_5^5\right)=98$.
			
		\end{itemize}
	}
\end{ex}

\begin{ex}%[0D8H3-2]%[Dự án đề kiểm tra Toán 10 GHKII NH23-24- Bùi Thanh Cương]
	Khai triển $(x+\sqrt{2})^4$. Khi đó:
	\choiceTF
	{\True Hệ số của $x^2$ là $12$}
	{ Hệ số của $x^3$ là $6 \sqrt{2}$}
	{\True Hệ số của $x$ là $8 \sqrt{2}$}
	{\True Hệ số không chứa $x$ trong khai triển trên bằng $4$} 
	\loigiai{
		Khai triển nhị thức 
		\allowdisplaybreaks
		\begin{eqnarray*}
			\left(x+\sqrt{2}\right)^4
			&=& \mathrm{C}_4^0 x^4+\mathrm{C}_4^1 x^3(\sqrt{2})+\mathrm{C}_4^2 x^2(\sqrt{2})^2+\mathrm{C}_4^3 x(\sqrt{2})^3+\mathrm{C}_4^4(\sqrt{2})^4 \\
			&=& x^4+4 \sqrt{2} x^3+12 x^2+8 \sqrt{2} x+4.
		\end{eqnarray*}
		\begin{itemize}
			\item  Hệ số của $x^2$ là $12$.
			\item Hệ số của $x^3$ là $4\sqrt{2}$.
			\item Hệ số của $x$ là $8 \sqrt{2}$.
			\item Hệ số không chứa $x$ trong khai triển trên bằng $4$.
		\end{itemize}
	}
\end{ex}

\begin{ex}%[0H9H1-3]%[Dự án đề kiểm tra Toán 10 GHKII NH23-24- Bùi Thanh Cương]
	Trong mặt phẳng tọa độ $Ox y$, cho các điểm $A(0 ; 2)$; $B(1 ; 1)$; $C(-1 ;-2)$. Các điểm $A'$, $B'$, $C'$ lần lượt chia các đoạn $BC$, $CA$, $AB$ theo các tỉ số $-1$; $\dfrac{1}{2}$; $-2$. Khi đó:
	\choiceTF
	{\True $A'=\left(0 ;-\dfrac{1}{2}\right)$}
	{ $B'=(2 ; 6)$}
	{$C'=\left(\dfrac{1}{3}; \dfrac{4}{3}\right)$}
	{\True Ba điểm $A'$, $B'$, $C'$ thẳng hàng} 
	\loigiai{
		\begin{itemize}
			\item $A'$  chia  đoạn $BC$ theo các tỉ số $-1$ nên $\overrightarrow{A'B}=-\overrightarrow{A'C}\Rightarrow A'$ là trung điểm đoạn $BC$.\\
			Gọi toạ độ điểm $A'\left(x_{A'};y_{A'}\right)$, ta có
			\[\heva{& x_{A'}=\dfrac{x_B+x_C}{2} \\ & y_{A'}=\dfrac{y_B+y_C}{2}}\Rightarrow \heva{& x_{A'}=\dfrac{1-1}{2}=0 \\ & y_{A'}=\dfrac{1-2}{2}=-\dfrac{1}{2}}\Rightarrow A'\left(0 ;-\dfrac{1}{2}\right).\]
			\item  $B'$ chia đoạn $CA$ theo các tỉ số  $\dfrac{1}{2}$ nên $\overrightarrow{B'C}=\dfrac{1}{2}\overrightarrow{B'A}\Leftrightarrow 2\overrightarrow{B'C}=\overrightarrow{B'A}$.\\
			Gọi toạ độ điểm $B'\left(x_{B'};y_{B'}\right)$, ta có $\overrightarrow{B'C}=\left(-1-x_{B'};-2-y_{B'}\right)$, $\overrightarrow{B'A}=\left(-x_{B'}; 2-y_{B'}\right)$.
			\[2\overrightarrow{B'C}=\overrightarrow{B'A}\Leftrightarrow  \heva{& 2\left(-1-x_{B'}\right)=-x_{B'}\\ & 2\left( -2-y_{B'}\right)=2-y_{B'}}\Leftrightarrow \heva{& x_{B'}=-2 \\ & y_{B'}=-6.}\]
			Suy ra $B'\left( -2 ;-6\right)$.
			\item $C'$  chia đoạn  $AB$ theo các tỉ số $-2$ nên $\overrightarrow{C'A}=-2\overrightarrow{C'B}$.\\
			Gọi toạ độ điểm $C'\left(x_{C'};y_{C'}\right)$, ta có $\overrightarrow{C'A}=\left(-x_{C'};2-y_{C'}\right)$, 
			$\overrightarrow{C'B}=\left(1-x_{C'}; 1-y_{C'}\right)$.\\
			\[ \overrightarrow{C'A}=-2\overrightarrow{C'B}
			\Leftrightarrow  \heva{& -x_{C'}=-2\left(1-x_{C'}\right) \\& 2-y_{C'}=-2\left(1-y_{C'} \right)& }\Leftrightarrow \heva{& x_{C'}= \dfrac{2}{3}\\ & y_{C'}=\dfrac{4}{3}.} \]
			Suy ra $C'\left( \dfrac{2}{3} ; \dfrac{4}{3}\right)$.
			\item Ta có $\overrightarrow{A'B'}=\left(-2 ;-\dfrac{11}{2}\right)$; $\overrightarrow{A'C'}=\left(\dfrac{2}{3} ; \dfrac{11}{6}\right)$.\\
			Ta thấy $\dfrac{-2}{\dfrac{2}{3}}=\dfrac{-\dfrac{11}{2}}{\dfrac{11}{6}}=-3$, suy ra $\overrightarrow{A'B'}=-3 \overrightarrow{A'C'}$ nên $A', B', C'$ thẳng hàng.
		\end{itemize}
	}
\end{ex}

\begin{ex}%[0H9H3-2]%[Dự án đề kiểm tra Toán 10 GHKII NH23-24- Bùi Thanh Cương]
	Trong mặt phẳng toạ độ $Oxy$, cho hai điểm $M(1 ; 2)$, $N(3 ;-1)$ và hai vec-tơ $\overrightarrow{n}=(2 ;-1)$, $\overrightarrow{u}=(1 ; 1)$. Khi đó:
	\choiceTF
	{\True Phương trình tổng quát của đường thẳng $d_1$ đi qua $M$ và có vectơ pháp tuyến $\overrightarrow{n}$ là $2x-y=0$}
	{\True Phương trình tham số của đường thẳng $d_2$ đi qua $N$ và có vectơ chỉ phương $\overrightarrow{u}$ là $\heva{& x=3+t \\ &y=-1+t}$}
	{Phương trình tham số của đường thẳng $d_3$ đi qua $N$ và có vectơ pháp tuyến $\overrightarrow{n}$ là $2 x-y+7=0$}
	{\True Phương trình tham số của đường thẳng $d_4$ đi qua $M$ và có vectơ chỉ phương $\overrightarrow{u}$ là $\heva{& x=1+t  \\ & y=2+t}$} 
	\loigiai{
		\begin{itemize}
			\item Đường thẳng $d_1$ đi qua $M(1 ; 2)$ và có vectơ pháp tuyến $\overrightarrow{n}=(2 ;-1)$ có phương trình tổng quát
			\[ 2(x-1)-(y-2)=0 \Leftrightarrow 2 x-y=0.\]
			\item Đường thẳng $d_2$ đi qua $N(3 ;-1)$  và có vectơ chỉ phương $\overrightarrow{u}=(1 ; 1)$ có phương trình tham số
			\[ \heva{& x=3+t \\ &y=-1+t.}\] 
			\item Đường thẳng $d_3$ đi qua $N(3 ;-1)$ và có vectơ pháp tuyến $\overrightarrow{n}=(2 ;-1)$ có phương trình tổng quát
			\[ 2(x-3)-(y+1)=0 \Leftrightarrow 2 x-y-7=0.\]
			\item Đường thẳng $d_4$ đi qua $M(1 ; 2)$ và có vectơ chỉ phương $\overrightarrow{u}=(1 ; 1)$ có phương trình tham số
			\[ \heva{& x=1+t  \\ & y=2+t.}\] 
		\end{itemize}
	}
\end{ex}

\Closesolutionfile{ans}
\subsection{Phần tự luận}

\hienthiloigiaibt

\begin{bt}%[0D8H1-2]%[Dự án đề kiểm tra Toán 10 Giữa HKII NH23-24- Phạm Hoài]%[Đề số 4]
	Có bao nhiêu số tự nhiên có $5$ chữ số khác nhau và chia hết cho $10$?
	\loigiai{
		Gọi số tự nhiên cần tìm là $\overline{abcde}$ ($a\neq b\neq c\neq d\neq e$).\\
		Vì số tự nhiên này chia hết cho $10$ nên $e=0$: có $1$ cách chọn $e$.\\
		$a\neq 0$ nên có $9$ cách chọn.\\
		$b$ có $8$ cách chọn.\\
		$c$ có $7$ cách chọn.\\
		$d$  có $6$ cách chọn.\\Theo quy tắc nhân ta được $1\cdot 9\cdot 8\cdot 7\cdot 6=3024$ cách.
		
	}
\end{bt}
\begin{bt}%[0D8H2-2]%[Dự án đề kiểm tra Toán 10 Giữa HKII NH23-24- Phạm Hoài]%[Đề số 4]
	Tìm tập nghiệm của phương trình $\mathrm{P}_x\cdot \mathrm{A}_x^2+72=6\left(\mathrm{A}_x^2+2\mathrm{P}_x\right)$
	\loigiai{
		Điều kiện $x\in \mathbb{N}, x\geq 2$.
		\begin{eqnarray*}
			P_x\cdot \mathrm{A}_x^2+72=6\left(\mathrm{A}_x^2+2\mathrm{P}_x\right)&\Leftrightarrow & \mathrm{A}_x^2\left(\mathrm{P}_x-6\right)-12\left(\mathrm{P}_x-6\right)=0\\
			&\Leftrightarrow & \left(\mathrm{P}_x-6\right)\left(\mathrm{A}_x^2-12\right)=0\\
			&\Leftrightarrow & \hoac{&\mathrm{P}_x=6\\&\mathrm{A}_x^2=12}\\
			&\Leftrightarrow & \hoac{&x!=6\\&x(x-1)=12}\\
			&\Leftrightarrow & \hoac{&x=3\\&x=4\\&x=-3.}
		\end{eqnarray*}
		Vì $x\in \mathbb{N}$ và $x\geq 2$ nên tập nghiệm của phương trình là $S=\{3;4\}$.
	}
\end{bt}
\begin{bt}%[0H9H1-3]%[Dự án đề kiểm tra Toán 10 Giữa HKII NH23-24- Phạm Hoài]%[Đề số 4]
	Trong mặt phẳng tọa độ $O x y$, cho hai điểm $A(3 ;-5)$, $B(1 ; 0)$. Tìm tọa độ điểm $C$ sao cho $\overrightarrow{O C}=-3 \overrightarrow{A B}$.
	\loigiai{
		Gọi $C\left(x_C; y_C\right)$.\\
		Ta có $\overrightarrow{OC}=\left(x_C;y_C\right)$, $\overrightarrow{ AB}=\left(-2;5\right)\Rightarrow -3\overrightarrow{AB}=\left(6;-15\right)$.\\
		Theo đề $\overrightarrow{OC}=-3\overrightarrow{AB}\Leftrightarrow \heva{&x_C=6\\&y_C=-15.}$\\
		Vậy $C\left(6;-15\right)$.}
\end{bt}
\begin{bt}%[0H9V3-2]%[Dự án đề kiểm tra Toán 10 Giữa HKII NH23-24- Phạm Hoài]%[Đề số 4]
	Viết phương trình đường thẳng $d$ song song với $\Delta\colon  x+4 y-2=0$ và cách điểm $A(-2 ; 3)$ một khoảng bằng $3$.
	\loigiai{
		Vì  $d\parallel  \Delta\colon  x+4 y-2=0 $ nên phương trình $d$ có dạng $x+4 y+c=0$.\\
		Mặt khác \begin{eqnarray*}
			\mathrm{d}(A, d)=3 &\Rightarrow &\dfrac{|-2+4\cdot 3+c|}{\sqrt{1+16}}=3 \\
			&\Rightarrow &|10+c|=3 \sqrt{17}\\
			&\Rightarrow &\hoac{& c = 3 \sqrt{1 7} - 1 0 \\&c = - 3 \sqrt{1 7} - 1 0 }\\
			&\Rightarrow &\hoac{&d_1\colon  x+4 y+3 \sqrt{17}-10=0 \\&d_2\colon  x+4 y-3 \sqrt{17}-10=0.}
		\end{eqnarray*}	
		Vậy có hai đường thẳng thỏa mãn là $x+4 y+3 \sqrt{17}-10=0$; $x+4 y-3 \sqrt{17}-10=0$.
	}
\end{bt}
\begin{bt}%[0H9C3-2]%[Dự án đề kiểm tra Toán 10 Giữa HKII NH23-24- Phạm Hoài]%[Đề số 4]
	Trong mặt phẳng với hệ tọa độ $O x y$ cho tam giác $A B C$ với $A(1 ; 2)$, $C(3 ; 1)$. Đường phân giác trong góc $A$ của tam giác $A B C$ có phương trình $2 x+y-4=0$. Viết phương trình tổng quát của đường thẳng chứa cạnh $AB$.
	\loigiai{
		\begin{itemize}
			\item Gọi $d$ là đường thẳng chứa cạnh $A B$ của tam giác $A B C$ và $\triangle$ là đường phân giác trong góc $A$.
			\item Nếu $C'$ là điểm đối xứng của $C$ qua đường thẳng $\Delta$ thì $C' \in d$. Ta tìm tọa độ điểm $C'$.
			\item Gọi $\Delta^{\prime}$ là đường thẳng đi qua $C$ và vuông góc với đường thẳng $\Delta$
			$\Rightarrow \Delta'$ có véc-tơ pháp tuyến $\overrightarrow{n}_d=(1 ;-2) \Rightarrow \Delta'$ có phương trình
			$(x-3)-2(y-1)=0$ hay $x-2 y-1=0$
			\item Gọi $I= \Delta' \cap \Delta \Rightarrow$ tọa độ của $I$ là nghiêm của hệ phương trình
			\begin{eqnarray*}
				\heva{&2 x+y=4 \\& x-2 y=1} \Leftrightarrow\heva{&x=\dfrac{9}{5} \\& y=\dfrac{2}{5}} \Rightarrow I\left(\dfrac{9}{5} ; \dfrac{2}{5}\right)
			\end{eqnarray*}
			Hơn nữa $I$ là trung điểm của $C C'\Rightarrow C'\left(2 \cdot \dfrac{9}{5}-3 ; 2 \cdot \dfrac{2}{5}-1\right)$.\\
			Hay $C'\left(\dfrac{3}{5} ;-\dfrac{1}{5}\right) \Rightarrow \overrightarrow{A C'}=\left(-\dfrac{2}{5} ;-\dfrac{11}{5}\right)=-\dfrac{1}{5}(2 ; 11)$.\\
			
			$\Rightarrow$ đường thẳng $d$ có véc-tơ pháp tuyến $\overrightarrow{ n}_d=(11 ;-2)$ nên $d$ có phương trình\\
			
			$11(x-1)-2(y-2)=0$  hay $11 x-2 y-7=0$.\\			
			Vậy đường thẳng $d$ có phương trình tổng quát là $11 x-2 y-7=0$.
		\end{itemize}
	}
\end{bt}
\begin{bt}%[0D8V3-5] %[Dự án đề kiểm tra Toán 10 Giữa HKII NH23-24- Phạm Hoài]%[Đề số 4]
	Tính giá trị của tổng $$T=\mathrm{C}_{2023}^1+\mathrm{C}_{2023}^2+\ldots+\mathrm{C}_{2023}^{2023}$$
	\loigiai{
		Ta có $\left(x+1\right)^{2023}=\mathrm{C}_{2023}^0 \cdot x^{2023}+\mathrm{C}_{2023}^1 \cdot x^{2022}+\mathrm{C}_{2023}^2 \cdot x^{2021}+\ldots+\mathrm{C}_{2023}^{2023} \cdot x^0$.
		\\
		Cho $x=1$, ta được $$(1+1)^{2023}=\mathrm{C}_{2023}^0+\mathrm{C}_{2023}^1+\mathrm{C}_{2023}^2+\ldots +\mathrm{C}_{2023}^{2023}.$$
		$$
		\Rightarrow T=\mathrm{C}_{2023}^1+\mathrm{C}_{203}^2+\ldots +\mathrm{C}_{2023}^{2023}=2^{2023}-\mathrm{C}_{2023}^0=2^{2023}-1.
		$$
	}
\end{bt}