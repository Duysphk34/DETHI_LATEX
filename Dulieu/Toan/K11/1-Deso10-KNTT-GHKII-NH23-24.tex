\section{Đề ôn thi giữa kỳ 2 toán 11}
\subsection{Phần trắc nghiệm}
Câu trắc nghiệm nhiều phương án lựa chọn. Học sinh trả lời từ
câu 1 đến câu 12. Mỗi câu hỏi học sinh \textit{chỉ chọn một} phương án.

\Opensolutionfile{ans}[Ans/Dapan]

\hienthiloigiaiex
%%%==============EX_1============%%%
\begin{ex}%[1D6H1-2]%[Dự án đề kiểm tra Toán khối 11 GHKII NH23-24-Dot-2- Nguyễn Đức Lợi]%[Deso 10 - Sach KNTT]
	Cho biểu thức $P=x^{-\tfrac{3}{4}}\cdot \sqrt{\sqrt{x^5}}$, $x > 0$. Khẳng định nào sau đây là đúng?
	\choice
	{$P=x^{-2}$}
	{$P=x^{-\dfrac{1}{2}}$}
	{\True $P=x^{\dfrac{1}{2}}$}
	{$P=x^2$}
	\loigiai{
		Ta có $P=x^{-\tfrac{3}{4}}\cdot\sqrt{\sqrt{x^5}}$ $=x^{-\tfrac{3}{4}}\cdot x^{\tfrac{5}{4}}=x^{-\tfrac{3}{4}+\tfrac{5}{4}}=x^{\tfrac{1}{2}}$.
	}
\end{ex}
%%%==============EX_2============%%%
\begin{ex}%[1D6H1-4]%[Dự án đề kiểm tra Toán khối 11 GHKII NH23-24-Dot-2- Nguyễn Đức Lợi]%[Deso 10 - Sach KNTT]
	Tìm tập tất cả các giá trị của $a$ để $\sqrt[{21}]{a^5} > \sqrt[{7}]{a^2}$.
	\choice
	{$a > 0$}
	{\True $0< a < 1$}
	{$a > 1$}
	{$\dfrac{5}{21} < a < \dfrac{2}{7}$}
	\loigiai{Ta có $\sqrt[{7}]{a^2}=\sqrt[{21}]{a^6}.$\\
		Khi đó $\sqrt[{21}]{a^5} > \sqrt[{7}]{a^2} \Leftrightarrow \sqrt[{21}]{a^5} > \sqrt[{21}]{a^6}$ mà $5< 6$ nên $0< a < 1$.
	}
\end{ex}
%%%==============EX_3============%%%
\begin{ex}%[1D6H2-3]%[Dự án đề kiểm tra Toán khối 11 GHKII NH23-24-Dot-2- Nguyễn Đức Lợi]%[Deso 10 - Sach KNTT]
	Cho các số thực dương $a,b$ với $a\ne 1$. Khẳng định nào sau đây là khẳng định đúng?
	\choice
	{$\log _{a^2} (ab)=\dfrac{1}{4} \log_a b$}
	{\True $\log _{a^2} (ab)=\dfrac{1}{2}+\dfrac{1}{2} \log_a b$}
	{$\log _{a^2} (ab)=\dfrac{1}{2} \log_a b$}
	{$\log _{a^2} (ab)=2+2\log_a b$}
	\loigiai{
		Ta có $\log _{a^2} (ab)=\log _{a^2} a+\log _{a^2} b=\dfrac{1}{2}\log_a a+\dfrac{1}{2}\log_a b=\dfrac{1}{2}+\dfrac{1}{2}\log_a b$.
	}
\end{ex}
%%%==============EX_4============%%%
\begin{ex}%[1D6N2-3]%[Dự án đề kiểm tra Toán khối 11 GHKII NH23-24-Dot-2- Nguyễn Đức Lợi]%[Deso 10 - Sach KNTT]
	Với mọi số thực $a$ dương, $\log_2 \dfrac{a}{2}$ bằng
	\choice
	{$\dfrac{1}{2} \log_2 a$}
	{$\log_2 a+1$}
	{\True $\log_2 a-1$}
	{$\log_2 a-2$}
	\loigiai{
		Có $\log_2 \dfrac{a}{2}=\log_2 a-\log_2 2=\log_2 a-1$.
	}
\end{ex}
%%%==============EX_5============%%%
\begin{ex}%[1D6H3-2]%[Dự án đề kiểm tra Toán khối 11 GHKII NH23-24-Dot-2- Nguyễn Đức Lợi]%[Deso 10 - Sach KNTT]
	Có bao nhiêu số nguyên thuộc tập xác định của hàm số $y=\log \left[\left(6-x\right)\left(x+2\right)\right]$?
	\choice
	{\True $7$}
	{$8$}
	{Vô số}
	{$9$}
	\loigiai{
		ĐKXĐ $\left(6-x\right)\left(x+2\right) > 0\Leftrightarrow -2< x < 6$.\\		
		Mà $x\in \mathbb{Z}\Rightarrow x\in \left\{-1;0;1;2;3;4; 5\right\}$.\\		
		Vậy có $7$ số nguyên thuộc tập xác định của hàm số $y=\log \left[\left(6-x\right)\left(x+2\right)\right]$.
	}
\end{ex}
%%%==============EX_6============%%%
\begin{ex}%[1D6H3-3]%[Dự án đề kiểm tra Toán khối 11 GHKII NH23-24-Dot-2- Nguyễn Đức Lợi]%[Deso 10 - Sach KNTT]
	\immini{
		Hàm số nào sau đây có đồ thị như hình bên?
		\choice
		{$y=\log_3 x$}
		{$y=\log_2 x+1$}
		{\True $y=\log_2 \left(x+1\right)$}
		{$y=\log_3 \left(x+1\right)$}
	}{
		\begin{tikzpicture}[scale=1, font=\footnotesize, line join=round, line cap=round, >=stealth]
			\def\xmin{-2}\def\xmax{3}
			\def\ymin{-2}\def\ymax{3}
			\draw[->] (\xmin-0.2,0)--(\xmax+0.2,0) node[below] {\footnotesize $x$};
			\draw[->] (0,\ymin-0.2)--(0,\ymax+0.2) node[right] {\footnotesize $y$};
			\draw (0,0) circle (1pt) node [below right] {$O$};
			\clip (\xmin,\ymin-0.2) rectangle (\xmax,\ymax);
			\draw[smooth,samples=200,domain=-0.99:\xmax] plot (\x,{ln((\x+1))/ln(2)});	
			\draw (-1,\ymin-0.2)--(-1,\ymax+0.2) ;
			\draw[dashed] (1,0)--(1,1)--(0,1);
			\draw (1,0) circle (1pt) node [below] {$1$};
			\draw (2,0) circle (1pt) node [below] {$2$};
			\draw (0,1) circle (1pt) node [left] {$1$};
			\draw (0,2) circle (1pt) node [left] {$2$};
		\end{tikzpicture}
	}
	\loigiai{
		Đồ thị hàm số đi qua điểm $\left(0;0\right)$  và $\left(1;1\right)$ nên $y=\log_2 \left(x+1\right)$.
	}
\end{ex}
%%%==============EX_7============%%%
\begin{ex}%[1D6N4-2]%[Dự án đề kiểm tra Toán khối 11 GHKII NH23-24-Dot-2- Nguyễn Đức Lợi]%[Deso 10 - Sach KNTT]
	Nghiệm của phương trình $\log_3 (x-2)=2$ là
	\choice
	{\True $x=11$}
	{$x=10$}
	{$x=7$}
	{$8$}
	\loigiai{
		Điều kiện: $x > 2$.\\		
		Phương trình tương đương với $x-2=3^2 \Leftrightarrow x=11$.
	}
\end{ex}
%%%==============EX_8============%%%
\begin{ex}%[1D6H4-5]%[Dự án đề kiểm tra Toán khối 11 GHKII NH23-24-Dot-2- Nguyễn Đức Lợi]%[Deso 10 - Sach KNTT]
	Tập nghiệm của bất phương trình $2^{x^2-7} < 4$ là
	\choice
	{\True $(-3;3)$}
	{$(0;3)$}
	{$(-\infty;3)$}
	{$(3;+\infty)$}
	\loigiai{
		Ta có $2^{x^2-7} < 4$ $\Rightarrow x^2-7< 2$ $\Leftrightarrow x^2 < 9$ $\Rightarrow x\in \left(-3;3\right)$.
	}
\end{ex}
%%%==============EX_9============%%%
\begin{ex}%[1H8H1-2]%[Dự án đề kiểm tra Toán khối 11 GHKII NH23-24-Dot-2- Nguyễn Đức Lợi]%[Deso 10 - Sach KNTT]
	Cho hình chóp $S.ABCD$ có đáy là hình thoi tâm $O$ và $SA=SC$, $SB=SD$. Trong các mệnh đề sau mệnh đề nào \textbf{sai}?
	\choice
	{$AC\perp SD$}
	{$BD\perp AC$}
	{$BD\perp SA$}
	{\True $AC\perp SA$}
	\loigiai{
		\immini{
			Ta có tam giác $SAC$ cân tại $S$ và $SO$ là đường trung tuyến cũng đồng thời là đường cao.\\		
			Do đó $SO\perp AC$.\\		
			Trong tam giác vuông $SOA$ thì $AC$ và $SA$ không thể vuông tại $A$.
		}{
			\begin{tikzpicture}[scale=0.7,line cap=round,line join=round,font=\footnotesize,>=stealth]
				\path
				(0,0) coordinate (A)
				($(A)+(-135:2)$) coordinate (B)
				($(A)+(0:4)$) coordinate (D)
				($(D)+(B)-(A)$) coordinate (C)
				($(B)!0.5!(D)$) coordinate (O)
				($(O)+(90:3)$) coordinate (S)
				;
				\draw (C)--(S)--(B)--(C)--(D)--(S)
				;
				\draw[dashed] (S)--(A)--(B) (A)--(D)  (A)--(C) (B)--(D) (S)--(O);
				\foreach \p/\r in {A/40, B/-90, C/-30, S/90,D/60,O/-90}
				\fill (\p) circle (1.25pt) node[shift={(\r:3mm)},scale=1]{$\p$};
			\end{tikzpicture}
		}
	}
\end{ex}
%%%==============EX_10============%%%
\begin{ex}%[1H8H1-3]%[Dự án đề kiểm tra Toán khối 11 GHKII NH23-24-Dot-2- Nguyễn Đức Lợi]%[Deso 10 - Sach KNTT]
	Cho tứ diện $OABC$ có $OA=OB=OC=a;$ $OA,OB,OC$ vuông góc với nhau từng đôi một. Gọi $I$ là trung điểm $BC$. Tính góc giữa hai đường thẳng $AB$ và $OI$.
	\choice
	{$45^\circ$}
	{$30^\circ$}
	{$90^\circ$}
	{\True $60^\circ$}
	\loigiai{
		\immini{
			Vì tứ diện $OABC$ có $OA=OB=OC=a;$ $OA,OB,OC$ vuông góc với nhau từng đôi một nên ta có thể dựng hình lập phương $AMNP.OBDC$ như hình vẽ với $I$ là trung điểm $BC$ nên \\
			$\left\{I\right\}=OD\cap BC$.\\		
			Cạnh của hình lập phương trên bằng $a$ nên\\
			$AB=AN=NB=a\sqrt{2}$. Suy ra tam giác $ABN$ đều.\\		
			Dễ thấy $OI \parallel  AN$ nên góc giữa hai đường thẳng $AB$ và $OI$ bằng góc giữa $AB$ và $AN$ bằng $60^\circ$.
		}{
			\begin{tikzpicture}[scale=0.7,line cap=round,line join=round,font=\footnotesize,>=stealth]
				\path
				(0,0) coordinate (O)
				($(O)+(-135:2)$) coordinate (B)
				($(O)+(0:4)$) coordinate (C)
				($(C)+(B)-(O)$) coordinate (D)
				($(B)!0.5!(C)$) coordinate (I)
				($(O)+(90:3)$) coordinate (A)
				($(B)+(90:3)$) coordinate (M)
				($(D)+(90:3)$) coordinate (N)
				($(C)+(90:3)$) coordinate (P)
				;
				\draw (M)--(N)--(D)--(B)--(M)--(A)--(P)--(N)--(A) (P)--(C)--(D)
				;
				\draw[dashed] (O)--(A)--(B)--(O)--(C)--(B) (O)--(D)
				;
				\foreach \p/\r in {A/120, B/-120, C/30, D/-80, O/170, I/-90, N/0, P/70, M/180}
				\fill (\p) circle (1.25pt) node[shift={(\r:3mm)},scale=1]{$\p$};
			\end{tikzpicture}
		}
	}
\end{ex}
%%%==============EX_11============%%%
\begin{ex}%[1H8H2-2]%[Dự án đề kiểm tra Toán khối 11 GHKII NH23-24-Dot-2- Nguyễn Đức Lợi]%[Deso 10 - Sach KNTT]
	Cho tứ diện $ABCD$ có hai mặt $ABC$ và $ABD$ là hai tam giác đều. Gọi $M$ là trung điểm của $AB$. Khẳng định nào sau đây đúng?
	\choice
	{$CM\perp \left(ABD\right)$}
	{\True $AB\perp \left(MCD\right)$}
	{$AB\perp \left(BCD\right)$}
	{$DM\perp \left(ABC\right)$}
	\loigiai{
		\immini{Ta có 
			$ \heva{& CM\perp AB  \\ & DM\perp AB \\& CM \cap DM = M}\Rightarrow AB\perp \left(CDM\right).
			$
		}{
			\begin{tikzpicture}[scale=0.7,line cap=round,line join=round,font=\footnotesize,>=stealth]
				\path
				(0,0) coordinate (A)
				($(A)+(-35:2)$) coordinate (B)
				($(A)+(0:4)$) coordinate (C)
				($(B)!0.5!(A)$) coordinate (M)
				($(M)+(90:3)$) coordinate (D)
				;
				\draw (D)--(A)--(B)--(C)--(D)--(M) (B)--(D)
				;
				\draw[dashed] (A)--(C)--(M)
				;
				\foreach \p/\r in {A/180, B/-90, C/0, D/90,M/-120}
				\fill (\p) circle (1.25pt) node[shift={(\r:3mm)},scale=1]{$\p$};
			\end{tikzpicture}
		}
	}
\end{ex}
%%%==============EX_12============%%%
\begin{ex}%[1H8V2-3]%[Dự án đề kiểm tra Toán khối 11 GHKII NH23-24-Dot-2- Nguyễn Đức Lợi]%[Deso 10 - Sach KNTT]
	Cho hình chóp $S.ABCD$ có đáy $ABCD$ là nửa lục giác đều với cạnh $a$. Cạnh $SA$ vuông góc với đáy và $SA=a\sqrt{3}$. $M$ là một điểm khác $B$ và ở trên $SB$ sao cho $AM$ vuông góc với $MD$. Khi đó, tỉ số $\dfrac{SM}{SB}$ bằng
	\choice
	{\True $\dfrac{3}{4}$}
	{$\dfrac{2}{3}$}
	{$\dfrac{3}{8}$}
	{$\dfrac{1}{3}$}
	\loigiai{
		\immini{
			Áp dụng tính chất nửa lục giác đều, ta có $BD\perp AB$.\\		
			Mặt khác, $BD\perp SA$. Suy ra $BD\perp \left(SAB\right)$, ta được $BD\perp AM$.\\		
			Kết hợp $AM\perp MD$, ta được $AM\perp \left(SBD\right)$. Suy ra $AM\perp SB$.\\		
			Khi đó $\dfrac{SM}{SB}=\dfrac{SM.SB}{SB^2}=\dfrac{SA^2}{SB^2}=\dfrac{3a^2}{4a^2}=\dfrac{3}{4}$.\\	
		}{
			\begin{tikzpicture}[scale=0.7,line cap=round,line join=round,font=\footnotesize,>=stealth]
				\path
				(0,0) coordinate (A)
				($(A)+(-35:2)$) coordinate (B)
				($(A)+(0:5)$) coordinate (D)
				($(B)+(0:2)$) coordinate (C)
				($(A)+(90:4)$) coordinate (S)
				($(S)!3/5!(B)$) coordinate (M)
				;
				\draw (S)--(A)--(B)--(C)--(D)--(S) (S)--(B) (S)--(C) (A)--(M)
				;
				\draw[dashed] (B)--(D)--(M) (A)--(D)
				;
				\foreach \p/\r in {A/180, B/-90, D/0, C/-30,M/70,S/90}
				\fill (\p) circle (1.25pt) node[shift={(\r:3mm)},scale=1]{$\p$};
			\end{tikzpicture}
		}
	}
\end{ex}



\Closesolutionfile{ans}
\bangdapan{Dapan}

\subsection{Câu trắc nghiệm đúng sai}
Học sinh trả lời từ câu 1 đến câu 4.
Trong mỗi ý \circlenum{A}, \circlenum{B}, \circlenum{C} và \circlenum{D} ở mỗi câu, học sinh chọn đúng hoặc sai.
\setcounter{ex}{0}
\LGexTF
\Opensolutionfile{ansbook}[ansbook/DapanDS]
\Opensolutionfile{ans}[Ans/DapanT]
%%%============EX_1==============%%%
\begin{ex}%[1D6H2-1]%[Dự án đề kiểm tra Toán khối 11 GHKII NH23-24-Dot-2- Nhật Thiện]%[Deso 10 - Sach KNTT]
	Biết ${a>0, a \neq 1}$.
	\choiceTF
	{Nếu biểu thức $A=2^{\log_23} - \log_{\sqrt{3}}3$ thì $A>2$}
	{\True Nếu biểu thức $B=\ln 2\cdot \log_24\cdot \log_43\cdot \log_32 - 5^{\log_5(\ln 2)}$ thì $B=0$}
	{Nếu biểu thức $C=\log_a\sqrt{a\sqrt{a\sqrt{a}}}$ thì $C>1$}
	{Nếu biểu thức $D=\log_a\dfrac{\sqrt{a^3}}{a\sqrt[4]{a}}$ thì $D>1$}
	\loigiai{
		\begin{itemize}
			\item Ta có $A=2^{\log _2 3} - \log _{\sqrt{3}} 3=3 - \log _{3^{\tfrac{1}{2}}} 3=3 - 2=1$.
			\item Ta có 
			\begin{eqnarray*}
				B&= &\ln 2 \cdot \log _2 4 \cdot \log _4 3 \cdot \log _3 2 - 5^{\log _5(\ln 2)}\\
				& =&\ln 2 \cdot \log _2 3 \cdot \log _3 2 - \ln 2 \\
				& =&\ln 2 - \ln 2=0.
			\end{eqnarray*}
			\item Ta có $C=\log _a \sqrt{a \sqrt{a \sqrt{a}}}=\log _a\left[a \cdot\left(a \cdot a^{\tfrac{1}{2}}\right)^{\tfrac{1}{2}}\right]^{\tfrac{1}{2}}=\log _a a^{\tfrac{7}{8}}=\dfrac{7}{8}$.
			\item Ta có $D=\log _a \dfrac{\sqrt{a^3}}{a \sqrt[4]{a}}=\log _a \dfrac{a^{\tfrac{3}{2}}}{a \cdot a^{\tfrac{1}{4}}}=\log _a a^{\tfrac{3}{2} - \left(1 + \tfrac{1}{4}\right)}=\log _a a^{\tfrac{1}{4}}=\dfrac{1}{4}$. 
		\end{itemize}
	}
\end{ex}
\begin{ex}%[1D6H4-5]%[Dự án đề kiểm tra Toán khối 11 GHKII NH23-24-Dot-2- Nhật Thiện]%[Deso 10 - Sach KNTT]
	Cho bất phương trình $\left(\dfrac{1}{9}\right)^x \geq 27\cdot 3^x$, có tập nghiệm là $S=\left(a; b\right]$.  
	\choiceTF
	{\True Bất phương trình có chung tập nghiệm với $3^{-2x}\ge 3^{3+x}$}
	{\True Có $A\left(0; b\right)$ giao điểm của đồ thị $y=x^3 + 2x - 1$ với trục tung $Oy$}
	{\True $\lim\limits_{x\to a} \left(3x + 2\right)=a$}
	{$\lim\limits_{x\to b} \left(3x + 2\right)=2$}
	\loigiai{
		Ta có 
		\begin{eqnarray*}
			& &\left(\dfrac{1}{9}\right)^x \geq 27\cdot 3^x \\
			&\Leftrightarrow& 3^{- 2 x} \geq 3^3 \cdot 3^x \Leftrightarrow 3^{- 2 x} \geq 3^{3 + x} \\
			&\Leftrightarrow& - 2 x \geq 3 + x\,(\text{do }3>1) \\
			&\Leftrightarrow& x \leq - 1.
		\end{eqnarray*}
		Vậy tập nghiệm của bất phương trình là $S=(-\infty;-1]$, suy ra $a=-\infty$; $b=-1$.\\
		Khi đó
		\begin{itemize}
			\item Ta có $3^{-2x}\ge 3^{3+x}\Leftrightarrow -2x\ge 3+x\Leftrightarrow x\le -1$.\\
			Bất phương trình đã cho có cùng tập nghiệm với $3^{-2x}\ge 3^{3+x}$.
			\item Với $x=0\Rightarrow y=0^3+2\cdot 0-1=-1$.\\
			Vậy đồ thị $y=x^3+2x-1$ cắt trục tung $Oy$ tại điểm $A(0;-1)$.
			\item $\lim\limits_{x\to -\infty} \left(3x + 2\right)=-\infty$.
			\item $\lim\limits_{x\to -1} \left(3x + 2\right)=3\cdot (-1)+2=-1$.
		\end{itemize}
	}
\end{ex}
\begin{ex}%[1H8V1-3]%[Dự án đề kiểm tra Toán khối 11 GHKII NH23-24-Dot-2- Nhật Thiện]%[Deso 10 - Sach KNTT]
	Cho hình lăng trụ tam giác $ABC.A'B'C'$ có $AA'\perp A B$, $AA'\perp A C$ và tất cả các cạnh đều bằng $a$. Gọi $M$ là trung điểm $AA'$. 
	\choiceTF
	{\True $\left(A'B, C'C\right)=\widehat{AA'B}$}
	{\True $\left(A'B, C'C\right)=45^{\circ} $}
	{$\left(A'C, MB\right)=\widehat{BAN}$}
	{$\widehat{BMN}\approx 42{,}6^{\circ}$}
	\loigiai{
		\begin{center}
			\begin{tikzpicture}[scale=1, font=\footnotesize, line join=round, line cap=round, >=stealth]
				\path (0,0) coordinate (A)	(1,-1) coordinate (B) (3,0) coordinate (C) ($(A)!.5!(C)$) coordinate (N);
				\foreach \x in {A,B,C} \path (\x)+(0,3) coordinate (\x');
				\path ($(A)!.5!(A')$) coordinate (M) ;
				\draw (A')--(A)--(B)--(C)--(C')--cycle (B)--(B') (A')--(B')--(C') (B)--(M);
				\draw[dashed] (A)--(C) (M)--(N)--(B);
				\foreach \p/\r in {A/-120,A'/120,B/-90,B'/-60,C/0,C'/0,M/180,N/45}
				\fill (\p) circle (1.5pt) node[shift={(\r:3mm)}]{$\p$};
			\end{tikzpicture}
		\end{center}
		\begin{itemize}
			\item Ta có: $A'A\parallel C'C \Rightarrow\left(A'B, C'C\right)=\left(A'B, A'A\right)=\widehat{AA'B}$
			\item $\Delta A'A B$ vuông cân tại $A$ nên $\widehat{AA'B}=45^{\circ}$ hay $\left(A'B, A'A\right)=45^\circ$.
			\item Gọi $N$ là trung điểm của $AC$.\\
			Ta có $A'C\parallel MN \Rightarrow\left(A'C, MB\right)=(MN, MB)=\widehat{BMN}$.
			\item 
			Xét $\triangle MNB$ có
			\begin{itemize}
				\item $MB=\sqrt{AB^2+AM^2}=\sqrt{a^2+\left(\dfrac{a}{2}\right)^2}=\dfrac{a\sqrt{5}}{2}$.
				\item $MN=\sqrt{AN^2+AM^2}=\sqrt{\left(\dfrac{a}{2}\right)^2 + \left(\dfrac{a}{2} \right)^2}=\dfrac{a\sqrt{2}}{2}$.
				\item $BN=\dfrac{a \sqrt{3}}{2}$.
			\end{itemize}
			Suy ra $\cos \widehat{BMN}=\dfrac{\left(\dfrac{\sqrt{5}}{2} a\right)^2 +\left(\dfrac{a\sqrt{2}}{2}\right)^2- \left(\dfrac{\sqrt{3}}{2} a\right)^2}{2\cdot \dfrac{\sqrt{5}}{2} a\cdot \dfrac{a\sqrt{2}}{2} }=\dfrac{\sqrt{10}}{5} \Rightarrow \widehat{BMN} \approx 50{,}76^{\circ}$.
		\end{itemize}
	}
\end{ex}
\begin{ex}%[1H8H2-3]%[1H8V5-3]%[1H8H1-3]%[Dự án đề kiểm tra Toán khối 11 GHKII NH23-24-Dot-2- Nhật Thiện]%[Deso 10 - Sach KNTT]
	Cho hình lập phương $ABCD.A'B'C'D'$ có cạnh $a$. 
	\choiceTF
	{\True $A'D'\bot \left(ABB'A'\right)$}
	{\True $\left(A'D', A B'\right)=90^{\circ}$}
	{\True $B'D' \perp\left(AA'O\right)$ với $O$ là tâm hình vuông $ABCD$}
	{Gọi hình chiếu của điểm ${A'}$ trên mặt phẳng $\left(A B'D'\right)$ là $H$. Khi đó $A'H=\dfrac{a\sqrt{2}}{3}$}
	\loigiai{
		\begin{center}
			\begin{tikzpicture}[scale=1, font=\footnotesize, line join=round, line cap=round, >=stealth]
				\path (0,0) coordinate (A)	(1.3,-1) coordinate (B) (3,0) coordinate (D) ($(B)+(D)-(A)$) coordinate (C) ($(A)!.5!(C)$) coordinate (O);
				\foreach \x in {A,B,C,D,O} \path (\x)+(0,-3) coordinate (\x');
				\path ($(A)!2/3!(O')$) coordinate (H);
				\draw (A)--(A')--(B')--(C')--(C)--(D)--cycle (A)--(B)--(C) (B)--(B') (A)--(C) (B)--(D) (A)--(B');
				\draw[dashed] (A')--(C') (B')--(D') (A')--(D')--(C') (D)--(D') (A)--(D') (A)--(O') (A')--(H);
				\foreach \p/\r in {A/-135,A'/135,B/90,B'/-90,C/0,C'/0,D/45,D'/45,O/-90,O'/-90,H/30}
				\fill (\p) circle (1.5pt) node[shift={(\r:3mm)}]{$\p$};
			\end{tikzpicture}
		\end{center}
		\begin{itemize}
			\item Ta có: $\heva{&A'D' \perp AA'\\&  A'D' \perp A'B'} \Rightarrow A'D' \perp\left(ABB'A'\right)$.\\
			\item Do $AB'\subset\left(ABB'A'\right)$ nên ${A'D' \perp AB'}$. \\
			Vậy $\left(A'D', AB'\right)=90^{\circ}$. 
			\item Ta có $\heva{&B'D'\perp A'C'\\&B'D'\perp AA'}\Rightarrow B'D'\perp (AA'C'C)$ hay $B'D'\perp (AA'O)$.
			\item Kẻ $A'H\perp AO'$ tại $H$ với $O'$ là tâm hình vuông $A'B'C'D'$.\\
			Do $B'D'\perp (AA'O)$ mà $A'H\subset (AA'O)$. Suy ra $A'H\perp B'D'$.\\
			Khi đó $\heva{&A'H\perp AO'\\&A'H\perp B'D'}\Rightarrow A'H'\perp (AB'D')$.\\
			Ta có $A'O'=\dfrac{A'C'}{2}=\dfrac{a\sqrt{2}}{2}$.\\
			Xét tam giác $AA'O$ vuông tại $A'$ có 
			$$\dfrac{1}{A'H^2}=\dfrac{1}{AA'^2}+\dfrac{1}{A'O'^2}\Rightarrow A'H=\dfrac{AA'\cdot A'O'}{\sqrt{AA'^2+A'O'^2}}=\dfrac{a\cdot \dfrac{a\sqrt{2}}{2}}{\sqrt{a^2+\left(\dfrac{a\sqrt{2}}{2}\right)^2}}=\dfrac{a\sqrt{3}}{3}.$$
		\end{itemize}
	}
\end{ex}

\Closesolutionfile{ans}
\Closesolutionfile{ansbook}

\begin{center}
	\textbf{\textsf{BẢNG ĐÁP ÁN ĐÚNG SAI}}
\end{center}
\input{Ansbook/DapanDS}

\subsection{Phần tự luận}

\hienthiloigiaibt
%%%==============BT_1==============%%%
\begin{bt}%[1D6V2-1]%[Dự án đề kiểm tra Toán khối 11 GHKII NH23-24-Dot 2-Tư Đô Nguyên]%[De10 - KNTT]
	Cho $f(x)=\mathrm{e}^{\sqrt{1+\tfrac{1}{x^2}+\tfrac{1}{(x+1)^2}}}$. Biết rằng $f(1)\cdot f(2)\cdot f(3)\ldots f(2025)=\mathrm{e}^{\tfrac{m}{n}}$ với $m$, $n$ là các số tự nhiên và $\dfrac{m}{n}$ là phân số tối giản. Tính $m-n^2$.
	\loigiai{
		Đặt $g(x)=\sqrt{1+\dfrac{1}{x^2}+\dfrac{1}{(1+x)^2}}$. \\
		Với $x > 0$ ta có
		\allowdisplaybreaks
		\begin{eqnarray*}
			g(x) & =&\sqrt{1+\frac{1}{x^2}+\dfrac{1}{(1+x)^2}}=\dfrac{\sqrt{x^2+(x+1)^2+x^2 \cdot(x+1)^2}}{x(x+1)}=\dfrac{\sqrt{\left(x^2+x+1\right)^2}}{x(x+1)} \\
			& =&\dfrac{x^2+x+1}{x(x+1)}=1+\dfrac{1}{x(x+1)}=1+\dfrac{1}{x}-\dfrac{1}{x+1}.
		\end{eqnarray*}
		Suy ra 
		\allowdisplaybreaks
		\begin{eqnarray*}
			g(1)+g(2)+\cdots+g(2025) &=& \left(1+\dfrac{1}{1}-\dfrac{1}{2} \right)+\left(1+\dfrac{1}{2}-\dfrac{1}{3} \right)+\cdots+\left(1+\dfrac{1}{2025}-\dfrac{1}{2026} \right) \\ 
			&=&2026-\dfrac{1}{2026}.
		\end{eqnarray*}
		Khi đó $f(1)\cdot f(2)\cdot f(3)\ldots f(2025)=\mathrm{e}^{g(1)+g(2)+g(3)+\cdots+g(2025)}=\mathrm{e}^{2026-\tfrac{1}{2026}}=\mathrm{e}^{\tfrac{2026^2-1}{2026}}=\mathrm{e}^{\tfrac{m}{n}}$. \\
		Do đó $m=2026^2-1$, $n=2026$. \\
		Vậy $m-n^2=2026^2-1-2026^2=-1$.
	}
\end{bt}
%%%==============BT_2==============%%%
\begin{bt}%[1D6V2-5]%[Dự án đề kiểm tra Toán khối 11 GHKII NH23-24-Dot 2-Tư Đô Nguyên]%[De10 - KNTT]
	Cường độ một trận động đất $M$ (độ Richter) được cho bởi công thức $M=\log A-\log A_0$, với $A$ là biên độ rung chấn tối đa và $A_0$ là một biên độ chuẩn (hằng số). Đầu thế kỉ $20$, một trận động đất ở San Francisco có cường độ $8$ độ Richter. Trong cùng năm đó, một trận động đất khác ở Nam Mỹ có biên độ rung chấn mạnh hơn gấp 4 lần. Hỏi cường độ của trận động đất ở Nam Mỹ là bao nhiêu (kết quả được làm tròn đến hàng phần chục)?
	\loigiai{
		Gọi $M_1$, $M_2$ lần lượt là cường độ của trận động đất ở San Francisco và ở Nam Mỹ.
		Trận động đất ở San Francisco có cường độ là 8 độ Richter nên
		\allowdisplaybreaks
		\begin{eqnarray*}
			M_1=\log A-\log A_0 \Leftrightarrow 8=\log A-\log A_0.
		\end{eqnarray*}
		Trận động đất ở Nam Mỹ có biên độ là $4A$, khi đó cường độ của trận động đất ở Nam Mỹ là
		\allowdisplaybreaks
		\begin{eqnarray*}
			M_2=\log (4A)-\log A_0=\log 4+\left(\log A-\log A_0 \right)=\log 4+8\approx 8{,}602 \text{ (độ Richter)}.
		\end{eqnarray*}
	}
\end{bt}

%%%==============BT_3==============%%%
\begin{bt}%[1D6V3-2]%[Dự án đề kiểm tra Toán khối 11 GHKII NH23-24-Dot 2-Tư Đô Nguyên]%[De10 - KNTT]
	Tìm tập xác định của hàm số $y=\left(x^2-x-2\right)^{-\log 100}.$
	\loigiai{
		Ta có $-\log 100=-2\in \mathbb{Z}^{-}$.
		Khi đó hàm số $y=\left(x^2-x-2\right)^{-\log 100}$ xác định khi và chỉ khi 
		\allowdisplaybreaks
		\begin{eqnarray*}
			x^2-x-2\neq 0\Leftrightarrow\heva{&x \neq-1\\
				&x \neq 2.}
		\end{eqnarray*}
		Vậy tập xác định của hàm số $y=\left(x^2-x-2\right)^{-\log 100}$ là $\mathscr{D}=\mathbb{R}\setminus\{-1;2\}$.
	}
\end{bt}

%%%==============BT_4==============%%%
\begin{bt}%[1D6V4-3]%[Dự án đề kiểm tra Toán khối 11 GHKII NH23-24-Dot 2-Tư Đô Nguyên]%[De10 - KNTT]
	Giải bất phương trình sau $5^{\left|x^2-2x\right|} > 125$.
	\loigiai{
		Ta có
		\allowdisplaybreaks
		\begin{eqnarray*}
			5^{\left|x^2-2x\right|}>125\Leftrightarrow\left|x^2-2x\right|>\log _5125\Leftrightarrow \left|x^2-2x\right|>3\Leftrightarrow\hoac{
				&	{x ^ {2}-2x > 3} \\
				&	{x ^ {2}-2x <-3}
			} \Leftrightarrow \hoac{&x<-1\\
				&x>3.}
		\end{eqnarray*}
		Tập nghiệm của bất phương trình là $S=(-\infty;-1) \cup (3;+\infty)$.}
\end{bt}

%%%==============BT_5==============%%%
\begin{bt}%[1H8V1-3]%[Dự án đề kiểm tra Toán khối 11 GHKII NH23-24-Dot 2-Tư Đô Nguyên]%[De10 - KNTT]
	Cho hình hộp $ABCD.A'B'C'D'$ có 6 mặt là hình vuông cạnh bằng $a$. Gọi $M,N$ lần lượt là trung điểm của cạnh $AA'$ và $A'B'$. Tính số đo góc giữa hai đường thẳng $MN$ và $BD$.
	\loigiai{
		\immini{
			Gọi $P$ là trung điểm cạnh $A'D'$. \\
			Vì $ABCD.A'B'C'D'$ là hình lập phương cạnh $a$ nên $$AB'=B'D'=D'A=a\sqrt{2}.$$
			Suy ra $MN=NP=PM=\dfrac{a\sqrt{2}}{2}$. \\
			Khi đó tam giác $MNP$ đều.\\
			Vậy $(MN,BD)=(MN,NP)=\widehat{MNP}=60^{^\circ}.$	
		}
		{
			\begin{tikzpicture}
				\def\a{3.5}
				\def\b{2}
				\def\h{3.5}
				\path 	(0:0) coordinate (A)
				++(0:\a) coordinate (D)
				++(-130:\b) coordinate (C)
				($(A)+(C)-(D)$) coordinate (B)
				($(A)+(90:\h)$) coordinate (A')
				($(B)+(90:\h)$) coordinate (B')
				($(C)+(90:\h)$) coordinate (C')
				($(D)+(90:\h)$) coordinate (D')
				($(A)!0.5!(A')$) coordinate (M)
				($(A')!0.5!(B')$) coordinate (N)
				($(A')!0.5!(D')$) coordinate (P)
				;
				\draw[dashed,thick] 	(B)--(A)--(D)	(A)--(A') (M)--(N)--(P)--cycle (A)--(B')--(D')--cycle;
				\draw[thick] 	(C)--(C') 	(D)--(D') 	(B)--(B')	(C)--(C') (N)--(P) (B')--(D')
				(B)--(C)--(D) 
				(A')--(B')--(C')--(D')--cycle;
				\foreach \x/\g in {A/180,B/180,C/0,D/0,A'/150,B'/180,C'/0,D'/0,M/-30,N/135,P/70}
				\fill[black] 	(\x) circle (1.5pt)
				($(\g:4mm)+(\x)$) node {$\x$};	
			\end{tikzpicture}	
		}
	}
\end{bt}

%%%==============BT_6==============%%%
\begin{bt}%[1H8V2-3]%[Dự án đề kiểm tra Toán khối 11 GHKII NH23-24-Dot 2-Tư Đô Nguyên]%[De10 - KNTT]
	Hình chóp $S.ABCD$ có cạnh $SA$ vuông góc với mặt phẳng $(ABCD)$ và đáy $ABCD$ là hình thang vuông tại $A$ và $D$ với $AD=CD=\dfrac{AB}{2}$. Gọi $I$ là trung điểm của đoạn $AB$. Hỏi các mặt bên của hình chóp $S.ABCD$ là tam giác gì?
	\loigiai{
		\immini{
			Đặt $AB=2a\Rightarrow AD=CD=a$. \\
			Do $AB=2CD$ nên $AI=AD=CD=CI=a$. \\
			Khi đó $AICD$ là hình vuông cạnh $a$. \\
			Do $SA\perp (ABCD)$ nên $\triangle SAD$, $\triangle SAB$ vuông tại $A$.\\
			Mặt khác $\heva{&{CD\perp AD} \\
				&{CD\perp SA}}\Rightarrow CD\perp (SAD)\Rightarrow CD\perp SD$ nên $\triangle SCD$ vuông tại $D$.\\
			Xét $\triangle ACB$ có trung tuyến $CI=\dfrac{1}{2} AB\Rightarrow \triangle ACB$ vuông tại $C\Rightarrow BC\perp AC$.\\
			Mặt khác $BC\perp SA\Rightarrow BC\perp (SAC)\Rightarrow BC\perp SC\Rightarrow \triangle SCB$ vuông tại $C$.
		}
		{
			\begin{tikzpicture}
				\def\a{4.5}
				\def\h{3.5}
				\path 	(0:0) coordinate (A)
				++(0:\a) coordinate (B)
				($(A)+(-130:\a/2)$) coordinate (D)
				($(B)+(D)-(A)$) coordinate (Ct)
				($(D)!1/2!(Ct)$) coordinate (C)
				($(A)+(90:\h)$) coordinate (S)
				($(A)!1/2!(B)$) coordinate (I)
				(intersection of A--C and I--D) coordinate (O);%giao điểm O
				\draw[dashed,thick] 	(B)--(A)--(D)	(A)--(S) (C)--(I) (A)--(C) (D)--(I);
				\draw[thick] 			(B)--(C)--(D)
				(B)--(S)	(C)--(S)	(D)--(S);
				\foreach \x/\g in {A/135,B/45,C/-45,D/-135,S/90,I/60,O/-110}
				\fill[black] 	(\x) circle (1.5pt)
				($(\g:3mm)+(\x)$) node {$\x$};	
				\draw pic[draw,angle radius=2mm]{right angle=B--A--S};%Theo chiều dương
			\end{tikzpicture}
		}
	}
	
\end{bt}