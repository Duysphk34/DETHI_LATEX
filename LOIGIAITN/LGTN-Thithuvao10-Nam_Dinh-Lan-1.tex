\protect \thongtin {Lời giải chiết phần trắc nghiệm}
\begin{loigiaiex}{1}
  Điều kiện xác định của $\sqrt {\dfrac {-1}{x-2}}$ là $x-2<0$ $\Leftrightarrow $ $x<2$  \phantom {a}\hfill { \faKey ~\circlenum {C}} 
\end{loigiaiex}
\begin{loigiaiex}{2}
  Ta có $\sqrt {(a-1)^2}-a = |a-1|-a = -(a-1) -a =1-2a$  \phantom {a}\hfill { \faKey ~\circlenum {A}} 
\end{loigiaiex}
\begin{loigiaiex}{3}
  Vì đồ thị hàm số $y=x+1-m$ cắt trục hoành tạo điểm có hoành độ $x=2$ nên ta có $2+1-m =0 $ $\Leftrightarrow $ $m=3$  \phantom {a}\hfill { \faKey ~\circlenum {B}} 
\end{loigiaiex}
\begin{loigiaiex}{4}
  Phương trình $x^2-2x-1=0$ có hai nghiệm $x_1; x_2$ nên theo hệ thức Vi-et ta có $\heva {x_1+x_2 &=2\\x_1\cdot x_2&=-1}$\\ Ta có $x_1^2 +x_2^2 =\left ( x_1+x_2\right )^2-2x_1\cdot x_2 =2^2-2\cdot (-1) =6$  \phantom {a}\hfill { \faKey ~\circlenum {C}} 
\end{loigiaiex}
\begin{loigiaiex}{5}
  Gọi $x$ (m) là chiều rộng hình chữ nhật lúc đầu $\left (x>0\right )$. \\ Chiều dài hình chữ nhật lúc đầu là $2x$ (m). \\ Theo đề bài ta có phương trình $2x-5=x+5$ $\Leftrightarrow $ $x=10$ (m).\\Vậy chu vi hình chữ nhật là $\left (x+2x\right )\cdot 2 = \left (10+20\right )\cdot 2 =60$ (m)  \phantom {a}\hfill { \faKey ~\circlenum {D}} 
\end{loigiaiex}
\begin{loigiaiex}{6}
  Ta có $|5-3|=2<7<8=5+3$ hay $|R-r|<OO^\prime <R+r$. \\ Vậy hai đường tròn cắt nhau.  \phantom {a}\hfill { \faKey ~\circlenum {A}} 
\end{loigiaiex}
\begin{loigiaiex}{7}
  \immini { Vì $AB$ là tiếp tuyến của $(O)$ nên $\triangle ABO $ vuông tại $O$. $\Rightarrow $ $\cos AOB =\dfrac {BO}{AO}= \dfrac {1}{2}$ $\Rightarrow AOB = 60^\circ $ \\ Theo tính chất hai tiếp tuyến cắt nhau ta có $OA$ là tia phân giác $BOC$ \\ $\Rightarrow $ $BOC=2AOB=120^\circ $ $\Rightarrow \text {sđ}\overarc {BC}=120^\circ $ \\ Độ dài cung $BC$ $l_{BC} =\dfrac {120}{360}\cdot 2\pi \cdot 1 = $ }{}  \phantom {a}\hfill { \faKey ~\circlenum {C}} 
\end{loigiaiex}
\begin{loigiaiex}{8}
  \phantom {a}\hfill { \faKey ~\circlenum {A}} 
\end{loigiaiex}
