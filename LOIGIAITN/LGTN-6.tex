\begin{loigiaiex}{1}
  \par Việc chọn 2 quyển sách khác bộ môn sẽ xảy ra một trong 3 trường hợp sau: \begin {itemize} \item Chọn sách toán và vật lý có $5 \cdot 3 =15$ (cách chọn) \item Chọn sách toán và Tiếng anh có $5 \cdot 6 =30$ (cách chọn) \item Chọn sách vật lý và tiếng anh có $3 \cdot 6 =18$ (cách chọn) \end {itemize} \par Vậy có tất cả $15 + 30 + 18 =63 $ (cách chọn)  \phantom {a}\hfill {\bfseries \sffamily Chọn~\circleTrue {B}} 
\end{loigiaiex}
\begin{loigiaiex}{2}
  \par Với 3 điểm không thẳng hàng sẽ tạo ra một tam giác. Từ đỉnh A và 2 điểm bất kì nằm trên đường thẳng d sẽ cho ta một tam giác. \par Như vậy số tam giác đuọc tạo ra là $\mathrm {C}_6^2$ (tam giác) \par  \phantom {a}\hfill {\bfseries \sffamily Chọn~\circleTrue {A}} 
\end{loigiaiex}
\begin{loigiaiex}{3}
  \par Số các số có 6 chữ số khác nhau được tạo ra từ $\left \{1, 2, 3, 4, 5, 6\right \}$ là $6!=720$ (số) \par Số các số có 6 chữ số khác nhau bắt đầu bởi 34 là $\mathrm {A}_4^3=24$ (số) \par Vậy số các số khác nhau không bắt đầu bởi $34$ là $720-24=696$ (số)  \phantom {a}\hfill {\bfseries \sffamily Chọn~\circleTrue {D}} 
\end{loigiaiex}
\begin{loigiaiex}{4}
  \begin {itemize} \item Chọn số 1 và 2 xếp vào 9 vị trí sẽ có $\mathrm {A}_9^2$. Tuy nhiên số 1 đứng trước số 2 nên không xét đến hoán vị số 1 và 2 do đó số cách chọn là $\dfrac {\mathrm {A}_9^2}{2!}= \mathrm {C}_9^2 $. \item Tương tự chọn số 3 và 4 xếp vào 7 vị trí còn lại sao cho số 3 đứng trước số 4 số cách chọn là $\mathrm {C}_7^2 $. \item Chọn chữ số 5 và 6 xếp vào 5 vị trí còn lại sao cho chữ số 5 đứng trước chữ số 6 số cách chọn là $\mathrm {C}_5^2 $. \item Xếp 3 chữ số 7, 8, 9 vào 3 vị trí còn lại chính là $3!$ \end {itemize} Vậy tổng cộng có $\mathrm {C}_9^2\cdot \mathrm {C}_7^2\cdot \mathrm {C}_5^2\cdot 3!=45360 $ (số)  \phantom {a}\hfill {\bfseries \sffamily Chọn~\circleTrue {C}} 
\end{loigiaiex}
\begin{loigiaiex}{5}
  Việc chọn hội đồng gồm hai bước: \begin {itemize} \item Bước 1. Chọn 2 giáo viên trong 5 giáo viên có $\mathrm {C}_5^2 =10$ (cách chọn) \item Bước 2. Chọn 3 học sinh trong 6 học sinh có $\mathrm {C}_6^3 =20$ (cách chọn) \item Vậy theo quy tắc nhân ta có $10\cdot 20=200$ (cách chọn) \end {itemize}  \phantom {a}\hfill {\bfseries \sffamily Chọn~\circleTrue {A}} 
\end{loigiaiex}
\begin{loigiaiex}{6}
  \par Ta có: \[(5x+2y)^4 =\mathrm {C}_4^0\cdot (5x)^4+\mathrm {C}_4^1\cdot (5x)^3\cdot (2y)+\mathrm {C}_4^2\cdot (5x)^2\cdot (2y)^2+\mathrm {C}_4^3\cdot (5x)\cdot (2y)^3+\mathrm {C}_4^4\cdot (2y)^4\] \par Vậy số hạng chính giữa trong khai triển ứng với $\mathrm {C}_4^2\cdot (5x)^2\cdot (2y)^2= 600x^2y^2$ \par  \phantom {a}\hfill {\bfseries \sffamily Chọn~\circleTrue {D}} 
\end{loigiaiex}
\begin{loigiaiex}{7}
  \phantom {a}\hfill {\bfseries \sffamily Chọn~\circleTrue {A}} 
\end{loigiaiex}
\begin{loigiaiex}{8}
  \par Ta có: \par $\overrightarrow {AB}=(-3;-5) $ \par $\overrightarrow {AC}=(2;-2) $ \par $AB=\sqrt {(-3)^2 +(-5)^2}= \sqrt {34}$ \par $AC=\sqrt {(2)^2 +(-2)^2}= \sqrt {8}$ \par Lại có: $\cos A$=$|\cos (\overrightarrow {AB},\overrightarrow {AC})|=\dfrac {|\overrightarrow {AB}\cdot \overrightarrow {AC}|}{|\overrightarrow {AB}|\cdot |\overrightarrow {AC}|}=\dfrac {|(-3)\cdot 2+ (-5)\cdot (-2)|}{\sqrt {(-3)^2+(-5)^2}\cdot \sqrt {(2)^2+(-2)^2}}=\dfrac {\sqrt {17}}{17}$ \par \par \par Mặt khác: $\sin ^2 \mathrm {~A}+\cos ^2 \mathrm {~A}=1$ $ \Rightarrow \sin \mathrm {A}=\sqrt {1-\cos ^2 \mathrm {~A}}=\sqrt {1-\left (\dfrac {\sqrt {17}}{17}\right )^2}=\dfrac {4\sqrt {17}}{17} $ \par Diện tích $\triangle ABC$ là $S_{\triangle ABC}=\dfrac {1}{2}AB\cdot AC \cdot \sin A = \sqrt {34}\cdot \sqrt {8}\cdot \dfrac {4\sqrt {17}}{17} =16$  \phantom {a}\hfill {\bfseries \sffamily Chọn~\circleTrue {C}} 
\end{loigiaiex}
\begin{loigiaiex}{9}
  \par Gọi $\overrightarrow {v_0}$ là vận tốc của dòng nước trên sông. \par Ta có $\overrightarrow {v_1}+ \overrightarrow {v_0}= \overrightarrow {v_2}$ $\Rightarrow \overrightarrow {v_0} =\overrightarrow {v_2} - \overrightarrow {v_1}= (2-3;1-4)=(-1;-3)$ \par Vậy vận tốc của dòng nước trên sông là $|\overrightarrow {v_0}| =\sqrt {(-1)^2+(-3)^2} \approx 3{,}2$ (m/s) \par  \phantom {a}\hfill {\bfseries \sffamily Chọn~\circleTrue {A}} 
\end{loigiaiex}
\begin{loigiaiex}{10}
  \par Ta có $\cos (\overrightarrow {OM},\overrightarrow {ON})=\dfrac {\overrightarrow {OM}\cdot \overrightarrow {ON}}{|\overrightarrow {OM}|\cdot |\overrightarrow {ON}|}=\dfrac {(-2)\cdot 3 +(-1)\cdot (-1)}{\sqrt {(-2)^2+(-1)^2}\cdot \sqrt {(3)^2+(-1)^2}}=\dfrac {\sqrt {2}}{2}$. \par Vậy góc giữa hai vec tơ $\overrightarrow {OM}$ và $\overrightarrow {ON}$ là $45^\circ $  \phantom {a}\hfill {\bfseries \sffamily Chọn~\circleTrue {B}} 
\end{loigiaiex}
\begin{loigiaiex}{11}
  \par Gọi $M(x;0)$ là điểm di động trên $Ox$. \par Ta có: $\overrightarrow {MA} = (1-x;2) \Rightarrow MA =\sqrt {(1-x)^2+2^2} $ \par $\overrightarrow {MB}=(4-x;1)\Rightarrow MB=\sqrt {(4-x)^2+1^2} $ \par $\Rightarrow MA + MB = \sqrt {(1-x)^2+2^2} + \sqrt {(4-x)^2+1^2}$ (*) \par Áp dụng bất đẳng thức Minkovsky \begin {eqnarray*} &\text {VP(*)}\;&= \sqrt {(1-x)^2+2^2} + \sqrt {(x-4)^2+1^2} \\ & &\ge \sqrt {(1-x+x-4)^2+(2+1)^2} = 3\sqrt {2} \end {eqnarray*} \par Dấu \lq \lq =\rq \rq có khi $\dfrac {1-x}{2}=\dfrac {x-4}{1} \Leftrightarrow x = 3 $. \par Khi đó tọa độ điểm $M(3,0)$ \par Vậy tọa độ trọng tâm tam giác $ABM$ khi $MA + MB$ nhỏ nhất là $\bigg (\dfrac {1+4+3}{3};\dfrac {2+1+3}{3}\bigg )=\bigg (\dfrac {8}{3};1\bigg )$  \phantom {a}\hfill {\bfseries \sffamily Chọn~\circleTrue {A}} 
\end{loigiaiex}
\begin{loigiaiex}{12}
  \par Ta có:$\overrightarrow {AB}=(-6;4)$ \par Gọi $I$ là trung điểm của $AB$ $\Rightarrow $ $I\bigg (\dfrac {2+(-4)}{2};\dfrac {-4+5}{2}\bigg )=(-1;3)$ \par Đường trung trực của $AB$ đi qua trung điểm $I$ của $AB$ vầ nhận $\overrightarrow {AB}$ làm vectơ pháp tuyến có phương trình là: \begin {eqnarray*} && -6(x+1)+4(y-3)=0\\ \Leftrightarrow && 3x-2y+9=0 \end {eqnarray*}  \phantom {a}\hfill {\bfseries \sffamily Chọn~\circleTrue {A}} 
\end{loigiaiex}
