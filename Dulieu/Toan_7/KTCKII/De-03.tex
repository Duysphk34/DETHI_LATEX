%%%Tùy chọn 1: Kì thi
%%%Tùy chọn 2: Môn
%%%Tùy chọn 3: lớp
%%%Tùy chọn 4: Sở/Phòng
%%%Tùy chọn 5: Ngày thi
%\pgfmathtruncatemacro{\x}{random(101,401)}
\begin{name}[Kiểm tra giữa kì II][Toán][7][Sở Giáo dục và Đào tạo]{Trường THCS Số 2 Mỹ Lợi }{2023 - 2024}
\end{name}
\tieumuc{Trắc nghiệm (3 điểm)}
\Opensolutionfile{ans}[Ans/DATN-7CKII-De-03]
%\luuloigiaiex
%\hienthiloigiaiex
\taoNdongke[4]{ex}
\Opensolutionfile{ansex}[LOIGIAITN/LGTN-7CKII-De-03]
%%%============EX_1==============%%%
\begin{ex}[NB]\immini{Loại nước uống được các bạn học sinh lớp 7A yêu thích nhất là:
		\choice[2]
		{Nước chanh}
		{Nước suối}
		{Trà sữa}
		{\True Nước cam}}{%
		\begin{tikzpicture}[line join=round,line cap=round,font=\scriptsize,declare function={R=2.5cm;r=0.67*R;h=0.6*R;}]
			\tikzset{pics/.cd,
				chuthich/.style args={#1}{code={
						\draw [fill=#1] (0:0.5) arc (0:45:0.5)--(0,0)--cycle;
				}}
			}
			\pgfmathsetmacro{\mysum}{90}
			\foreach \i/\mau [evaluate=\i as \goc using (3.6*\i)] in {25/\mauphu,30/cyan,25/orange,10/gray!50!black,10/red!70!blue}
			{\path[fill=\mau,draw=white,thick] (\mysum:R) arc (\mysum:\mysum+\goc:R)--(0,0)--cycle;
				\pgfmathsetmacro{\tb}{\mysum+0.5*\goc}
				\path (\tb:r) node[fill=white,inner sep=1pt]{\i\%};
				\pgfmathsetmacro{\mysum}{\mysum+\goc}
				\xdef\mysum{\mysum}
			}
			\foreach \n/\p/\c/\x in {1/1*h/\mauphu/nước chanh,2/0.5*h/cyan/nước cam,3/0*h/orange/chuối,4/-0.5*h/gray!50!black/trà sữa,2/-1*h/red!70!blue/sinh tố}{%
				\pic[local bounding box=gc\n] at (3,\p) {chuthich={\c}};
				\node [right=3pt of gc\n,anchor= mid west ] {\x};
			}
	\end{tikzpicture}}
	\loigiai{}
\end{ex}
%%%============EX_2==============%%%
\begin{ex}[NB]
	Gieo ngẫu nhiên xúc xắc một lần, kết quả có thể xảy ra đối với mặt xuất hiện là mặt 1 chấm, mặt 2 chấm, mặt 3 chấm, mặt 4 chấm, mặt 5 chấm, mặt 6 chấm. Xét biến cố \lq\lq Mặt xuất hiện của xúc xắc có số chấm là số lẻ\rq\rq thì xác suất của biến cố này là
	\choice 
	{$\dfrac{4}{6}$}
	{$\dfrac{1}{6}$}
	{$\dfrac{5}{6}$}
	{\True $\dfrac{3}{6}$}
	\loigiai{}
\end{ex}
%%%============EX_3==============%%%
\begin{ex}[NB]
Dựa vào bảng số liệu sau, hãy cho biết trong năm 2019, ngành dệt may Việt Nam đạt kim ngạch xuất khẩu là bao nhiêu?\\
\begin{center}
	\begin{tabular}{|c|*{4}{C{0.17\textwidth}|}}
	\hline \rowcolor{\mauphu!20}\textbf { Năm } & \textbf{2 0 1 7} & \textbf{2 0 1 8} & \textbf{2 0 1 9} & \textbf{2 0 2 0} \\
	\hline \text { Ngành dệt may } & 31,8 & 36,2 & 38,8 & 35,0 \\
	\hline
	\end{tabular}
\end{center}
\choice
{31,8}
{36,2}
{\True 38,8}
{35,0}
\loigiai{}
\end{ex}
%%%============EX_4==============%%%
\begin{ex}[NB]
	Một hình chữ nhật có chiều dài là $5\mathrm{~cm}$, chiều rộng $3\mathrm{~cm}$. Biểu thức nào sau đây biểu thị chu vi của hình chữ nhật đó:
	\choice
	{$5+3$}
	{$5\cdot3$}
	{$2\cdot5+3$}
	{\True$2\cdot(5+3)$}
	\loigiai{}
\end{ex}
%%%============EX_5==============%%%
\begin{ex}[NB]
Trong các biểu thức sau, biểu thức nào là đơn thức một biến:
\choice
{$\dfrac{2}{5}$+x $^2$ y $^2$}
{\True$2x$}
{$1-\dfrac{5}{9} x^2$}
{$3x^2y^3z$}
\loigiai{}
\end{ex}
%%%============EX_6==============%%%
\begin{ex}[NB]
Đa thức nào sau đây là đa thức một biến?
\choice
{$x^2 y+3x-5$}
{$2xy-3x+1$}
{\True$2x^3-3x+1$}
{$2x^3-4z+1$}
\loigiai{}
\end{ex}
%%%============EX_7==============%%%
\begin{ex}[NB]$x=1$ là nghiệm của đa thức:
	\choice 
	{$x+1$}
	{\True$x-1$}
	{$x^3+1$}
	{$x^2+1$}
	\loigiai{}
\end{ex}
%%%============EX_8==============%%%
\begin{ex}[NB]
Bậc của đa thức $P(x)=-x^5-3x^4-x^2+3$ là
	\choice 
	{\True$5$}
	{$4$}
	{$2$}
	{$3$}
\loigiai{}
\end{ex}
%%%============EX_9==============%%%
\begin{ex}[NB]\immini{%
		Các tam giác cân trong hình vẽ dưới đây là
		\choice
		{$\triangle MNP;\; \triangle MNQ$}
		{\True$\triangle MNP;\; \triangle PMQ$}
		{$\triangle MPQ;\; \triangle MNQ$}
		{$\triangle MPQ$}
		}{%
		\begin{tikzpicture}[declare function={r=3cm;gm=120;gh=180-2*(180-gm);}, font=\scriptsize\bfseries\sffamily]
			\path (0,0) coordinate (P)
			 (0:r) coordinate (Q)
			 (gm:r) coordinate (M)
			 ($(M)!1!-gh:(P)$) coordinate (N)
			;
			\foreach \x/\y in {M/N,M/P,P/Q}{
				\path[draw=cyan] (\x)--(\y) node[pos=0.5,midway,sloped]{\tikz{
				\draw (0pt,0pt)--(0pt,4pt) (1.5pt,0pt)--(1.5pt,4pt);
			}};
			}
			
			\draw (N)--(M)--(Q)--cycle (M)--(P);
			\foreach \d/\g in {P/-90,Q/-45,M/90,N/-165}{
			 \path[draw=black,fill=red] (\d) circle (1.3pt) node [shift={(\g:7pt)}]{\d};
			}
		\end{tikzpicture}
		}
	\loigiai{}
\end{ex}
%%%============EX_10==============%%%
\begin{ex}[NB]
\immini{%
	Cho ba điểm $A, B, C$ thẳng hàng và $B$ nằm giữa $A$ và $C$.Trên đường thẳng vuông góc với $AC$ tại $B$ ta lấy điểm $H$. Khi đó:
	\choice
	{$AH<BH$}
	{$AH<AB$}
	{\True$AH>BH$}
	{$AH=BH$}
	}{%
	 \begin{tikzpicture}[declare function={r=2;}]
	  \path(0,0) coordinate (A)
	  (r,0) coordinate (B)
	  ({2.5*r},0) coordinate (C)
	  ($(B)!1!90:(C)$)  coordinate (H)
	  ;
	  \path pic[draw,angle radius=5pt,fill=cyan!15]{right angle= H--B--A};
	  \draw (A)--(B)--(C)--(H)--cycle (H)--(B);
	  \foreach \d/\g in {A/-175,B/-90,C/-15,H/90}{
	  	\path[draw=black,fill=red] (\d) circle (1.3pt) node [shift={(\g:7pt)}]{\d};}
	 \end{tikzpicture}
	}
	\loigiai{}
\end{ex}
%%%============EX_11==============%%%
\begin{ex}[NB]
Cho tam giác $EHK$ có: $EH< EK, EF\bot HK$ tại $F$. Chọn câu đúng:
	\choice
	{$FH=FK$}
	{$FH> FK$}
	{\True$FH< FK$}
	{$FH\ge FK$}
\loigiai{}
\end{ex}
%%%============EX_12==============%%%
\begin{ex}[NB]
Các đường cao của tam giác ABC cắt nhau tại H thì:
	\choice
	{Điểm H là trọng tâm của tam giác ABC}
	{Điểm H cách đều ba cạnh của tam giác ABC}
	{Điểm H cách đều ba đỉnh của tam giác ABC}
	{\True Điểm H là trực tâm của tam giác ABC}
	\loigiai{}
\end{ex}
\Closesolutionfile{ansex}
\Closesolutionfile{ans}

%%%==========Phần trắc nghiệm đúng sai============%%%

%\tieumuc{Bài Tập Trắc Nghiệm Đúng Sai}-- \textit{Trong mỗi câu có 4 ý tương ứng A, B, C, D; Học sinh chọn đúng hoặc sai.}
%\Opensolutionfile{ans}[Ans/DATAM1]
%\Opensolutionfile{ansbook}[Ans/DATNTF-1]
%\luulgEXTF
%\Opensolutionfile{ansex}[LOIGIAITN/LGTNTF-1]
%
%
%\Closesolutionfile{ansex}
%\Closesolutionfile{ansbook}
%\Closesolutionfile{ans}	

%%%==========Phần tự luận============%%%
\tieumuc{Tự Luận (7 điểm)}
\Opensolutionfile{ansbt}[LOIGIAITL/LGTL-7CKII-De-03]
%\luuloigiaibt
\taoNdongke[10]{bt}
%%%==============BT_1==============%%%
\begin{bt}[2,0 điểm]
	Cho hai đa thức: $P(x)=2 x^2+5 x-1$ và $Q(x)=2 x^2-5 x-7$
	\begin{enumerate}
		\item Tính $P(x)+Q(x)$
		\item Tính $P(x)-Q(x)$
	\end{enumerate}
	\loigiai{}
\end{bt}
\taoNdongke[10]{bt}
%%%==============BT_2==============%%%
\begin{bt}[1,5 điểm]
	Tính
	\begin{enumerate}
		\item $(x+3)(x-1)$
		\item $\left(3x^3-2x^2\right):(3x^2)$
	\end{enumerate}
	\loigiai{}
\end{bt}
\taoNdongke[20]{bt}
%%%==============BT_3==============%%%
\begin{bt}[2,5 điểm]
	Cho tam giác $ABC$ cân tại $A$, hai đường cao $BD$ và $CE$ cắt nhau tại $H$.
	\begin{enumerate}
		\item Chứng minh $\triangle BCD=\triangle CBE$
		\item Chứng minh tam giác $BHC$ cân.
		\item Chứng minh tia $AH$ là tia phân giác của góc $BAC$.
	\end{enumerate}
	\loigiai{}
\end{bt}
\taoNdongke[10]{bt}
%%%==============BT_4==============%%%
\begin{bt}[1.0 điểm]Có một mảnh gỗ hình tròn cần đục một lỗ ở tâm, làm thế nào để xác định được tâm của mảnh gồ đó.
\loigiai{}
\end{bt}
\Closesolutionfile{ansbt}

\fileend


\begin{center}
	\rule[4pt]{2cm}{1pt}\large \textbf{HẾT}\rule[4pt]{2cm}{1pt}
\end{center}











