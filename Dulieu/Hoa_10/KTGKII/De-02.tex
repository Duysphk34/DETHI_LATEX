\def\x{110}
\setcounter{bt}{0}
\setcounter{ex}{0}
\setcounter{tieumuc}{0}
%%%Tùy chọn 1: Kì thi
%%%Tùy chọn 2: Môn
%%%Tùy chọn 3: lớp
%%%Tùy chọn 4: Sở/Phòng
%%%Tùy chọn 5: Ngày thi
\begin{tcolorbox}
	\begin{name}[Kiểm tra giữa kì II][Hóa học][10][Sở Giáo dục và Đào tạo]{Trường THCS }{2023 - 2024}
	\end{name}
\end{tcolorbox}
%%%==========Phần trắc nghiệm 1 phương án============%%%
\tieumuc{Bài Tập Trắc Nghiệm}-- \textit{Mỗi câu chỉ chọn một phương án.}
\Opensolutionfile{ans}[Ans/DATN-2]
\luuloigiaiex
\Opensolutionfile{ansex}[LOIGIAITN/LGTN-2]
%%%=========ex_1=========%%%
\begin{ex}Số oxi hóa của nitơ trong $NH_{3}$ là
	\choice
	{+3}
	{+5}
	{-4}
	{\True -3}
	\loigiai{
		Trong $NH_{3}$, nitơ có số oxi hóa là -3.
	}
\end{ex}

\Closesolutionfile{ansex}
\Closesolutionfile{ans}

%%%==========Phần trắc nghiệm đúng sai============%%%
\tieumuc{Bài Tập Trắc Nghiệm Đúng Sai}-- \textit{Trong mỗi câu có 4 ý tương ứng A, B, C, D; Học sinh chọn đúng hoặc sai.}
\Opensolutionfile{ans}[Ans/DATAM1]
\Opensolutionfile{ansbook}[Ans/DATNTF-2]
\luulgEXTF
\Opensolutionfile{ansex}[LOIGIAITN/LGTNTF-2]
\begin{ex} Đánh giá tính đúng sai của các phát biểu sau về phản ứng oxi hóa - khử và số oxi hóa:
	\choiceTF
	{\True Chất oxi hóa là chất nhận electron và chất khử là chất nhường electron.}
	{\True Số oxi hóa của một nguyên tố có thể tăng hoặc giảm trong phản ứng oxi hóa - khử, phản ánh sự chuyển giao electron.}
	{\True Một chất có thể vừa là chất oxi hóa vừa là chất khử trong cùng một phản ứng, được biết đến là phản ứng tự oxi hóa - khử.}
	{Trong mọi phản ứng hóa học, luôn có sự thay đổi số oxi hóa của ít nhất một nguyên tố tham gia phản ứng.}
	\loigiai{}
\end{ex}
\Closesolutionfile{ansex}
\Closesolutionfile{ansbook}
\Closesolutionfile{ans}	

%%%==========Phần tự luận============%%%
\tieumuc{Bài tập tự luận}
\Opensolutionfile{ansbt}[LOIGIAITL/LGTL-2]
\luuloigiaibt

\Closesolutionfile{ansbt}

\label{\x}