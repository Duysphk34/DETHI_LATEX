\section{Đề ôn thi giữa kỳ 2 toán 11}
\subsection{Phần trắc nghiệm}
\textbf{Câu trắc nghiệm nhiều phương án lựa chọn. Học sinh trả lời từ
câu 1 đến câu 12. Mỗi câu hỏi học sinh \textit{chỉ chọn một} phương án.}

\Opensolutionfile{ans}[Ans/Ans-1-Deso9-KNTT]
\hienthiloigiaiex
%%%=============EX_1=============%%%
\begin{ex}%[1D6N1-2]%[Dự án đề kiểm tra Toán 11 GHKII NH23-24- Bùi Thanh Cương]
	Cho $a$ là số thực dương khác $1$. Khi đó $\sqrt[4]{a^{\tfrac{2}{3}}}$ bằng
	\choice
	{$\sqrt[3]{a^2}$}
	{$a^{\tfrac{8}{3}}$}
	{$a^{\tfrac{3}{8}}$}
	{\True $\sqrt[6]{a}$}
	\loigiai{
		Ta có $\sqrt[4]{a^{\tfrac{2}{3}}}=\left(a^{\tfrac{2}{3}}\right)^{\tfrac{1}{4}}=a^{\tfrac{2}{3}\cdot \tfrac{1}{4}}=a^{\tfrac{1}{6}}=\sqrt[6]{a}$.
	}
\end{ex}

\begin{ex}%[1D6N1-4]%[Dự án đề kiểm tra Toán 11 GHKII NH23-24- Bùi Thanh Cương]
	Cho $a>1$ . Mệnh đề nào sau đây là đúng?
	\choice
	{\True $a^{-\sqrt{3}}>\dfrac{1}{a^{\sqrt{5}}}$}
	{$a^{\dfrac{1}{3}}>\sqrt{a}$}
	{$\dfrac{\sqrt[3]{a^2}}{a}>1$}
	{$\dfrac{1}{a^{2016}}<\dfrac{1}{a^{2017}}$}
	\loigiai{
		Vì $a>1;-\sqrt{3}>-\sqrt{5} \Rightarrow a^{-\sqrt{3}}>a^{-\sqrt{5}}\Rightarrow  a^{-\sqrt{3}}>\dfrac{1}{a^{\sqrt{5}}}$.\\
	Vậy khi $a>1$ thì $a^{-\sqrt{3}}>\dfrac{1}{a^{\sqrt{5}}}$.
}
\end{ex}

\begin{ex}%[1D6N2-1]%[Dự án đề kiểm tra Toán 11 GHKII NH23-24- Bùi Thanh Cương]
	Cho $a$ là số thực dương $a\ne 1$ và $P=\log_{\sqrt[3]{a}}a^3$. Mệnh đề nào sau đây đúng?
	\choice
	{$P=\dfrac{1}{3}$}
	{$P=3$}
	{$P=1$}
	{\True $P=9$}
	\loigiai{
		$\log_{\sqrt[3]{a}}a^3=\log_{a^{\frac{1}{3}}}a^3=3\cdot 3\log_{a}a=9$.}
\end{ex}

\begin{ex}%[1D6H2-2]%[Dự án đề kiểm tra Toán 11 GHKII NH23-24- Bùi Thanh Cương]
	Cho $a$ và $b$ là hai số thực dương thỏa mãn $a^4b=16$. Giá trị của $4\log_2a+\log_2b$ bằng
	\choice
	{\True $4$}
	{$2$}
	{$16$}
	{$8$}
	\loigiai{
		Ta có $4\log_2a+\log_2b=\log_2a^4+\log_2b=\log_2\left(a^4b\right)=\log_216=\log_22^4=4$.}
\end{ex}

\begin{ex}%[1D6N3-2]%[Dự án đề kiểm tra Toán 11 GHKII NH23-24- Bùi Thanh Cương]
	Tập xác định của hàm số $y=\log_3\left(x-4\right)$ là.
	\choice
	{$\left(-\infty ;4\right)$}
	{\True $\left(4;+\infty\right)$}
	{$\left(5;+\infty\right)$}
	{$\left(-\infty ;+\infty\right)$}
	\loigiai{
		Điều kiện $x-4>0\Leftrightarrow x>4$.\\
		Vậy tập xác định của hàm số $y=\log_3\left(x-4\right)$ là $\mathscr{D}=\left(4;+\infty\right)$.}
\end{ex}

\begin{ex}%[1D6H3-3]%[Dự án đề kiểm tra Toán 11 GHKII NH23-24- Bùi Thanh Cương]
	Trong các mệnh đề sau, mệnh đề nào đúng?
	\choice
	{Đồ thị của hàm số $y=2^x$ và $y=\log_2x$ đối xứng với nhau qua đường thẳng $y=-x$}
	{\True Đồ thị của hai hàm số $y=\mathrm{e}^x$ và $y=\ln x$ đối xứng với nhau qua đường thẳng $y=x$}
	{Đồ thị của hai hàm số $y=2^x$ và hàm số $y=\dfrac{1}{2^x}$ đối xứng với nhau qua trục hoành}
	{Đồ thị của hai hàm số $y=\log_2x$ và $y=\log_2\dfrac{1}{x}$ đối xứng với nhau qua trục tung}
	\loigiai{
		Đồ thị hàm số $y=a^x$ và đồ thị hàm số $y=\log_ax$ đối xứng với nhau qua đường phân giác góc
		phần tư thứ nhất ($y=x$).\\
	Suy ra đồ thị của hai hàm số $y=\mathrm{e}^x$ và $y=\ln x$ đối xứng với nhau qua đường thẳng $y=x$.}
\end{ex}
\begin{ex}%[1D6H4-2]%[Dự án đề kiểm tra Toán 11 GHKII NH23-24- Bùi Thanh Cương]
	Nghiệm của phương trình $\log_2\left(x-2\right)=3$ là
	\choice
	{$x=6$}
	{$x=8$}
	{$x=11$}
	{\True $x=10$}
	\loigiai{
		Điều kiện: $x-2>0\Leftrightarrow x>2$.\\
		Phương trình $\log_2\left(x-2\right)=3\Leftrightarrow x-2=8\Leftrightarrow x=10$ (thỏa).\\
		Vậy phương trình có nghiệm $x=10$.}
\end{ex}

\begin{ex}%[1D6H4-3]%[Dự án đề kiểm tra Toán 11 GHKII NH23-24- Bùi Thanh Cương]
	Tập nghiệm của bất phương trình $3^{x^2-23}<9$ là
	\choice
	{\True $(-5 ; 5)$}
	{$(-\infty ; 5)$}
	{$(5 ;+\infty)$}
	{$(0 ; 5)$}
	\loigiai{
		Ta có $3^{x^2-23}<9 \Leftrightarrow x^2-23<2 \Leftrightarrow x^2<25 \Leftrightarrow-5<x<5$.\\
		Vậy tập nghiệm của bất phương trình  là $(-5 ; 5)$.}
\end{ex}

\begin{ex}%[1H8N1-2]%[Dự án đề kiểm tra Toán 11 GHKII NH23-24- Bùi Thanh Cương]
	Cho hình lập phương $ABCD.A'B'C'D'$. Đường thẳng nào sau đây vuông góc với đường thẳng $BC'$ ?
	\choice
	{\True $A'D$}
	{$AC$}
	{$BB'$}
	{$AD'$}
	\loigiai{
			\immini
			{
				Ta có: $\heva{&B'C \parallel   A'D\\ & BC'\perp B'C}  \Rightarrow BC'\perp A'D$.
			}
			{
			\begin{tikzpicture}[line join=round, line cap=round,thick,scale=0.7]
				\def\h{3}
				\path
				(0,0) coordinate (A)
				(3.5,0) coordinate (D)
				(-1.5,-1.5) coordinate (B)
				;
				\path  ($(B)+(D)-(A)$) coordinate (C);
				%\path (A)--(C) node[pos=.5,scale=.5,red]{//}
				\foreach \diem/\anh in {A/A', B/B',C/C',D/D'} \coordinate (\anh) at ($(0,\h)+(\diem)-(A)$);
				\path (intersection of A'--C and A--C') coordinate (O);
				\draw (A')--(B')--(C')--(D')--cycle;
				\draw (B)--(C)--(D);
				\draw (B)--(B') (C)--(C') (D)--(D') (B')--(C) (B)--(C');
				%\draw (A')--(C');
				\draw[dashed] (A)--(B) (A)--(D) (A)--(A') (A')--(D);
				%\path (A')--(C) node[pos=.5,right]{$a\sqrt{3}$};
				\foreach \i/\j in {A/140,B/-160,C/-20,D/-20,A'/140,B'/-160,C'/-20,D'/40}\fill  (\i) circle(1pt) ($(\i) + (\j:5mm)$)node{$\i$};
				% \path (B)--(C) node[midway,yshift=10pt]{$a\sqrt{2}$};
				% \path (B')--(A) node[left,midway,xshift=1pt]{$3a$};
				%\draw pic [draw, angle radius = 6 mm] {angle = O--A--S} node[shift={(20:4mm)}]{$60^\circ$};
				%\draw pic[draw, angle radius=2mm]{right angle=S--O--A};
			\end{tikzpicture}	
			}
			}
\end{ex}

\begin{ex}%[1H8H2-3]%[Dự án đề kiểm tra Toán 11 GHKII NH23-24- Bùi Thanh Cương]
	Cho hình lập phương $ABCD.A'B'C'D'$ cạnh $a$. Gọi $M$ là trung điểm của $CD$ và $N$ là trung điểm của $A'D'$. Góc giữa hai đường thẳng $B'M$ và $C'N$ bằng
	\choice
	{$30^\circ $}
	{$45^\circ $}
	{$60^\circ $}
	{\True $90^\circ $}
	\loigiai{
		\immini
		{	Gọi $I$ là trung điểm của $C'D'$ khi đó $IB'$ là hình chiếu vuông góc của $B'M$ trên $\left(A'B'C'D'\right)$.\\
			Hai tam giác vuông $\triangle C'ID$ và $\triangle D'NC'$ bằng nhau.\\
			 Suy ra $\widehat{IB'C'}+\widehat{NC'B'}=\widehat{NC'D'}+\widehat{NC'B'}=90^\circ  \Rightarrow C'N\perp IB'$.\\
			Do đó $C'N \perp B'M$. \\
			Vậy góc giữa $B'M$ và $C'N$ bằng $90^\circ $.
		}
		{
			\begin{tikzpicture}[line join=round, line cap=round,thick,scale=0.75]
				\def\h{-3}
				\path
				(0,0) coordinate (A)
				(3.5,0) coordinate (B)
				(-1.5,-1.5) coordinate (D);
				\path  ($(B)+(D)-(A)$) coordinate (C);
				%\path (A)--(C) node[pos=.5,scale=.5,red]{//}
				\foreach \diem/\anh in {A/A', B/B',C/C',D/D'} \coordinate (\anh) at ($(0,\h)+(\diem)-(A)$);
				\path 
				(barycentric cs:D=1,C=1)coordinate(M)
				(barycentric cs:D'=1,C'=1)coordinate(I)
				(barycentric cs:A'=1,D'=1)coordinate(N)
				;
				\draw (A)--(B)--(C)--(D)--cycle;
				\draw (B')--(C')--(D');
				\draw (B)--(B') (C)--(C') (D)--(D') (M)--(I);
				%\draw (A')--(C');
				\draw[dashed] (A')--(B') (A')--(D') (A)--(A') (C')--(N) (B')--(I) (B')--(M);
				%\path (A')--(C) node[pos=.5,right]{$a\sqrt{3}$};
				\foreach \i/\j in {A/140,B/-20,C/-20,D/-160,A'/140,B'/40,C'/-120,D'/-160,M/90,N/160,I/220}\fill  (\i) circle(1pt) ($(\i) + (\j:5mm)$)node{$\i$};
			\end{tikzpicture}
		}
		}
\end{ex}

\begin{ex}%[1H8H2-2]%[Dự án đề kiểm tra Toán 11 GHKII NH23-24- Bùi Thanh Cương]
	Cho hình chóp $S.ABCD$ có đáy là hình vuông, cạnh bên $SA$ vuông góc với đáy $(ABCD)$.\\
	Khẳng định nào sau đây sai?
	\choice
	{\True $CD\perp (SBC)$}
	{$SA\perp (ABC)$}
	{$BC\perp (SAB)$}
	{$BD\perp (SAC)$}
	\loigiai{
		\immini{
			Từ giả thiết, ta có : $SA\perp (ABCD)\Rightarrow SA\perp BC,\, SA\perp CD$.\\
			Suy ra các mệnh đề đúng: 
			\begin{itemize}
				\item $(ABC)\equiv (ABCD)\Rightarrow SA\perp (ABC)$.
				\item $\heva{& BC\perp AB\\ & BC\perp SA } \Rightarrow BC\perp (SAB)$.
			\item $\heva{& BD\perp AC\\ & BD\perp SA } \Rightarrow BD\perp (SAC)$.
			\end{itemize}
			Ta có $\heva{
				& CD\perp AD\\ 
				& CD\perp SA} \Rightarrow CD\perp (SAD)$.\\
			Mà $(SCD)$ và $(SAD)$ không song song nên $CD\perp (SCD)$ là mệnh đề sai.}
{
		\begin{tikzpicture}[line join=round, line cap=round,thick,scale=0.7]
			\def\h{4}
			\path
			(0,0) coordinate (A)
			(-2,-2) coordinate (B)
			(5,0) coordinate (D)
			($(B)+(D)-(A)$) coordinate (C)
			(intersection of A--C and B--D) coordinate (O)
			;
			\path ($(0,\h)+(A)$) coordinate (S)
			%(barycentric cs:B=1,S=1.3)coordinate(H)
			%(barycentric cs:S=1.3,D=1)coordinate(K)
			% (barycentric cs:A=1,B=1)coordinate(E)
			% (barycentric cs:O=1,N=1)coordinate(F)
			%(intersection of S--O and H--K) coordinate (J)
			%(intersection of S--C and A--J) coordinate (I)
			;
			\draw[dashed] (A)--(D) (A)--(B) (S)--(A) (A)--(C) (B)--(D);
			\draw (B)--(C) --(D);
			\draw (S)--(B) (S)--(C) (S)--(D);
			\foreach \i/\j in {A/180,B/-120,C/-45,D/0,S/90}
			\draw[fill=white] (\i) circle(1pt) ($(\i) + (\j:4mm)$)node{$\i$};
			\draw pic[draw, angle radius=2mm]{right angle=S--A--D};
			%\draw pic [draw," ", angle eccentricity=1.2,angle radius =8 mm] {angle = S--C--A};
			%\draw pic[draw, angle radius=2mm]{right angle=A--K--D};
		\end{tikzpicture}	
		}
			}
\end{ex}
	
\begin{ex}%[1H8H2-3]%[Dự án đề kiểm tra Toán 11 GHKII NH23-24- Bùi Thanh Cương]
		Cho tứ diện đều $ABCD$ có $M$, $N$ lần lượt là trung điểm của các cạnh $AB$ và $CD$. Mệnh đề nào sau đây sai?
		\choice
		{$MN\perp AB$}
		{\True $MN\perp BD$}
		{$MN\perp CD$}
		{$AB\perp CD$}
	\loigiai{
		\immini
		{Từ giả thiết ta có các mệnh đề đúng:
			\begin{itemize}
				\item $\Delta NAB$ cân tại $N$ nên $MN\perp AB$ .
				\item $\Delta MCD$ cân tại $M$ nên $MN\perp CD$ .
				\item $CD\perp\left(ABN\right)$ $\Rightarrow CD\perp AB$.
			\end{itemize}
			Giả sử $MN\perp BD$
			mà $MN\perp AB$.\\
			 Suy ra $MN\perp\left(ABD\right)$ (Vô lí vì $ABCD$ là tứ diện đều)\\
			Vậy mệnh đề sai $MN\perp BD$.
		}
		{
		\begin{tikzpicture}[line join=round, line cap=round,thick,scale=0.8]
			\def\h{4}
			\path
			(0,0) coordinate (B)
			(5,0) coordinate (D)
			(1.5,-1.5) coordinate (C)
			(barycentric cs:B=1,D=1,C=1)coordinate(H);
			;
			\path ($(0,\h)+(H)$) coordinate (A);
			\path 
			(barycentric cs:A=1,B=1)coordinate(M)
			(barycentric cs:C=1,D=1)coordinate(N);
			;
			\draw[dashed] (B)--(D) (B)--(N) (M)--(D) (M)--(N);
			\draw (A) -- (B)-- (C)--(D) --cycle (A)--(C) (A)--(N) (C)--(M);
			\foreach \i/\j in {B/160,C/-90,D/-45,A/90,M/180,N/-90}\fill  (\i) circle(1pt) ($(\i) + (\j:3mm)$)node{$\i$};
			%\draw pic [draw,"$30^\circ$", angle eccentricity=1.5,angle radius =6 mm] {angle = S--B--A};
			%\draw pic[draw, angle radius=2mm]{right angle=A--I--B};
			%\draw pic[draw, angle radius=2mm]{right angle=C--I--D};
			%\path (A)--(B) node[midway]{$\setminus$} (A)--(C) node[midway]{$\setminus$};
			%\path (D)--(B) node[midway]{$//$} (D)--(C) node[midway]{$//$};
			% (A)--(O)node[below,midway,yshift=2.5pt]{$ r $}(S)--(A) node[left,midway]{$\ell$}
		\end{tikzpicture}
		}
	}
\end{ex}
\Closesolutionfile{ans}
%\bangdapan{Ans-1-Deso9-KNTT}



\subsection{Câu trắc nghiệm đúng sai}
Học sinh trả lời từ câu 1 đến câu 4.
Trong mỗi ý \circlenum{A}, \circlenum{B}, \circlenum{C} và \circlenum{D} ở mỗi câu, học sinh chọn đúng hoặc sai.
\setcounter{ex}{0}
\LGexTF
\Opensolutionfile{ansbook}[ansbook/DapanDS]
\Opensolutionfile{ans}[Ans/DapanT]
%%%============EX_1==============%%%
\begin{ex}%[1D6H3-2]%[Dự án đề kiểm tra Toán khối 11-GHK2-NH23-24-Đợt 1-Nguyễn Ngọc Dũng]%[Deso9-KNTT]
Các phát biểu sau đúng hay sai về điều kiện của $x$ để biểu thức có nghĩa
\choiceTF
{\True $\log (x+1)$ có nghĩa khi và chỉ khi $x>-1$}
{\True $\ln (x-1)^2$ có nghĩa khi và chỉ khi $x \neq 1$}
{\True $\log _{x-1} x$ có nghĩa khi và chỉ khi $\heva {x>1 \\ x \neq 2}$}
{\True $\log ^2 \dfrac{1}{x-x^2}$ có nghĩa khi và chi khi $0<x<1$}
\loigiai{
\begin{itemize}
\item $\log (x+1)$ có nghĩa khi và chỉ khi $x+1>0 \Leftrightarrow x > -1.$
\item $\ln (x-1)^2$ có nghĩa khi và chỉ khi $(x-1)^2 > 0 \Leftrightarrow x \neq  1.$
\item $\log _{x-1} x$ có nghĩa khi và chỉ khi $\heva{&x-1>0 \\ &x-1\neq 1 \\ & x >0 } \Leftrightarrow \heva{&x>1 \\ &x \neq 2 \\ &x>0} \Leftrightarrow \heva {x>1 \\ x \neq 2}.$
\item $\log ^2 \dfrac{1}{x-x^2}$ có nghĩa khi và chi khi $x-x^2 > 0 \Leftrightarrow 0<x<1.$
\end{itemize}
}
\end{ex} 

%%%============EX_2==============%%%
\begin{ex}%[1D6H4-2]%[Dự án đề kiểm tra Toán khối 11-GHK2-NH23-24-Đợt 1-Nguyễn Ngọc Dũng]%[Deso9-KNTT]
Cho bất phương trình $\log_{\frac{1}{10}} \left(x^2-5x+7\right) \ge 0$ có tập nghiệm là $S=[a;b]$. Khi đó:
\choiceTF
{\True Điều kiện xác định $x\in \mathbb{R}$}
{\True Bất phương trình có chung tập nghiệm với  $x^2-5x+6 \le 0$}
{$a$; $b$; $5$ là một cấp số cộng}
{\True $[a;b] \cup (2;9)=[2;9)$ } 
\loigiai{
\begin{itemize}
\item Điều kiện xác định $x^2-5x+7>0  $ (luôn đúng). \\
Tập xác định $\mathscr{D}=\mathbb{R}$.
\item Ta có  $\log_{\frac{1}{10}} \left(x^2-5x+7\right) \ge 0 \Leftrightarrow x^2-5x+7 \le 1 \Leftrightarrow x^2-5x+6 \le 0 \Leftrightarrow 2 \le x \le 3$.\\
Do đó tập nghiệm của bất phương trình đã cho là $S=[2;3]$.\\
Vậy bất phương trình có chung tập nghiệm với  $x^2-5x+6 \le 0$.
\item Vì tập nghiệm của bất phương trình đã cho là $S=[2;3]$ nên $a=2$; $b=3$. \\
Vậy $a$; $b$; $5$ không phải là một cấp số cộng.
\item Ta có $[a;b] \cup (2;9)=[2;3] \cup (2;9)=[2;9)$.
\end{itemize}
}
\end{ex}

%%%============EX_3==============%%%

\begin{ex}%[1H8H1-3]%[Dự án đề kiểm tra Toán khối 11-GHK2-NH23-24-Đợt 1-Nguyễn Ngọc Dũng]%[Deso9-KNTT]
Cho hình hộp $ABCD.A'B'C'D'$ có $6$ mặt là hình vuông cạnh $a$. Khi đó
\choiceTF
{\True $BC' \parallel AD'$}
{\True $\left( AD',B'C\right)=90^\circ$}
{\True $\left( AD',DC'\right)=\left( BC',DC'\right)$}
{$\widehat{BC'D}=90^\circ$} 
\loigiai{
\begin{center}
\begin{tikzpicture}[line cap=round,line join=round,font=\footnotesize,>=stealth,scale=.7]
\coordinate[label=left:$A$] (A) at (0,0);
\coordinate[label=below:$D$] (D) at(1.5,-1.5);
\coordinate[label=right:$B$] (B) at(5,0);
\coordinate[label=below:$C$] (C) at($(B)+(D)-(A)$); 
\coordinate[label=above:$A'$] (A') at	($(A)+(0,5)$);
\coordinate[label=left:$D'$] (D') at ($(D)+(A')-(A)$);
\coordinate[label=above:$B'$] (B') at ($(B)+(A')-(A)$);
\coordinate[label=right:$C'$] (C') at ($(D')+(B')-(A')$);
\draw (A')--(D')--(C')--(B')--cycle (D)--(C) (D')--(C')--(C) (D) --(A)--(A') (D')--(D) (D)--(C') (A)--(D') ;
\draw[dashed] (A)--(B)--(B') (B)--(C) (B)--(C') (B')--(C) (D)--(B);
\end{tikzpicture}
\end{center}
\begin{itemize}
\item Ta có $ABC'D'$ là hình bình hành nên $BC' \parallel AD'$.
\item Vì $BC' \parallel AD'$ nên $\left( AD',B'C \right)=\left( BC',B'C\right)=90^\circ$ ($BCC'B'$ là hình vuông).
\item Vì $BC' \parallel AD'$ nên $\left( AD',DC'\right)=\left( BC',DC'\right)$.
\item $\widehat{BC'D}=60^\circ$ ($\triangle BC'D$ đều).
\end{itemize}
}
\end{ex}

%%%============EX_4==============%%%
\begin{ex}%[1H8V2-3]%[Dự án đề kiểm tra Toán khối 11-GHK2-NH23-24-Đợt 1-Nguyễn Ngọc Dũng]%[Deso9-KNTT]
Cho hình chóp $S.ABCD$ có đáy $ABCD$ là hình vuông có tâm $O$. Cạnh bên $SA$ vuông góc với đáy $ABCD$, $H$ là hình chiếu vuông góc của $A$ trên $SO$. Khi đó:
\choiceTF
{\True $BD\perp(SAC)$}
{\True $BD\perp SC$}
{\True $CD\perp (SAD)$}
{\True $AH\perp SB$}
\loigiai
{
\begin{center}
\begin{tikzpicture}[scale=1, font=\footnotesize, line join=round, line cap=round, >=stealth]
\path
(0,0) coordinate (A)node[left]{$A$}
(4,0) coordinate (D)node[right]{$D$}
(-1.5,-1.5) coordinate (B)node[below left]{$B$}
($(B)+(D)-(A)$) coordinate (C) node[below right]{$C$}
(0,3) coordinate (S) node[above]{$S$}
($(B)!0.5!(D)$) coordinate (O)node[below]{$O$}
($(S)!0.6!(O)$) coordinate (H) node[left]{$H$}
;
\draw 
(S)--(B)--(C)--(D)--cycle
(S)--(C)
;
\draw[dashed]
(S)--(A)--(B)
(A)--(D)
(A)--(C)
(B)--(D)
(S)--(O)
(A)--(H)
;
\end{tikzpicture}
\end{center}
\begin{enumerate}
\item Ta thấy rằng
\begin{itemize}
\item $BD\perp AC$ do đáy $ABCD$ là hình vuông.
\item $SA\perp BD$ do $SA\perp(ABCD)$.
\end{itemize}
Mà $AC$ và $SA$ cắt nhau và nằm trong mặt phẳng $(SAC)$ do đó $BD\perp (SAC)$.
\item Vì $BD\perp (SAC)$ nên $BD\perp SC$.
\item Ta thấy rằng
\begin{itemize}
\item $CD\perp AD$ do đáy $ABCD$ là hình vuông.
\item $SA\perp CD$ do $SA\perp(ABCD)$.
\end{itemize}
Mà $AD$ và $CD$ cắt nhau và nằm trong mặt phẳng $(SAD)$ do đó $CD\perp (SAD)$.
\item Ta có
\begin{itemize}
\item $AH\perp SO$ do giả thiết.
\item $AH\perp BD$ do $BD\perp(SAC)$ mà $AH\subset (SAC)$.
\end{itemize}
Mà $SO$ và $BD$ cắt nhau và nằm trong mặt phẳng $(SBD)$ do đó $AH\perp (SBD)$.\\
Từ đây suy ra $AH\perp SB$.
\end{enumerate}
}
\end{ex}

\Closesolutionfile{ans}
\Closesolutionfile{ansbook}

\begin{center}
\textbf{\textsf{BẢNG ĐÁP ÁN ĐÚNG SAI}}
\end{center}
\input{Ansbook/DapanDS}



\subsection{Phần tự luận}

\hienthiloigiaibt

%%%=============BT_1=============%%%
\begin{bt}%[1D6H3-5]
	Công ty FTK về mua bán xe ô tô đã qua sử dụng, sau khi khảo sát thị trường $6$ tháng đã đưa ra công thức chung về giá trị còn lại của ô tô $4$ chỗ kể từ khi đưa vào sử dụng (các loại xe $4$ chỗ không sử dụng mục đích kinh doanh) được tính $P(t)=A \cdot\left(\dfrac{3}{4}\right)^{\tfrac{t}{4}}$. Trong đó $A$ là giá tiền ban đầu mua xe, $t$ là số năm kể từ khi đưa vào sử dụng. Tính giá trị còn lại của xe ô tô sau $30$ tháng đưa vào sử dụng. Biết giá trị mua xe ban đầu là $920$ triệu.
	\loigiai{
		Ta có $A=920$ triệu; $t=2{,}5$ năm.\\
		Vậy giá trị còn lại của xe ô tô sau 30 tháng đưa vào sử dụng là
		$$
		P(2{,}5)=920 \cdot\left(\dfrac{3}{4}\right)^{\tfrac{2,5}{4}}=768\, 601 \, 304 \, \text { (triệu đồng). }
		$$
	}
\end{bt}
%%%=============BT_2=============%%%
\begin{bt}%[1D6V2-2]
	Cho $a=\log 2$, $b=\ln 2$. Hãy biểu diễn $\ln 800$ theo $a$ và $b$.
	\loigiai{
		
		Ta có \begin{eqnarray*}
			\ln 800&=&\ln \left(2^3 \cdot 10^2\right)\\
			&=&3 \ln 2+2 \ln 10\\
			&=&3 \ln 2+2 \ln 2 \cdot \log _2 10 \\
			& =&3\ln 2+\dfrac{2 \ln 2}{\log  2}\\
			&=&3 b+\dfrac{2 b}{a}.
		\end{eqnarray*} 
	}
\end{bt}
%%%=============BT_3=============%%%
\begin{bt}%[1D6V3-5]
	Nếu $D_0$ là chênh lệch nhiệt độ ban đầu giữa một vật $M$ và các vật xung quanh, và nếu các vật xung quanh có nhiệt độ $T_S$, thì nhiệt độ của vật $M$ tại thời điểm $t$ được mô hình hóa bởi hàm số
	$$T(t)=T_S+D_0 \cdot e^{-k t}\, \hfill{(1)}$$ (trong đó $k$ là hằng số dương phụ thuộc vào vật $M$). Một con gà tây nướng được lấy từ lò nướng khi nhiệt độ của nó đã đạt đến $195^{\circ}$ F và được đặt trên một bàn trong một căn phòng có nhiệt độ là $65^{\circ}$ F. Nếu nhiệt độ của gà tây là $150^{\circ}$ F sau nửa giờ, nhiệt độ của nó sau $60$ phút là bao nhiêu?
	\loigiai{
		Ta có $T_S=65^{\circ}$ và độ chênh lệch nhiệt độ là $D_0=195^{\circ}-65^{\circ}=130^{\circ}$.\\
		Sau nửa giờ $(t=0{,}5)$ thì nhiệt độ của gà là $T=150^{\circ}$.\\
		Áp dụng công thức $(1)$ ta được  $150=65+130 \cdot e^{-k(0,5)} \Leftrightarrow e^{-k}=\left(\dfrac{17}{26}\right)^2$.\\
		Vậy $T(t)=65+130 \cdot\left(\dfrac{17}{26}\right)^{2 t}$.\\
		Suy ra nhiệt độ của gà sau $ 60$ phút ($t=1$  giờ)  là  $65+130 \cdot \left(\dfrac{17}{26}\right)^{2{,}1} \approx 121^{\circ}$ F.}
\end{bt}
%%%=============BT_4=============%%%
\begin{bt}%[1D6H4-2]
	Giải phương trình sau  $2^{x^2-x+8}=4^{1-3 x}$
	\loigiai{
		\begin{eqnarray*}
			2^{x^2-x+8}=4^{1-3 x} &\Leftrightarrow& 2^{x^2-x+8}=2^{2(1-3 x)}\\ &\Leftrightarrow& x^2-x+8=2-6 x\\
			& \Leftrightarrow &x^2+5 x+6=0\\
			& \Leftrightarrow & \hoac{&x=-2\\&x=-3.}
		\end{eqnarray*}
		Vậy tập nghiệm của phương trình là $S=\{-3;2\}$.}
\end{bt}
%%%=============BT_5=============%%%
\begin{bt} %[1H8H1-3]
	Cho tứ diện đều $A B C D$. Gọi $M$ là trung điểm của cạnh $B C$. Tính côsin của góc giữa hai đường thẳng $A B$ và $D M$ bằng?
	\loigiai{
		\immini{	Kẻ $M N \parallel  A B$, có $M N$ là đường trung bình của $\triangle A B C$.\\
			Suy ra $M N=\dfrac{A B}{2}$.\\
			Do đó $(A B, D M)=(M N, D M)=\widehat{D M} N=\alpha$.\\
			Gọi tứ diện đều $A B C D$ có cạnh bằng $a$.\\
			Ta có $M N=\dfrac{a}{2}$; $D N=D M=\dfrac{a \sqrt{3}}{2}$. \\
			$\Rightarrow \cos \alpha=\dfrac{M N^2+D M^2-D N^2}{2 \cdot M N \cdot D M}=\dfrac{\sqrt{3}}{6}$.}
		{\begin{tikzpicture}[>=stealth,line join=round,line cap=round,font=\footnotesize,scale=0.75]
				\path 
				(0,0) coordinate (B)
				(2,-1.6) coordinate (C)
				(5,0) coordinate (D)
				($(B)!1/2!(C)$) coordinate (M)
				($(D)!2/3!(M)$) coordinate (O)
				($(O)+(0,4)$) coordinate (A)
				($(A)!1/2!(C)$) coordinate (N)
				
				
				;
				\draw[dashed](M)--(D)--(B);
				\draw (A)--(B)--(C)--(A)--(D)-- (C)--(N)--(D) (M)--(N);
				\foreach \x/\y in {A/90,B/180,D/10,C/-90,M/180,N/0}
				\draw[fill=black] (\x) circle (1.1pt) + (\y:0.5cm) node{$\x$};
				\draw pic[draw, angle radius=3mm, angle eccentricity=1.5]{ angle = D--M--N};
				\node at ($(M)+(24:0.75)$){\small $\alpha$};
		\end{tikzpicture}}
	}
\end{bt}
%%%=============BT_6=============%%%
\begin{bt} %[1H8V3-2]
	Cho hình chóp $S.ABCD$ có đáy là hình thang vuông tại $A$ và $D$, $AB=2AD=2CD=2a$. Biết  $S A \perp(A B C D)$, $S A=3 a$. Tính diện tích hình chiếu vuông góc của tam giác $S B C$ lên mặt phẳng $(S A B)$.
	\loigiai{
		\immini{Gọi $I$ là trung điểm $A B$.\\
			Dễ dàng chứng minh $A I C D$ là hình vuông.\\
			$\Rightarrow C I=\dfrac{1}{2} A B \Rightarrow \triangle A B C$ vuông tại $C$.\\
			Ta có $\heva{&B C \perp S A \\& B C \perp A C} \Rightarrow B C \perp\left(S A C\right).$\\
			Ta có $\heva{&C I \perp A B \\& C I \perp S A} \Rightarrow C I \perp \left(S A B\right).$\\
			Hình chiếu của $\triangle S B C$ trên mp$(S A B)$ là $\triangle S I B$.}
		{\begin{tikzpicture}[>=stealth,line join=round,line cap=round,font=\footnotesize,scale=0.75]
				\path 
				(0,0) coordinate (A)
				(-1.5,-2) coordinate (D)
				(5,0) coordinate (I)
				($(D)+(I)-(A)$) coordinate (C)
				($(I)!1!180:(A)$) coordinate (B)
				($(A)+(0,4)$) coordinate (S)
				;
				\draw[dashed](D)--(A)node[pos=0.5, left]{$a$};
				\draw[dashed](S)--(I)node[pos=0.5, right]{$2a$};
				\draw[dashed](S)--(A)node[pos=0.5, right]{$3a$};
				\draw(D)--(C)node[pos=0.5, below]{$a$};
				\draw[dashed](A)--(C)--(I);
				\draw (D)--(S)--(C)--(B)--(S);
				\draw[dashed] (B)edge node[midway, sloped, rotate=90, anchor=center] {$ = $}(I);
				\draw[dashed] (I)edge node[midway, sloped, rotate=90, anchor=center] {$ = $}(A);
				\foreach \x/\y in {A/180,B/0,D/-90,C/-90,S/90,I/70}
				\draw[fill=black] (\x) circle (1.1pt) + (\y:0.5cm) node{$\x$};
		\end{tikzpicture}}
		$$
		S_{\triangle S I B}=\dfrac{1}{2} S_{\triangle S A B}=\dfrac{1}{2} \cdot \dfrac{1}{2} \cdot S A \cdot A B=\dfrac{1}{4} \cdot 3 a \cdot 2 a=\dfrac{3}{2} a^2.
		$$	
	}
\end{bt}
