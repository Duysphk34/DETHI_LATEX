%\newpage
\def\x{220}

%%%Tùy chọn 1: Kì thi
%%%Tùy chọn 2: Môn
%%%Tùy chọn 3: lớp
%%%Tùy chọn 4: Sở/Phòng
%%%Tùy chọn 5: Ngày thi
\begin{name}[Kiểm tra giữa kì II][Toán][7][Sở Giáo dục và Đào tạo]{Trường THCS }{2023 - 2024}
\end{name}

%%%==========Phần trắc nghiệm 1 phương án============%%%
\tieumuc{Bài Tập Trắc Nghiệm}-- \textit{Mỗi câu chỉ chọn một phương án.}
\Opensolutionfile{ans}[Ans/DATN-De-01]
\luuloigiaiex
%\hienthiloigiaiex
\taoNdongke[6]{ex}
\Opensolutionfile{ansex}[LOIGIAITN/LGTN-De-01]
%%%============EX_1==============%%%
\begin{ex}
	Cho bảng sau:
	\begin{center}
		\begin{tabular}{|C{4cm}|*{5}{C{0.09\textwidth}|}}
			\hline
			\textbf{Năm} & $\mathbf{1979}$ & $\mathbf{1989}$ & $\mathbf{1999}$ & $\mathbf{2009}$ & $\mathbf{2019}$ \\
			\hline
			\makecell[c]{
					Dân số Việt Nam \\
					(triệu người)} & 53 & 67 & 79 & 87 & 96 \\
			\hline
			\makecell[c]{
				Dân số Thái Lan \\
				(triệu người)} & 46 & 56 & 62 & 67 & 70 \\
			\hline
		\end{tabular}
	\end{center}
	Theo các số liệu ở bảng trên, khẳng định nào sau đây là đúng:
	\choice
	{Năm 1989 dân số Việt Nam ít hơn dân số Thái Lan}
	{Năm 2009 dân số Việt Nam nhiều hơn dân số Thái Lan 20 triệu người}
	{Dân số Việt Nam luôn ít hơn dân số Thái Lan}
	{Từ 1979 đến 2019 dân số Thái Lan nhiều nhất là 96 triệu người}
	\loigiai{}
\end{ex}

%%%============EX_2==============%%%
\begin{ex}
	Xếp loại thi đua năm 2021-2022 của lớp 6A được thể hiện ở bảng sau
	\begin{center}
		\begin{tabular}{|c|*{4}{C{2cm}|}}
			\hline
			\textbf{Loại} & Giỏi & Khá & Đạt & Chưa đạt \\
			\hline
			\textbf{Số lượng} & 9 & 15 & 20 & 2 \\
			\hline
		\end{tabular}
	\end{center}
	Loại nào chiếm số lượng nhiều nhất?
	\choice
	{Giỏi}
	{Khá}
	{Đạt}
	{Chưa đạt}
	\loigiai{}
\end{ex}
%%%============EX_3==============%%%
\begin{ex}
	Cho bảng thống kê lượng mưa trung bình 6 tháng đầu năm ở Hà Tĩnh như sau:
	\begin{center}
		\begin{tabular}{|*{7}{c|}}
			\hline
			Tháng & 1 & 2 & 3 & 4 & 5 & 6 \\
			\hline
			\makecell[c]{Lượng mưa} & 36,5 & 22,6 & 16,5 & 18,7 & 12,7 & 13,1 \\
			\hline
		\end{tabular}
	\end{center}
	Ba tháng có lượng mưa ít nhất là:
	\choice
	{$3; 5; 6$}
	{$1; 3; 4$}
	{$2; 4; 6$}
	{$1; 5; 6$}
	\loigiai{}
\end{ex}
%%%============EX_4==============%%%
\begin{ex}
	\hinhphai{Cho biểu đồ hình quạt tròn biểu diễn kết quả thống kê chọn môn thể thao yêu thích nhất trong bốn môn: Bóng đá, bóng bàn, bóng chuyền, đá cầu của một lớp 7. (Mỗi bạn chỉ được chọn một môn yêu thích nhất). Môn có nhiều bạn yêu thích là:
		\choice
		{Bóng chuyền}
		{Đá cầu}
		{Bóng bàn}
		{Bóng đá}}{
		%%=========Hình 1=========%%
		\includegraphics[width=5cm]{Images/KTGK-TOAN7-DE01/cau-4}
		}
	\loigiai{}
\end{ex}
%%%============EX_5==============%%%
\begin{ex}
	Một hộp có 5 quả bóng gồm các màu: xanh, đỏ, vàng, hồng, tím. Lấy ngẫu nhiên một quả bóng trong hộp. Tập hợp $M$ gồm các kết quả có thể xảy ra đối với màu của quả bóng là:
	\choice
	{$M=\{5\}$}
	{$M=\{$ xanh, đỏ, vàng, hồng, tím $\}$}
	{$M=\{$x a n h, đỏ, hồng, tím $\}$}
	{$M=\{1; 2; 3; 4; 5\}$}
	\loigiai{}
\end{ex}
%%%============EX_6==============%%%
\begin{ex}
	Cho tam giác $ABC$ cân tại $A$, với $\widehat{B}=50^{\circ}$. Số đo góc $C$ là:
	\choice
	{$50^{\circ}$}
	{$60^{\circ}$}
	{$70^{\circ}$}
	{$80^{\circ}$}
	\loigiai{}
\end{ex}
%%%============EX_7==============%%%
\begin{ex}
	Cho tam giác nhọn $ABC$ biết $\widehat{B} < \widehat{C}$. Gọi $H$ là hình chiếu của $A$ trên $BC$. Các đoạn thẳng sau được sắp xếp theo thứ tự giảm dần là:
	\choice
	{$AB; AC; AH$}
	{$AC; AB; AH$}
	{$AH; AB; AC$}
	{$AH; AC; AB$}
	\loigiai{}
\end{ex}
%%%============EX_8==============%%%
\begin{ex}
	Cho tam giác $MNP$ và tam giác $DEF$ có $MN=DE; \widehat{N}=\widehat{E}$; cần thêm điều kiện nào để $\triangle MNP=\triangle DEF$:
	\choice
	{$MP=DE$}
	{$NP=DF$}
	{$\widehat{P}=\widehat{F}$}
	{$NP=EF$}
	\loigiai{}
\end{ex}
%%%============EX_9==============%%%
\begin{ex}
	\hinhphai{Trong các đoạn thẳng $OM, ON, OP, OQ$ (hình bên) đoạn thẳng nào ngắn nhất:
	\choice
	{OQ}
	{OP}
	{ON}
	{OM}}{
	 \begin{tikzpicture}[declare function={r=3;}]
	 	\path 
	 	(0,0) coordinate (P)
	 	($(P)+(90:r)$) coordinate (O)
	 	($(P)+(0:r-1)$) coordinate (Q)
	 	($(P)+(0:-r+2)$) coordinate (N)
	 	($(P)+(0:-r-1)$) coordinate (M)
	 	;
	 	\path pic[draw=\maunhan,thick,angle radius=5pt]{right angle= Q--P--O};
	 	\draw[shorten <=-1cm,shorten >=-1cm,\mycolor,thick](M) node [above left,xshift=-5pt,\maunhan]{d}--(Q);
	 	\draw[\mauphu,ultra thick](O)--(M) (O)--(N) (O)--(P)(O)--(Q);
	 	\foreach \x/\y in{P/-90,O/90,Q/-90,N/-90,M/-90}{
	 		\path[fill=white,draw=\maunhan,thick](\x) circle (1pt) node[shift={(\y:8pt)},font=\color{\maunhan}\bfseries\sffamily]{\x};}
	 \end{tikzpicture}
	}
	\loigiai{}
\end{ex}

%%%============EX_10==============%%%
\begin{ex}
	Cho $\triangle ABC$ và $\triangle MNP$ có $AB=MN; AC=MP$. Cần thêm điều kiện nào về cạnh để $\triangle ABC=\triangle MNP$:
	\choice
	{$AB=MP$}
	{$BC=MP$}
	{$BC=NP$}
	{$AC=MN$}
	\loigiai{}
\end{ex}
%%%============EX_11==============%%%
\begin{ex}
	Cho $\triangle ABC=\triangle MNP$ biết $\widehat{A}=100^{\circ}; \widehat{B}=50^{\circ}$. Số đo góc $P$ là:
	\choice
	{$180^{\circ}$}
	{$100^{\circ}$}
	{$50^{\circ}$}
	{$30^{\circ}$}
\loigiai{}
\end{ex}
%%%============EX_12==============%%%
\begin{ex}
	Tổ 1 của lớp 7A có 4 bạn nữ: Mai, Hà, An, Ngân và 5 bạn nam: Hùng, Trung, Phong, Nam, Bảo. Chọn ra ngẫu nhiên một học sinh trong tổ 1 của lớp 7
	Xét biến cố "Học sinh được chọn ra là học sinh nữ". Những kết quả thuận lợi cho biến cố đó là:
		\choice
		{Mai, Hà, An, Ngân}
		{Mai, An, Ngân}
		{Hùng, Trung, Mai}
		{Cả 9 bạn trong tổ 1}
		\loigiai{}
\end{ex}

\Closesolutionfile{ansex}
\Closesolutionfile{ans}

%%%==========Phần trắc nghiệm đúng sai============%%%
%\tieumuc{Bài Tập Trắc Nghiệm Đúng Sai}-- \textit{Trong mỗi câu có 4 ý tương ứng A, B, C, D; Học sinh chọn đúng hoặc sai.}
%\Opensolutionfile{ans}[Ans/DATAM1]
%\Opensolutionfile{ansbook}[Ans/DATNTF-1]
%\luulgEXTF
%\Opensolutionfile{ansex}[LOIGIAITN/LGTNTF-1]
%
%
%\Closesolutionfile{ansex}
%\Closesolutionfile{ansbook}
%\Closesolutionfile{ans}	

%%%==========Phần tự luận============%%%
\tieumuc{Bài tập tự luận}
\Opensolutionfile{ansbt}[LOIGIAITL/LGTL-De-01]
%\luuloigiaibt
\taoNdongke[15]{bt}
%%==============BT_1==============%%%
\begin{bt}
      	Viết ngẫu nhiên một số tự nhiên có hai chữ số không vượt quá 50. Gọi $\mathscr{D}$ là tập hợp gồm các kết quả có thể xảy ra đối với số tự nhiên được viết ra.
      	\begin{enumerate}[a)]
      		\item Tìm số phần tử của tập hợp $\mathscr{D}$
      		\item Hãy tính xác suất của mỗi biến cố sau:
      		    \begin{enumerate}[b1)]
      			\item "Số tự nhiên được viết ra chia hết cho 5 "
      			
      			\item "Số tự nhiên được viết ra là bội của 11 "
      			
      			\item "Số tự nhiên được viết ra là ước của 60 "	 
      			\end{enumerate}
      	\end{enumerate}
	\loigiai{}
\end{bt}
%%%==============BT_2==============%%%
\taoNdongke[15]{bt}
\begin{bt}
	 Cho tam giác $ABC$ cân ở $A\left(\hat{A} < 90^{\circ}\right)$. Hai đường cao $BD$ và $CE$ cắt nhau tại $I$. Chứng minh rằng:
	   \begin{enumerate}[a)]
	      		\item $\triangle AEC=\triangle ADB$
	      		\item AI là tia phân giác của góc $A$
	      		\item $ED \parallel BC$.
	   \end{enumerate}
	\loigiai{}
\end{bt}

\Closesolutionfile{ansbt}

\fileend

\begin{center}
	\rule[4pt]{2cm}{1pt}\large \textbf{HẾT}\rule[4pt]{2cm}{1pt}
\end{center}











