\begin{loigiaibt}{1}
  Ta có $$\left (x-\dfrac {1}{x}\right )^4=\mathrm {C}_4^0 x^4+\mathrm {C}_4^1 x^3\left (-\dfrac {1}{x}\right )+\mathrm {C}_4^2 x^2\left (-\dfrac {1}{x}\right )^2+\mathrm {C}_4^3 x\left (-\dfrac {1}{x}\right )^3+\mathrm {C}_4^4\left (-\dfrac {1}{x}\right )^4.$$ Số hạng không chứa $x$ là $\mathrm {C}_4^2 x^2\left (-\dfrac {1}{x}\right )^2=\mathrm {C}_4^2=6$.  
\end{loigiaibt}
\begin{loigiaibt}{2}
 Xếp $A$ lên một trong $3$ toa tàu: có $3$ cách.\\ Xếp $B$ lên một trong $3$ toa tàu: có $3$ cách.\\ Tương tự, số cách xếp $C$ và $D$ cũng là $3$ cách.\\ Với mỗi cách xếp $\mathrm {A}$ ta có $3$ cách xếp $B$ lên toa tàu.\\ Vậy số cách xếp thỏa mãn là $3 \times 3 \times 3 \times 3=81$ (cách).  
\end{loigiaibt}
\begin{loigiaibt}{3}
 Xét các số thoả mãn điều kiện có mặt chữ số $1$ và $5$, có $3$ trường hợp sau:\begin {itemize} \item Chọn $4$ số trong $6$ số còn lại cho vào $4$ vị trí còn lại có $\mathrm {A}_6^4$ cách. Vậy có $5 \cdot \mathrm {A}_6^4=1800$ số. \item Số có dạng $\overline {5 a b c d e}$. Tương tự cũng có $5 \cdot \mathrm {A}_6^4=1800$ số. \item Số $1$ và số $5$ không ở vị trí đầu tiên.\\ Có $\mathrm {A}_5^2$ cách chọn vị trí cho số $1$ và số $5$.\\ Chữ số đầu tiên khác $0$ và chọn trong $\{2 ; 3 ; 4 ; 6 ; 7\}$ nên có $5$ cách chọn.\\ Chọn $3$ số trong $5$ số cho $3$ vị trí còn lại có $\mathrm {A}_5^3$ cách.\\ Do đó tạo được $\mathrm {A}_5^2 \cdot 5 \cdot \mathrm {A}_5^3=6000$ số. \end {itemize} Vậy có $1800+1800+6000=9600$ số. 
\end{loigiaibt}
\begin{loigiaibt}{4}
 a) Hai đường đi (giả sử là hai đường thẳng $d_1, d_2$) của hai tàu có cặp vectơ chỉ phương $\vec {u}_1=(-33 ; 25), \vec {u}_2=(-30 ;-40) ;$ côsin góc tạo bởi hai đường thẳng là $$\cos \left (d_1, d_2\right )=\dfrac {\left |\vec {u}_1 \cdot \vec {u}_2\right |}{\left |\vec {u}_1\right | \cdot \left |\vec {u}_2\right |}=\dfrac {|-33 \cdot (-30)+25(-40)|}{\sqrt {(-33)^2+25^2} \cdot \sqrt {(-30)^2+(-40)^2}} \approx 0,00483.$$ b) Tại thời điểm $t$, vị trí tàu $A$ là $M(3-33 t ;-4+25 t)$, vị trí của tàu $B$ là $N(4-30 t ; 3-40 t)$.\\ Ta có $M N=\sqrt {(1+3 t)^2+(7-65 t)^2}=\sqrt {4234 t^2-904 t+50}$.\\ Khi đó $M N$ nhỏ nhất khi hàm bậc hai $f(t)=4234 t^2-904 t+50$ đạt giá trị nhỏ nhất, lúc đó $$x=-\dfrac {b}{2 a}=-\dfrac {-904}{2\cdot 4234}=\dfrac {226}{2117} \approx 0,107 \text { (giây).}$$ c) Khi tàu $A$ đứng yên, vị trí ban đầu của nó có tọa độ $P(3 ;-4)$; vị trí tàu $B$ ứng với thời gian $t$ là $Q(4-30 t ; 3-40 t)$, suy ra $$ P Q=\sqrt {(1-30 t)^2+(7-40 t)^2}=\sqrt {2500 t^2-620 t+50} . $$ Đoạn $P Q$ ngắn nhất ứng với $t=-\dfrac {b}{2 a}=\dfrac {620}{2 \cdot 2500}=\dfrac {31}{250}=0,124$ (giây).\\ Khi đó: $P Q_{\min }=\sqrt {2500 \cdot (0,124)^2-620 \cdot (0,124)+50}=\dfrac {17}{5}=3,4(\mathrm {~km})$. 
\end{loigiaibt}
