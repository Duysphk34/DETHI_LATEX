\begin{name}[Kiểm tra cuối kì II][Hóa][9][Sở Giáo dục và Đào tạo]{Trường THCS}{2023 - 2024}
\end{name}
\tieumuc{Trắc nghiệm (5 điểm)}
\Opensolutionfile{ans}[Ans/DATN-9CKII-De-02]
%\luuloigiaiex
%\hienthiloigiaiex
%\taoNdongke[5]{ex}
%\tatloigiaiex
\Opensolutionfile{ansex}[LOIGIAITN/LGTN-9CKII-De-02]
%%%============EX_1==============%%%
\begin{ex}[QG. 18-203]Xenlulozơ thuộc loại polisaccarit, là thành phần chính tạo nên màng tế bào thực vật, có nhiều trong gỗ, bông gòn. Công thức của xenlulozơ là:
	\choice
	{\True $\left(C_6{H}_{10} O_5\right)_n$}
	{$C_{12} H_{22} O_{11}$}
	{$C_6H_{12} O_6$}
	{$C_2H_4O_2$}
	\loigiai{Xenlulozơ có công thức hóa học là $(C_6H_{10}O_5)_n$}
\end{ex}
%%%============EX_2==============%%%
\begin{ex}
	Ở nhiệt độ thường, nhỏ vài giọt dung dịch iot vào hồ tinh bột thấy xuất hiện màu
	\choice
	{nâu đỏ}
	{vàng}
	{\True xanh tím}
	{hồng}
	\loigiai{Tinh bột phản ứng màu với hồ tinh bột xuất hiện màu xanh tím}
\end{ex}
%%%============EX_3==============%%%
\begin{ex}
	Cho sơ đồ phản ứng: 
	\begin{enumerate}[a)]
		\item \puhh[axit,$t^\circ$]{$X$ \+ $H_2O$}{->}{$Y$}
		\item \puhh[$NH_3$, $t^\circ$]{$Y$ \+ $Ag_2O$}{->}{$C_6H_{12}O_7$ \+ $2Ag$}
		\item \puhh[\text{lên men}]{$Y$}{->}{$E$ \+ $Z$}
		\item \puhh[\text{ánh sáng}][\text{clorophin}][0.9]{$Z$\+$H_2O$}{->}{$X$ \+ $G$}
	\end{enumerate}
	X, Y, Z lần lượt là:
	\choice
	{Xenlulozơ, fructozơ, cacbon đioxit}
	{Xenlulozơ, saccarozơ, cacbon đioxit}
	{Tinh bột, glucozơ, rượu etylic}
	{\True Tinh bột, glucozơ, cacbon đioxit}
	\loigiai{
	\begin{enumerate}[a)]
		\item \puhh[axit][$t^\circ$][][][-Stealth]{$(C_6H_{10}O_5)_n$ \+ $nH_2O$}{->}{$nC_6H_{12}O_6$}
		\item \puhh[$NH_3$, $t^\circ$]{$Y$ \+ $Ag_2O$}{->}{$C_6H_{12}O_7$ \+ $2Ag$}
		\item \puhh[lên men ][$30-35^\circ$][0.9][][-Stealth]{$C_6H_{12}O_6$}{->}{$2C_2H_5OH$\+$2CO_2$}
		\item \puhh[\text{ánh sáng}][\text{clorophin}][0.9][]{$6nCO_2$\+ $5nH_2O$}{<=>}{$(C_6H_{10}O_5)_n$\+$6nCO_2$}
	\end{enumerate} 
	Vậy X, Y, Z lần lượt là tinh bột, glucozơ, cacbon đioxit.
	}
\end{ex}
%%%============EX_4==============%%%
\begin{ex}
	Protein được tạo từ
	\choice
	{\True các amino axit}
	{các axit amin}
	{các axit hữu cơ}
	{các axit axetic}
	\loigiai{Protein được tạo nên từ các amino axit}
\end{ex}
%%%============EX_5==============%%%
\begin{ex}
	Monome nào sau đây tham gia phản ứng trùng hợp để tạo ra PE?
	\choice
	{Metan}
	{\True Etilen}
	{Axetilen}
	{Vinyl clorua}
	\loigiai{Etilen tham gia phản ứng trùng hợp tạo ra polietilen (PE)}
\end{ex}
%%%============EX_6==============%%%
\begin{ex}
	Loại tơ có nguồn gốc từ xenlulozơ là
	\choice
	{tơ tằm, bông vải}
	{tơ tằm, sợi đay}
	{\True bông vải, sợi đay}
	{tơ tằm, tơ nilon-6,6}
	\loigiai{bông vải, sợi đay là các loại tơ có thành phần là xenlulozo}
\end{ex}
%%%============EX_7==============%%%
\begin{ex}
	Khi cho hỗn hợp các ancol tác dụng với m gam K (vừa đủ), thu được 4,48 lít H $_2$ (đktc) Giá trị của m là
	\choice
	{7,8}
	{3,9}
	{9,75}
	{\True 15,6}
	\loigiai{}
\end{ex}
%%%============EX_8==============%%%
\begin{ex}
	Trung hòa 400 ml dung dịch axit axetic 0,5M bằng dung dịch NaOH 0,5M. Thể tích dung dịch NaOH cần dùng là
	\choice
	{100 ml}
	{200 ml}
	{300 ml}
	{\True 400 ml}
	\loigiai{}
\end{ex}
%%%============EX_9==============%%%
\begin{ex}Xà phòng hóa hoàn toàn 178 gam chất béo X trong dung dịch KOH, thu được m gam muối kali stearat ($C_{17}H_{35}COOK$ ). Giá trị của m là
	\choice
	{200,8}
	{183,6}
	{211,6}
	{\True 193,2}
	\loigiai{}
\end{ex}
%%%============EX_10==============%%%
\begin{ex}
	Tiến hành sản xuất ancol etylic từ xenlulozơ với hiệu suất của toàn bộ quá trình là 70\%. Để sản xuất 2 tấn ancol etylic, khối lượng xenlulozơ cần dùng là
	\choice
	{\True 5,031 tấn}
	{10,062 tấn}
	{3,521 tấn}
	{2,515 tấn}
	\loigiai{}
\end{ex}
%%%============EX_11==============%%%
\begin{ex}
	Trong gas, dùng để đun, nấu thức ăn trong gia đình, người ta thêm một lượng nhỏ khí có công thức hoá học  $C_2H_5S$  có mùi hôi. Mục đích của việc thêm hoá chất này vào gas là nhằm:
	\choice
	{Tăng năng suất toả nhiệt của gas}
	{\True Phát hiện nhậnh chóng sự cố rò rỉ gas}
	{Hạ giá thành sản xuất gas}
	{Phòng chống cháy nổ khi sử dụng gas}
	\loigiai{}
\end{ex}
%%%============EX_12==============%%%
\begin{ex}
	Điều khẳng định nào sau đây không đúng?
	\choice
	{Metan là chất khí nhẹ hơn không khí}
	{Metan là nguồn cung cấp hiđro cho công nghiệp sản xuất phân bón hoá học}
	{Metan là chất khí cháy được trong không khí, toả nhiều nhiệt}
	{\True Metan là chất khí nhẹ hơn khí hiđro}
	\loigiai{}
\end{ex}
%%%============EX_13==============%%%
\begin{ex}
	Điểm khác biệt cơ bản trong cấu tạo phân tử của etilen so với metan là:
	\choice
	{Hoá trị của nguyên tố cacbon}
	{Liên kết giữa hai nguyên tử cacbon}
	{Hóa trị của hiđro}
	{\True Liên kết đôi của etilen so với liên kết đơn của metan}
	\loigiai{}
\end{ex}
%%%============EX_14==============%%%
\begin{ex}
	Cho các phát biểu sau:
	\begin{enumerate}[(a)]
		\item  Phân tử etilen chỉ chứa một liên kết đôi.
		\item  Metan, etilen và axetilen đều có khả năng làm mất màu dung dịch brom.
		\item  Metan có khả năng tham gia phản ứng trùng hợp.
		\item  Khí etilen có nhiều trong khí thiên nhiên, khí mỏ dầu và khí biogaz.
		\item  Axetilen là nguyên liệu để điều chế nhựa polietilen, rượu etylic,{\dots}
	\end{enumerate}
	Số phát biểu sai là
	\choice
	{\True $5$}
	{$2$}
	{$3$}
	{$4$}
	\loigiai{}
\end{ex}
%%%============EX_15==============%%%
\begin{ex}
	Để loại bỏ khí axetilen trong hỗn hợp với metan người ta dùng
	\choice
	{nước}
	{khí hiđro}
	{\True dung dịch brom}
	{khí oxi}
	\loigiai{}
\end{ex}
%%%============EX_16==============%%%
\begin{ex}
	Hợp chất hữu cơ là chất khí ít tan trong nước, tham gia phản ứng thế, không tham gia phản ứng cộng. Hợp chất đó là
	\choice
	{\True metan}
	{etilen}
	{axetilen}
	{benzen}
	\loigiai{}
\end{ex}
%%%============EX_17==============%%%
\begin{ex}
	Nguyên tử của nguyên tố X có 3 lớp electron, lớp electron ngoài cùng có 7 electron. Vị trí của nguyên tố X là
	\choice
	{thuộc chu kỳ 3, nhóm VI}
	{thuộc chu kỳ 7, nhóm III}
	{\True thuộc chu kỳ 3, nhóm VII}
	{thuộc chu kỳ 7, nhóm VI}
	\loigiai{}
\end{ex}
%%%============EX_18==============%%%
\begin{ex}
	Số thứ tự chu kì trong bảng hệ thống tuần hoàn cho biết:
	\choice
	{Số thứ tự của nguyên tố}
	{Số hiệu nguyên tử}
	{Số electron lớp ngoài cùng}
	{\True Số lớp electron}
	\loigiai{}
\end{ex}
%%%============EX_19==============%%%
\begin{ex}
	Khi nung nóng hỗn hợp bột gồm 9,6 gam lưu huỳnh và 22,4 gam sắt trong ống nghiệm kín, không chứa không khí, sau khi phản ứng hoàn toàn thu được rắn Y. Thành phần của rắn Y là
	\choice
	{Fe}
	{\True Fe và FeS}
	{FeS}
	{S và FeS}
	\loigiai{}
\end{ex}
%%%============EX_20==============%%%
\begin{ex}
	Nung 13,4 gam hỗn hợp 2 muối cacbonat của 2 kim loại hóa trị 2, thu được $6{,}8$ gam chất rắn và khí X. Lượng khí X sinh ra cho hấp thụ vào $75$ ml dung dịch NaOH $1$M, khối lượng muối khan thu được sau phản ứng là
	\choice
	{$5{,}8$ gam}
	{$6{,}5$ gam}
	{$4{,}2 $gam}
	{\True $6{,}3$ gam}
	\loigiai{}
\end{ex}
\Closesolutionfile{ansex}
\Closesolutionfile{ans}

%%%==========Phần trắc nghiệm đúng sai============%%%

%\tieumuc{Trắc Nghiệm Đúng Sai}-- \textit{Trong mỗi câu có 4 ý tương ứng A, B, C, D; Học sinh chọn đúng hoặc sai.}
%\Opensolutionfile{ans}[Ans/DATAM1]
%\Opensolutionfile{ansbook}[Ans/DATNTF-1]
%\luulgEXTF
%\Opensolutionfile{ansex}[LOIGIAITN/LGTNTF-1]
%Soạn ở đây
%\Closesolutionfile{ansex}
%\Closesolutionfile{ansbook}
%\Closesolutionfile{ans}	
\newpage
%%%==========Phần tự luận============%%%
\tieumuc{Tự Luận (5 điểm)}
\Opensolutionfile{ansbt}[LOIGIAITL/LGTL-9CKII-De-02]
%\luuloigiaibt
%\hienthiloigiaibt
%\taoNdongke[15]{bt}
\begin{bt}[$1{,}5$ điểm]
	Viết các PTHH thụcc hiện chuyển đối hóa học sau (ghi rõ điều kiện phản ứng nếu có):
	\begin{center}
		$CH_4$ \MuiTen[(1)] $C_2H_2$ \MuiTen[(2)] $C_2H_4$ \MuiTen[(3)] $C_2H_5OH$ \MuiTen[(4)] $CH_3COOC_2H_5$
	\end{center}
	\loigiai{}
\end{bt}
%%%============BT_02==============%%%
\begin{bt}[$1{,}0$ điểm]Rượu Bàu Đá là một trong những đặc sản nổi tiếng của tỉnh Bình Định, có xuất xứ từ xóm Bàu Đá, thôn Cù Lâm, xã Nhơn Lộc, huyện An Nhơn . Rượu Bàu Đá được các gia đình trong vùng chưng cất từ gạo như một nghề gia truyền theo sơ đồ sau:
	\begin{center}
		\begin{tikzpicture}[declare function={gocm=0;goch=90;gocb=-90;r=4;}]
			\tikzset{mystyle/.style={inner sep =2pt,outer sep=2pt,draw=\mycolor,rectangle,minimum width=2.0cm,text width=2.0cm,align=center,rounded corners=1mm,minimum height=1.5cm,font=\bfseries}}
			\tikzset{mynode/.style={pos=0.5,sloped,font=\itshape\scriptsize\sffamily,align=center,text width=2.5cm}}
			\path(0,0) node[mystyle](gao){Gạo}++(gocm:r) node[mystyle] (com){Cơm}++(gocm:{r}) node[mystyle] (comruou){Cơm rượu (rượu cái)}++(gocm:r+1) node[mystyle] (ruoutrang) {Rượu trắng};
			\foreach \x/\y in {
				gao/com,com/comruou,comruou/ruoutrang}
			\draw[-stealth](\x.east)--(\y.west);
			\path (gao.east)--(com.west) node[mynode,above]{1.Làm sạch} node[mynode,below]{2.nấu}
			(com.east)--(comruou.west) node[mynode,above]{1.trộn men} node[mynode,below]{2.ủ}
			(comruou.east)--(ruoutrang.west) node[mynode,above]{1.chưng cất} node[mynode,below]{2.chưng tái và lọc}
			;
		\end{tikzpicture}\par
	\end{center}
	Biết rằng mỗi một mẻ gạo  sau khi nấu ta thu được $7{,}2\mathrm{~kg}$ cơm chứa $75\%$ tinh bột. Hỏi mỗi một mẻ gạo như vậy ta có thể thu được bao nhiêu lít dung dịch rượu $46^\circ$. Biết hiệu suất cả quá trình là $60\%$ và khối lượng riêng của rượu là $0{,78}\mathrm{~gam/ml}$.
	\loigiai{}
\end{bt}

%%%==============BT_1==============%%%
\begin{bt}[$2{,}5$ điểm]
	: Cho 16,6 gam hỗn hợp $X$ gồm axit axetic và ancol etylic tác dụng vừa đủ với $200\mathrm{ml}$ dung dịch $KOH 1M$.
	\begin{enumerate}
		\item Viết PTHH xảy ra và tính phần trăm khối lượng mỗi chất trong hỗn hợp X.
		\item Tính thể tích $H_2$ thu được (ở đktc) khi cho 16,6 gam hỗn hợp $X$ tác dụng với $\mathrm{Na}$ dư.
		\item Đun nóng 16,6 gam hỗn hợp $X$ ở trên với $H_2SO_4$ đặc, sau phản ứng thu được 4,4 gam etyl axetat. Hãy tính hiệu suất của phản ứng este hóa.
	\end{enumerate}
	\loigiai{}
\end{bt}

\Closesolutionfile{ansbt}

\fileend

\begin{center}
	\rule[4pt]{2cm}{1pt}\large \textbf{HẾT}\rule[4pt]{2cm}{1pt}
\end{center}