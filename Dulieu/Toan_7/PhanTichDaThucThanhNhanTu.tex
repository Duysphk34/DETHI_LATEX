%%%=========Bai_1=========%%%
\begin{bt}
	Phân tích đa thức sau thành nhân tử: $x^2 + 19 x + 90$.
	\loigiai{ Ta có: $x^2 + 19 x + 90 = \left(x + 9\right) \left(x + 10\right)$
	}
\end{bt}
%%%=========Bai_2=========%%%
\begin{bt}
	Phân tích đa thức sau thành nhân tử: $x^2 + 18 x + 80$.
	\loigiai{ Ta có: $x^2 + 18 x + 80 = \left(x + 8\right) \left(x + 10\right)$
	}
\end{bt}
%%%=========Bai_3=========%%%
\begin{bt}
	Phân tích đa thức sau thành nhân tử: $x^2 + 17 x + 70$.
	\loigiai{ Ta có: $x^2 + 17 x + 70 = \left(x + 7\right) \left(x + 10\right)$
	}
\end{bt}
%%%=========Bai_4=========%%%
\begin{bt}
	Phân tích đa thức sau thành nhân tử: $x^2 + 16 x + 60$.
	\loigiai{ Ta có: $x^2 + 16 x + 60 = \left(x + 6\right) \left(x + 10\right)$
	}
\end{bt}
%%%=========Bai_5=========%%%
\begin{bt}
	Phân tích đa thức sau thành nhân tử: $x^2 + 15 x + 50$.
	\loigiai{ Ta có: $x^2 + 15 x + 50 = \left(x + 5\right) \left(x + 10\right)$
	}
\end{bt}
%%%=========Bai_6=========%%%
\begin{bt}
	Phân tích đa thức sau thành nhân tử: $x^2 + 14 x + 40$.
	\loigiai{ Ta có: $x^2 + 14 x + 40 = \left(x + 4\right) \left(x + 10\right)$
	}
\end{bt}
%%%=========Bai_7=========%%%
\begin{bt}
	Phân tích đa thức sau thành nhân tử: $x^2 + 13 x + 30$.
	\loigiai{ Ta có: $x^2 + 13 x + 30 = \left(x + 3\right) \left(x + 10\right)$
	}
\end{bt}
%%%=========Bai_8=========%%%
\begin{bt}
	Phân tích đa thức sau thành nhân tử: $x^2 + 12 x + 20$.
	\loigiai{ Ta có: $x^2 + 12 x + 20 = \left(x + 2\right) \left(x + 10\right)$
	}
\end{bt}
%%%=========Bai_9=========%%%
\begin{bt}
	Phân tích đa thức sau thành nhân tử: $x^2 + 11 x + 10$.
	\loigiai{ Ta có: $x^2 + 11 x + 10 = \left(x + 1\right) \left(x + 10\right)$
	}
\end{bt}
%%%=========Bai_10=========%%%
\begin{bt}
	Phân tích đa thức sau thành nhân tử: $x^2 + 9 x - 10$.
	\loigiai{ Ta có: $x^2 + 9 x - 10 = \left(x - 1\right) \left(x + 10\right)$
	}
\end{bt}
%%%=========Bai_11=========%%%
\begin{bt}
	Phân tích đa thức sau thành nhân tử: $x^2 + 8 x - 20$.
	\loigiai{ Ta có: $x^2 + 8 x - 20 = \left(x - 2\right) \left(x + 10\right)$
	}
\end{bt}
%%%=========Bai_12=========%%%
\begin{bt}
	Phân tích đa thức sau thành nhân tử: $x^2 + 7 x - 30$.
	\loigiai{ Ta có: $x^2 + 7 x - 30 = \left(x - 3\right) \left(x + 10\right)$
	}
\end{bt}
%%%=========Bai_13=========%%%
\begin{bt}
	Phân tích đa thức sau thành nhân tử: $x^2 + 6 x - 40$.
	\loigiai{ Ta có: $x^2 + 6 x - 40 = \left(x - 4\right) \left(x + 10\right)$
	}
\end{bt}
%%%=========Bai_14=========%%%
\begin{bt}
	Phân tích đa thức sau thành nhân tử: $x^2 + 5 x - 50$.
	\loigiai{ Ta có: $x^2 + 5 x - 50 = \left(x - 5\right) \left(x + 10\right)$
	}
\end{bt}
%%%=========Bai_15=========%%%
\begin{bt}
	Phân tích đa thức sau thành nhân tử: $x^2 + 4 x - 60$.
	\loigiai{ Ta có: $x^2 + 4 x - 60 = \left(x - 6\right) \left(x + 10\right)$
	}
\end{bt}
%%%=========Bai_16=========%%%
\begin{bt}
	Phân tích đa thức sau thành nhân tử: $x^2 + 3 x - 70$.
	\loigiai{ Ta có: $x^2 + 3 x - 70 = \left(x - 7\right) \left(x + 10\right)$
	}
\end{bt}
%%%=========Bai_17=========%%%
\begin{bt}
	Phân tích đa thức sau thành nhân tử: $x^2 + 2 x - 80$.
	\loigiai{ Ta có: $x^2 + 2 x - 80 = \left(x - 8\right) \left(x + 10\right)$
	}
\end{bt}
%%%=========Bai_18=========%%%
\begin{bt}
	Phân tích đa thức sau thành nhân tử: $x^2 + x - 90$.
	\loigiai{ Ta có: $x^2 + x - 90 = \left(x - 9\right) \left(x + 10\right)$
	}
\end{bt}
%%%=========Bai_19=========%%%
\begin{bt}
	Phân tích đa thức sau thành nhân tử: $x^2 + 19 x + 90$.
	\loigiai{ Ta có: $x^2 + 19 x + 90 = \left(x + 9\right) \left(x + 10\right)$
	}
\end{bt}
%%%=========Bai_20=========%%%
\begin{bt}
	Phân tích đa thức sau thành nhân tử: $x^2 + 17 x + 72$.
	\loigiai{ Ta có: $x^2 + 17 x + 72 = \left(x + 8\right) \left(x + 9\right)$
	}
\end{bt}
%%%=========Bai_21=========%%%
\begin{bt}
	Phân tích đa thức sau thành nhân tử: $x^2 + 16 x + 63$.
	\loigiai{ Ta có: $x^2 + 16 x + 63 = \left(x + 7\right) \left(x + 9\right)$
	}
\end{bt}
%%%=========Bai_22=========%%%
\begin{bt}
	Phân tích đa thức sau thành nhân tử: $x^2 + 15 x + 54$.
	\loigiai{ Ta có: $x^2 + 15 x + 54 = \left(x + 6\right) \left(x + 9\right)$
	}
\end{bt}
%%%=========Bai_23=========%%%
\begin{bt}
	Phân tích đa thức sau thành nhân tử: $x^2 + 14 x + 45$.
	\loigiai{ Ta có: $x^2 + 14 x + 45 = \left(x + 5\right) \left(x + 9\right)$
	}
\end{bt}
%%%=========Bai_24=========%%%
\begin{bt}
	Phân tích đa thức sau thành nhân tử: $x^2 + 13 x + 36$.
	\loigiai{ Ta có: $x^2 + 13 x + 36 = \left(x + 4\right) \left(x + 9\right)$
	}
\end{bt}
%%%=========Bai_25=========%%%
\begin{bt}
	Phân tích đa thức sau thành nhân tử: $x^2 + 12 x + 27$.
	\loigiai{ Ta có: $x^2 + 12 x + 27 = \left(x + 3\right) \left(x + 9\right)$
	}
\end{bt}
%%%=========Bai_26=========%%%
\begin{bt}
	Phân tích đa thức sau thành nhân tử: $x^2 + 11 x + 18$.
	\loigiai{ Ta có: $x^2 + 11 x + 18 = \left(x + 2\right) \left(x + 9\right)$
	}
\end{bt}
%%%=========Bai_27=========%%%
\begin{bt}
	Phân tích đa thức sau thành nhân tử: $x^2 + 10 x + 9$.
	\loigiai{ Ta có: $x^2 + 10 x + 9 = \left(x + 1\right) \left(x + 9\right)$
	}
\end{bt}
%%%=========Bai_28=========%%%
\begin{bt}
	Phân tích đa thức sau thành nhân tử: $x^2 + 8 x - 9$.
	\loigiai{ Ta có: $x^2 + 8 x - 9 = \left(x - 1\right) \left(x + 9\right)$
	}
\end{bt}
%%%=========Bai_29=========%%%
\begin{bt}
	Phân tích đa thức sau thành nhân tử: $x^2 + 7 x - 18$.
	\loigiai{ Ta có: $x^2 + 7 x - 18 = \left(x - 2\right) \left(x + 9\right)$
	}
\end{bt}
%%%=========Bai_30=========%%%
\begin{bt}
	Phân tích đa thức sau thành nhân tử: $x^2 + 6 x - 27$.
	\loigiai{ Ta có: $x^2 + 6 x - 27 = \left(x - 3\right) \left(x + 9\right)$
	}
\end{bt}
%%%=========Bai_31=========%%%
\begin{bt}
	Phân tích đa thức sau thành nhân tử: $x^2 + 5 x - 36$.
	\loigiai{ Ta có: $x^2 + 5 x - 36 = \left(x - 4\right) \left(x + 9\right)$
	}
\end{bt}
%%%=========Bai_32=========%%%
\begin{bt}
	Phân tích đa thức sau thành nhân tử: $x^2 + 4 x - 45$.
	\loigiai{ Ta có: $x^2 + 4 x - 45 = \left(x - 5\right) \left(x + 9\right)$
	}
\end{bt}
%%%=========Bai_33=========%%%
\begin{bt}
	Phân tích đa thức sau thành nhân tử: $x^2 + 3 x - 54$.
	\loigiai{ Ta có: $x^2 + 3 x - 54 = \left(x - 6\right) \left(x + 9\right)$
	}
\end{bt}
%%%=========Bai_34=========%%%
\begin{bt}
	Phân tích đa thức sau thành nhân tử: $x^2 + 2 x - 63$.
	\loigiai{ Ta có: $x^2 + 2 x - 63 = \left(x - 7\right) \left(x + 9\right)$
	}
\end{bt}
%%%=========Bai_35=========%%%
\begin{bt}
	Phân tích đa thức sau thành nhân tử: $x^2 + x - 72$.
	\loigiai{ Ta có: $x^2 + x - 72 = \left(x - 8\right) \left(x + 9\right)$
	}
\end{bt}
%%%=========Bai_36=========%%%
\begin{bt}
	Phân tích đa thức sau thành nhân tử: $x^2 + 18 x + 80$.
	\loigiai{ Ta có: $x^2 + 18 x + 80 = \left(x + 8\right) \left(x + 10\right)$
	}
\end{bt}
%%%=========Bai_37=========%%%
\begin{bt}
	Phân tích đa thức sau thành nhân tử: $x^2 + 17 x + 72$.
	\loigiai{ Ta có: $x^2 + 17 x + 72 = \left(x + 8\right) \left(x + 9\right)$
	}
\end{bt}
%%%=========Bai_38=========%%%
\begin{bt}
	Phân tích đa thức sau thành nhân tử: $x^2 + 15 x + 56$.
	\loigiai{ Ta có: $x^2 + 15 x + 56 = \left(x + 7\right) \left(x + 8\right)$
	}
\end{bt}
%%%=========Bai_39=========%%%
\begin{bt}
	Phân tích đa thức sau thành nhân tử: $x^2 + 14 x + 48$.
	\loigiai{ Ta có: $x^2 + 14 x + 48 = \left(x + 6\right) \left(x + 8\right)$
	}
\end{bt}
%%%=========Bai_40=========%%%
\begin{bt}
	Phân tích đa thức sau thành nhân tử: $x^2 + 13 x + 40$.
	\loigiai{ Ta có: $x^2 + 13 x + 40 = \left(x + 5\right) \left(x + 8\right)$
	}
\end{bt}
%%%=========Bai_41=========%%%
\begin{bt}
	Phân tích đa thức sau thành nhân tử: $x^2 + 12 x + 32$.
	\loigiai{ Ta có: $x^2 + 12 x + 32 = \left(x + 4\right) \left(x + 8\right)$
	}
\end{bt}
%%%=========Bai_42=========%%%
\begin{bt}
	Phân tích đa thức sau thành nhân tử: $x^2 + 11 x + 24$.
	\loigiai{ Ta có: $x^2 + 11 x + 24 = \left(x + 3\right) \left(x + 8\right)$
	}
\end{bt}
%%%=========Bai_43=========%%%
\begin{bt}
	Phân tích đa thức sau thành nhân tử: $x^2 + 10 x + 16$.
	\loigiai{ Ta có: $x^2 + 10 x + 16 = \left(x + 2\right) \left(x + 8\right)$
	}
\end{bt}
%%%=========Bai_44=========%%%
\begin{bt}
	Phân tích đa thức sau thành nhân tử: $x^2 + 9 x + 8$.
	\loigiai{ Ta có: $x^2 + 9 x + 8 = \left(x + 1\right) \left(x + 8\right)$
	}
\end{bt}
%%%=========Bai_45=========%%%
\begin{bt}
	Phân tích đa thức sau thành nhân tử: $x^2 + 7 x - 8$.
	\loigiai{ Ta có: $x^2 + 7 x - 8 = \left(x - 1\right) \left(x + 8\right)$
	}
\end{bt}
%%%=========Bai_46=========%%%
\begin{bt}
	Phân tích đa thức sau thành nhân tử: $x^2 + 6 x - 16$.
	\loigiai{ Ta có: $x^2 + 6 x - 16 = \left(x - 2\right) \left(x + 8\right)$
	}
\end{bt}
%%%=========Bai_47=========%%%
\begin{bt}
	Phân tích đa thức sau thành nhân tử: $x^2 + 5 x - 24$.
	\loigiai{ Ta có: $x^2 + 5 x - 24 = \left(x - 3\right) \left(x + 8\right)$
	}
\end{bt}
%%%=========Bai_48=========%%%
\begin{bt}
	Phân tích đa thức sau thành nhân tử: $x^2 + 4 x - 32$.
	\loigiai{ Ta có: $x^2 + 4 x - 32 = \left(x - 4\right) \left(x + 8\right)$
	}
\end{bt}
%%%=========Bai_49=========%%%
\begin{bt}
	Phân tích đa thức sau thành nhân tử: $x^2 + 3 x - 40$.
	\loigiai{ Ta có: $x^2 + 3 x - 40 = \left(x - 5\right) \left(x + 8\right)$
	}
\end{bt}
%%%=========Bai_50=========%%%
\begin{bt}
	Phân tích đa thức sau thành nhân tử: $x^2 + 2 x - 48$.
	\loigiai{ Ta có: $x^2 + 2 x - 48 = \left(x - 6\right) \left(x + 8\right)$
	}
\end{bt}
%%%=========Bai_51=========%%%
\begin{bt}
	Phân tích đa thức sau thành nhân tử: $x^2 + x - 56$.
	\loigiai{ Ta có: $x^2 + x - 56 = \left(x - 7\right) \left(x + 8\right)$
	}
\end{bt}
%%%=========Bai_52=========%%%
\begin{bt}
	Phân tích đa thức sau thành nhân tử: $x^2 - x - 72$.
	\loigiai{ Ta có: $x^2 - x - 72 = \left(x - 9\right) \left(x + 8\right)$
	}
\end{bt}
%%%=========Bai_53=========%%%
\begin{bt}
	Phân tích đa thức sau thành nhân tử: $x^2 + 17 x + 70$.
	\loigiai{ Ta có: $x^2 + 17 x + 70 = \left(x + 7\right) \left(x + 10\right)$
	}
\end{bt}
%%%=========Bai_54=========%%%
\begin{bt}
	Phân tích đa thức sau thành nhân tử: $x^2 + 16 x + 63$.
	\loigiai{ Ta có: $x^2 + 16 x + 63 = \left(x + 7\right) \left(x + 9\right)$
	}
\end{bt}
%%%=========Bai_55=========%%%
\begin{bt}
	Phân tích đa thức sau thành nhân tử: $x^2 + 15 x + 56$.
	\loigiai{ Ta có: $x^2 + 15 x + 56 = \left(x + 7\right) \left(x + 8\right)$
	}
\end{bt}
%%%=========Bai_56=========%%%
\begin{bt}
	Phân tích đa thức sau thành nhân tử: $x^2 + 13 x + 42$.
	\loigiai{ Ta có: $x^2 + 13 x + 42 = \left(x + 6\right) \left(x + 7\right)$
	}
\end{bt}
%%%=========Bai_57=========%%%
\begin{bt}
	Phân tích đa thức sau thành nhân tử: $x^2 + 12 x + 35$.
	\loigiai{ Ta có: $x^2 + 12 x + 35 = \left(x + 5\right) \left(x + 7\right)$
	}
\end{bt}
%%%=========Bai_58=========%%%
\begin{bt}
	Phân tích đa thức sau thành nhân tử: $x^2 + 11 x + 28$.
	\loigiai{ Ta có: $x^2 + 11 x + 28 = \left(x + 4\right) \left(x + 7\right)$
	}
\end{bt}
%%%=========Bai_59=========%%%
\begin{bt}
	Phân tích đa thức sau thành nhân tử: $x^2 + 10 x + 21$.
	\loigiai{ Ta có: $x^2 + 10 x + 21 = \left(x + 3\right) \left(x + 7\right)$
	}
\end{bt}
%%%=========Bai_60=========%%%
\begin{bt}
	Phân tích đa thức sau thành nhân tử: $x^2 + 9 x + 14$.
	\loigiai{ Ta có: $x^2 + 9 x + 14 = \left(x + 2\right) \left(x + 7\right)$
	}
\end{bt}
%%%=========Bai_61=========%%%
\begin{bt}
	Phân tích đa thức sau thành nhân tử: $x^2 + 8 x + 7$.
	\loigiai{ Ta có: $x^2 + 8 x + 7 = \left(x + 1\right) \left(x + 7\right)$
	}
\end{bt}
%%%=========Bai_62=========%%%
\begin{bt}
	Phân tích đa thức sau thành nhân tử: $x^2 + 6 x - 7$.
	\loigiai{ Ta có: $x^2 + 6 x - 7 = \left(x - 1\right) \left(x + 7\right)$
	}
\end{bt}
%%%=========Bai_63=========%%%
\begin{bt}
	Phân tích đa thức sau thành nhân tử: $x^2 + 5 x - 14$.
	\loigiai{ Ta có: $x^2 + 5 x - 14 = \left(x - 2\right) \left(x + 7\right)$
	}
\end{bt}
%%%=========Bai_64=========%%%
\begin{bt}
	Phân tích đa thức sau thành nhân tử: $x^2 + 4 x - 21$.
	\loigiai{ Ta có: $x^2 + 4 x - 21 = \left(x - 3\right) \left(x + 7\right)$
	}
\end{bt}
%%%=========Bai_65=========%%%
\begin{bt}
	Phân tích đa thức sau thành nhân tử: $x^2 + 3 x - 28$.
	\loigiai{ Ta có: $x^2 + 3 x - 28 = \left(x - 4\right) \left(x + 7\right)$
	}
\end{bt}
%%%=========Bai_66=========%%%
\begin{bt}
	Phân tích đa thức sau thành nhân tử: $x^2 + 2 x - 35$.
	\loigiai{ Ta có: $x^2 + 2 x - 35 = \left(x - 5\right) \left(x + 7\right)$
	}
\end{bt}
%%%=========Bai_67=========%%%
\begin{bt}
	Phân tích đa thức sau thành nhân tử: $x^2 + x - 42$.
	\loigiai{ Ta có: $x^2 + x - 42 = \left(x - 6\right) \left(x + 7\right)$
	}
\end{bt}
%%%=========Bai_68=========%%%
\begin{bt}
	Phân tích đa thức sau thành nhân tử: $x^2 - x - 56$.
	\loigiai{ Ta có: $x^2 - x - 56 = \left(x - 8\right) \left(x + 7\right)$
	}
\end{bt}
%%%=========Bai_69=========%%%
\begin{bt}
	Phân tích đa thức sau thành nhân tử: $x^2 - 2 x - 63$.
	\loigiai{ Ta có: $x^2 - 2 x - 63 = \left(x - 9\right) \left(x + 7\right)$
	}
\end{bt}
%%%=========Bai_70=========%%%
\begin{bt}
	Phân tích đa thức sau thành nhân tử: $x^2 + 16 x + 60$.
	\loigiai{ Ta có: $x^2 + 16 x + 60 = \left(x + 6\right) \left(x + 10\right)$
	}
\end{bt}
%%%=========Bai_71=========%%%
\begin{bt}
	Phân tích đa thức sau thành nhân tử: $x^2 + 15 x + 54$.
	\loigiai{ Ta có: $x^2 + 15 x + 54 = \left(x + 6\right) \left(x + 9\right)$
	}
\end{bt}
%%%=========Bai_72=========%%%
\begin{bt}
	Phân tích đa thức sau thành nhân tử: $x^2 + 14 x + 48$.
	\loigiai{ Ta có: $x^2 + 14 x + 48 = \left(x + 6\right) \left(x + 8\right)$
	}
\end{bt}
%%%=========Bai_73=========%%%
\begin{bt}
	Phân tích đa thức sau thành nhân tử: $x^2 + 13 x + 42$.
	\loigiai{ Ta có: $x^2 + 13 x + 42 = \left(x + 6\right) \left(x + 7\right)$
	}
\end{bt}
%%%=========Bai_74=========%%%
\begin{bt}
	Phân tích đa thức sau thành nhân tử: $x^2 + 11 x + 30$.
	\loigiai{ Ta có: $x^2 + 11 x + 30 = \left(x + 5\right) \left(x + 6\right)$
	}
\end{bt}
%%%=========Bai_75=========%%%
\begin{bt}
	Phân tích đa thức sau thành nhân tử: $x^2 + 10 x + 24$.
	\loigiai{ Ta có: $x^2 + 10 x + 24 = \left(x + 4\right) \left(x + 6\right)$
	}
\end{bt}
%%%=========Bai_76=========%%%
\begin{bt}
	Phân tích đa thức sau thành nhân tử: $x^2 + 9 x + 18$.
	\loigiai{ Ta có: $x^2 + 9 x + 18 = \left(x + 3\right) \left(x + 6\right)$
	}
\end{bt}
%%%=========Bai_77=========%%%
\begin{bt}
	Phân tích đa thức sau thành nhân tử: $x^2 + 8 x + 12$.
	\loigiai{ Ta có: $x^2 + 8 x + 12 = \left(x + 2\right) \left(x + 6\right)$
	}
\end{bt}
%%%=========Bai_78=========%%%
\begin{bt}
	Phân tích đa thức sau thành nhân tử: $x^2 + 7 x + 6$.
	\loigiai{ Ta có: $x^2 + 7 x + 6 = \left(x + 1\right) \left(x + 6\right)$
	}
\end{bt}
%%%=========Bai_79=========%%%
\begin{bt}
	Phân tích đa thức sau thành nhân tử: $x^2 + 5 x - 6$.
	\loigiai{ Ta có: $x^2 + 5 x - 6 = \left(x - 1\right) \left(x + 6\right)$
	}
\end{bt}
%%%=========Bai_80=========%%%
\begin{bt}
	Phân tích đa thức sau thành nhân tử: $x^2 + 4 x - 12$.
	\loigiai{ Ta có: $x^2 + 4 x - 12 = \left(x - 2\right) \left(x + 6\right)$
	}
\end{bt}
%%%=========Bai_81=========%%%
\begin{bt}
	Phân tích đa thức sau thành nhân tử: $x^2 + 3 x - 18$.
	\loigiai{ Ta có: $x^2 + 3 x - 18 = \left(x - 3\right) \left(x + 6\right)$
	}
\end{bt}
%%%=========Bai_82=========%%%
\begin{bt}
	Phân tích đa thức sau thành nhân tử: $x^2 + 2 x - 24$.
	\loigiai{ Ta có: $x^2 + 2 x - 24 = \left(x - 4\right) \left(x + 6\right)$
	}
\end{bt}
%%%=========Bai_83=========%%%
\begin{bt}
	Phân tích đa thức sau thành nhân tử: $x^2 + x - 30$.
	\loigiai{ Ta có: $x^2 + x - 30 = \left(x - 5\right) \left(x + 6\right)$
	}
\end{bt}
%%%=========Bai_84=========%%%
\begin{bt}
	Phân tích đa thức sau thành nhân tử: $x^2 - x - 42$.
	\loigiai{ Ta có: $x^2 - x - 42 = \left(x - 7\right) \left(x + 6\right)$
	}
\end{bt}
%%%=========Bai_85=========%%%
\begin{bt}
	Phân tích đa thức sau thành nhân tử: $x^2 - 2 x - 48$.
	\loigiai{ Ta có: $x^2 - 2 x - 48 = \left(x - 8\right) \left(x + 6\right)$
	}
\end{bt}
%%%=========Bai_86=========%%%
\begin{bt}
	Phân tích đa thức sau thành nhân tử: $x^2 - 3 x - 54$.
	\loigiai{ Ta có: $x^2 - 3 x - 54 = \left(x - 9\right) \left(x + 6\right)$
	}
\end{bt}
%%%=========Bai_87=========%%%
\begin{bt}
	Phân tích đa thức sau thành nhân tử: $x^2 + 15 x + 50$.
	\loigiai{ Ta có: $x^2 + 15 x + 50 = \left(x + 5\right) \left(x + 10\right)$
	}
\end{bt}
%%%=========Bai_88=========%%%
\begin{bt}
	Phân tích đa thức sau thành nhân tử: $x^2 + 14 x + 45$.
	\loigiai{ Ta có: $x^2 + 14 x + 45 = \left(x + 5\right) \left(x + 9\right)$
	}
\end{bt}
%%%=========Bai_89=========%%%
\begin{bt}
	Phân tích đa thức sau thành nhân tử: $x^2 + 13 x + 40$.
	\loigiai{ Ta có: $x^2 + 13 x + 40 = \left(x + 5\right) \left(x + 8\right)$
	}
\end{bt}
%%%=========Bai_90=========%%%
\begin{bt}
	Phân tích đa thức sau thành nhân tử: $x^2 + 12 x + 35$.
	\loigiai{ Ta có: $x^2 + 12 x + 35 = \left(x + 5\right) \left(x + 7\right)$
	}
\end{bt}
%%%=========Bai_91=========%%%
\begin{bt}
	Phân tích đa thức sau thành nhân tử: $x^2 + 11 x + 30$.
	\loigiai{ Ta có: $x^2 + 11 x + 30 = \left(x + 5\right) \left(x + 6\right)$
	}
\end{bt}
%%%=========Bai_92=========%%%
\begin{bt}
	Phân tích đa thức sau thành nhân tử: $x^2 + 9 x + 20$.
	\loigiai{ Ta có: $x^2 + 9 x + 20 = \left(x + 4\right) \left(x + 5\right)$
	}
\end{bt}
%%%=========Bai_93=========%%%
\begin{bt}
	Phân tích đa thức sau thành nhân tử: $x^2 + 8 x + 15$.
	\loigiai{ Ta có: $x^2 + 8 x + 15 = \left(x + 3\right) \left(x + 5\right)$
	}
\end{bt}
%%%=========Bai_94=========%%%
\begin{bt}
	Phân tích đa thức sau thành nhân tử: $x^2 + 7 x + 10$.
	\loigiai{ Ta có: $x^2 + 7 x + 10 = \left(x + 2\right) \left(x + 5\right)$
	}
\end{bt}
%%%=========Bai_95=========%%%
\begin{bt}
	Phân tích đa thức sau thành nhân tử: $x^2 + 6 x + 5$.
	\loigiai{ Ta có: $x^2 + 6 x + 5 = \left(x + 1\right) \left(x + 5\right)$
	}
\end{bt}
%%%=========Bai_96=========%%%
\begin{bt}
	Phân tích đa thức sau thành nhân tử: $x^2 + 4 x - 5$.
	\loigiai{ Ta có: $x^2 + 4 x - 5 = \left(x - 1\right) \left(x + 5\right)$
	}
\end{bt}
%%%=========Bai_97=========%%%
\begin{bt}
	Phân tích đa thức sau thành nhân tử: $x^2 + 3 x - 10$.
	\loigiai{ Ta có: $x^2 + 3 x - 10 = \left(x - 2\right) \left(x + 5\right)$
	}
\end{bt}
%%%=========Bai_98=========%%%
\begin{bt}
	Phân tích đa thức sau thành nhân tử: $x^2 + 2 x - 15$.
	\loigiai{ Ta có: $x^2 + 2 x - 15 = \left(x - 3\right) \left(x + 5\right)$
	}
\end{bt}
%%%=========Bai_99=========%%%
\begin{bt}
	Phân tích đa thức sau thành nhân tử: $x^2 + x - 20$.
	\loigiai{ Ta có: $x^2 + x - 20 = \left(x - 4\right) \left(x + 5\right)$
	}
\end{bt}
%%%=========Bai_100=========%%%
\begin{bt}
	Phân tích đa thức sau thành nhân tử: $x^2 - x - 30$.
	\loigiai{ Ta có: $x^2 - x - 30 = \left(x - 6\right) \left(x + 5\right)$
	}
\end{bt}
%%%=========Bai_101=========%%%
\begin{bt}
	Phân tích đa thức sau thành nhân tử: $x^2 - 2 x - 35$.
	\loigiai{ Ta có: $x^2 - 2 x - 35 = \left(x - 7\right) \left(x + 5\right)$
	}
\end{bt}
%%%=========Bai_102=========%%%
\begin{bt}
	Phân tích đa thức sau thành nhân tử: $x^2 - 3 x - 40$.
	\loigiai{ Ta có: $x^2 - 3 x - 40 = \left(x - 8\right) \left(x + 5\right)$
	}
\end{bt}
%%%=========Bai_103=========%%%
\begin{bt}
	Phân tích đa thức sau thành nhân tử: $x^2 - 4 x - 45$.
	\loigiai{ Ta có: $x^2 - 4 x - 45 = \left(x - 9\right) \left(x + 5\right)$
	}
\end{bt}
%%%=========Bai_104=========%%%
\begin{bt}
	Phân tích đa thức sau thành nhân tử: $x^2 + 14 x + 40$.
	\loigiai{ Ta có: $x^2 + 14 x + 40 = \left(x + 4\right) \left(x + 10\right)$
	}
\end{bt}
%%%=========Bai_105=========%%%
\begin{bt}
	Phân tích đa thức sau thành nhân tử: $x^2 + 13 x + 36$.
	\loigiai{ Ta có: $x^2 + 13 x + 36 = \left(x + 4\right) \left(x + 9\right)$
	}
\end{bt}
%%%=========Bai_106=========%%%
\begin{bt}
	Phân tích đa thức sau thành nhân tử: $x^2 + 12 x + 32$.
	\loigiai{ Ta có: $x^2 + 12 x + 32 = \left(x + 4\right) \left(x + 8\right)$
	}
\end{bt}
%%%=========Bai_107=========%%%
\begin{bt}
	Phân tích đa thức sau thành nhân tử: $x^2 + 11 x + 28$.
	\loigiai{ Ta có: $x^2 + 11 x + 28 = \left(x + 4\right) \left(x + 7\right)$
	}
\end{bt}
%%%=========Bai_108=========%%%
\begin{bt}
	Phân tích đa thức sau thành nhân tử: $x^2 + 10 x + 24$.
	\loigiai{ Ta có: $x^2 + 10 x + 24 = \left(x + 4\right) \left(x + 6\right)$
	}
\end{bt}
%%%=========Bai_109=========%%%
\begin{bt}
	Phân tích đa thức sau thành nhân tử: $x^2 + 9 x + 20$.
	\loigiai{ Ta có: $x^2 + 9 x + 20 = \left(x + 4\right) \left(x + 5\right)$
	}
\end{bt}
%%%=========Bai_110=========%%%
\begin{bt}
	Phân tích đa thức sau thành nhân tử: $x^2 + 7 x + 12$.
	\loigiai{ Ta có: $x^2 + 7 x + 12 = \left(x + 3\right) \left(x + 4\right)$
	}
\end{bt}
%%%=========Bai_111=========%%%
\begin{bt}
	Phân tích đa thức sau thành nhân tử: $x^2 + 6 x + 8$.
	\loigiai{ Ta có: $x^2 + 6 x + 8 = \left(x + 2\right) \left(x + 4\right)$
	}
\end{bt}
%%%=========Bai_112=========%%%
\begin{bt}
	Phân tích đa thức sau thành nhân tử: $x^2 + 5 x + 4$.
	\loigiai{ Ta có: $x^2 + 5 x + 4 = \left(x + 1\right) \left(x + 4\right)$
	}
\end{bt}
%%%=========Bai_113=========%%%
\begin{bt}
	Phân tích đa thức sau thành nhân tử: $x^2 + 3 x - 4$.
	\loigiai{ Ta có: $x^2 + 3 x - 4 = \left(x - 1\right) \left(x + 4\right)$
	}
\end{bt}
%%%=========Bai_114=========%%%
\begin{bt}
	Phân tích đa thức sau thành nhân tử: $x^2 + 2 x - 8$.
	\loigiai{ Ta có: $x^2 + 2 x - 8 = \left(x - 2\right) \left(x + 4\right)$
	}
\end{bt}
%%%=========Bai_115=========%%%
\begin{bt}
	Phân tích đa thức sau thành nhân tử: $x^2 + x - 12$.
	\loigiai{ Ta có: $x^2 + x - 12 = \left(x - 3\right) \left(x + 4\right)$
	}
\end{bt}
%%%=========Bai_116=========%%%
\begin{bt}
	Phân tích đa thức sau thành nhân tử: $x^2 - x - 20$.
	\loigiai{ Ta có: $x^2 - x - 20 = \left(x - 5\right) \left(x + 4\right)$
	}
\end{bt}
%%%=========Bai_117=========%%%
\begin{bt}
	Phân tích đa thức sau thành nhân tử: $x^2 - 2 x - 24$.
	\loigiai{ Ta có: $x^2 - 2 x - 24 = \left(x - 6\right) \left(x + 4\right)$
	}
\end{bt}
%%%=========Bai_118=========%%%
\begin{bt}
	Phân tích đa thức sau thành nhân tử: $x^2 - 3 x - 28$.
	\loigiai{ Ta có: $x^2 - 3 x - 28 = \left(x - 7\right) \left(x + 4\right)$
	}
\end{bt}
%%%=========Bai_119=========%%%
\begin{bt}
	Phân tích đa thức sau thành nhân tử: $x^2 - 4 x - 32$.
	\loigiai{ Ta có: $x^2 - 4 x - 32 = \left(x - 8\right) \left(x + 4\right)$
	}
\end{bt}
%%%=========Bai_120=========%%%
\begin{bt}
	Phân tích đa thức sau thành nhân tử: $x^2 - 5 x - 36$.
	\loigiai{ Ta có: $x^2 - 5 x - 36 = \left(x - 9\right) \left(x + 4\right)$
	}
\end{bt}
%%%=========Bai_121=========%%%
\begin{bt}
	Phân tích đa thức sau thành nhân tử: $x^2 + 13 x + 30$.
	\loigiai{ Ta có: $x^2 + 13 x + 30 = \left(x + 3\right) \left(x + 10\right)$
	}
\end{bt}
%%%=========Bai_122=========%%%
\begin{bt}
	Phân tích đa thức sau thành nhân tử: $x^2 + 12 x + 27$.
	\loigiai{ Ta có: $x^2 + 12 x + 27 = \left(x + 3\right) \left(x + 9\right)$
	}
\end{bt}
%%%=========Bai_123=========%%%
\begin{bt}
	Phân tích đa thức sau thành nhân tử: $x^2 + 11 x + 24$.
	\loigiai{ Ta có: $x^2 + 11 x + 24 = \left(x + 3\right) \left(x + 8\right)$
	}
\end{bt}
%%%=========Bai_124=========%%%
\begin{bt}
	Phân tích đa thức sau thành nhân tử: $x^2 + 10 x + 21$.
	\loigiai{ Ta có: $x^2 + 10 x + 21 = \left(x + 3\right) \left(x + 7\right)$
	}
\end{bt}
%%%=========Bai_125=========%%%
\begin{bt}
	Phân tích đa thức sau thành nhân tử: $x^2 + 9 x + 18$.
	\loigiai{ Ta có: $x^2 + 9 x + 18 = \left(x + 3\right) \left(x + 6\right)$
	}
\end{bt}
%%%=========Bai_126=========%%%
\begin{bt}
	Phân tích đa thức sau thành nhân tử: $x^2 + 8 x + 15$.
	\loigiai{ Ta có: $x^2 + 8 x + 15 = \left(x + 3\right) \left(x + 5\right)$
	}
\end{bt}
%%%=========Bai_127=========%%%
\begin{bt}
	Phân tích đa thức sau thành nhân tử: $x^2 + 7 x + 12$.
	\loigiai{ Ta có: $x^2 + 7 x + 12 = \left(x + 3\right) \left(x + 4\right)$
	}
\end{bt}
%%%=========Bai_128=========%%%
\begin{bt}
	Phân tích đa thức sau thành nhân tử: $x^2 + 5 x + 6$.
	\loigiai{ Ta có: $x^2 + 5 x + 6 = \left(x + 2\right) \left(x + 3\right)$
	}
\end{bt}
%%%=========Bai_129=========%%%
\begin{bt}
	Phân tích đa thức sau thành nhân tử: $x^2 + 4 x + 3$.
	\loigiai{ Ta có: $x^2 + 4 x + 3 = \left(x + 1\right) \left(x + 3\right)$
	}
\end{bt}
%%%=========Bai_130=========%%%
\begin{bt}
	Phân tích đa thức sau thành nhân tử: $x^2 + 2 x - 3$.
	\loigiai{ Ta có: $x^2 + 2 x - 3 = \left(x - 1\right) \left(x + 3\right)$
	}
\end{bt}
%%%=========Bai_131=========%%%
\begin{bt}
	Phân tích đa thức sau thành nhân tử: $x^2 + x - 6$.
	\loigiai{ Ta có: $x^2 + x - 6 = \left(x - 2\right) \left(x + 3\right)$
	}
\end{bt}
%%%=========Bai_132=========%%%
\begin{bt}
	Phân tích đa thức sau thành nhân tử: $x^2 - x - 12$.
	\loigiai{ Ta có: $x^2 - x - 12 = \left(x - 4\right) \left(x + 3\right)$
	}
\end{bt}
%%%=========Bai_133=========%%%
\begin{bt}
	Phân tích đa thức sau thành nhân tử: $x^2 - 2 x - 15$.
	\loigiai{ Ta có: $x^2 - 2 x - 15 = \left(x - 5\right) \left(x + 3\right)$
	}
\end{bt}
%%%=========Bai_134=========%%%
\begin{bt}
	Phân tích đa thức sau thành nhân tử: $x^2 - 3 x - 18$.
	\loigiai{ Ta có: $x^2 - 3 x - 18 = \left(x - 6\right) \left(x + 3\right)$
	}
\end{bt}
%%%=========Bai_135=========%%%
\begin{bt}
	Phân tích đa thức sau thành nhân tử: $x^2 - 4 x - 21$.
	\loigiai{ Ta có: $x^2 - 4 x - 21 = \left(x - 7\right) \left(x + 3\right)$
	}
\end{bt}
%%%=========Bai_136=========%%%
\begin{bt}
	Phân tích đa thức sau thành nhân tử: $x^2 - 5 x - 24$.
	\loigiai{ Ta có: $x^2 - 5 x - 24 = \left(x - 8\right) \left(x + 3\right)$
	}
\end{bt}
%%%=========Bai_137=========%%%
\begin{bt}
	Phân tích đa thức sau thành nhân tử: $x^2 - 6 x - 27$.
	\loigiai{ Ta có: $x^2 - 6 x - 27 = \left(x - 9\right) \left(x + 3\right)$
	}
\end{bt}
%%%=========Bai_138=========%%%
\begin{bt}
	Phân tích đa thức sau thành nhân tử: $x^2 + 12 x + 20$.
	\loigiai{ Ta có: $x^2 + 12 x + 20 = \left(x + 2\right) \left(x + 10\right)$
	}
\end{bt}
%%%=========Bai_139=========%%%
\begin{bt}
	Phân tích đa thức sau thành nhân tử: $x^2 + 11 x + 18$.
	\loigiai{ Ta có: $x^2 + 11 x + 18 = \left(x + 2\right) \left(x + 9\right)$
	}
\end{bt}
%%%=========Bai_140=========%%%
\begin{bt}
	Phân tích đa thức sau thành nhân tử: $x^2 + 10 x + 16$.
	\loigiai{ Ta có: $x^2 + 10 x + 16 = \left(x + 2\right) \left(x + 8\right)$
	}
\end{bt}
%%%=========Bai_141=========%%%
\begin{bt}
	Phân tích đa thức sau thành nhân tử: $x^2 + 9 x + 14$.
	\loigiai{ Ta có: $x^2 + 9 x + 14 = \left(x + 2\right) \left(x + 7\right)$
	}
\end{bt}
%%%=========Bai_142=========%%%
\begin{bt}
	Phân tích đa thức sau thành nhân tử: $x^2 + 8 x + 12$.
	\loigiai{ Ta có: $x^2 + 8 x + 12 = \left(x + 2\right) \left(x + 6\right)$
	}
\end{bt}
%%%=========Bai_143=========%%%
\begin{bt}
	Phân tích đa thức sau thành nhân tử: $x^2 + 7 x + 10$.
	\loigiai{ Ta có: $x^2 + 7 x + 10 = \left(x + 2\right) \left(x + 5\right)$
	}
\end{bt}
%%%=========Bai_144=========%%%
\begin{bt}
	Phân tích đa thức sau thành nhân tử: $x^2 + 6 x + 8$.
	\loigiai{ Ta có: $x^2 + 6 x + 8 = \left(x + 2\right) \left(x + 4\right)$
	}
\end{bt}
%%%=========Bai_145=========%%%
\begin{bt}
	Phân tích đa thức sau thành nhân tử: $x^2 + 5 x + 6$.
	\loigiai{ Ta có: $x^2 + 5 x + 6 = \left(x + 2\right) \left(x + 3\right)$
	}
\end{bt}
%%%=========Bai_146=========%%%
\begin{bt}
	Phân tích đa thức sau thành nhân tử: $x^2 + 3 x + 2$.
	\loigiai{ Ta có: $x^2 + 3 x + 2 = \left(x + 1\right) \left(x + 2\right)$
	}
\end{bt}
%%%=========Bai_147=========%%%
\begin{bt}
	Phân tích đa thức sau thành nhân tử: $x^2 + x - 2$.
	\loigiai{ Ta có: $x^2 + x - 2 = \left(x - 1\right) \left(x + 2\right)$
	}
\end{bt}
%%%=========Bai_148=========%%%
\begin{bt}
	Phân tích đa thức sau thành nhân tử: $x^2 - x - 6$.
	\loigiai{ Ta có: $x^2 - x - 6 = \left(x - 3\right) \left(x + 2\right)$
	}
\end{bt}
%%%=========Bai_149=========%%%
\begin{bt}
	Phân tích đa thức sau thành nhân tử: $x^2 - 2 x - 8$.
	\loigiai{ Ta có: $x^2 - 2 x - 8 = \left(x - 4\right) \left(x + 2\right)$
	}
\end{bt}
%%%=========Bai_150=========%%%
\begin{bt}
	Phân tích đa thức sau thành nhân tử: $x^2 - 3 x - 10$.
	\loigiai{ Ta có: $x^2 - 3 x - 10 = \left(x - 5\right) \left(x + 2\right)$
	}
\end{bt}
%%%=========Bai_151=========%%%
\begin{bt}
	Phân tích đa thức sau thành nhân tử: $x^2 - 4 x - 12$.
	\loigiai{ Ta có: $x^2 - 4 x - 12 = \left(x - 6\right) \left(x + 2\right)$
	}
\end{bt}
%%%=========Bai_152=========%%%
\begin{bt}
	Phân tích đa thức sau thành nhân tử: $x^2 - 5 x - 14$.
	\loigiai{ Ta có: $x^2 - 5 x - 14 = \left(x - 7\right) \left(x + 2\right)$
	}
\end{bt}
%%%=========Bai_153=========%%%
\begin{bt}
	Phân tích đa thức sau thành nhân tử: $x^2 - 6 x - 16$.
	\loigiai{ Ta có: $x^2 - 6 x - 16 = \left(x - 8\right) \left(x + 2\right)$
	}
\end{bt}
%%%=========Bai_154=========%%%
\begin{bt}
	Phân tích đa thức sau thành nhân tử: $x^2 - 7 x - 18$.
	\loigiai{ Ta có: $x^2 - 7 x - 18 = \left(x - 9\right) \left(x + 2\right)$
	}
\end{bt}
%%%=========Bai_155=========%%%
\begin{bt}
	Phân tích đa thức sau thành nhân tử: $x^2 + 11 x + 10$.
	\loigiai{ Ta có: $x^2 + 11 x + 10 = \left(x + 1\right) \left(x + 10\right)$
	}
\end{bt}
%%%=========Bai_156=========%%%
\begin{bt}
	Phân tích đa thức sau thành nhân tử: $x^2 + 10 x + 9$.
	\loigiai{ Ta có: $x^2 + 10 x + 9 = \left(x + 1\right) \left(x + 9\right)$
	}
\end{bt}
%%%=========Bai_157=========%%%
\begin{bt}
	Phân tích đa thức sau thành nhân tử: $x^2 + 9 x + 8$.
	\loigiai{ Ta có: $x^2 + 9 x + 8 = \left(x + 1\right) \left(x + 8\right)$
	}
\end{bt}
%%%=========Bai_158=========%%%
\begin{bt}
	Phân tích đa thức sau thành nhân tử: $x^2 + 8 x + 7$.
	\loigiai{ Ta có: $x^2 + 8 x + 7 = \left(x + 1\right) \left(x + 7\right)$
	}
\end{bt}
%%%=========Bai_159=========%%%
\begin{bt}
	Phân tích đa thức sau thành nhân tử: $x^2 + 7 x + 6$.
	\loigiai{ Ta có: $x^2 + 7 x + 6 = \left(x + 1\right) \left(x + 6\right)$
	}
\end{bt}
%%%=========Bai_160=========%%%
\begin{bt}
	Phân tích đa thức sau thành nhân tử: $x^2 + 6 x + 5$.
	\loigiai{ Ta có: $x^2 + 6 x + 5 = \left(x + 1\right) \left(x + 5\right)$
	}
\end{bt}
%%%=========Bai_161=========%%%
\begin{bt}
	Phân tích đa thức sau thành nhân tử: $x^2 + 5 x + 4$.
	\loigiai{ Ta có: $x^2 + 5 x + 4 = \left(x + 1\right) \left(x + 4\right)$
	}
\end{bt}
%%%=========Bai_162=========%%%
\begin{bt}
	Phân tích đa thức sau thành nhân tử: $x^2 + 4 x + 3$.
	\loigiai{ Ta có: $x^2 + 4 x + 3 = \left(x + 1\right) \left(x + 3\right)$
	}
\end{bt}
%%%=========Bai_163=========%%%
\begin{bt}
	Phân tích đa thức sau thành nhân tử: $x^2 + 3 x + 2$.
	\loigiai{ Ta có: $x^2 + 3 x + 2 = \left(x + 1\right) \left(x + 2\right)$
	}
\end{bt}
%%%=========Bai_164=========%%%
\begin{bt}
	Phân tích đa thức sau thành nhân tử: $x^2 - x - 2$.
	\loigiai{ Ta có: $x^2 - x - 2 = \left(x - 2\right) \left(x + 1\right)$
	}
\end{bt}
%%%=========Bai_165=========%%%
\begin{bt}
	Phân tích đa thức sau thành nhân tử: $x^2 - 2 x - 3$.
	\loigiai{ Ta có: $x^2 - 2 x - 3 = \left(x - 3\right) \left(x + 1\right)$
	}
\end{bt}
%%%=========Bai_166=========%%%
\begin{bt}
	Phân tích đa thức sau thành nhân tử: $x^2 - 3 x - 4$.
	\loigiai{ Ta có: $x^2 - 3 x - 4 = \left(x - 4\right) \left(x + 1\right)$
	}
\end{bt}
%%%=========Bai_167=========%%%
\begin{bt}
	Phân tích đa thức sau thành nhân tử: $x^2 - 4 x - 5$.
	\loigiai{ Ta có: $x^2 - 4 x - 5 = \left(x - 5\right) \left(x + 1\right)$
	}
\end{bt}
%%%=========Bai_168=========%%%
\begin{bt}
	Phân tích đa thức sau thành nhân tử: $x^2 - 5 x - 6$.
	\loigiai{ Ta có: $x^2 - 5 x - 6 = \left(x - 6\right) \left(x + 1\right)$
	}
\end{bt}
%%%=========Bai_169=========%%%
\begin{bt}
	Phân tích đa thức sau thành nhân tử: $x^2 - 6 x - 7$.
	\loigiai{ Ta có: $x^2 - 6 x - 7 = \left(x - 7\right) \left(x + 1\right)$
	}
\end{bt}
%%%=========Bai_170=========%%%
\begin{bt}
	Phân tích đa thức sau thành nhân tử: $x^2 - 7 x - 8$.
	\loigiai{ Ta có: $x^2 - 7 x - 8 = \left(x - 8\right) \left(x + 1\right)$
	}
\end{bt}
%%%=========Bai_171=========%%%
\begin{bt}
	Phân tích đa thức sau thành nhân tử: $x^2 - 8 x - 9$.
	\loigiai{ Ta có: $x^2 - 8 x - 9 = \left(x - 9\right) \left(x + 1\right)$
	}
\end{bt}
%%%=========Bai_172=========%%%
\begin{bt}
	Phân tích đa thức sau thành nhân tử: $x^2 + 9 x - 10$.
	\loigiai{ Ta có: $x^2 + 9 x - 10 = \left(x - 1\right) \left(x + 10\right)$
	}
\end{bt}
%%%=========Bai_173=========%%%
\begin{bt}
	Phân tích đa thức sau thành nhân tử: $x^2 + 8 x - 9$.
	\loigiai{ Ta có: $x^2 + 8 x - 9 = \left(x - 1\right) \left(x + 9\right)$
	}
\end{bt}
%%%=========Bai_174=========%%%
\begin{bt}
	Phân tích đa thức sau thành nhân tử: $x^2 + 7 x - 8$.
	\loigiai{ Ta có: $x^2 + 7 x - 8 = \left(x - 1\right) \left(x + 8\right)$
	}
\end{bt}
%%%=========Bai_175=========%%%
\begin{bt}
	Phân tích đa thức sau thành nhân tử: $x^2 + 6 x - 7$.
	\loigiai{ Ta có: $x^2 + 6 x - 7 = \left(x - 1\right) \left(x + 7\right)$
	}
\end{bt}
%%%=========Bai_176=========%%%
\begin{bt}
	Phân tích đa thức sau thành nhân tử: $x^2 + 5 x - 6$.
	\loigiai{ Ta có: $x^2 + 5 x - 6 = \left(x - 1\right) \left(x + 6\right)$
	}
\end{bt}
%%%=========Bai_177=========%%%
\begin{bt}
	Phân tích đa thức sau thành nhân tử: $x^2 + 4 x - 5$.
	\loigiai{ Ta có: $x^2 + 4 x - 5 = \left(x - 1\right) \left(x + 5\right)$
	}
\end{bt}
%%%=========Bai_178=========%%%
\begin{bt}
	Phân tích đa thức sau thành nhân tử: $x^2 + 3 x - 4$.
	\loigiai{ Ta có: $x^2 + 3 x - 4 = \left(x - 1\right) \left(x + 4\right)$
	}
\end{bt}
%%%=========Bai_179=========%%%
\begin{bt}
	Phân tích đa thức sau thành nhân tử: $x^2 + 2 x - 3$.
	\loigiai{ Ta có: $x^2 + 2 x - 3 = \left(x - 1\right) \left(x + 3\right)$
	}
\end{bt}
%%%=========Bai_180=========%%%
\begin{bt}
	Phân tích đa thức sau thành nhân tử: $x^2 + x - 2$.
	\loigiai{ Ta có: $x^2 + x - 2 = \left(x - 1\right) \left(x + 2\right)$
	}
\end{bt}
%%%=========Bai_181=========%%%
\begin{bt}
	Phân tích đa thức sau thành nhân tử: $x^2 - 3 x + 2$.
	\loigiai{ Ta có: $x^2 - 3 x + 2 = \left(x - 2\right) \left(x - 1\right)$
	}
\end{bt}
%%%=========Bai_182=========%%%
\begin{bt}
	Phân tích đa thức sau thành nhân tử: $x^2 - 4 x + 3$.
	\loigiai{ Ta có: $x^2 - 4 x + 3 = \left(x - 3\right) \left(x - 1\right)$
	}
\end{bt}
%%%=========Bai_183=========%%%
\begin{bt}
	Phân tích đa thức sau thành nhân tử: $x^2 - 5 x + 4$.
	\loigiai{ Ta có: $x^2 - 5 x + 4 = \left(x - 4\right) \left(x - 1\right)$
	}
\end{bt}
%%%=========Bai_184=========%%%
\begin{bt}
	Phân tích đa thức sau thành nhân tử: $x^2 - 6 x + 5$.
	\loigiai{ Ta có: $x^2 - 6 x + 5 = \left(x - 5\right) \left(x - 1\right)$
	}
\end{bt}
%%%=========Bai_185=========%%%
\begin{bt}
	Phân tích đa thức sau thành nhân tử: $x^2 - 7 x + 6$.
	\loigiai{ Ta có: $x^2 - 7 x + 6 = \left(x - 6\right) \left(x - 1\right)$
	}
\end{bt}
%%%=========Bai_186=========%%%
\begin{bt}
	Phân tích đa thức sau thành nhân tử: $x^2 - 8 x + 7$.
	\loigiai{ Ta có: $x^2 - 8 x + 7 = \left(x - 7\right) \left(x - 1\right)$
	}
\end{bt}
%%%=========Bai_187=========%%%
\begin{bt}
	Phân tích đa thức sau thành nhân tử: $x^2 - 9 x + 8$.
	\loigiai{ Ta có: $x^2 - 9 x + 8 = \left(x - 8\right) \left(x - 1\right)$
	}
\end{bt}
%%%=========Bai_188=========%%%
\begin{bt}
	Phân tích đa thức sau thành nhân tử: $x^2 - 10 x + 9$.
	\loigiai{ Ta có: $x^2 - 10 x + 9 = \left(x - 9\right) \left(x - 1\right)$
	}
\end{bt}
%%%=========Bai_189=========%%%
\begin{bt}
	Phân tích đa thức sau thành nhân tử: $x^2 + 8 x - 20$.
	\loigiai{ Ta có: $x^2 + 8 x - 20 = \left(x - 2\right) \left(x + 10\right)$
	}
\end{bt}
%%%=========Bai_190=========%%%
\begin{bt}
	Phân tích đa thức sau thành nhân tử: $x^2 + 7 x - 18$.
	\loigiai{ Ta có: $x^2 + 7 x - 18 = \left(x - 2\right) \left(x + 9\right)$
	}
\end{bt}
%%%=========Bai_191=========%%%
\begin{bt}
	Phân tích đa thức sau thành nhân tử: $x^2 + 6 x - 16$.
	\loigiai{ Ta có: $x^2 + 6 x - 16 = \left(x - 2\right) \left(x + 8\right)$
	}
\end{bt}
%%%=========Bai_192=========%%%
\begin{bt}
	Phân tích đa thức sau thành nhân tử: $x^2 + 5 x - 14$.
	\loigiai{ Ta có: $x^2 + 5 x - 14 = \left(x - 2\right) \left(x + 7\right)$
	}
\end{bt}
%%%=========Bai_193=========%%%
\begin{bt}
	Phân tích đa thức sau thành nhân tử: $x^2 + 4 x - 12$.
	\loigiai{ Ta có: $x^2 + 4 x - 12 = \left(x - 2\right) \left(x + 6\right)$
	}
\end{bt}
%%%=========Bai_194=========%%%
\begin{bt}
	Phân tích đa thức sau thành nhân tử: $x^2 + 3 x - 10$.
	\loigiai{ Ta có: $x^2 + 3 x - 10 = \left(x - 2\right) \left(x + 5\right)$
	}
\end{bt}
%%%=========Bai_195=========%%%
\begin{bt}
	Phân tích đa thức sau thành nhân tử: $x^2 + 2 x - 8$.
	\loigiai{ Ta có: $x^2 + 2 x - 8 = \left(x - 2\right) \left(x + 4\right)$
	}
\end{bt}
%%%=========Bai_196=========%%%
\begin{bt}
	Phân tích đa thức sau thành nhân tử: $x^2 + x - 6$.
	\loigiai{ Ta có: $x^2 + x - 6 = \left(x - 2\right) \left(x + 3\right)$
	}
\end{bt}
%%%=========Bai_197=========%%%
\begin{bt}
	Phân tích đa thức sau thành nhân tử: $x^2 - x - 2$.
	\loigiai{ Ta có: $x^2 - x - 2 = \left(x - 2\right) \left(x + 1\right)$
	}
\end{bt}
%%%=========Bai_198=========%%%
\begin{bt}
	Phân tích đa thức sau thành nhân tử: $x^2 - 3 x + 2$.
	\loigiai{ Ta có: $x^2 - 3 x + 2 = \left(x - 2\right) \left(x - 1\right)$
	}
\end{bt}
%%%=========Bai_199=========%%%
\begin{bt}
	Phân tích đa thức sau thành nhân tử: $x^2 - 5 x + 6$.
	\loigiai{ Ta có: $x^2 - 5 x + 6 = \left(x - 3\right) \left(x - 2\right)$
	}
\end{bt}
%%%=========Bai_200=========%%%
\begin{bt}
	Phân tích đa thức sau thành nhân tử: $x^2 - 6 x + 8$.
	\loigiai{ Ta có: $x^2 - 6 x + 8 = \left(x - 4\right) \left(x - 2\right)$
	}
\end{bt}
%%%=========Bai_201=========%%%
\begin{bt}
	Phân tích đa thức sau thành nhân tử: $x^2 - 7 x + 10$.
	\loigiai{ Ta có: $x^2 - 7 x + 10 = \left(x - 5\right) \left(x - 2\right)$
	}
\end{bt}
%%%=========Bai_202=========%%%
\begin{bt}
	Phân tích đa thức sau thành nhân tử: $x^2 - 8 x + 12$.
	\loigiai{ Ta có: $x^2 - 8 x + 12 = \left(x - 6\right) \left(x - 2\right)$
	}
\end{bt}
%%%=========Bai_203=========%%%
\begin{bt}
	Phân tích đa thức sau thành nhân tử: $x^2 - 9 x + 14$.
	\loigiai{ Ta có: $x^2 - 9 x + 14 = \left(x - 7\right) \left(x - 2\right)$
	}
\end{bt}
%%%=========Bai_204=========%%%
\begin{bt}
	Phân tích đa thức sau thành nhân tử: $x^2 - 10 x + 16$.
	\loigiai{ Ta có: $x^2 - 10 x + 16 = \left(x - 8\right) \left(x - 2\right)$
	}
\end{bt}
%%%=========Bai_205=========%%%
\begin{bt}
	Phân tích đa thức sau thành nhân tử: $x^2 - 11 x + 18$.
	\loigiai{ Ta có: $x^2 - 11 x + 18 = \left(x - 9\right) \left(x - 2\right)$
	}
\end{bt}
%%%=========Bai_206=========%%%
\begin{bt}
	Phân tích đa thức sau thành nhân tử: $x^2 + 7 x - 30$.
	\loigiai{ Ta có: $x^2 + 7 x - 30 = \left(x - 3\right) \left(x + 10\right)$
	}
\end{bt}
%%%=========Bai_207=========%%%
\begin{bt}
	Phân tích đa thức sau thành nhân tử: $x^2 + 6 x - 27$.
	\loigiai{ Ta có: $x^2 + 6 x - 27 = \left(x - 3\right) \left(x + 9\right)$
	}
\end{bt}
%%%=========Bai_208=========%%%
\begin{bt}
	Phân tích đa thức sau thành nhân tử: $x^2 + 5 x - 24$.
	\loigiai{ Ta có: $x^2 + 5 x - 24 = \left(x - 3\right) \left(x + 8\right)$
	}
\end{bt}
%%%=========Bai_209=========%%%
\begin{bt}
	Phân tích đa thức sau thành nhân tử: $x^2 + 4 x - 21$.
	\loigiai{ Ta có: $x^2 + 4 x - 21 = \left(x - 3\right) \left(x + 7\right)$
	}
\end{bt}
%%%=========Bai_210=========%%%
\begin{bt}
	Phân tích đa thức sau thành nhân tử: $x^2 + 3 x - 18$.
	\loigiai{ Ta có: $x^2 + 3 x - 18 = \left(x - 3\right) \left(x + 6\right)$
	}
\end{bt}
%%%=========Bai_211=========%%%
\begin{bt}
	Phân tích đa thức sau thành nhân tử: $x^2 + 2 x - 15$.
	\loigiai{ Ta có: $x^2 + 2 x - 15 = \left(x - 3\right) \left(x + 5\right)$
	}
\end{bt}
%%%=========Bai_212=========%%%
\begin{bt}
	Phân tích đa thức sau thành nhân tử: $x^2 + x - 12$.
	\loigiai{ Ta có: $x^2 + x - 12 = \left(x - 3\right) \left(x + 4\right)$
	}
\end{bt}
%%%=========Bai_213=========%%%
\begin{bt}
	Phân tích đa thức sau thành nhân tử: $x^2 - x - 6$.
	\loigiai{ Ta có: $x^2 - x - 6 = \left(x - 3\right) \left(x + 2\right)$
	}
\end{bt}
%%%=========Bai_214=========%%%
\begin{bt}
	Phân tích đa thức sau thành nhân tử: $x^2 - 2 x - 3$.
	\loigiai{ Ta có: $x^2 - 2 x - 3 = \left(x - 3\right) \left(x + 1\right)$
	}
\end{bt}
%%%=========Bai_215=========%%%
\begin{bt}
	Phân tích đa thức sau thành nhân tử: $x^2 - 4 x + 3$.
	\loigiai{ Ta có: $x^2 - 4 x + 3 = \left(x - 3\right) \left(x - 1\right)$
	}
\end{bt}
%%%=========Bai_216=========%%%
\begin{bt}
	Phân tích đa thức sau thành nhân tử: $x^2 - 5 x + 6$.
	\loigiai{ Ta có: $x^2 - 5 x + 6 = \left(x - 3\right) \left(x - 2\right)$
	}
\end{bt}
%%%=========Bai_217=========%%%
\begin{bt}
	Phân tích đa thức sau thành nhân tử: $x^2 - 7 x + 12$.
	\loigiai{ Ta có: $x^2 - 7 x + 12 = \left(x - 4\right) \left(x - 3\right)$
	}
\end{bt}
%%%=========Bai_218=========%%%
\begin{bt}
	Phân tích đa thức sau thành nhân tử: $x^2 - 8 x + 15$.
	\loigiai{ Ta có: $x^2 - 8 x + 15 = \left(x - 5\right) \left(x - 3\right)$
	}
\end{bt}
%%%=========Bai_219=========%%%
\begin{bt}
	Phân tích đa thức sau thành nhân tử: $x^2 - 9 x + 18$.
	\loigiai{ Ta có: $x^2 - 9 x + 18 = \left(x - 6\right) \left(x - 3\right)$
	}
\end{bt}
%%%=========Bai_220=========%%%
\begin{bt}
	Phân tích đa thức sau thành nhân tử: $x^2 - 10 x + 21$.
	\loigiai{ Ta có: $x^2 - 10 x + 21 = \left(x - 7\right) \left(x - 3\right)$
	}
\end{bt}
%%%=========Bai_221=========%%%
\begin{bt}
	Phân tích đa thức sau thành nhân tử: $x^2 - 11 x + 24$.
	\loigiai{ Ta có: $x^2 - 11 x + 24 = \left(x - 8\right) \left(x - 3\right)$
	}
\end{bt}
%%%=========Bai_222=========%%%
\begin{bt}
	Phân tích đa thức sau thành nhân tử: $x^2 - 12 x + 27$.
	\loigiai{ Ta có: $x^2 - 12 x + 27 = \left(x - 9\right) \left(x - 3\right)$
	}
\end{bt}
%%%=========Bai_223=========%%%
\begin{bt}
	Phân tích đa thức sau thành nhân tử: $x^2 + 6 x - 40$.
	\loigiai{ Ta có: $x^2 + 6 x - 40 = \left(x - 4\right) \left(x + 10\right)$
	}
\end{bt}
%%%=========Bai_224=========%%%
\begin{bt}
	Phân tích đa thức sau thành nhân tử: $x^2 + 5 x - 36$.
	\loigiai{ Ta có: $x^2 + 5 x - 36 = \left(x - 4\right) \left(x + 9\right)$
	}
\end{bt}
%%%=========Bai_225=========%%%
\begin{bt}
	Phân tích đa thức sau thành nhân tử: $x^2 + 4 x - 32$.
	\loigiai{ Ta có: $x^2 + 4 x - 32 = \left(x - 4\right) \left(x + 8\right)$
	}
\end{bt}
%%%=========Bai_226=========%%%
\begin{bt}
	Phân tích đa thức sau thành nhân tử: $x^2 + 3 x - 28$.
	\loigiai{ Ta có: $x^2 + 3 x - 28 = \left(x - 4\right) \left(x + 7\right)$
	}
\end{bt}
%%%=========Bai_227=========%%%
\begin{bt}
	Phân tích đa thức sau thành nhân tử: $x^2 + 2 x - 24$.
	\loigiai{ Ta có: $x^2 + 2 x - 24 = \left(x - 4\right) \left(x + 6\right)$
	}
\end{bt}
%%%=========Bai_228=========%%%
\begin{bt}
	Phân tích đa thức sau thành nhân tử: $x^2 + x - 20$.
	\loigiai{ Ta có: $x^2 + x - 20 = \left(x - 4\right) \left(x + 5\right)$
	}
\end{bt}
%%%=========Bai_229=========%%%
\begin{bt}
	Phân tích đa thức sau thành nhân tử: $x^2 - x - 12$.
	\loigiai{ Ta có: $x^2 - x - 12 = \left(x - 4\right) \left(x + 3\right)$
	}
\end{bt}
%%%=========Bai_230=========%%%
\begin{bt}
	Phân tích đa thức sau thành nhân tử: $x^2 - 2 x - 8$.
	\loigiai{ Ta có: $x^2 - 2 x - 8 = \left(x - 4\right) \left(x + 2\right)$
	}
\end{bt}
%%%=========Bai_231=========%%%
\begin{bt}
	Phân tích đa thức sau thành nhân tử: $x^2 - 3 x - 4$.
	\loigiai{ Ta có: $x^2 - 3 x - 4 = \left(x - 4\right) \left(x + 1\right)$
	}
\end{bt}
%%%=========Bai_232=========%%%
\begin{bt}
	Phân tích đa thức sau thành nhân tử: $x^2 - 5 x + 4$.
	\loigiai{ Ta có: $x^2 - 5 x + 4 = \left(x - 4\right) \left(x - 1\right)$
	}
\end{bt}
%%%=========Bai_233=========%%%
\begin{bt}
	Phân tích đa thức sau thành nhân tử: $x^2 - 6 x + 8$.
	\loigiai{ Ta có: $x^2 - 6 x + 8 = \left(x - 4\right) \left(x - 2\right)$
	}
\end{bt}
%%%=========Bai_234=========%%%
\begin{bt}
	Phân tích đa thức sau thành nhân tử: $x^2 - 7 x + 12$.
	\loigiai{ Ta có: $x^2 - 7 x + 12 = \left(x - 4\right) \left(x - 3\right)$
	}
\end{bt}
%%%=========Bai_235=========%%%
\begin{bt}
	Phân tích đa thức sau thành nhân tử: $x^2 - 9 x + 20$.
	\loigiai{ Ta có: $x^2 - 9 x + 20 = \left(x - 5\right) \left(x - 4\right)$
	}
\end{bt}
%%%=========Bai_236=========%%%
\begin{bt}
	Phân tích đa thức sau thành nhân tử: $x^2 - 10 x + 24$.
	\loigiai{ Ta có: $x^2 - 10 x + 24 = \left(x - 6\right) \left(x - 4\right)$
	}
\end{bt}
%%%=========Bai_237=========%%%
\begin{bt}
	Phân tích đa thức sau thành nhân tử: $x^2 - 11 x + 28$.
	\loigiai{ Ta có: $x^2 - 11 x + 28 = \left(x - 7\right) \left(x - 4\right)$
	}
\end{bt}
%%%=========Bai_238=========%%%
\begin{bt}
	Phân tích đa thức sau thành nhân tử: $x^2 - 12 x + 32$.
	\loigiai{ Ta có: $x^2 - 12 x + 32 = \left(x - 8\right) \left(x - 4\right)$
	}
\end{bt}
%%%=========Bai_239=========%%%
\begin{bt}
	Phân tích đa thức sau thành nhân tử: $x^2 - 13 x + 36$.
	\loigiai{ Ta có: $x^2 - 13 x + 36 = \left(x - 9\right) \left(x - 4\right)$
	}
\end{bt}
%%%=========Bai_240=========%%%
\begin{bt}
	Phân tích đa thức sau thành nhân tử: $x^2 + 5 x - 50$.
	\loigiai{ Ta có: $x^2 + 5 x - 50 = \left(x - 5\right) \left(x + 10\right)$
	}
\end{bt}
%%%=========Bai_241=========%%%
\begin{bt}
	Phân tích đa thức sau thành nhân tử: $x^2 + 4 x - 45$.
	\loigiai{ Ta có: $x^2 + 4 x - 45 = \left(x - 5\right) \left(x + 9\right)$
	}
\end{bt}
%%%=========Bai_242=========%%%
\begin{bt}
	Phân tích đa thức sau thành nhân tử: $x^2 + 3 x - 40$.
	\loigiai{ Ta có: $x^2 + 3 x - 40 = \left(x - 5\right) \left(x + 8\right)$
	}
\end{bt}
%%%=========Bai_243=========%%%
\begin{bt}
	Phân tích đa thức sau thành nhân tử: $x^2 + 2 x - 35$.
	\loigiai{ Ta có: $x^2 + 2 x - 35 = \left(x - 5\right) \left(x + 7\right)$
	}
\end{bt}
%%%=========Bai_244=========%%%
\begin{bt}
	Phân tích đa thức sau thành nhân tử: $x^2 + x - 30$.
	\loigiai{ Ta có: $x^2 + x - 30 = \left(x - 5\right) \left(x + 6\right)$
	}
\end{bt}
%%%=========Bai_245=========%%%
\begin{bt}
	Phân tích đa thức sau thành nhân tử: $x^2 - x - 20$.
	\loigiai{ Ta có: $x^2 - x - 20 = \left(x - 5\right) \left(x + 4\right)$
	}
\end{bt}
%%%=========Bai_246=========%%%
\begin{bt}
	Phân tích đa thức sau thành nhân tử: $x^2 - 2 x - 15$.
	\loigiai{ Ta có: $x^2 - 2 x - 15 = \left(x - 5\right) \left(x + 3\right)$
	}
\end{bt}
%%%=========Bai_247=========%%%
\begin{bt}
	Phân tích đa thức sau thành nhân tử: $x^2 - 3 x - 10$.
	\loigiai{ Ta có: $x^2 - 3 x - 10 = \left(x - 5\right) \left(x + 2\right)$
	}
\end{bt}
%%%=========Bai_248=========%%%
\begin{bt}
	Phân tích đa thức sau thành nhân tử: $x^2 - 4 x - 5$.
	\loigiai{ Ta có: $x^2 - 4 x - 5 = \left(x - 5\right) \left(x + 1\right)$
	}
\end{bt}
%%%=========Bai_249=========%%%
\begin{bt}
	Phân tích đa thức sau thành nhân tử: $x^2 - 6 x + 5$.
	\loigiai{ Ta có: $x^2 - 6 x + 5 = \left(x - 5\right) \left(x - 1\right)$
	}
\end{bt}
%%%=========Bai_250=========%%%
\begin{bt}
	Phân tích đa thức sau thành nhân tử: $x^2 - 7 x + 10$.
	\loigiai{ Ta có: $x^2 - 7 x + 10 = \left(x - 5\right) \left(x - 2\right)$
	}
\end{bt}
%%%=========Bai_251=========%%%
\begin{bt}
	Phân tích đa thức sau thành nhân tử: $x^2 - 8 x + 15$.
	\loigiai{ Ta có: $x^2 - 8 x + 15 = \left(x - 5\right) \left(x - 3\right)$
	}
\end{bt}
%%%=========Bai_252=========%%%
\begin{bt}
	Phân tích đa thức sau thành nhân tử: $x^2 - 9 x + 20$.
	\loigiai{ Ta có: $x^2 - 9 x + 20 = \left(x - 5\right) \left(x - 4\right)$
	}
\end{bt}
%%%=========Bai_253=========%%%
\begin{bt}
	Phân tích đa thức sau thành nhân tử: $x^2 - 11 x + 30$.
	\loigiai{ Ta có: $x^2 - 11 x + 30 = \left(x - 6\right) \left(x - 5\right)$
	}
\end{bt}
%%%=========Bai_254=========%%%
\begin{bt}
	Phân tích đa thức sau thành nhân tử: $x^2 - 12 x + 35$.
	\loigiai{ Ta có: $x^2 - 12 x + 35 = \left(x - 7\right) \left(x - 5\right)$
	}
\end{bt}
%%%=========Bai_255=========%%%
\begin{bt}
	Phân tích đa thức sau thành nhân tử: $x^2 - 13 x + 40$.
	\loigiai{ Ta có: $x^2 - 13 x + 40 = \left(x - 8\right) \left(x - 5\right)$
	}
\end{bt}
%%%=========Bai_256=========%%%
\begin{bt}
	Phân tích đa thức sau thành nhân tử: $x^2 - 14 x + 45$.
	\loigiai{ Ta có: $x^2 - 14 x + 45 = \left(x - 9\right) \left(x - 5\right)$
	}
\end{bt}
%%%=========Bai_257=========%%%
\begin{bt}
	Phân tích đa thức sau thành nhân tử: $x^2 + 4 x - 60$.
	\loigiai{ Ta có: $x^2 + 4 x - 60 = \left(x - 6\right) \left(x + 10\right)$
	}
\end{bt}
%%%=========Bai_258=========%%%
\begin{bt}
	Phân tích đa thức sau thành nhân tử: $x^2 + 3 x - 54$.
	\loigiai{ Ta có: $x^2 + 3 x - 54 = \left(x - 6\right) \left(x + 9\right)$
	}
\end{bt}
%%%=========Bai_259=========%%%
\begin{bt}
	Phân tích đa thức sau thành nhân tử: $x^2 + 2 x - 48$.
	\loigiai{ Ta có: $x^2 + 2 x - 48 = \left(x - 6\right) \left(x + 8\right)$
	}
\end{bt}
%%%=========Bai_260=========%%%
\begin{bt}
	Phân tích đa thức sau thành nhân tử: $x^2 + x - 42$.
	\loigiai{ Ta có: $x^2 + x - 42 = \left(x - 6\right) \left(x + 7\right)$
	}
\end{bt}
%%%=========Bai_261=========%%%
\begin{bt}
	Phân tích đa thức sau thành nhân tử: $x^2 - x - 30$.
	\loigiai{ Ta có: $x^2 - x - 30 = \left(x - 6\right) \left(x + 5\right)$
	}
\end{bt}
%%%=========Bai_262=========%%%
\begin{bt}
	Phân tích đa thức sau thành nhân tử: $x^2 - 2 x - 24$.
	\loigiai{ Ta có: $x^2 - 2 x - 24 = \left(x - 6\right) \left(x + 4\right)$
	}
\end{bt}
%%%=========Bai_263=========%%%
\begin{bt}
	Phân tích đa thức sau thành nhân tử: $x^2 - 3 x - 18$.
	\loigiai{ Ta có: $x^2 - 3 x - 18 = \left(x - 6\right) \left(x + 3\right)$
	}
\end{bt}
%%%=========Bai_264=========%%%
\begin{bt}
	Phân tích đa thức sau thành nhân tử: $x^2 - 4 x - 12$.
	\loigiai{ Ta có: $x^2 - 4 x - 12 = \left(x - 6\right) \left(x + 2\right)$
	}
\end{bt}
%%%=========Bai_265=========%%%
\begin{bt}
	Phân tích đa thức sau thành nhân tử: $x^2 - 5 x - 6$.
	\loigiai{ Ta có: $x^2 - 5 x - 6 = \left(x - 6\right) \left(x + 1\right)$
	}
\end{bt}
%%%=========Bai_266=========%%%
\begin{bt}
	Phân tích đa thức sau thành nhân tử: $x^2 - 7 x + 6$.
	\loigiai{ Ta có: $x^2 - 7 x + 6 = \left(x - 6\right) \left(x - 1\right)$
	}
\end{bt}
%%%=========Bai_267=========%%%
\begin{bt}
	Phân tích đa thức sau thành nhân tử: $x^2 - 8 x + 12$.
	\loigiai{ Ta có: $x^2 - 8 x + 12 = \left(x - 6\right) \left(x - 2\right)$
	}
\end{bt}
%%%=========Bai_268=========%%%
\begin{bt}
	Phân tích đa thức sau thành nhân tử: $x^2 - 9 x + 18$.
	\loigiai{ Ta có: $x^2 - 9 x + 18 = \left(x - 6\right) \left(x - 3\right)$
	}
\end{bt}
%%%=========Bai_269=========%%%
\begin{bt}
	Phân tích đa thức sau thành nhân tử: $x^2 - 10 x + 24$.
	\loigiai{ Ta có: $x^2 - 10 x + 24 = \left(x - 6\right) \left(x - 4\right)$
	}
\end{bt}
%%%=========Bai_270=========%%%
\begin{bt}
	Phân tích đa thức sau thành nhân tử: $x^2 - 11 x + 30$.
	\loigiai{ Ta có: $x^2 - 11 x + 30 = \left(x - 6\right) \left(x - 5\right)$
	}
\end{bt}
%%%=========Bai_271=========%%%
\begin{bt}
	Phân tích đa thức sau thành nhân tử: $x^2 - 13 x + 42$.
	\loigiai{ Ta có: $x^2 - 13 x + 42 = \left(x - 7\right) \left(x - 6\right)$
	}
\end{bt}
%%%=========Bai_272=========%%%
\begin{bt}
	Phân tích đa thức sau thành nhân tử: $x^2 - 14 x + 48$.
	\loigiai{ Ta có: $x^2 - 14 x + 48 = \left(x - 8\right) \left(x - 6\right)$
	}
\end{bt}
%%%=========Bai_273=========%%%
\begin{bt}
	Phân tích đa thức sau thành nhân tử: $x^2 - 15 x + 54$.
	\loigiai{ Ta có: $x^2 - 15 x + 54 = \left(x - 9\right) \left(x - 6\right)$
	}
\end{bt}
%%%=========Bai_274=========%%%
\begin{bt}
	Phân tích đa thức sau thành nhân tử: $x^2 + 3 x - 70$.
	\loigiai{ Ta có: $x^2 + 3 x - 70 = \left(x - 7\right) \left(x + 10\right)$
	}
\end{bt}
%%%=========Bai_275=========%%%
\begin{bt}
	Phân tích đa thức sau thành nhân tử: $x^2 + 2 x - 63$.
	\loigiai{ Ta có: $x^2 + 2 x - 63 = \left(x - 7\right) \left(x + 9\right)$
	}
\end{bt}
%%%=========Bai_276=========%%%
\begin{bt}
	Phân tích đa thức sau thành nhân tử: $x^2 + x - 56$.
	\loigiai{ Ta có: $x^2 + x - 56 = \left(x - 7\right) \left(x + 8\right)$
	}
\end{bt}
%%%=========Bai_277=========%%%
\begin{bt}
	Phân tích đa thức sau thành nhân tử: $x^2 - x - 42$.
	\loigiai{ Ta có: $x^2 - x - 42 = \left(x - 7\right) \left(x + 6\right)$
	}
\end{bt}
%%%=========Bai_278=========%%%
\begin{bt}
	Phân tích đa thức sau thành nhân tử: $x^2 - 2 x - 35$.
	\loigiai{ Ta có: $x^2 - 2 x - 35 = \left(x - 7\right) \left(x + 5\right)$
	}
\end{bt}
%%%=========Bai_279=========%%%
\begin{bt}
	Phân tích đa thức sau thành nhân tử: $x^2 - 3 x - 28$.
	\loigiai{ Ta có: $x^2 - 3 x - 28 = \left(x - 7\right) \left(x + 4\right)$
	}
\end{bt}
%%%=========Bai_280=========%%%
\begin{bt}
	Phân tích đa thức sau thành nhân tử: $x^2 - 4 x - 21$.
	\loigiai{ Ta có: $x^2 - 4 x - 21 = \left(x - 7\right) \left(x + 3\right)$
	}
\end{bt}
%%%=========Bai_281=========%%%
\begin{bt}
	Phân tích đa thức sau thành nhân tử: $x^2 - 5 x - 14$.
	\loigiai{ Ta có: $x^2 - 5 x - 14 = \left(x - 7\right) \left(x + 2\right)$
	}
\end{bt}
%%%=========Bai_282=========%%%
\begin{bt}
	Phân tích đa thức sau thành nhân tử: $x^2 - 6 x - 7$.
	\loigiai{ Ta có: $x^2 - 6 x - 7 = \left(x - 7\right) \left(x + 1\right)$
	}
\end{bt}
%%%=========Bai_283=========%%%
\begin{bt}
	Phân tích đa thức sau thành nhân tử: $x^2 - 8 x + 7$.
	\loigiai{ Ta có: $x^2 - 8 x + 7 = \left(x - 7\right) \left(x - 1\right)$
	}
\end{bt}
%%%=========Bai_284=========%%%
\begin{bt}
	Phân tích đa thức sau thành nhân tử: $x^2 - 9 x + 14$.
	\loigiai{ Ta có: $x^2 - 9 x + 14 = \left(x - 7\right) \left(x - 2\right)$
	}
\end{bt}
%%%=========Bai_285=========%%%
\begin{bt}
	Phân tích đa thức sau thành nhân tử: $x^2 - 10 x + 21$.
	\loigiai{ Ta có: $x^2 - 10 x + 21 = \left(x - 7\right) \left(x - 3\right)$
	}
\end{bt}
%%%=========Bai_286=========%%%
\begin{bt}
	Phân tích đa thức sau thành nhân tử: $x^2 - 11 x + 28$.
	\loigiai{ Ta có: $x^2 - 11 x + 28 = \left(x - 7\right) \left(x - 4\right)$
	}
\end{bt}
%%%=========Bai_287=========%%%
\begin{bt}
	Phân tích đa thức sau thành nhân tử: $x^2 - 12 x + 35$.
	\loigiai{ Ta có: $x^2 - 12 x + 35 = \left(x - 7\right) \left(x - 5\right)$
	}
\end{bt}
%%%=========Bai_288=========%%%
\begin{bt}
	Phân tích đa thức sau thành nhân tử: $x^2 - 13 x + 42$.
	\loigiai{ Ta có: $x^2 - 13 x + 42 = \left(x - 7\right) \left(x - 6\right)$
	}
\end{bt}
%%%=========Bai_289=========%%%
\begin{bt}
	Phân tích đa thức sau thành nhân tử: $x^2 - 15 x + 56$.
	\loigiai{ Ta có: $x^2 - 15 x + 56 = \left(x - 8\right) \left(x - 7\right)$
	}
\end{bt}
%%%=========Bai_290=========%%%
\begin{bt}
	Phân tích đa thức sau thành nhân tử: $x^2 - 16 x + 63$.
	\loigiai{ Ta có: $x^2 - 16 x + 63 = \left(x - 9\right) \left(x - 7\right)$
	}
\end{bt}
%%%=========Bai_291=========%%%
\begin{bt}
	Phân tích đa thức sau thành nhân tử: $x^2 + 2 x - 80$.
	\loigiai{ Ta có: $x^2 + 2 x - 80 = \left(x - 8\right) \left(x + 10\right)$
	}
\end{bt}
%%%=========Bai_292=========%%%
\begin{bt}
	Phân tích đa thức sau thành nhân tử: $x^2 + x - 72$.
	\loigiai{ Ta có: $x^2 + x - 72 = \left(x - 8\right) \left(x + 9\right)$
	}
\end{bt}
%%%=========Bai_293=========%%%
\begin{bt}
	Phân tích đa thức sau thành nhân tử: $x^2 - x - 56$.
	\loigiai{ Ta có: $x^2 - x - 56 = \left(x - 8\right) \left(x + 7\right)$
	}
\end{bt}
%%%=========Bai_294=========%%%
\begin{bt}
	Phân tích đa thức sau thành nhân tử: $x^2 - 2 x - 48$.
	\loigiai{ Ta có: $x^2 - 2 x - 48 = \left(x - 8\right) \left(x + 6\right)$
	}
\end{bt}
%%%=========Bai_295=========%%%
\begin{bt}
	Phân tích đa thức sau thành nhân tử: $x^2 - 3 x - 40$.
	\loigiai{ Ta có: $x^2 - 3 x - 40 = \left(x - 8\right) \left(x + 5\right)$
	}
\end{bt}
%%%=========Bai_296=========%%%
\begin{bt}
	Phân tích đa thức sau thành nhân tử: $x^2 - 4 x - 32$.
	\loigiai{ Ta có: $x^2 - 4 x - 32 = \left(x - 8\right) \left(x + 4\right)$
	}
\end{bt}
%%%=========Bai_297=========%%%
\begin{bt}
	Phân tích đa thức sau thành nhân tử: $x^2 - 5 x - 24$.
	\loigiai{ Ta có: $x^2 - 5 x - 24 = \left(x - 8\right) \left(x + 3\right)$
	}
\end{bt}
%%%=========Bai_298=========%%%
\begin{bt}
	Phân tích đa thức sau thành nhân tử: $x^2 - 6 x - 16$.
	\loigiai{ Ta có: $x^2 - 6 x - 16 = \left(x - 8\right) \left(x + 2\right)$
	}
\end{bt}
%%%=========Bai_299=========%%%
\begin{bt}
	Phân tích đa thức sau thành nhân tử: $x^2 - 7 x - 8$.
	\loigiai{ Ta có: $x^2 - 7 x - 8 = \left(x - 8\right) \left(x + 1\right)$
	}
\end{bt}
%%%=========Bai_300=========%%%
\begin{bt}
	Phân tích đa thức sau thành nhân tử: $x^2 - 9 x + 8$.
	\loigiai{ Ta có: $x^2 - 9 x + 8 = \left(x - 8\right) \left(x - 1\right)$
	}
\end{bt}
%%%=========Bai_301=========%%%
\begin{bt}
	Phân tích đa thức sau thành nhân tử: $x^2 - 10 x + 16$.
	\loigiai{ Ta có: $x^2 - 10 x + 16 = \left(x - 8\right) \left(x - 2\right)$
	}
\end{bt}
%%%=========Bai_302=========%%%
\begin{bt}
	Phân tích đa thức sau thành nhân tử: $x^2 - 11 x + 24$.
	\loigiai{ Ta có: $x^2 - 11 x + 24 = \left(x - 8\right) \left(x - 3\right)$
	}
\end{bt}
%%%=========Bai_303=========%%%
\begin{bt}
	Phân tích đa thức sau thành nhân tử: $x^2 - 12 x + 32$.
	\loigiai{ Ta có: $x^2 - 12 x + 32 = \left(x - 8\right) \left(x - 4\right)$
	}
\end{bt}
%%%=========Bai_304=========%%%
\begin{bt}
	Phân tích đa thức sau thành nhân tử: $x^2 - 13 x + 40$.
	\loigiai{ Ta có: $x^2 - 13 x + 40 = \left(x - 8\right) \left(x - 5\right)$
	}
\end{bt}
%%%=========Bai_305=========%%%
\begin{bt}
	Phân tích đa thức sau thành nhân tử: $x^2 - 14 x + 48$.
	\loigiai{ Ta có: $x^2 - 14 x + 48 = \left(x - 8\right) \left(x - 6\right)$
	}
\end{bt}
%%%=========Bai_306=========%%%
\begin{bt}
	Phân tích đa thức sau thành nhân tử: $x^2 - 15 x + 56$.
	\loigiai{ Ta có: $x^2 - 15 x + 56 = \left(x - 8\right) \left(x - 7\right)$
	}
\end{bt}
%%%=========Bai_307=========%%%
\begin{bt}
	Phân tích đa thức sau thành nhân tử: $x^2 - 17 x + 72$.
	\loigiai{ Ta có: $x^2 - 17 x + 72 = \left(x - 9\right) \left(x - 8\right)$
	}
\end{bt}
%%%=========Bai_308=========%%%
\begin{bt}
	Phân tích đa thức sau thành nhân tử: $x^2 + x - 90$.
	\loigiai{ Ta có: $x^2 + x - 90 = \left(x - 9\right) \left(x + 10\right)$
	}
\end{bt}
%%%=========Bai_309=========%%%
\begin{bt}
	Phân tích đa thức sau thành nhân tử: $x^2 - x - 72$.
	\loigiai{ Ta có: $x^2 - x - 72 = \left(x - 9\right) \left(x + 8\right)$
	}
\end{bt}
%%%=========Bai_310=========%%%
\begin{bt}
	Phân tích đa thức sau thành nhân tử: $x^2 - 2 x - 63$.
	\loigiai{ Ta có: $x^2 - 2 x - 63 = \left(x - 9\right) \left(x + 7\right)$
	}
\end{bt}
%%%=========Bai_311=========%%%
\begin{bt}
	Phân tích đa thức sau thành nhân tử: $x^2 - 3 x - 54$.
	\loigiai{ Ta có: $x^2 - 3 x - 54 = \left(x - 9\right) \left(x + 6\right)$
	}
\end{bt}
%%%=========Bai_312=========%%%
\begin{bt}
	Phân tích đa thức sau thành nhân tử: $x^2 - 4 x - 45$.
	\loigiai{ Ta có: $x^2 - 4 x - 45 = \left(x - 9\right) \left(x + 5\right)$
	}
\end{bt}
%%%=========Bai_313=========%%%
\begin{bt}
	Phân tích đa thức sau thành nhân tử: $x^2 - 5 x - 36$.
	\loigiai{ Ta có: $x^2 - 5 x - 36 = \left(x - 9\right) \left(x + 4\right)$
	}
\end{bt}
%%%=========Bai_314=========%%%
\begin{bt}
	Phân tích đa thức sau thành nhân tử: $x^2 - 6 x - 27$.
	\loigiai{ Ta có: $x^2 - 6 x - 27 = \left(x - 9\right) \left(x + 3\right)$
	}
\end{bt}
%%%=========Bai_315=========%%%
\begin{bt}
	Phân tích đa thức sau thành nhân tử: $x^2 - 7 x - 18$.
	\loigiai{ Ta có: $x^2 - 7 x - 18 = \left(x - 9\right) \left(x + 2\right)$
	}
\end{bt}
%%%=========Bai_316=========%%%
\begin{bt}
	Phân tích đa thức sau thành nhân tử: $x^2 - 8 x - 9$.
	\loigiai{ Ta có: $x^2 - 8 x - 9 = \left(x - 9\right) \left(x + 1\right)$
	}
\end{bt}
%%%=========Bai_317=========%%%
\begin{bt}
	Phân tích đa thức sau thành nhân tử: $x^2 - 10 x + 9$.
	\loigiai{ Ta có: $x^2 - 10 x + 9 = \left(x - 9\right) \left(x - 1\right)$
	}
\end{bt}
%%%=========Bai_318=========%%%
\begin{bt}
	Phân tích đa thức sau thành nhân tử: $x^2 - 11 x + 18$.
	\loigiai{ Ta có: $x^2 - 11 x + 18 = \left(x - 9\right) \left(x - 2\right)$
	}
\end{bt}
%%%=========Bai_319=========%%%
\begin{bt}
	Phân tích đa thức sau thành nhân tử: $x^2 - 12 x + 27$.
	\loigiai{ Ta có: $x^2 - 12 x + 27 = \left(x - 9\right) \left(x - 3\right)$
	}
\end{bt}
%%%=========Bai_320=========%%%
\begin{bt}
	Phân tích đa thức sau thành nhân tử: $x^2 - 13 x + 36$.
	\loigiai{ Ta có: $x^2 - 13 x + 36 = \left(x - 9\right) \left(x - 4\right)$
	}
\end{bt}
%%%=========Bai_321=========%%%
\begin{bt}
	Phân tích đa thức sau thành nhân tử: $x^2 - 14 x + 45$.
	\loigiai{ Ta có: $x^2 - 14 x + 45 = \left(x - 9\right) \left(x - 5\right)$
	}
\end{bt}
%%%=========Bai_322=========%%%
\begin{bt}
	Phân tích đa thức sau thành nhân tử: $x^2 - 15 x + 54$.
	\loigiai{ Ta có: $x^2 - 15 x + 54 = \left(x - 9\right) \left(x - 6\right)$
	}
\end{bt}
%%%=========Bai_323=========%%%
\begin{bt}
	Phân tích đa thức sau thành nhân tử: $x^2 - 16 x + 63$.
	\loigiai{ Ta có: $x^2 - 16 x + 63 = \left(x - 9\right) \left(x - 7\right)$
	}
\end{bt}
%%%=========Bai_324=========%%%
\begin{bt}
	Phân tích đa thức sau thành nhân tử: $x^2 - 17 x + 72$.
	\loigiai{ Ta có: $x^2 - 17 x + 72 = \left(x - 9\right) \left(x - 8\right)$
	}
\end{bt}
%%%=========Bai_325=========%%%
\begin{bt}
	Phân tích đa thức sau thành nhân tử: $2 x^2 + 38 x + 180$.
	\loigiai{ Ta có: $2 x^2 + 38 x + 180 = 2 \left(x + 9\right) \left(x + 10\right)$
	}
\end{bt}
%%%=========Bai_326=========%%%
\begin{bt}
	Phân tích đa thức sau thành nhân tử: $2 x^2 + 36 x + 160$.
	\loigiai{ Ta có: $2 x^2 + 36 x + 160 = 2 \left(x + 8\right) \left(x + 10\right)$
	}
\end{bt}
%%%=========Bai_327=========%%%
\begin{bt}
	Phân tích đa thức sau thành nhân tử: $2 x^2 + 34 x + 140$.
	\loigiai{ Ta có: $2 x^2 + 34 x + 140 = 2 \left(x + 7\right) \left(x + 10\right)$
	}
\end{bt}
%%%=========Bai_328=========%%%
\begin{bt}
	Phân tích đa thức sau thành nhân tử: $2 x^2 + 32 x + 120$.
	\loigiai{ Ta có: $2 x^2 + 32 x + 120 = 2 \left(x + 6\right) \left(x + 10\right)$
	}
\end{bt}
%%%=========Bai_329=========%%%
\begin{bt}
	Phân tích đa thức sau thành nhân tử: $2 x^2 + 30 x + 100$.
	\loigiai{ Ta có: $2 x^2 + 30 x + 100 = 2 \left(x + 5\right) \left(x + 10\right)$
	}
\end{bt}
%%%=========Bai_330=========%%%
\begin{bt}
	Phân tích đa thức sau thành nhân tử: $2 x^2 + 28 x + 80$.
	\loigiai{ Ta có: $2 x^2 + 28 x + 80 = 2 \left(x + 4\right) \left(x + 10\right)$
	}
\end{bt}
%%%=========Bai_331=========%%%
\begin{bt}
	Phân tích đa thức sau thành nhân tử: $2 x^2 + 26 x + 60$.
	\loigiai{ Ta có: $2 x^2 + 26 x + 60 = 2 \left(x + 3\right) \left(x + 10\right)$
	}
\end{bt}
%%%=========Bai_332=========%%%
\begin{bt}
	Phân tích đa thức sau thành nhân tử: $2 x^2 + 24 x + 40$.
	\loigiai{ Ta có: $2 x^2 + 24 x + 40 = 2 \left(x + 2\right) \left(x + 10\right)$
	}
\end{bt}
%%%=========Bai_333=========%%%
\begin{bt}
	Phân tích đa thức sau thành nhân tử: $2 x^2 + 22 x + 20$.
	\loigiai{ Ta có: $2 x^2 + 22 x + 20 = 2 \left(x + 1\right) \left(x + 10\right)$
	}
\end{bt}
%%%=========Bai_334=========%%%
\begin{bt}
	Phân tích đa thức sau thành nhân tử: $2 x^2 + 18 x - 20$.
	\loigiai{ Ta có: $2 x^2 + 18 x - 20 = 2 \left(x - 1\right) \left(x + 10\right)$
	}
\end{bt}
%%%=========Bai_335=========%%%
\begin{bt}
	Phân tích đa thức sau thành nhân tử: $2 x^2 + 16 x - 40$.
	\loigiai{ Ta có: $2 x^2 + 16 x - 40 = 2 \left(x - 2\right) \left(x + 10\right)$
	}
\end{bt}
%%%=========Bai_336=========%%%
\begin{bt}
	Phân tích đa thức sau thành nhân tử: $2 x^2 + 14 x - 60$.
	\loigiai{ Ta có: $2 x^2 + 14 x - 60 = 2 \left(x - 3\right) \left(x + 10\right)$
	}
\end{bt}
%%%=========Bai_337=========%%%
\begin{bt}
	Phân tích đa thức sau thành nhân tử: $2 x^2 + 12 x - 80$.
	\loigiai{ Ta có: $2 x^2 + 12 x - 80 = 2 \left(x - 4\right) \left(x + 10\right)$
	}
\end{bt}
%%%=========Bai_338=========%%%
\begin{bt}
	Phân tích đa thức sau thành nhân tử: $2 x^2 + 10 x - 100$.
	\loigiai{ Ta có: $2 x^2 + 10 x - 100 = 2 \left(x - 5\right) \left(x + 10\right)$
	}
\end{bt}
%%%=========Bai_339=========%%%
\begin{bt}
	Phân tích đa thức sau thành nhân tử: $2 x^2 + 8 x - 120$.
	\loigiai{ Ta có: $2 x^2 + 8 x - 120 = 2 \left(x - 6\right) \left(x + 10\right)$
	}
\end{bt}
%%%=========Bai_340=========%%%
\begin{bt}
	Phân tích đa thức sau thành nhân tử: $2 x^2 + 6 x - 140$.
	\loigiai{ Ta có: $2 x^2 + 6 x - 140 = 2 \left(x - 7\right) \left(x + 10\right)$
	}
\end{bt}
%%%=========Bai_341=========%%%
\begin{bt}
	Phân tích đa thức sau thành nhân tử: $2 x^2 + 4 x - 160$.
	\loigiai{ Ta có: $2 x^2 + 4 x - 160 = 2 \left(x - 8\right) \left(x + 10\right)$
	}
\end{bt}
%%%=========Bai_342=========%%%
\begin{bt}
	Phân tích đa thức sau thành nhân tử: $2 x^2 + 2 x - 180$.
	\loigiai{ Ta có: $2 x^2 + 2 x - 180 = 2 \left(x - 9\right) \left(x + 10\right)$
	}
\end{bt}
%%%=========Bai_343=========%%%
\begin{bt}
	Phân tích đa thức sau thành nhân tử: $2 x^2 + 38 x + 180$.
	\loigiai{ Ta có: $2 x^2 + 38 x + 180 = 2 \left(x + 9\right) \left(x + 10\right)$
	}
\end{bt}
%%%=========Bai_344=========%%%
\begin{bt}
	Phân tích đa thức sau thành nhân tử: $2 x^2 + 34 x + 144$.
	\loigiai{ Ta có: $2 x^2 + 34 x + 144 = 2 \left(x + 8\right) \left(x + 9\right)$
	}
\end{bt}
%%%=========Bai_345=========%%%
\begin{bt}
	Phân tích đa thức sau thành nhân tử: $2 x^2 + 32 x + 126$.
	\loigiai{ Ta có: $2 x^2 + 32 x + 126 = 2 \left(x + 7\right) \left(x + 9\right)$
	}
\end{bt}
%%%=========Bai_346=========%%%
\begin{bt}
	Phân tích đa thức sau thành nhân tử: $2 x^2 + 30 x + 108$.
	\loigiai{ Ta có: $2 x^2 + 30 x + 108 = 2 \left(x + 6\right) \left(x + 9\right)$
	}
\end{bt}
%%%=========Bai_347=========%%%
\begin{bt}
	Phân tích đa thức sau thành nhân tử: $2 x^2 + 28 x + 90$.
	\loigiai{ Ta có: $2 x^2 + 28 x + 90 = 2 \left(x + 5\right) \left(x + 9\right)$
	}
\end{bt}
%%%=========Bai_348=========%%%
\begin{bt}
	Phân tích đa thức sau thành nhân tử: $2 x^2 + 26 x + 72$.
	\loigiai{ Ta có: $2 x^2 + 26 x + 72 = 2 \left(x + 4\right) \left(x + 9\right)$
	}
\end{bt}
%%%=========Bai_349=========%%%
\begin{bt}
	Phân tích đa thức sau thành nhân tử: $2 x^2 + 24 x + 54$.
	\loigiai{ Ta có: $2 x^2 + 24 x + 54 = 2 \left(x + 3\right) \left(x + 9\right)$
	}
\end{bt}
%%%=========Bai_350=========%%%
\begin{bt}
	Phân tích đa thức sau thành nhân tử: $2 x^2 + 22 x + 36$.
	\loigiai{ Ta có: $2 x^2 + 22 x + 36 = 2 \left(x + 2\right) \left(x + 9\right)$
	}
\end{bt}
%%%=========Bai_351=========%%%
\begin{bt}
	Phân tích đa thức sau thành nhân tử: $2 x^2 + 20 x + 18$.
	\loigiai{ Ta có: $2 x^2 + 20 x + 18 = 2 \left(x + 1\right) \left(x + 9\right)$
	}
\end{bt}
%%%=========Bai_352=========%%%
\begin{bt}
	Phân tích đa thức sau thành nhân tử: $2 x^2 + 16 x - 18$.
	\loigiai{ Ta có: $2 x^2 + 16 x - 18 = 2 \left(x - 1\right) \left(x + 9\right)$
	}
\end{bt}
%%%=========Bai_353=========%%%
\begin{bt}
	Phân tích đa thức sau thành nhân tử: $2 x^2 + 14 x - 36$.
	\loigiai{ Ta có: $2 x^2 + 14 x - 36 = 2 \left(x - 2\right) \left(x + 9\right)$
	}
\end{bt}
%%%=========Bai_354=========%%%
\begin{bt}
	Phân tích đa thức sau thành nhân tử: $2 x^2 + 12 x - 54$.
	\loigiai{ Ta có: $2 x^2 + 12 x - 54 = 2 \left(x - 3\right) \left(x + 9\right)$
	}
\end{bt}
%%%=========Bai_355=========%%%
\begin{bt}
	Phân tích đa thức sau thành nhân tử: $2 x^2 + 10 x - 72$.
	\loigiai{ Ta có: $2 x^2 + 10 x - 72 = 2 \left(x - 4\right) \left(x + 9\right)$
	}
\end{bt}
%%%=========Bai_356=========%%%
\begin{bt}
	Phân tích đa thức sau thành nhân tử: $2 x^2 + 8 x - 90$.
	\loigiai{ Ta có: $2 x^2 + 8 x - 90 = 2 \left(x - 5\right) \left(x + 9\right)$
	}
\end{bt}
%%%=========Bai_357=========%%%
\begin{bt}
	Phân tích đa thức sau thành nhân tử: $2 x^2 + 6 x - 108$.
	\loigiai{ Ta có: $2 x^2 + 6 x - 108 = 2 \left(x - 6\right) \left(x + 9\right)$
	}
\end{bt}
%%%=========Bai_358=========%%%
\begin{bt}
	Phân tích đa thức sau thành nhân tử: $2 x^2 + 4 x - 126$.
	\loigiai{ Ta có: $2 x^2 + 4 x - 126 = 2 \left(x - 7\right) \left(x + 9\right)$
	}
\end{bt}
%%%=========Bai_359=========%%%
\begin{bt}
	Phân tích đa thức sau thành nhân tử: $2 x^2 + 2 x - 144$.
	\loigiai{ Ta có: $2 x^2 + 2 x - 144 = 2 \left(x - 8\right) \left(x + 9\right)$
	}
\end{bt}
%%%=========Bai_360=========%%%
\begin{bt}
	Phân tích đa thức sau thành nhân tử: $2 x^2 + 36 x + 160$.
	\loigiai{ Ta có: $2 x^2 + 36 x + 160 = 2 \left(x + 8\right) \left(x + 10\right)$
	}
\end{bt}
%%%=========Bai_361=========%%%
\begin{bt}
	Phân tích đa thức sau thành nhân tử: $2 x^2 + 34 x + 144$.
	\loigiai{ Ta có: $2 x^2 + 34 x + 144 = 2 \left(x + 8\right) \left(x + 9\right)$
	}
\end{bt}
%%%=========Bai_362=========%%%
\begin{bt}
	Phân tích đa thức sau thành nhân tử: $2 x^2 + 30 x + 112$.
	\loigiai{ Ta có: $2 x^2 + 30 x + 112 = 2 \left(x + 7\right) \left(x + 8\right)$
	}
\end{bt}
%%%=========Bai_363=========%%%
\begin{bt}
	Phân tích đa thức sau thành nhân tử: $2 x^2 + 28 x + 96$.
	\loigiai{ Ta có: $2 x^2 + 28 x + 96 = 2 \left(x + 6\right) \left(x + 8\right)$
	}
\end{bt}
%%%=========Bai_364=========%%%
\begin{bt}
	Phân tích đa thức sau thành nhân tử: $2 x^2 + 26 x + 80$.
	\loigiai{ Ta có: $2 x^2 + 26 x + 80 = 2 \left(x + 5\right) \left(x + 8\right)$
	}
\end{bt}
%%%=========Bai_365=========%%%
\begin{bt}
	Phân tích đa thức sau thành nhân tử: $2 x^2 + 24 x + 64$.
	\loigiai{ Ta có: $2 x^2 + 24 x + 64 = 2 \left(x + 4\right) \left(x + 8\right)$
	}
\end{bt}
%%%=========Bai_366=========%%%
\begin{bt}
	Phân tích đa thức sau thành nhân tử: $2 x^2 + 22 x + 48$.
	\loigiai{ Ta có: $2 x^2 + 22 x + 48 = 2 \left(x + 3\right) \left(x + 8\right)$
	}
\end{bt}
%%%=========Bai_367=========%%%
\begin{bt}
	Phân tích đa thức sau thành nhân tử: $2 x^2 + 20 x + 32$.
	\loigiai{ Ta có: $2 x^2 + 20 x + 32 = 2 \left(x + 2\right) \left(x + 8\right)$
	}
\end{bt}
%%%=========Bai_368=========%%%
\begin{bt}
	Phân tích đa thức sau thành nhân tử: $2 x^2 + 18 x + 16$.
	\loigiai{ Ta có: $2 x^2 + 18 x + 16 = 2 \left(x + 1\right) \left(x + 8\right)$
	}
\end{bt}
%%%=========Bai_369=========%%%
\begin{bt}
	Phân tích đa thức sau thành nhân tử: $2 x^2 + 14 x - 16$.
	\loigiai{ Ta có: $2 x^2 + 14 x - 16 = 2 \left(x - 1\right) \left(x + 8\right)$
	}
\end{bt}
%%%=========Bai_370=========%%%
\begin{bt}
	Phân tích đa thức sau thành nhân tử: $2 x^2 + 12 x - 32$.
	\loigiai{ Ta có: $2 x^2 + 12 x - 32 = 2 \left(x - 2\right) \left(x + 8\right)$
	}
\end{bt}
%%%=========Bai_371=========%%%
\begin{bt}
	Phân tích đa thức sau thành nhân tử: $2 x^2 + 10 x - 48$.
	\loigiai{ Ta có: $2 x^2 + 10 x - 48 = 2 \left(x - 3\right) \left(x + 8\right)$
	}
\end{bt}
%%%=========Bai_372=========%%%
\begin{bt}
	Phân tích đa thức sau thành nhân tử: $2 x^2 + 8 x - 64$.
	\loigiai{ Ta có: $2 x^2 + 8 x - 64 = 2 \left(x - 4\right) \left(x + 8\right)$
	}
\end{bt}
%%%=========Bai_373=========%%%
\begin{bt}
	Phân tích đa thức sau thành nhân tử: $2 x^2 + 6 x - 80$.
	\loigiai{ Ta có: $2 x^2 + 6 x - 80 = 2 \left(x - 5\right) \left(x + 8\right)$
	}
\end{bt}
%%%=========Bai_374=========%%%
\begin{bt}
	Phân tích đa thức sau thành nhân tử: $2 x^2 + 4 x - 96$.
	\loigiai{ Ta có: $2 x^2 + 4 x - 96 = 2 \left(x - 6\right) \left(x + 8\right)$
	}
\end{bt}
%%%=========Bai_375=========%%%
\begin{bt}
	Phân tích đa thức sau thành nhân tử: $2 x^2 + 2 x - 112$.
	\loigiai{ Ta có: $2 x^2 + 2 x - 112 = 2 \left(x - 7\right) \left(x + 8\right)$
	}
\end{bt}
%%%=========Bai_376=========%%%
\begin{bt}
	Phân tích đa thức sau thành nhân tử: $2 x^2 - 2 x - 144$.
	\loigiai{ Ta có: $2 x^2 - 2 x - 144 = 2 \left(x - 9\right) \left(x + 8\right)$
	}
\end{bt}
%%%=========Bai_377=========%%%
\begin{bt}
	Phân tích đa thức sau thành nhân tử: $2 x^2 + 34 x + 140$.
	\loigiai{ Ta có: $2 x^2 + 34 x + 140 = 2 \left(x + 7\right) \left(x + 10\right)$
	}
\end{bt}
%%%=========Bai_378=========%%%
\begin{bt}
	Phân tích đa thức sau thành nhân tử: $2 x^2 + 32 x + 126$.
	\loigiai{ Ta có: $2 x^2 + 32 x + 126 = 2 \left(x + 7\right) \left(x + 9\right)$
	}
\end{bt}
%%%=========Bai_379=========%%%
\begin{bt}
	Phân tích đa thức sau thành nhân tử: $2 x^2 + 30 x + 112$.
	\loigiai{ Ta có: $2 x^2 + 30 x + 112 = 2 \left(x + 7\right) \left(x + 8\right)$
	}
\end{bt}
%%%=========Bai_380=========%%%
\begin{bt}
	Phân tích đa thức sau thành nhân tử: $2 x^2 + 26 x + 84$.
	\loigiai{ Ta có: $2 x^2 + 26 x + 84 = 2 \left(x + 6\right) \left(x + 7\right)$
	}
\end{bt}
%%%=========Bai_381=========%%%
\begin{bt}
	Phân tích đa thức sau thành nhân tử: $2 x^2 + 24 x + 70$.
	\loigiai{ Ta có: $2 x^2 + 24 x + 70 = 2 \left(x + 5\right) \left(x + 7\right)$
	}
\end{bt}
%%%=========Bai_382=========%%%
\begin{bt}
	Phân tích đa thức sau thành nhân tử: $2 x^2 + 22 x + 56$.
	\loigiai{ Ta có: $2 x^2 + 22 x + 56 = 2 \left(x + 4\right) \left(x + 7\right)$
	}
\end{bt}
%%%=========Bai_383=========%%%
\begin{bt}
	Phân tích đa thức sau thành nhân tử: $2 x^2 + 20 x + 42$.
	\loigiai{ Ta có: $2 x^2 + 20 x + 42 = 2 \left(x + 3\right) \left(x + 7\right)$
	}
\end{bt}
%%%=========Bai_384=========%%%
\begin{bt}
	Phân tích đa thức sau thành nhân tử: $2 x^2 + 18 x + 28$.
	\loigiai{ Ta có: $2 x^2 + 18 x + 28 = 2 \left(x + 2\right) \left(x + 7\right)$
	}
\end{bt}
%%%=========Bai_385=========%%%
\begin{bt}
	Phân tích đa thức sau thành nhân tử: $2 x^2 + 16 x + 14$.
	\loigiai{ Ta có: $2 x^2 + 16 x + 14 = 2 \left(x + 1\right) \left(x + 7\right)$
	}
\end{bt}
%%%=========Bai_386=========%%%
\begin{bt}
	Phân tích đa thức sau thành nhân tử: $2 x^2 + 12 x - 14$.
	\loigiai{ Ta có: $2 x^2 + 12 x - 14 = 2 \left(x - 1\right) \left(x + 7\right)$
	}
\end{bt}
%%%=========Bai_387=========%%%
\begin{bt}
	Phân tích đa thức sau thành nhân tử: $2 x^2 + 10 x - 28$.
	\loigiai{ Ta có: $2 x^2 + 10 x - 28 = 2 \left(x - 2\right) \left(x + 7\right)$
	}
\end{bt}
%%%=========Bai_388=========%%%
\begin{bt}
	Phân tích đa thức sau thành nhân tử: $2 x^2 + 8 x - 42$.
	\loigiai{ Ta có: $2 x^2 + 8 x - 42 = 2 \left(x - 3\right) \left(x + 7\right)$
	}
\end{bt}
%%%=========Bai_389=========%%%
\begin{bt}
	Phân tích đa thức sau thành nhân tử: $2 x^2 + 6 x - 56$.
	\loigiai{ Ta có: $2 x^2 + 6 x - 56 = 2 \left(x - 4\right) \left(x + 7\right)$
	}
\end{bt}
%%%=========Bai_390=========%%%
\begin{bt}
	Phân tích đa thức sau thành nhân tử: $2 x^2 + 4 x - 70$.
	\loigiai{ Ta có: $2 x^2 + 4 x - 70 = 2 \left(x - 5\right) \left(x + 7\right)$
	}
\end{bt}
%%%=========Bai_391=========%%%
\begin{bt}
	Phân tích đa thức sau thành nhân tử: $2 x^2 + 2 x - 84$.
	\loigiai{ Ta có: $2 x^2 + 2 x - 84 = 2 \left(x - 6\right) \left(x + 7\right)$
	}
\end{bt}
%%%=========Bai_392=========%%%
\begin{bt}
	Phân tích đa thức sau thành nhân tử: $2 x^2 - 2 x - 112$.
	\loigiai{ Ta có: $2 x^2 - 2 x - 112 = 2 \left(x - 8\right) \left(x + 7\right)$
	}
\end{bt}
%%%=========Bai_393=========%%%
\begin{bt}
	Phân tích đa thức sau thành nhân tử: $2 x^2 - 4 x - 126$.
	\loigiai{ Ta có: $2 x^2 - 4 x - 126 = 2 \left(x - 9\right) \left(x + 7\right)$
	}
\end{bt}
%%%=========Bai_394=========%%%
\begin{bt}
	Phân tích đa thức sau thành nhân tử: $2 x^2 + 32 x + 120$.
	\loigiai{ Ta có: $2 x^2 + 32 x + 120 = 2 \left(x + 6\right) \left(x + 10\right)$
	}
\end{bt}
%%%=========Bai_395=========%%%
\begin{bt}
	Phân tích đa thức sau thành nhân tử: $2 x^2 + 30 x + 108$.
	\loigiai{ Ta có: $2 x^2 + 30 x + 108 = 2 \left(x + 6\right) \left(x + 9\right)$
	}
\end{bt}
%%%=========Bai_396=========%%%
\begin{bt}
	Phân tích đa thức sau thành nhân tử: $2 x^2 + 28 x + 96$.
	\loigiai{ Ta có: $2 x^2 + 28 x + 96 = 2 \left(x + 6\right) \left(x + 8\right)$
	}
\end{bt}
%%%=========Bai_397=========%%%
\begin{bt}
	Phân tích đa thức sau thành nhân tử: $2 x^2 + 26 x + 84$.
	\loigiai{ Ta có: $2 x^2 + 26 x + 84 = 2 \left(x + 6\right) \left(x + 7\right)$
	}
\end{bt}
%%%=========Bai_398=========%%%
\begin{bt}
	Phân tích đa thức sau thành nhân tử: $2 x^2 + 22 x + 60$.
	\loigiai{ Ta có: $2 x^2 + 22 x + 60 = 2 \left(x + 5\right) \left(x + 6\right)$
	}
\end{bt}
%%%=========Bai_399=========%%%
\begin{bt}
	Phân tích đa thức sau thành nhân tử: $2 x^2 + 20 x + 48$.
	\loigiai{ Ta có: $2 x^2 + 20 x + 48 = 2 \left(x + 4\right) \left(x + 6\right)$
	}
\end{bt}
%%%=========Bai_400=========%%%
\begin{bt}
	Phân tích đa thức sau thành nhân tử: $2 x^2 + 18 x + 36$.
	\loigiai{ Ta có: $2 x^2 + 18 x + 36 = 2 \left(x + 3\right) \left(x + 6\right)$
	}
\end{bt}
%%%=========Bai_401=========%%%
\begin{bt}
	Phân tích đa thức sau thành nhân tử: $2 x^2 + 16 x + 24$.
	\loigiai{ Ta có: $2 x^2 + 16 x + 24 = 2 \left(x + 2\right) \left(x + 6\right)$
	}
\end{bt}
%%%=========Bai_402=========%%%
\begin{bt}
	Phân tích đa thức sau thành nhân tử: $2 x^2 + 14 x + 12$.
	\loigiai{ Ta có: $2 x^2 + 14 x + 12 = 2 \left(x + 1\right) \left(x + 6\right)$
	}
\end{bt}
%%%=========Bai_403=========%%%
\begin{bt}
	Phân tích đa thức sau thành nhân tử: $2 x^2 + 10 x - 12$.
	\loigiai{ Ta có: $2 x^2 + 10 x - 12 = 2 \left(x - 1\right) \left(x + 6\right)$
	}
\end{bt}
%%%=========Bai_404=========%%%
\begin{bt}
	Phân tích đa thức sau thành nhân tử: $2 x^2 + 8 x - 24$.
	\loigiai{ Ta có: $2 x^2 + 8 x - 24 = 2 \left(x - 2\right) \left(x + 6\right)$
	}
\end{bt}
%%%=========Bai_405=========%%%
\begin{bt}
	Phân tích đa thức sau thành nhân tử: $2 x^2 + 6 x - 36$.
	\loigiai{ Ta có: $2 x^2 + 6 x - 36 = 2 \left(x - 3\right) \left(x + 6\right)$
	}
\end{bt}
%%%=========Bai_406=========%%%
\begin{bt}
	Phân tích đa thức sau thành nhân tử: $2 x^2 + 4 x - 48$.
	\loigiai{ Ta có: $2 x^2 + 4 x - 48 = 2 \left(x - 4\right) \left(x + 6\right)$
	}
\end{bt}
%%%=========Bai_407=========%%%
\begin{bt}
	Phân tích đa thức sau thành nhân tử: $2 x^2 + 2 x - 60$.
	\loigiai{ Ta có: $2 x^2 + 2 x - 60 = 2 \left(x - 5\right) \left(x + 6\right)$
	}
\end{bt}
%%%=========Bai_408=========%%%
\begin{bt}
	Phân tích đa thức sau thành nhân tử: $2 x^2 - 2 x - 84$.
	\loigiai{ Ta có: $2 x^2 - 2 x - 84 = 2 \left(x - 7\right) \left(x + 6\right)$
	}
\end{bt}
%%%=========Bai_409=========%%%
\begin{bt}
	Phân tích đa thức sau thành nhân tử: $2 x^2 - 4 x - 96$.
	\loigiai{ Ta có: $2 x^2 - 4 x - 96 = 2 \left(x - 8\right) \left(x + 6\right)$
	}
\end{bt}
%%%=========Bai_410=========%%%
\begin{bt}
	Phân tích đa thức sau thành nhân tử: $2 x^2 - 6 x - 108$.
	\loigiai{ Ta có: $2 x^2 - 6 x - 108 = 2 \left(x - 9\right) \left(x + 6\right)$
	}
\end{bt}
%%%=========Bai_411=========%%%
\begin{bt}
	Phân tích đa thức sau thành nhân tử: $2 x^2 + 30 x + 100$.
	\loigiai{ Ta có: $2 x^2 + 30 x + 100 = 2 \left(x + 5\right) \left(x + 10\right)$
	}
\end{bt}
%%%=========Bai_412=========%%%
\begin{bt}
	Phân tích đa thức sau thành nhân tử: $2 x^2 + 28 x + 90$.
	\loigiai{ Ta có: $2 x^2 + 28 x + 90 = 2 \left(x + 5\right) \left(x + 9\right)$
	}
\end{bt}
%%%=========Bai_413=========%%%
\begin{bt}
	Phân tích đa thức sau thành nhân tử: $2 x^2 + 26 x + 80$.
	\loigiai{ Ta có: $2 x^2 + 26 x + 80 = 2 \left(x + 5\right) \left(x + 8\right)$
	}
\end{bt}
%%%=========Bai_414=========%%%
\begin{bt}
	Phân tích đa thức sau thành nhân tử: $2 x^2 + 24 x + 70$.
	\loigiai{ Ta có: $2 x^2 + 24 x + 70 = 2 \left(x + 5\right) \left(x + 7\right)$
	}
\end{bt}
%%%=========Bai_415=========%%%
\begin{bt}
	Phân tích đa thức sau thành nhân tử: $2 x^2 + 22 x + 60$.
	\loigiai{ Ta có: $2 x^2 + 22 x + 60 = 2 \left(x + 5\right) \left(x + 6\right)$
	}
\end{bt}
%%%=========Bai_416=========%%%
\begin{bt}
	Phân tích đa thức sau thành nhân tử: $2 x^2 + 18 x + 40$.
	\loigiai{ Ta có: $2 x^2 + 18 x + 40 = 2 \left(x + 4\right) \left(x + 5\right)$
	}
\end{bt}
%%%=========Bai_417=========%%%
\begin{bt}
	Phân tích đa thức sau thành nhân tử: $2 x^2 + 16 x + 30$.
	\loigiai{ Ta có: $2 x^2 + 16 x + 30 = 2 \left(x + 3\right) \left(x + 5\right)$
	}
\end{bt}
%%%=========Bai_418=========%%%
\begin{bt}
	Phân tích đa thức sau thành nhân tử: $2 x^2 + 14 x + 20$.
	\loigiai{ Ta có: $2 x^2 + 14 x + 20 = 2 \left(x + 2\right) \left(x + 5\right)$
	}
\end{bt}
%%%=========Bai_419=========%%%
\begin{bt}
	Phân tích đa thức sau thành nhân tử: $2 x^2 + 12 x + 10$.
	\loigiai{ Ta có: $2 x^2 + 12 x + 10 = 2 \left(x + 1\right) \left(x + 5\right)$
	}
\end{bt}
%%%=========Bai_420=========%%%
\begin{bt}
	Phân tích đa thức sau thành nhân tử: $2 x^2 + 8 x - 10$.
	\loigiai{ Ta có: $2 x^2 + 8 x - 10 = 2 \left(x - 1\right) \left(x + 5\right)$
	}
\end{bt}
%%%=========Bai_421=========%%%
\begin{bt}
	Phân tích đa thức sau thành nhân tử: $2 x^2 + 6 x - 20$.
	\loigiai{ Ta có: $2 x^2 + 6 x - 20 = 2 \left(x - 2\right) \left(x + 5\right)$
	}
\end{bt}
%%%=========Bai_422=========%%%
\begin{bt}
	Phân tích đa thức sau thành nhân tử: $2 x^2 + 4 x - 30$.
	\loigiai{ Ta có: $2 x^2 + 4 x - 30 = 2 \left(x - 3\right) \left(x + 5\right)$
	}
\end{bt}
%%%=========Bai_423=========%%%
\begin{bt}
	Phân tích đa thức sau thành nhân tử: $2 x^2 + 2 x - 40$.
	\loigiai{ Ta có: $2 x^2 + 2 x - 40 = 2 \left(x - 4\right) \left(x + 5\right)$
	}
\end{bt}
%%%=========Bai_424=========%%%
\begin{bt}
	Phân tích đa thức sau thành nhân tử: $2 x^2 - 2 x - 60$.
	\loigiai{ Ta có: $2 x^2 - 2 x - 60 = 2 \left(x - 6\right) \left(x + 5\right)$
	}
\end{bt}
%%%=========Bai_425=========%%%
\begin{bt}
	Phân tích đa thức sau thành nhân tử: $2 x^2 - 4 x - 70$.
	\loigiai{ Ta có: $2 x^2 - 4 x - 70 = 2 \left(x - 7\right) \left(x + 5\right)$
	}
\end{bt}
%%%=========Bai_426=========%%%
\begin{bt}
	Phân tích đa thức sau thành nhân tử: $2 x^2 - 6 x - 80$.
	\loigiai{ Ta có: $2 x^2 - 6 x - 80 = 2 \left(x - 8\right) \left(x + 5\right)$
	}
\end{bt}
%%%=========Bai_427=========%%%
\begin{bt}
	Phân tích đa thức sau thành nhân tử: $2 x^2 - 8 x - 90$.
	\loigiai{ Ta có: $2 x^2 - 8 x - 90 = 2 \left(x - 9\right) \left(x + 5\right)$
	}
\end{bt}
%%%=========Bai_428=========%%%
\begin{bt}
	Phân tích đa thức sau thành nhân tử: $2 x^2 + 28 x + 80$.
	\loigiai{ Ta có: $2 x^2 + 28 x + 80 = 2 \left(x + 4\right) \left(x + 10\right)$
	}
\end{bt}
%%%=========Bai_429=========%%%
\begin{bt}
	Phân tích đa thức sau thành nhân tử: $2 x^2 + 26 x + 72$.
	\loigiai{ Ta có: $2 x^2 + 26 x + 72 = 2 \left(x + 4\right) \left(x + 9\right)$
	}
\end{bt}
%%%=========Bai_430=========%%%
\begin{bt}
	Phân tích đa thức sau thành nhân tử: $2 x^2 + 24 x + 64$.
	\loigiai{ Ta có: $2 x^2 + 24 x + 64 = 2 \left(x + 4\right) \left(x + 8\right)$
	}
\end{bt}
%%%=========Bai_431=========%%%
\begin{bt}
	Phân tích đa thức sau thành nhân tử: $2 x^2 + 22 x + 56$.
	\loigiai{ Ta có: $2 x^2 + 22 x + 56 = 2 \left(x + 4\right) \left(x + 7\right)$
	}
\end{bt}
%%%=========Bai_432=========%%%
\begin{bt}
	Phân tích đa thức sau thành nhân tử: $2 x^2 + 20 x + 48$.
	\loigiai{ Ta có: $2 x^2 + 20 x + 48 = 2 \left(x + 4\right) \left(x + 6\right)$
	}
\end{bt}
%%%=========Bai_433=========%%%
\begin{bt}
	Phân tích đa thức sau thành nhân tử: $2 x^2 + 18 x + 40$.
	\loigiai{ Ta có: $2 x^2 + 18 x + 40 = 2 \left(x + 4\right) \left(x + 5\right)$
	}
\end{bt}
%%%=========Bai_434=========%%%
\begin{bt}
	Phân tích đa thức sau thành nhân tử: $2 x^2 + 14 x + 24$.
	\loigiai{ Ta có: $2 x^2 + 14 x + 24 = 2 \left(x + 3\right) \left(x + 4\right)$
	}
\end{bt}
%%%=========Bai_435=========%%%
\begin{bt}
	Phân tích đa thức sau thành nhân tử: $2 x^2 + 12 x + 16$.
	\loigiai{ Ta có: $2 x^2 + 12 x + 16 = 2 \left(x + 2\right) \left(x + 4\right)$
	}
\end{bt}
%%%=========Bai_436=========%%%
\begin{bt}
	Phân tích đa thức sau thành nhân tử: $2 x^2 + 10 x + 8$.
	\loigiai{ Ta có: $2 x^2 + 10 x + 8 = 2 \left(x + 1\right) \left(x + 4\right)$
	}
\end{bt}
%%%=========Bai_437=========%%%
\begin{bt}
	Phân tích đa thức sau thành nhân tử: $2 x^2 + 6 x - 8$.
	\loigiai{ Ta có: $2 x^2 + 6 x - 8 = 2 \left(x - 1\right) \left(x + 4\right)$
	}
\end{bt}
%%%=========Bai_438=========%%%
\begin{bt}
	Phân tích đa thức sau thành nhân tử: $2 x^2 + 4 x - 16$.
	\loigiai{ Ta có: $2 x^2 + 4 x - 16 = 2 \left(x - 2\right) \left(x + 4\right)$
	}
\end{bt}
%%%=========Bai_439=========%%%
\begin{bt}
	Phân tích đa thức sau thành nhân tử: $2 x^2 + 2 x - 24$.
	\loigiai{ Ta có: $2 x^2 + 2 x - 24 = 2 \left(x - 3\right) \left(x + 4\right)$
	}
\end{bt}
%%%=========Bai_440=========%%%
\begin{bt}
	Phân tích đa thức sau thành nhân tử: $2 x^2 - 2 x - 40$.
	\loigiai{ Ta có: $2 x^2 - 2 x - 40 = 2 \left(x - 5\right) \left(x + 4\right)$
	}
\end{bt}
%%%=========Bai_441=========%%%
\begin{bt}
	Phân tích đa thức sau thành nhân tử: $2 x^2 - 4 x - 48$.
	\loigiai{ Ta có: $2 x^2 - 4 x - 48 = 2 \left(x - 6\right) \left(x + 4\right)$
	}
\end{bt}
%%%=========Bai_442=========%%%
\begin{bt}
	Phân tích đa thức sau thành nhân tử: $2 x^2 - 6 x - 56$.
	\loigiai{ Ta có: $2 x^2 - 6 x - 56 = 2 \left(x - 7\right) \left(x + 4\right)$
	}
\end{bt}
%%%=========Bai_443=========%%%
\begin{bt}
	Phân tích đa thức sau thành nhân tử: $2 x^2 - 8 x - 64$.
	\loigiai{ Ta có: $2 x^2 - 8 x - 64 = 2 \left(x - 8\right) \left(x + 4\right)$
	}
\end{bt}
%%%=========Bai_444=========%%%
\begin{bt}
	Phân tích đa thức sau thành nhân tử: $2 x^2 - 10 x - 72$.
	\loigiai{ Ta có: $2 x^2 - 10 x - 72 = 2 \left(x - 9\right) \left(x + 4\right)$
	}
\end{bt}
%%%=========Bai_445=========%%%
\begin{bt}
	Phân tích đa thức sau thành nhân tử: $2 x^2 + 26 x + 60$.
	\loigiai{ Ta có: $2 x^2 + 26 x + 60 = 2 \left(x + 3\right) \left(x + 10\right)$
	}
\end{bt}
%%%=========Bai_446=========%%%
\begin{bt}
	Phân tích đa thức sau thành nhân tử: $2 x^2 + 24 x + 54$.
	\loigiai{ Ta có: $2 x^2 + 24 x + 54 = 2 \left(x + 3\right) \left(x + 9\right)$
	}
\end{bt}
%%%=========Bai_447=========%%%
\begin{bt}
	Phân tích đa thức sau thành nhân tử: $2 x^2 + 22 x + 48$.
	\loigiai{ Ta có: $2 x^2 + 22 x + 48 = 2 \left(x + 3\right) \left(x + 8\right)$
	}
\end{bt}
%%%=========Bai_448=========%%%
\begin{bt}
	Phân tích đa thức sau thành nhân tử: $2 x^2 + 20 x + 42$.
	\loigiai{ Ta có: $2 x^2 + 20 x + 42 = 2 \left(x + 3\right) \left(x + 7\right)$
	}
\end{bt}
%%%=========Bai_449=========%%%
\begin{bt}
	Phân tích đa thức sau thành nhân tử: $2 x^2 + 18 x + 36$.
	\loigiai{ Ta có: $2 x^2 + 18 x + 36 = 2 \left(x + 3\right) \left(x + 6\right)$
	}
\end{bt}
%%%=========Bai_450=========%%%
\begin{bt}
	Phân tích đa thức sau thành nhân tử: $2 x^2 + 16 x + 30$.
	\loigiai{ Ta có: $2 x^2 + 16 x + 30 = 2 \left(x + 3\right) \left(x + 5\right)$
	}
\end{bt}
%%%=========Bai_451=========%%%
\begin{bt}
	Phân tích đa thức sau thành nhân tử: $2 x^2 + 14 x + 24$.
	\loigiai{ Ta có: $2 x^2 + 14 x + 24 = 2 \left(x + 3\right) \left(x + 4\right)$
	}
\end{bt}
%%%=========Bai_452=========%%%
\begin{bt}
	Phân tích đa thức sau thành nhân tử: $2 x^2 + 10 x + 12$.
	\loigiai{ Ta có: $2 x^2 + 10 x + 12 = 2 \left(x + 2\right) \left(x + 3\right)$
	}
\end{bt}
%%%=========Bai_453=========%%%
\begin{bt}
	Phân tích đa thức sau thành nhân tử: $2 x^2 + 8 x + 6$.
	\loigiai{ Ta có: $2 x^2 + 8 x + 6 = 2 \left(x + 1\right) \left(x + 3\right)$
	}
\end{bt}
%%%=========Bai_454=========%%%
\begin{bt}
	Phân tích đa thức sau thành nhân tử: $2 x^2 + 4 x - 6$.
	\loigiai{ Ta có: $2 x^2 + 4 x - 6 = 2 \left(x - 1\right) \left(x + 3\right)$
	}
\end{bt}
%%%=========Bai_455=========%%%
\begin{bt}
	Phân tích đa thức sau thành nhân tử: $2 x^2 + 2 x - 12$.
	\loigiai{ Ta có: $2 x^2 + 2 x - 12 = 2 \left(x - 2\right) \left(x + 3\right)$
	}
\end{bt}
%%%=========Bai_456=========%%%
\begin{bt}
	Phân tích đa thức sau thành nhân tử: $2 x^2 - 2 x - 24$.
	\loigiai{ Ta có: $2 x^2 - 2 x - 24 = 2 \left(x - 4\right) \left(x + 3\right)$
	}
\end{bt}
%%%=========Bai_457=========%%%
\begin{bt}
	Phân tích đa thức sau thành nhân tử: $2 x^2 - 4 x - 30$.
	\loigiai{ Ta có: $2 x^2 - 4 x - 30 = 2 \left(x - 5\right) \left(x + 3\right)$
	}
\end{bt}
%%%=========Bai_458=========%%%
\begin{bt}
	Phân tích đa thức sau thành nhân tử: $2 x^2 - 6 x - 36$.
	\loigiai{ Ta có: $2 x^2 - 6 x - 36 = 2 \left(x - 6\right) \left(x + 3\right)$
	}
\end{bt}
%%%=========Bai_459=========%%%
\begin{bt}
	Phân tích đa thức sau thành nhân tử: $2 x^2 - 8 x - 42$.
	\loigiai{ Ta có: $2 x^2 - 8 x - 42 = 2 \left(x - 7\right) \left(x + 3\right)$
	}
\end{bt}
%%%=========Bai_460=========%%%
\begin{bt}
	Phân tích đa thức sau thành nhân tử: $2 x^2 - 10 x - 48$.
	\loigiai{ Ta có: $2 x^2 - 10 x - 48 = 2 \left(x - 8\right) \left(x + 3\right)$
	}
\end{bt}
%%%=========Bai_461=========%%%
\begin{bt}
	Phân tích đa thức sau thành nhân tử: $2 x^2 - 12 x - 54$.
	\loigiai{ Ta có: $2 x^2 - 12 x - 54 = 2 \left(x - 9\right) \left(x + 3\right)$
	}
\end{bt}
%%%=========Bai_462=========%%%
\begin{bt}
	Phân tích đa thức sau thành nhân tử: $2 x^2 + 24 x + 40$.
	\loigiai{ Ta có: $2 x^2 + 24 x + 40 = 2 \left(x + 2\right) \left(x + 10\right)$
	}
\end{bt}
%%%=========Bai_463=========%%%
\begin{bt}
	Phân tích đa thức sau thành nhân tử: $2 x^2 + 22 x + 36$.
	\loigiai{ Ta có: $2 x^2 + 22 x + 36 = 2 \left(x + 2\right) \left(x + 9\right)$
	}
\end{bt}
%%%=========Bai_464=========%%%
\begin{bt}
	Phân tích đa thức sau thành nhân tử: $2 x^2 + 20 x + 32$.
	\loigiai{ Ta có: $2 x^2 + 20 x + 32 = 2 \left(x + 2\right) \left(x + 8\right)$
	}
\end{bt}
%%%=========Bai_465=========%%%
\begin{bt}
	Phân tích đa thức sau thành nhân tử: $2 x^2 + 18 x + 28$.
	\loigiai{ Ta có: $2 x^2 + 18 x + 28 = 2 \left(x + 2\right) \left(x + 7\right)$
	}
\end{bt}
%%%=========Bai_466=========%%%
\begin{bt}
	Phân tích đa thức sau thành nhân tử: $2 x^2 + 16 x + 24$.
	\loigiai{ Ta có: $2 x^2 + 16 x + 24 = 2 \left(x + 2\right) \left(x + 6\right)$
	}
\end{bt}
%%%=========Bai_467=========%%%
\begin{bt}
	Phân tích đa thức sau thành nhân tử: $2 x^2 + 14 x + 20$.
	\loigiai{ Ta có: $2 x^2 + 14 x + 20 = 2 \left(x + 2\right) \left(x + 5\right)$
	}
\end{bt}
%%%=========Bai_468=========%%%
\begin{bt}
	Phân tích đa thức sau thành nhân tử: $2 x^2 + 12 x + 16$.
	\loigiai{ Ta có: $2 x^2 + 12 x + 16 = 2 \left(x + 2\right) \left(x + 4\right)$
	}
\end{bt}
%%%=========Bai_469=========%%%
\begin{bt}
	Phân tích đa thức sau thành nhân tử: $2 x^2 + 10 x + 12$.
	\loigiai{ Ta có: $2 x^2 + 10 x + 12 = 2 \left(x + 2\right) \left(x + 3\right)$
	}
\end{bt}
%%%=========Bai_470=========%%%
\begin{bt}
	Phân tích đa thức sau thành nhân tử: $2 x^2 + 6 x + 4$.
	\loigiai{ Ta có: $2 x^2 + 6 x + 4 = 2 \left(x + 1\right) \left(x + 2\right)$
	}
\end{bt}
%%%=========Bai_471=========%%%
\begin{bt}
	Phân tích đa thức sau thành nhân tử: $2 x^2 + 2 x - 4$.
	\loigiai{ Ta có: $2 x^2 + 2 x - 4 = 2 \left(x - 1\right) \left(x + 2\right)$
	}
\end{bt}
%%%=========Bai_472=========%%%
\begin{bt}
	Phân tích đa thức sau thành nhân tử: $2 x^2 - 2 x - 12$.
	\loigiai{ Ta có: $2 x^2 - 2 x - 12 = 2 \left(x - 3\right) \left(x + 2\right)$
	}
\end{bt}
%%%=========Bai_473=========%%%
\begin{bt}
	Phân tích đa thức sau thành nhân tử: $2 x^2 - 4 x - 16$.
	\loigiai{ Ta có: $2 x^2 - 4 x - 16 = 2 \left(x - 4\right) \left(x + 2\right)$
	}
\end{bt}
%%%=========Bai_474=========%%%
\begin{bt}
	Phân tích đa thức sau thành nhân tử: $2 x^2 - 6 x - 20$.
	\loigiai{ Ta có: $2 x^2 - 6 x - 20 = 2 \left(x - 5\right) \left(x + 2\right)$
	}
\end{bt}
%%%=========Bai_475=========%%%
\begin{bt}
	Phân tích đa thức sau thành nhân tử: $2 x^2 - 8 x - 24$.
	\loigiai{ Ta có: $2 x^2 - 8 x - 24 = 2 \left(x - 6\right) \left(x + 2\right)$
	}
\end{bt}
%%%=========Bai_476=========%%%
\begin{bt}
	Phân tích đa thức sau thành nhân tử: $2 x^2 - 10 x - 28$.
	\loigiai{ Ta có: $2 x^2 - 10 x - 28 = 2 \left(x - 7\right) \left(x + 2\right)$
	}
\end{bt}
%%%=========Bai_477=========%%%
\begin{bt}
	Phân tích đa thức sau thành nhân tử: $2 x^2 - 12 x - 32$.
	\loigiai{ Ta có: $2 x^2 - 12 x - 32 = 2 \left(x - 8\right) \left(x + 2\right)$
	}
\end{bt}
%%%=========Bai_478=========%%%
\begin{bt}
	Phân tích đa thức sau thành nhân tử: $2 x^2 - 14 x - 36$.
	\loigiai{ Ta có: $2 x^2 - 14 x - 36 = 2 \left(x - 9\right) \left(x + 2\right)$
	}
\end{bt}
%%%=========Bai_479=========%%%
\begin{bt}
	Phân tích đa thức sau thành nhân tử: $2 x^2 + 22 x + 20$.
	\loigiai{ Ta có: $2 x^2 + 22 x + 20 = 2 \left(x + 1\right) \left(x + 10\right)$
	}
\end{bt}
%%%=========Bai_480=========%%%
\begin{bt}
	Phân tích đa thức sau thành nhân tử: $2 x^2 + 20 x + 18$.
	\loigiai{ Ta có: $2 x^2 + 20 x + 18 = 2 \left(x + 1\right) \left(x + 9\right)$
	}
\end{bt}
%%%=========Bai_481=========%%%
\begin{bt}
	Phân tích đa thức sau thành nhân tử: $2 x^2 + 18 x + 16$.
	\loigiai{ Ta có: $2 x^2 + 18 x + 16 = 2 \left(x + 1\right) \left(x + 8\right)$
	}
\end{bt}
%%%=========Bai_482=========%%%
\begin{bt}
	Phân tích đa thức sau thành nhân tử: $2 x^2 + 16 x + 14$.
	\loigiai{ Ta có: $2 x^2 + 16 x + 14 = 2 \left(x + 1\right) \left(x + 7\right)$
	}
\end{bt}
%%%=========Bai_483=========%%%
\begin{bt}
	Phân tích đa thức sau thành nhân tử: $2 x^2 + 14 x + 12$.
	\loigiai{ Ta có: $2 x^2 + 14 x + 12 = 2 \left(x + 1\right) \left(x + 6\right)$
	}
\end{bt}
%%%=========Bai_484=========%%%
\begin{bt}
	Phân tích đa thức sau thành nhân tử: $2 x^2 + 12 x + 10$.
	\loigiai{ Ta có: $2 x^2 + 12 x + 10 = 2 \left(x + 1\right) \left(x + 5\right)$
	}
\end{bt}
%%%=========Bai_485=========%%%
\begin{bt}
	Phân tích đa thức sau thành nhân tử: $2 x^2 + 10 x + 8$.
	\loigiai{ Ta có: $2 x^2 + 10 x + 8 = 2 \left(x + 1\right) \left(x + 4\right)$
	}
\end{bt}
%%%=========Bai_486=========%%%
\begin{bt}
	Phân tích đa thức sau thành nhân tử: $2 x^2 + 8 x + 6$.
	\loigiai{ Ta có: $2 x^2 + 8 x + 6 = 2 \left(x + 1\right) \left(x + 3\right)$
	}
\end{bt}
%%%=========Bai_487=========%%%
\begin{bt}
	Phân tích đa thức sau thành nhân tử: $2 x^2 + 6 x + 4$.
	\loigiai{ Ta có: $2 x^2 + 6 x + 4 = 2 \left(x + 1\right) \left(x + 2\right)$
	}
\end{bt}
%%%=========Bai_488=========%%%
\begin{bt}
	Phân tích đa thức sau thành nhân tử: $2 x^2 - 2 x - 4$.
	\loigiai{ Ta có: $2 x^2 - 2 x - 4 = 2 \left(x - 2\right) \left(x + 1\right)$
	}
\end{bt}
%%%=========Bai_489=========%%%
\begin{bt}
	Phân tích đa thức sau thành nhân tử: $2 x^2 - 4 x - 6$.
	\loigiai{ Ta có: $2 x^2 - 4 x - 6 = 2 \left(x - 3\right) \left(x + 1\right)$
	}
\end{bt}
%%%=========Bai_490=========%%%
\begin{bt}
	Phân tích đa thức sau thành nhân tử: $2 x^2 - 6 x - 8$.
	\loigiai{ Ta có: $2 x^2 - 6 x - 8 = 2 \left(x - 4\right) \left(x + 1\right)$
	}
\end{bt}
%%%=========Bai_491=========%%%
\begin{bt}
	Phân tích đa thức sau thành nhân tử: $2 x^2 - 8 x - 10$.
	\loigiai{ Ta có: $2 x^2 - 8 x - 10 = 2 \left(x - 5\right) \left(x + 1\right)$
	}
\end{bt}
%%%=========Bai_492=========%%%
\begin{bt}
	Phân tích đa thức sau thành nhân tử: $2 x^2 - 10 x - 12$.
	\loigiai{ Ta có: $2 x^2 - 10 x - 12 = 2 \left(x - 6\right) \left(x + 1\right)$
	}
\end{bt}
%%%=========Bai_493=========%%%
\begin{bt}
	Phân tích đa thức sau thành nhân tử: $2 x^2 - 12 x - 14$.
	\loigiai{ Ta có: $2 x^2 - 12 x - 14 = 2 \left(x - 7\right) \left(x + 1\right)$
	}
\end{bt}
%%%=========Bai_494=========%%%
\begin{bt}
	Phân tích đa thức sau thành nhân tử: $2 x^2 - 14 x - 16$.
	\loigiai{ Ta có: $2 x^2 - 14 x - 16 = 2 \left(x - 8\right) \left(x + 1\right)$
	}
\end{bt}
%%%=========Bai_495=========%%%
\begin{bt}
	Phân tích đa thức sau thành nhân tử: $2 x^2 - 16 x - 18$.
	\loigiai{ Ta có: $2 x^2 - 16 x - 18 = 2 \left(x - 9\right) \left(x + 1\right)$
	}
\end{bt}
%%%=========Bai_496=========%%%
\begin{bt}
	Phân tích đa thức sau thành nhân tử: $2 x^2 + 18 x - 20$.
	\loigiai{ Ta có: $2 x^2 + 18 x - 20 = 2 \left(x - 1\right) \left(x + 10\right)$
	}
\end{bt}
%%%=========Bai_497=========%%%
\begin{bt}
	Phân tích đa thức sau thành nhân tử: $2 x^2 + 16 x - 18$.
	\loigiai{ Ta có: $2 x^2 + 16 x - 18 = 2 \left(x - 1\right) \left(x + 9\right)$
	}
\end{bt}
%%%=========Bai_498=========%%%
\begin{bt}
	Phân tích đa thức sau thành nhân tử: $2 x^2 + 14 x - 16$.
	\loigiai{ Ta có: $2 x^2 + 14 x - 16 = 2 \left(x - 1\right) \left(x + 8\right)$
	}
\end{bt}
%%%=========Bai_499=========%%%
\begin{bt}
	Phân tích đa thức sau thành nhân tử: $2 x^2 + 12 x - 14$.
	\loigiai{ Ta có: $2 x^2 + 12 x - 14 = 2 \left(x - 1\right) \left(x + 7\right)$
	}
\end{bt}
%%%=========Bai_500=========%%%
\begin{bt}
	Phân tích đa thức sau thành nhân tử: $2 x^2 + 10 x - 12$.
	\loigiai{ Ta có: $2 x^2 + 10 x - 12 = 2 \left(x - 1\right) \left(x + 6\right)$
	}
\end{bt}
%%%=========Bai_501=========%%%
\begin{bt}
	Phân tích đa thức sau thành nhân tử: $2 x^2 + 8 x - 10$.
	\loigiai{ Ta có: $2 x^2 + 8 x - 10 = 2 \left(x - 1\right) \left(x + 5\right)$
	}
\end{bt}
%%%=========Bai_502=========%%%
\begin{bt}
	Phân tích đa thức sau thành nhân tử: $2 x^2 + 6 x - 8$.
	\loigiai{ Ta có: $2 x^2 + 6 x - 8 = 2 \left(x - 1\right) \left(x + 4\right)$
	}
\end{bt}
%%%=========Bai_503=========%%%
\begin{bt}
	Phân tích đa thức sau thành nhân tử: $2 x^2 + 4 x - 6$.
	\loigiai{ Ta có: $2 x^2 + 4 x - 6 = 2 \left(x - 1\right) \left(x + 3\right)$
	}
\end{bt}
%%%=========Bai_504=========%%%
\begin{bt}
	Phân tích đa thức sau thành nhân tử: $2 x^2 + 2 x - 4$.
	\loigiai{ Ta có: $2 x^2 + 2 x - 4 = 2 \left(x - 1\right) \left(x + 2\right)$
	}
\end{bt}
%%%=========Bai_505=========%%%
\begin{bt}
	Phân tích đa thức sau thành nhân tử: $2 x^2 - 6 x + 4$.
	\loigiai{ Ta có: $2 x^2 - 6 x + 4 = 2 \left(x - 2\right) \left(x - 1\right)$
	}
\end{bt}
%%%=========Bai_506=========%%%
\begin{bt}
	Phân tích đa thức sau thành nhân tử: $2 x^2 - 8 x + 6$.
	\loigiai{ Ta có: $2 x^2 - 8 x + 6 = 2 \left(x - 3\right) \left(x - 1\right)$
	}
\end{bt}
%%%=========Bai_507=========%%%
\begin{bt}
	Phân tích đa thức sau thành nhân tử: $2 x^2 - 10 x + 8$.
	\loigiai{ Ta có: $2 x^2 - 10 x + 8 = 2 \left(x - 4\right) \left(x - 1\right)$
	}
\end{bt}
%%%=========Bai_508=========%%%
\begin{bt}
	Phân tích đa thức sau thành nhân tử: $2 x^2 - 12 x + 10$.
	\loigiai{ Ta có: $2 x^2 - 12 x + 10 = 2 \left(x - 5\right) \left(x - 1\right)$
	}
\end{bt}
%%%=========Bai_509=========%%%
\begin{bt}
	Phân tích đa thức sau thành nhân tử: $2 x^2 - 14 x + 12$.
	\loigiai{ Ta có: $2 x^2 - 14 x + 12 = 2 \left(x - 6\right) \left(x - 1\right)$
	}
\end{bt}
%%%=========Bai_510=========%%%
\begin{bt}
	Phân tích đa thức sau thành nhân tử: $2 x^2 - 16 x + 14$.
	\loigiai{ Ta có: $2 x^2 - 16 x + 14 = 2 \left(x - 7\right) \left(x - 1\right)$
	}
\end{bt}
%%%=========Bai_511=========%%%
\begin{bt}
	Phân tích đa thức sau thành nhân tử: $2 x^2 - 18 x + 16$.
	\loigiai{ Ta có: $2 x^2 - 18 x + 16 = 2 \left(x - 8\right) \left(x - 1\right)$
	}
\end{bt}
%%%=========Bai_512=========%%%
\begin{bt}
	Phân tích đa thức sau thành nhân tử: $2 x^2 - 20 x + 18$.
	\loigiai{ Ta có: $2 x^2 - 20 x + 18 = 2 \left(x - 9\right) \left(x - 1\right)$
	}
\end{bt}
%%%=========Bai_513=========%%%
\begin{bt}
	Phân tích đa thức sau thành nhân tử: $2 x^2 + 16 x - 40$.
	\loigiai{ Ta có: $2 x^2 + 16 x - 40 = 2 \left(x - 2\right) \left(x + 10\right)$
	}
\end{bt}
%%%=========Bai_514=========%%%
\begin{bt}
	Phân tích đa thức sau thành nhân tử: $2 x^2 + 14 x - 36$.
	\loigiai{ Ta có: $2 x^2 + 14 x - 36 = 2 \left(x - 2\right) \left(x + 9\right)$
	}
\end{bt}
%%%=========Bai_515=========%%%
\begin{bt}
	Phân tích đa thức sau thành nhân tử: $2 x^2 + 12 x - 32$.
	\loigiai{ Ta có: $2 x^2 + 12 x - 32 = 2 \left(x - 2\right) \left(x + 8\right)$
	}
\end{bt}
%%%=========Bai_516=========%%%
\begin{bt}
	Phân tích đa thức sau thành nhân tử: $2 x^2 + 10 x - 28$.
	\loigiai{ Ta có: $2 x^2 + 10 x - 28 = 2 \left(x - 2\right) \left(x + 7\right)$
	}
\end{bt}
%%%=========Bai_517=========%%%
\begin{bt}
	Phân tích đa thức sau thành nhân tử: $2 x^2 + 8 x - 24$.
	\loigiai{ Ta có: $2 x^2 + 8 x - 24 = 2 \left(x - 2\right) \left(x + 6\right)$
	}
\end{bt}
%%%=========Bai_518=========%%%
\begin{bt}
	Phân tích đa thức sau thành nhân tử: $2 x^2 + 6 x - 20$.
	\loigiai{ Ta có: $2 x^2 + 6 x - 20 = 2 \left(x - 2\right) \left(x + 5\right)$
	}
\end{bt}
%%%=========Bai_519=========%%%
\begin{bt}
	Phân tích đa thức sau thành nhân tử: $2 x^2 + 4 x - 16$.
	\loigiai{ Ta có: $2 x^2 + 4 x - 16 = 2 \left(x - 2\right) \left(x + 4\right)$
	}
\end{bt}
%%%=========Bai_520=========%%%
\begin{bt}
	Phân tích đa thức sau thành nhân tử: $2 x^2 + 2 x - 12$.
	\loigiai{ Ta có: $2 x^2 + 2 x - 12 = 2 \left(x - 2\right) \left(x + 3\right)$
	}
\end{bt}
%%%=========Bai_521=========%%%
\begin{bt}
	Phân tích đa thức sau thành nhân tử: $2 x^2 - 2 x - 4$.
	\loigiai{ Ta có: $2 x^2 - 2 x - 4 = 2 \left(x - 2\right) \left(x + 1\right)$
	}
\end{bt}
%%%=========Bai_522=========%%%
\begin{bt}
	Phân tích đa thức sau thành nhân tử: $2 x^2 - 6 x + 4$.
	\loigiai{ Ta có: $2 x^2 - 6 x + 4 = 2 \left(x - 2\right) \left(x - 1\right)$
	}
\end{bt}
%%%=========Bai_523=========%%%
\begin{bt}
	Phân tích đa thức sau thành nhân tử: $2 x^2 - 10 x + 12$.
	\loigiai{ Ta có: $2 x^2 - 10 x + 12 = 2 \left(x - 3\right) \left(x - 2\right)$
	}
\end{bt}
%%%=========Bai_524=========%%%
\begin{bt}
	Phân tích đa thức sau thành nhân tử: $2 x^2 - 12 x + 16$.
	\loigiai{ Ta có: $2 x^2 - 12 x + 16 = 2 \left(x - 4\right) \left(x - 2\right)$
	}
\end{bt}
%%%=========Bai_525=========%%%
\begin{bt}
	Phân tích đa thức sau thành nhân tử: $2 x^2 - 14 x + 20$.
	\loigiai{ Ta có: $2 x^2 - 14 x + 20 = 2 \left(x - 5\right) \left(x - 2\right)$
	}
\end{bt}
%%%=========Bai_526=========%%%
\begin{bt}
	Phân tích đa thức sau thành nhân tử: $2 x^2 - 16 x + 24$.
	\loigiai{ Ta có: $2 x^2 - 16 x + 24 = 2 \left(x - 6\right) \left(x - 2\right)$
	}
\end{bt}
%%%=========Bai_527=========%%%
\begin{bt}
	Phân tích đa thức sau thành nhân tử: $2 x^2 - 18 x + 28$.
	\loigiai{ Ta có: $2 x^2 - 18 x + 28 = 2 \left(x - 7\right) \left(x - 2\right)$
	}
\end{bt}
%%%=========Bai_528=========%%%
\begin{bt}
	Phân tích đa thức sau thành nhân tử: $2 x^2 - 20 x + 32$.
	\loigiai{ Ta có: $2 x^2 - 20 x + 32 = 2 \left(x - 8\right) \left(x - 2\right)$
	}
\end{bt}
%%%=========Bai_529=========%%%
\begin{bt}
	Phân tích đa thức sau thành nhân tử: $2 x^2 - 22 x + 36$.
	\loigiai{ Ta có: $2 x^2 - 22 x + 36 = 2 \left(x - 9\right) \left(x - 2\right)$
	}
\end{bt}
%%%=========Bai_530=========%%%
\begin{bt}
	Phân tích đa thức sau thành nhân tử: $2 x^2 + 14 x - 60$.
	\loigiai{ Ta có: $2 x^2 + 14 x - 60 = 2 \left(x - 3\right) \left(x + 10\right)$
	}
\end{bt}
%%%=========Bai_531=========%%%
\begin{bt}
	Phân tích đa thức sau thành nhân tử: $2 x^2 + 12 x - 54$.
	\loigiai{ Ta có: $2 x^2 + 12 x - 54 = 2 \left(x - 3\right) \left(x + 9\right)$
	}
\end{bt}
%%%=========Bai_532=========%%%
\begin{bt}
	Phân tích đa thức sau thành nhân tử: $2 x^2 + 10 x - 48$.
	\loigiai{ Ta có: $2 x^2 + 10 x - 48 = 2 \left(x - 3\right) \left(x + 8\right)$
	}
\end{bt}
%%%=========Bai_533=========%%%
\begin{bt}
	Phân tích đa thức sau thành nhân tử: $2 x^2 + 8 x - 42$.
	\loigiai{ Ta có: $2 x^2 + 8 x - 42 = 2 \left(x - 3\right) \left(x + 7\right)$
	}
\end{bt}
%%%=========Bai_534=========%%%
\begin{bt}
	Phân tích đa thức sau thành nhân tử: $2 x^2 + 6 x - 36$.
	\loigiai{ Ta có: $2 x^2 + 6 x - 36 = 2 \left(x - 3\right) \left(x + 6\right)$
	}
\end{bt}
%%%=========Bai_535=========%%%
\begin{bt}
	Phân tích đa thức sau thành nhân tử: $2 x^2 + 4 x - 30$.
	\loigiai{ Ta có: $2 x^2 + 4 x - 30 = 2 \left(x - 3\right) \left(x + 5\right)$
	}
\end{bt}
%%%=========Bai_536=========%%%
\begin{bt}
	Phân tích đa thức sau thành nhân tử: $2 x^2 + 2 x - 24$.
	\loigiai{ Ta có: $2 x^2 + 2 x - 24 = 2 \left(x - 3\right) \left(x + 4\right)$
	}
\end{bt}
%%%=========Bai_537=========%%%
\begin{bt}
	Phân tích đa thức sau thành nhân tử: $2 x^2 - 2 x - 12$.
	\loigiai{ Ta có: $2 x^2 - 2 x - 12 = 2 \left(x - 3\right) \left(x + 2\right)$
	}
\end{bt}
%%%=========Bai_538=========%%%
\begin{bt}
	Phân tích đa thức sau thành nhân tử: $2 x^2 - 4 x - 6$.
	\loigiai{ Ta có: $2 x^2 - 4 x - 6 = 2 \left(x - 3\right) \left(x + 1\right)$
	}
\end{bt}
%%%=========Bai_539=========%%%
\begin{bt}
	Phân tích đa thức sau thành nhân tử: $2 x^2 - 8 x + 6$.
	\loigiai{ Ta có: $2 x^2 - 8 x + 6 = 2 \left(x - 3\right) \left(x - 1\right)$
	}
\end{bt}
%%%=========Bai_540=========%%%
\begin{bt}
	Phân tích đa thức sau thành nhân tử: $2 x^2 - 10 x + 12$.
	\loigiai{ Ta có: $2 x^2 - 10 x + 12 = 2 \left(x - 3\right) \left(x - 2\right)$
	}
\end{bt}
%%%=========Bai_541=========%%%
\begin{bt}
	Phân tích đa thức sau thành nhân tử: $2 x^2 - 14 x + 24$.
	\loigiai{ Ta có: $2 x^2 - 14 x + 24 = 2 \left(x - 4\right) \left(x - 3\right)$
	}
\end{bt}
%%%=========Bai_542=========%%%
\begin{bt}
	Phân tích đa thức sau thành nhân tử: $2 x^2 - 16 x + 30$.
	\loigiai{ Ta có: $2 x^2 - 16 x + 30 = 2 \left(x - 5\right) \left(x - 3\right)$
	}
\end{bt}
%%%=========Bai_543=========%%%
\begin{bt}
	Phân tích đa thức sau thành nhân tử: $2 x^2 - 18 x + 36$.
	\loigiai{ Ta có: $2 x^2 - 18 x + 36 = 2 \left(x - 6\right) \left(x - 3\right)$
	}
\end{bt}
%%%=========Bai_544=========%%%
\begin{bt}
	Phân tích đa thức sau thành nhân tử: $2 x^2 - 20 x + 42$.
	\loigiai{ Ta có: $2 x^2 - 20 x + 42 = 2 \left(x - 7\right) \left(x - 3\right)$
	}
\end{bt}
%%%=========Bai_545=========%%%
\begin{bt}
	Phân tích đa thức sau thành nhân tử: $2 x^2 - 22 x + 48$.
	\loigiai{ Ta có: $2 x^2 - 22 x + 48 = 2 \left(x - 8\right) \left(x - 3\right)$
	}
\end{bt}
%%%=========Bai_546=========%%%
\begin{bt}
	Phân tích đa thức sau thành nhân tử: $2 x^2 - 24 x + 54$.
	\loigiai{ Ta có: $2 x^2 - 24 x + 54 = 2 \left(x - 9\right) \left(x - 3\right)$
	}
\end{bt}
%%%=========Bai_547=========%%%
\begin{bt}
	Phân tích đa thức sau thành nhân tử: $2 x^2 + 12 x - 80$.
	\loigiai{ Ta có: $2 x^2 + 12 x - 80 = 2 \left(x - 4\right) \left(x + 10\right)$
	}
\end{bt}
%%%=========Bai_548=========%%%
\begin{bt}
	Phân tích đa thức sau thành nhân tử: $2 x^2 + 10 x - 72$.
	\loigiai{ Ta có: $2 x^2 + 10 x - 72 = 2 \left(x - 4\right) \left(x + 9\right)$
	}
\end{bt}
%%%=========Bai_549=========%%%
\begin{bt}
	Phân tích đa thức sau thành nhân tử: $2 x^2 + 8 x - 64$.
	\loigiai{ Ta có: $2 x^2 + 8 x - 64 = 2 \left(x - 4\right) \left(x + 8\right)$
	}
\end{bt}
%%%=========Bai_550=========%%%
\begin{bt}
	Phân tích đa thức sau thành nhân tử: $2 x^2 + 6 x - 56$.
	\loigiai{ Ta có: $2 x^2 + 6 x - 56 = 2 \left(x - 4\right) \left(x + 7\right)$
	}
\end{bt}
%%%=========Bai_551=========%%%
\begin{bt}
	Phân tích đa thức sau thành nhân tử: $2 x^2 + 4 x - 48$.
	\loigiai{ Ta có: $2 x^2 + 4 x - 48 = 2 \left(x - 4\right) \left(x + 6\right)$
	}
\end{bt}
%%%=========Bai_552=========%%%
\begin{bt}
	Phân tích đa thức sau thành nhân tử: $2 x^2 + 2 x - 40$.
	\loigiai{ Ta có: $2 x^2 + 2 x - 40 = 2 \left(x - 4\right) \left(x + 5\right)$
	}
\end{bt}
%%%=========Bai_553=========%%%
\begin{bt}
	Phân tích đa thức sau thành nhân tử: $2 x^2 - 2 x - 24$.
	\loigiai{ Ta có: $2 x^2 - 2 x - 24 = 2 \left(x - 4\right) \left(x + 3\right)$
	}
\end{bt}
%%%=========Bai_554=========%%%
\begin{bt}
	Phân tích đa thức sau thành nhân tử: $2 x^2 - 4 x - 16$.
	\loigiai{ Ta có: $2 x^2 - 4 x - 16 = 2 \left(x - 4\right) \left(x + 2\right)$
	}
\end{bt}
%%%=========Bai_555=========%%%
\begin{bt}
	Phân tích đa thức sau thành nhân tử: $2 x^2 - 6 x - 8$.
	\loigiai{ Ta có: $2 x^2 - 6 x - 8 = 2 \left(x - 4\right) \left(x + 1\right)$
	}
\end{bt}
%%%=========Bai_556=========%%%
\begin{bt}
	Phân tích đa thức sau thành nhân tử: $2 x^2 - 10 x + 8$.
	\loigiai{ Ta có: $2 x^2 - 10 x + 8 = 2 \left(x - 4\right) \left(x - 1\right)$
	}
\end{bt}
%%%=========Bai_557=========%%%
\begin{bt}
	Phân tích đa thức sau thành nhân tử: $2 x^2 - 12 x + 16$.
	\loigiai{ Ta có: $2 x^2 - 12 x + 16 = 2 \left(x - 4\right) \left(x - 2\right)$
	}
\end{bt}
%%%=========Bai_558=========%%%
\begin{bt}
	Phân tích đa thức sau thành nhân tử: $2 x^2 - 14 x + 24$.
	\loigiai{ Ta có: $2 x^2 - 14 x + 24 = 2 \left(x - 4\right) \left(x - 3\right)$
	}
\end{bt}
%%%=========Bai_559=========%%%
\begin{bt}
	Phân tích đa thức sau thành nhân tử: $2 x^2 - 18 x + 40$.
	\loigiai{ Ta có: $2 x^2 - 18 x + 40 = 2 \left(x - 5\right) \left(x - 4\right)$
	}
\end{bt}
%%%=========Bai_560=========%%%
\begin{bt}
	Phân tích đa thức sau thành nhân tử: $2 x^2 - 20 x + 48$.
	\loigiai{ Ta có: $2 x^2 - 20 x + 48 = 2 \left(x - 6\right) \left(x - 4\right)$
	}
\end{bt}
%%%=========Bai_561=========%%%
\begin{bt}
	Phân tích đa thức sau thành nhân tử: $2 x^2 - 22 x + 56$.
	\loigiai{ Ta có: $2 x^2 - 22 x + 56 = 2 \left(x - 7\right) \left(x - 4\right)$
	}
\end{bt}
%%%=========Bai_562=========%%%
\begin{bt}
	Phân tích đa thức sau thành nhân tử: $2 x^2 - 24 x + 64$.
	\loigiai{ Ta có: $2 x^2 - 24 x + 64 = 2 \left(x - 8\right) \left(x - 4\right)$
	}
\end{bt}
%%%=========Bai_563=========%%%
\begin{bt}
	Phân tích đa thức sau thành nhân tử: $2 x^2 - 26 x + 72$.
	\loigiai{ Ta có: $2 x^2 - 26 x + 72 = 2 \left(x - 9\right) \left(x - 4\right)$
	}
\end{bt}
%%%=========Bai_564=========%%%
\begin{bt}
	Phân tích đa thức sau thành nhân tử: $2 x^2 + 10 x - 100$.
	\loigiai{ Ta có: $2 x^2 + 10 x - 100 = 2 \left(x - 5\right) \left(x + 10\right)$
	}
\end{bt}
%%%=========Bai_565=========%%%
\begin{bt}
	Phân tích đa thức sau thành nhân tử: $2 x^2 + 8 x - 90$.
	\loigiai{ Ta có: $2 x^2 + 8 x - 90 = 2 \left(x - 5\right) \left(x + 9\right)$
	}
\end{bt}
%%%=========Bai_566=========%%%
\begin{bt}
	Phân tích đa thức sau thành nhân tử: $2 x^2 + 6 x - 80$.
	\loigiai{ Ta có: $2 x^2 + 6 x - 80 = 2 \left(x - 5\right) \left(x + 8\right)$
	}
\end{bt}
%%%=========Bai_567=========%%%
\begin{bt}
	Phân tích đa thức sau thành nhân tử: $2 x^2 + 4 x - 70$.
	\loigiai{ Ta có: $2 x^2 + 4 x - 70 = 2 \left(x - 5\right) \left(x + 7\right)$
	}
\end{bt}
%%%=========Bai_568=========%%%
\begin{bt}
	Phân tích đa thức sau thành nhân tử: $2 x^2 + 2 x - 60$.
	\loigiai{ Ta có: $2 x^2 + 2 x - 60 = 2 \left(x - 5\right) \left(x + 6\right)$
	}
\end{bt}
%%%=========Bai_569=========%%%
\begin{bt}
	Phân tích đa thức sau thành nhân tử: $2 x^2 - 2 x - 40$.
	\loigiai{ Ta có: $2 x^2 - 2 x - 40 = 2 \left(x - 5\right) \left(x + 4\right)$
	}
\end{bt}
%%%=========Bai_570=========%%%
\begin{bt}
	Phân tích đa thức sau thành nhân tử: $2 x^2 - 4 x - 30$.
	\loigiai{ Ta có: $2 x^2 - 4 x - 30 = 2 \left(x - 5\right) \left(x + 3\right)$
	}
\end{bt}
%%%=========Bai_571=========%%%
\begin{bt}
	Phân tích đa thức sau thành nhân tử: $2 x^2 - 6 x - 20$.
	\loigiai{ Ta có: $2 x^2 - 6 x - 20 = 2 \left(x - 5\right) \left(x + 2\right)$
	}
\end{bt}
%%%=========Bai_572=========%%%
\begin{bt}
	Phân tích đa thức sau thành nhân tử: $2 x^2 - 8 x - 10$.
	\loigiai{ Ta có: $2 x^2 - 8 x - 10 = 2 \left(x - 5\right) \left(x + 1\right)$
	}
\end{bt}
%%%=========Bai_573=========%%%
\begin{bt}
	Phân tích đa thức sau thành nhân tử: $2 x^2 - 12 x + 10$.
	\loigiai{ Ta có: $2 x^2 - 12 x + 10 = 2 \left(x - 5\right) \left(x - 1\right)$
	}
\end{bt}
%%%=========Bai_574=========%%%
\begin{bt}
	Phân tích đa thức sau thành nhân tử: $2 x^2 - 14 x + 20$.
	\loigiai{ Ta có: $2 x^2 - 14 x + 20 = 2 \left(x - 5\right) \left(x - 2\right)$
	}
\end{bt}
%%%=========Bai_575=========%%%
\begin{bt}
	Phân tích đa thức sau thành nhân tử: $2 x^2 - 16 x + 30$.
	\loigiai{ Ta có: $2 x^2 - 16 x + 30 = 2 \left(x - 5\right) \left(x - 3\right)$
	}
\end{bt}
%%%=========Bai_576=========%%%
\begin{bt}
	Phân tích đa thức sau thành nhân tử: $2 x^2 - 18 x + 40$.
	\loigiai{ Ta có: $2 x^2 - 18 x + 40 = 2 \left(x - 5\right) \left(x - 4\right)$
	}
\end{bt}
%%%=========Bai_577=========%%%
\begin{bt}
	Phân tích đa thức sau thành nhân tử: $2 x^2 - 22 x + 60$.
	\loigiai{ Ta có: $2 x^2 - 22 x + 60 = 2 \left(x - 6\right) \left(x - 5\right)$
	}
\end{bt}
%%%=========Bai_578=========%%%
\begin{bt}
	Phân tích đa thức sau thành nhân tử: $2 x^2 - 24 x + 70$.
	\loigiai{ Ta có: $2 x^2 - 24 x + 70 = 2 \left(x - 7\right) \left(x - 5\right)$
	}
\end{bt}
%%%=========Bai_579=========%%%
\begin{bt}
	Phân tích đa thức sau thành nhân tử: $2 x^2 - 26 x + 80$.
	\loigiai{ Ta có: $2 x^2 - 26 x + 80 = 2 \left(x - 8\right) \left(x - 5\right)$
	}
\end{bt}
%%%=========Bai_580=========%%%
\begin{bt}
	Phân tích đa thức sau thành nhân tử: $2 x^2 - 28 x + 90$.
	\loigiai{ Ta có: $2 x^2 - 28 x + 90 = 2 \left(x - 9\right) \left(x - 5\right)$
	}
\end{bt}
%%%=========Bai_581=========%%%
\begin{bt}
	Phân tích đa thức sau thành nhân tử: $2 x^2 + 8 x - 120$.
	\loigiai{ Ta có: $2 x^2 + 8 x - 120 = 2 \left(x - 6\right) \left(x + 10\right)$
	}
\end{bt}
%%%=========Bai_582=========%%%
\begin{bt}
	Phân tích đa thức sau thành nhân tử: $2 x^2 + 6 x - 108$.
	\loigiai{ Ta có: $2 x^2 + 6 x - 108 = 2 \left(x - 6\right) \left(x + 9\right)$
	}
\end{bt}
%%%=========Bai_583=========%%%
\begin{bt}
	Phân tích đa thức sau thành nhân tử: $2 x^2 + 4 x - 96$.
	\loigiai{ Ta có: $2 x^2 + 4 x - 96 = 2 \left(x - 6\right) \left(x + 8\right)$
	}
\end{bt}
%%%=========Bai_584=========%%%
\begin{bt}
	Phân tích đa thức sau thành nhân tử: $2 x^2 + 2 x - 84$.
	\loigiai{ Ta có: $2 x^2 + 2 x - 84 = 2 \left(x - 6\right) \left(x + 7\right)$
	}
\end{bt}
%%%=========Bai_585=========%%%
\begin{bt}
	Phân tích đa thức sau thành nhân tử: $2 x^2 - 2 x - 60$.
	\loigiai{ Ta có: $2 x^2 - 2 x - 60 = 2 \left(x - 6\right) \left(x + 5\right)$
	}
\end{bt}
%%%=========Bai_586=========%%%
\begin{bt}
	Phân tích đa thức sau thành nhân tử: $2 x^2 - 4 x - 48$.
	\loigiai{ Ta có: $2 x^2 - 4 x - 48 = 2 \left(x - 6\right) \left(x + 4\right)$
	}
\end{bt}
%%%=========Bai_587=========%%%
\begin{bt}
	Phân tích đa thức sau thành nhân tử: $2 x^2 - 6 x - 36$.
	\loigiai{ Ta có: $2 x^2 - 6 x - 36 = 2 \left(x - 6\right) \left(x + 3\right)$
	}
\end{bt}
%%%=========Bai_588=========%%%
\begin{bt}
	Phân tích đa thức sau thành nhân tử: $2 x^2 - 8 x - 24$.
	\loigiai{ Ta có: $2 x^2 - 8 x - 24 = 2 \left(x - 6\right) \left(x + 2\right)$
	}
\end{bt}
%%%=========Bai_589=========%%%
\begin{bt}
	Phân tích đa thức sau thành nhân tử: $2 x^2 - 10 x - 12$.
	\loigiai{ Ta có: $2 x^2 - 10 x - 12 = 2 \left(x - 6\right) \left(x + 1\right)$
	}
\end{bt}
%%%=========Bai_590=========%%%
\begin{bt}
	Phân tích đa thức sau thành nhân tử: $2 x^2 - 14 x + 12$.
	\loigiai{ Ta có: $2 x^2 - 14 x + 12 = 2 \left(x - 6\right) \left(x - 1\right)$
	}
\end{bt}
%%%=========Bai_591=========%%%
\begin{bt}
	Phân tích đa thức sau thành nhân tử: $2 x^2 - 16 x + 24$.
	\loigiai{ Ta có: $2 x^2 - 16 x + 24 = 2 \left(x - 6\right) \left(x - 2\right)$
	}
\end{bt}
%%%=========Bai_592=========%%%
\begin{bt}
	Phân tích đa thức sau thành nhân tử: $2 x^2 - 18 x + 36$.
	\loigiai{ Ta có: $2 x^2 - 18 x + 36 = 2 \left(x - 6\right) \left(x - 3\right)$
	}
\end{bt}
%%%=========Bai_593=========%%%
\begin{bt}
	Phân tích đa thức sau thành nhân tử: $2 x^2 - 20 x + 48$.
	\loigiai{ Ta có: $2 x^2 - 20 x + 48 = 2 \left(x - 6\right) \left(x - 4\right)$
	}
\end{bt}
%%%=========Bai_594=========%%%
\begin{bt}
	Phân tích đa thức sau thành nhân tử: $2 x^2 - 22 x + 60$.
	\loigiai{ Ta có: $2 x^2 - 22 x + 60 = 2 \left(x - 6\right) \left(x - 5\right)$
	}
\end{bt}
%%%=========Bai_595=========%%%
\begin{bt}
	Phân tích đa thức sau thành nhân tử: $2 x^2 - 26 x + 84$.
	\loigiai{ Ta có: $2 x^2 - 26 x + 84 = 2 \left(x - 7\right) \left(x - 6\right)$
	}
\end{bt}
%%%=========Bai_596=========%%%
\begin{bt}
	Phân tích đa thức sau thành nhân tử: $2 x^2 - 28 x + 96$.
	\loigiai{ Ta có: $2 x^2 - 28 x + 96 = 2 \left(x - 8\right) \left(x - 6\right)$
	}
\end{bt}
%%%=========Bai_597=========%%%
\begin{bt}
	Phân tích đa thức sau thành nhân tử: $2 x^2 - 30 x + 108$.
	\loigiai{ Ta có: $2 x^2 - 30 x + 108 = 2 \left(x - 9\right) \left(x - 6\right)$
	}
\end{bt}
%%%=========Bai_598=========%%%
\begin{bt}
	Phân tích đa thức sau thành nhân tử: $2 x^2 + 6 x - 140$.
	\loigiai{ Ta có: $2 x^2 + 6 x - 140 = 2 \left(x - 7\right) \left(x + 10\right)$
	}
\end{bt}
%%%=========Bai_599=========%%%
\begin{bt}
	Phân tích đa thức sau thành nhân tử: $2 x^2 + 4 x - 126$.
	\loigiai{ Ta có: $2 x^2 + 4 x - 126 = 2 \left(x - 7\right) \left(x + 9\right)$
	}
\end{bt}
%%%=========Bai_600=========%%%
\begin{bt}
	Phân tích đa thức sau thành nhân tử: $2 x^2 + 2 x - 112$.
	\loigiai{ Ta có: $2 x^2 + 2 x - 112 = 2 \left(x - 7\right) \left(x + 8\right)$
	}
\end{bt}
%%%=========Bai_601=========%%%
\begin{bt}
	Phân tích đa thức sau thành nhân tử: $2 x^2 - 2 x - 84$.
	\loigiai{ Ta có: $2 x^2 - 2 x - 84 = 2 \left(x - 7\right) \left(x + 6\right)$
	}
\end{bt}
%%%=========Bai_602=========%%%
\begin{bt}
	Phân tích đa thức sau thành nhân tử: $2 x^2 - 4 x - 70$.
	\loigiai{ Ta có: $2 x^2 - 4 x - 70 = 2 \left(x - 7\right) \left(x + 5\right)$
	}
\end{bt}
%%%=========Bai_603=========%%%
\begin{bt}
	Phân tích đa thức sau thành nhân tử: $2 x^2 - 6 x - 56$.
	\loigiai{ Ta có: $2 x^2 - 6 x - 56 = 2 \left(x - 7\right) \left(x + 4\right)$
	}
\end{bt}
%%%=========Bai_604=========%%%
\begin{bt}
	Phân tích đa thức sau thành nhân tử: $2 x^2 - 8 x - 42$.
	\loigiai{ Ta có: $2 x^2 - 8 x - 42 = 2 \left(x - 7\right) \left(x + 3\right)$
	}
\end{bt}
%%%=========Bai_605=========%%%
\begin{bt}
	Phân tích đa thức sau thành nhân tử: $2 x^2 - 10 x - 28$.
	\loigiai{ Ta có: $2 x^2 - 10 x - 28 = 2 \left(x - 7\right) \left(x + 2\right)$
	}
\end{bt}
%%%=========Bai_606=========%%%
\begin{bt}
	Phân tích đa thức sau thành nhân tử: $2 x^2 - 12 x - 14$.
	\loigiai{ Ta có: $2 x^2 - 12 x - 14 = 2 \left(x - 7\right) \left(x + 1\right)$
	}
\end{bt}
%%%=========Bai_607=========%%%
\begin{bt}
	Phân tích đa thức sau thành nhân tử: $2 x^2 - 16 x + 14$.
	\loigiai{ Ta có: $2 x^2 - 16 x + 14 = 2 \left(x - 7\right) \left(x - 1\right)$
	}
\end{bt}
%%%=========Bai_608=========%%%
\begin{bt}
	Phân tích đa thức sau thành nhân tử: $2 x^2 - 18 x + 28$.
	\loigiai{ Ta có: $2 x^2 - 18 x + 28 = 2 \left(x - 7\right) \left(x - 2\right)$
	}
\end{bt}
%%%=========Bai_609=========%%%
\begin{bt}
	Phân tích đa thức sau thành nhân tử: $2 x^2 - 20 x + 42$.
	\loigiai{ Ta có: $2 x^2 - 20 x + 42 = 2 \left(x - 7\right) \left(x - 3\right)$
	}
\end{bt}
%%%=========Bai_610=========%%%
\begin{bt}
	Phân tích đa thức sau thành nhân tử: $2 x^2 - 22 x + 56$.
	\loigiai{ Ta có: $2 x^2 - 22 x + 56 = 2 \left(x - 7\right) \left(x - 4\right)$
	}
\end{bt}
%%%=========Bai_611=========%%%
\begin{bt}
	Phân tích đa thức sau thành nhân tử: $2 x^2 - 24 x + 70$.
	\loigiai{ Ta có: $2 x^2 - 24 x + 70 = 2 \left(x - 7\right) \left(x - 5\right)$
	}
\end{bt}
%%%=========Bai_612=========%%%
\begin{bt}
	Phân tích đa thức sau thành nhân tử: $2 x^2 - 26 x + 84$.
	\loigiai{ Ta có: $2 x^2 - 26 x + 84 = 2 \left(x - 7\right) \left(x - 6\right)$
	}
\end{bt}
%%%=========Bai_613=========%%%
\begin{bt}
	Phân tích đa thức sau thành nhân tử: $2 x^2 - 30 x + 112$.
	\loigiai{ Ta có: $2 x^2 - 30 x + 112 = 2 \left(x - 8\right) \left(x - 7\right)$
	}
\end{bt}
%%%=========Bai_614=========%%%
\begin{bt}
	Phân tích đa thức sau thành nhân tử: $2 x^2 - 32 x + 126$.
	\loigiai{ Ta có: $2 x^2 - 32 x + 126 = 2 \left(x - 9\right) \left(x - 7\right)$
	}
\end{bt}
%%%=========Bai_615=========%%%
\begin{bt}
	Phân tích đa thức sau thành nhân tử: $2 x^2 + 4 x - 160$.
	\loigiai{ Ta có: $2 x^2 + 4 x - 160 = 2 \left(x - 8\right) \left(x + 10\right)$
	}
\end{bt}
%%%=========Bai_616=========%%%
\begin{bt}
	Phân tích đa thức sau thành nhân tử: $2 x^2 + 2 x - 144$.
	\loigiai{ Ta có: $2 x^2 + 2 x - 144 = 2 \left(x - 8\right) \left(x + 9\right)$
	}
\end{bt}
%%%=========Bai_617=========%%%
\begin{bt}
	Phân tích đa thức sau thành nhân tử: $2 x^2 - 2 x - 112$.
	\loigiai{ Ta có: $2 x^2 - 2 x - 112 = 2 \left(x - 8\right) \left(x + 7\right)$
	}
\end{bt}
%%%=========Bai_618=========%%%
\begin{bt}
	Phân tích đa thức sau thành nhân tử: $2 x^2 - 4 x - 96$.
	\loigiai{ Ta có: $2 x^2 - 4 x - 96 = 2 \left(x - 8\right) \left(x + 6\right)$
	}
\end{bt}
%%%=========Bai_619=========%%%
\begin{bt}
	Phân tích đa thức sau thành nhân tử: $2 x^2 - 6 x - 80$.
	\loigiai{ Ta có: $2 x^2 - 6 x - 80 = 2 \left(x - 8\right) \left(x + 5\right)$
	}
\end{bt}
%%%=========Bai_620=========%%%
\begin{bt}
	Phân tích đa thức sau thành nhân tử: $2 x^2 - 8 x - 64$.
	\loigiai{ Ta có: $2 x^2 - 8 x - 64 = 2 \left(x - 8\right) \left(x + 4\right)$
	}
\end{bt}
%%%=========Bai_621=========%%%
\begin{bt}
	Phân tích đa thức sau thành nhân tử: $2 x^2 - 10 x - 48$.
	\loigiai{ Ta có: $2 x^2 - 10 x - 48 = 2 \left(x - 8\right) \left(x + 3\right)$
	}
\end{bt}
%%%=========Bai_622=========%%%
\begin{bt}
	Phân tích đa thức sau thành nhân tử: $2 x^2 - 12 x - 32$.
	\loigiai{ Ta có: $2 x^2 - 12 x - 32 = 2 \left(x - 8\right) \left(x + 2\right)$
	}
\end{bt}
%%%=========Bai_623=========%%%
\begin{bt}
	Phân tích đa thức sau thành nhân tử: $2 x^2 - 14 x - 16$.
	\loigiai{ Ta có: $2 x^2 - 14 x - 16 = 2 \left(x - 8\right) \left(x + 1\right)$
	}
\end{bt}
%%%=========Bai_624=========%%%
\begin{bt}
	Phân tích đa thức sau thành nhân tử: $2 x^2 - 18 x + 16$.
	\loigiai{ Ta có: $2 x^2 - 18 x + 16 = 2 \left(x - 8\right) \left(x - 1\right)$
	}
\end{bt}
%%%=========Bai_625=========%%%
\begin{bt}
	Phân tích đa thức sau thành nhân tử: $2 x^2 - 20 x + 32$.
	\loigiai{ Ta có: $2 x^2 - 20 x + 32 = 2 \left(x - 8\right) \left(x - 2\right)$
	}
\end{bt}
%%%=========Bai_626=========%%%
\begin{bt}
	Phân tích đa thức sau thành nhân tử: $2 x^2 - 22 x + 48$.
	\loigiai{ Ta có: $2 x^2 - 22 x + 48 = 2 \left(x - 8\right) \left(x - 3\right)$
	}
\end{bt}
%%%=========Bai_627=========%%%
\begin{bt}
	Phân tích đa thức sau thành nhân tử: $2 x^2 - 24 x + 64$.
	\loigiai{ Ta có: $2 x^2 - 24 x + 64 = 2 \left(x - 8\right) \left(x - 4\right)$
	}
\end{bt}
%%%=========Bai_628=========%%%
\begin{bt}
	Phân tích đa thức sau thành nhân tử: $2 x^2 - 26 x + 80$.
	\loigiai{ Ta có: $2 x^2 - 26 x + 80 = 2 \left(x - 8\right) \left(x - 5\right)$
	}
\end{bt}
%%%=========Bai_629=========%%%
\begin{bt}
	Phân tích đa thức sau thành nhân tử: $2 x^2 - 28 x + 96$.
	\loigiai{ Ta có: $2 x^2 - 28 x + 96 = 2 \left(x - 8\right) \left(x - 6\right)$
	}
\end{bt}
%%%=========Bai_630=========%%%
\begin{bt}
	Phân tích đa thức sau thành nhân tử: $2 x^2 - 30 x + 112$.
	\loigiai{ Ta có: $2 x^2 - 30 x + 112 = 2 \left(x - 8\right) \left(x - 7\right)$
	}
\end{bt}
%%%=========Bai_631=========%%%
\begin{bt}
	Phân tích đa thức sau thành nhân tử: $2 x^2 - 34 x + 144$.
	\loigiai{ Ta có: $2 x^2 - 34 x + 144 = 2 \left(x - 9\right) \left(x - 8\right)$
	}
\end{bt}
%%%=========Bai_632=========%%%
\begin{bt}
	Phân tích đa thức sau thành nhân tử: $2 x^2 + 2 x - 180$.
	\loigiai{ Ta có: $2 x^2 + 2 x - 180 = 2 \left(x - 9\right) \left(x + 10\right)$
	}
\end{bt}
%%%=========Bai_633=========%%%
\begin{bt}
	Phân tích đa thức sau thành nhân tử: $2 x^2 - 2 x - 144$.
	\loigiai{ Ta có: $2 x^2 - 2 x - 144 = 2 \left(x - 9\right) \left(x + 8\right)$
	}
\end{bt}
%%%=========Bai_634=========%%%
\begin{bt}
	Phân tích đa thức sau thành nhân tử: $2 x^2 - 4 x - 126$.
	\loigiai{ Ta có: $2 x^2 - 4 x - 126 = 2 \left(x - 9\right) \left(x + 7\right)$
	}
\end{bt}
%%%=========Bai_635=========%%%
\begin{bt}
	Phân tích đa thức sau thành nhân tử: $2 x^2 - 6 x - 108$.
	\loigiai{ Ta có: $2 x^2 - 6 x - 108 = 2 \left(x - 9\right) \left(x + 6\right)$
	}
\end{bt}
%%%=========Bai_636=========%%%
\begin{bt}
	Phân tích đa thức sau thành nhân tử: $2 x^2 - 8 x - 90$.
	\loigiai{ Ta có: $2 x^2 - 8 x - 90 = 2 \left(x - 9\right) \left(x + 5\right)$
	}
\end{bt}
%%%=========Bai_637=========%%%
\begin{bt}
	Phân tích đa thức sau thành nhân tử: $2 x^2 - 10 x - 72$.
	\loigiai{ Ta có: $2 x^2 - 10 x - 72 = 2 \left(x - 9\right) \left(x + 4\right)$
	}
\end{bt}
%%%=========Bai_638=========%%%
\begin{bt}
	Phân tích đa thức sau thành nhân tử: $2 x^2 - 12 x - 54$.
	\loigiai{ Ta có: $2 x^2 - 12 x - 54 = 2 \left(x - 9\right) \left(x + 3\right)$
	}
\end{bt}
%%%=========Bai_639=========%%%
\begin{bt}
	Phân tích đa thức sau thành nhân tử: $2 x^2 - 14 x - 36$.
	\loigiai{ Ta có: $2 x^2 - 14 x - 36 = 2 \left(x - 9\right) \left(x + 2\right)$
	}
\end{bt}
%%%=========Bai_640=========%%%
\begin{bt}
	Phân tích đa thức sau thành nhân tử: $2 x^2 - 16 x - 18$.
	\loigiai{ Ta có: $2 x^2 - 16 x - 18 = 2 \left(x - 9\right) \left(x + 1\right)$
	}
\end{bt}
%%%=========Bai_641=========%%%
\begin{bt}
	Phân tích đa thức sau thành nhân tử: $2 x^2 - 20 x + 18$.
	\loigiai{ Ta có: $2 x^2 - 20 x + 18 = 2 \left(x - 9\right) \left(x - 1\right)$
	}
\end{bt}
%%%=========Bai_642=========%%%
\begin{bt}
	Phân tích đa thức sau thành nhân tử: $2 x^2 - 22 x + 36$.
	\loigiai{ Ta có: $2 x^2 - 22 x + 36 = 2 \left(x - 9\right) \left(x - 2\right)$
	}
\end{bt}
%%%=========Bai_643=========%%%
\begin{bt}
	Phân tích đa thức sau thành nhân tử: $2 x^2 - 24 x + 54$.
	\loigiai{ Ta có: $2 x^2 - 24 x + 54 = 2 \left(x - 9\right) \left(x - 3\right)$
	}
\end{bt}
%%%=========Bai_644=========%%%
\begin{bt}
	Phân tích đa thức sau thành nhân tử: $2 x^2 - 26 x + 72$.
	\loigiai{ Ta có: $2 x^2 - 26 x + 72 = 2 \left(x - 9\right) \left(x - 4\right)$
	}
\end{bt}
%%%=========Bai_645=========%%%
\begin{bt}
	Phân tích đa thức sau thành nhân tử: $2 x^2 - 28 x + 90$.
	\loigiai{ Ta có: $2 x^2 - 28 x + 90 = 2 \left(x - 9\right) \left(x - 5\right)$
	}
\end{bt}
%%%=========Bai_646=========%%%
\begin{bt}
	Phân tích đa thức sau thành nhân tử: $2 x^2 - 30 x + 108$.
	\loigiai{ Ta có: $2 x^2 - 30 x + 108 = 2 \left(x - 9\right) \left(x - 6\right)$
	}
\end{bt}
%%%=========Bai_647=========%%%
\begin{bt}
	Phân tích đa thức sau thành nhân tử: $2 x^2 - 32 x + 126$.
	\loigiai{ Ta có: $2 x^2 - 32 x + 126 = 2 \left(x - 9\right) \left(x - 7\right)$
	}
\end{bt}
%%%=========Bai_648=========%%%
\begin{bt}
	Phân tích đa thức sau thành nhân tử: $2 x^2 - 34 x + 144$.
	\loigiai{ Ta có: $2 x^2 - 34 x + 144 = 2 \left(x - 9\right) \left(x - 8\right)$
	}
\end{bt}
%%%=========Bai_649=========%%%
\begin{bt}
	Phân tích đa thức sau thành nhân tử: $3 x^2 + 57 x + 270$.
	\loigiai{ Ta có: $3 x^2 + 57 x + 270 = 3 \left(x + 9\right) \left(x + 10\right)$
	}
\end{bt}
%%%=========Bai_650=========%%%
\begin{bt}
	Phân tích đa thức sau thành nhân tử: $3 x^2 + 54 x + 240$.
	\loigiai{ Ta có: $3 x^2 + 54 x + 240 = 3 \left(x + 8\right) \left(x + 10\right)$
	}
\end{bt}
%%%=========Bai_651=========%%%
\begin{bt}
	Phân tích đa thức sau thành nhân tử: $3 x^2 + 51 x + 210$.
	\loigiai{ Ta có: $3 x^2 + 51 x + 210 = 3 \left(x + 7\right) \left(x + 10\right)$
	}
\end{bt}
%%%=========Bai_652=========%%%
\begin{bt}
	Phân tích đa thức sau thành nhân tử: $3 x^2 + 48 x + 180$.
	\loigiai{ Ta có: $3 x^2 + 48 x + 180 = 3 \left(x + 6\right) \left(x + 10\right)$
	}
\end{bt}
%%%=========Bai_653=========%%%
\begin{bt}
	Phân tích đa thức sau thành nhân tử: $3 x^2 + 45 x + 150$.
	\loigiai{ Ta có: $3 x^2 + 45 x + 150 = 3 \left(x + 5\right) \left(x + 10\right)$
	}
\end{bt}
%%%=========Bai_654=========%%%
\begin{bt}
	Phân tích đa thức sau thành nhân tử: $3 x^2 + 42 x + 120$.
	\loigiai{ Ta có: $3 x^2 + 42 x + 120 = 3 \left(x + 4\right) \left(x + 10\right)$
	}
\end{bt}
%%%=========Bai_655=========%%%
\begin{bt}
	Phân tích đa thức sau thành nhân tử: $3 x^2 + 39 x + 90$.
	\loigiai{ Ta có: $3 x^2 + 39 x + 90 = 3 \left(x + 3\right) \left(x + 10\right)$
	}
\end{bt}
%%%=========Bai_656=========%%%
\begin{bt}
	Phân tích đa thức sau thành nhân tử: $3 x^2 + 36 x + 60$.
	\loigiai{ Ta có: $3 x^2 + 36 x + 60 = 3 \left(x + 2\right) \left(x + 10\right)$
	}
\end{bt}
%%%=========Bai_657=========%%%
\begin{bt}
	Phân tích đa thức sau thành nhân tử: $3 x^2 + 33 x + 30$.
	\loigiai{ Ta có: $3 x^2 + 33 x + 30 = 3 \left(x + 1\right) \left(x + 10\right)$
	}
\end{bt}
%%%=========Bai_658=========%%%
\begin{bt}
	Phân tích đa thức sau thành nhân tử: $3 x^2 + 27 x - 30$.
	\loigiai{ Ta có: $3 x^2 + 27 x - 30 = 3 \left(x - 1\right) \left(x + 10\right)$
	}
\end{bt}
%%%=========Bai_659=========%%%
\begin{bt}
	Phân tích đa thức sau thành nhân tử: $3 x^2 + 24 x - 60$.
	\loigiai{ Ta có: $3 x^2 + 24 x - 60 = 3 \left(x - 2\right) \left(x + 10\right)$
	}
\end{bt}
%%%=========Bai_660=========%%%
\begin{bt}
	Phân tích đa thức sau thành nhân tử: $3 x^2 + 21 x - 90$.
	\loigiai{ Ta có: $3 x^2 + 21 x - 90 = 3 \left(x - 3\right) \left(x + 10\right)$
	}
\end{bt}
%%%=========Bai_661=========%%%
\begin{bt}
	Phân tích đa thức sau thành nhân tử: $3 x^2 + 18 x - 120$.
	\loigiai{ Ta có: $3 x^2 + 18 x - 120 = 3 \left(x - 4\right) \left(x + 10\right)$
	}
\end{bt}
%%%=========Bai_662=========%%%
\begin{bt}
	Phân tích đa thức sau thành nhân tử: $3 x^2 + 15 x - 150$.
	\loigiai{ Ta có: $3 x^2 + 15 x - 150 = 3 \left(x - 5\right) \left(x + 10\right)$
	}
\end{bt}
%%%=========Bai_663=========%%%
\begin{bt}
	Phân tích đa thức sau thành nhân tử: $3 x^2 + 12 x - 180$.
	\loigiai{ Ta có: $3 x^2 + 12 x - 180 = 3 \left(x - 6\right) \left(x + 10\right)$
	}
\end{bt}
%%%=========Bai_664=========%%%
\begin{bt}
	Phân tích đa thức sau thành nhân tử: $3 x^2 + 9 x - 210$.
	\loigiai{ Ta có: $3 x^2 + 9 x - 210 = 3 \left(x - 7\right) \left(x + 10\right)$
	}
\end{bt}
%%%=========Bai_665=========%%%
\begin{bt}
	Phân tích đa thức sau thành nhân tử: $3 x^2 + 6 x - 240$.
	\loigiai{ Ta có: $3 x^2 + 6 x - 240 = 3 \left(x - 8\right) \left(x + 10\right)$
	}
\end{bt}
%%%=========Bai_666=========%%%
\begin{bt}
	Phân tích đa thức sau thành nhân tử: $3 x^2 + 3 x - 270$.
	\loigiai{ Ta có: $3 x^2 + 3 x - 270 = 3 \left(x - 9\right) \left(x + 10\right)$
	}
\end{bt}
%%%=========Bai_667=========%%%
\begin{bt}
	Phân tích đa thức sau thành nhân tử: $3 x^2 + 57 x + 270$.
	\loigiai{ Ta có: $3 x^2 + 57 x + 270 = 3 \left(x + 9\right) \left(x + 10\right)$
	}
\end{bt}
%%%=========Bai_668=========%%%
\begin{bt}
	Phân tích đa thức sau thành nhân tử: $3 x^2 + 51 x + 216$.
	\loigiai{ Ta có: $3 x^2 + 51 x + 216 = 3 \left(x + 8\right) \left(x + 9\right)$
	}
\end{bt}
%%%=========Bai_669=========%%%
\begin{bt}
	Phân tích đa thức sau thành nhân tử: $3 x^2 + 48 x + 189$.
	\loigiai{ Ta có: $3 x^2 + 48 x + 189 = 3 \left(x + 7\right) \left(x + 9\right)$
	}
\end{bt}
%%%=========Bai_670=========%%%
\begin{bt}
	Phân tích đa thức sau thành nhân tử: $3 x^2 + 45 x + 162$.
	\loigiai{ Ta có: $3 x^2 + 45 x + 162 = 3 \left(x + 6\right) \left(x + 9\right)$
	}
\end{bt}
%%%=========Bai_671=========%%%
\begin{bt}
	Phân tích đa thức sau thành nhân tử: $3 x^2 + 42 x + 135$.
	\loigiai{ Ta có: $3 x^2 + 42 x + 135 = 3 \left(x + 5\right) \left(x + 9\right)$
	}
\end{bt}
%%%=========Bai_672=========%%%
\begin{bt}
	Phân tích đa thức sau thành nhân tử: $3 x^2 + 39 x + 108$.
	\loigiai{ Ta có: $3 x^2 + 39 x + 108 = 3 \left(x + 4\right) \left(x + 9\right)$
	}
\end{bt}
%%%=========Bai_673=========%%%
\begin{bt}
	Phân tích đa thức sau thành nhân tử: $3 x^2 + 36 x + 81$.
	\loigiai{ Ta có: $3 x^2 + 36 x + 81 = 3 \left(x + 3\right) \left(x + 9\right)$
	}
\end{bt}
%%%=========Bai_674=========%%%
\begin{bt}
	Phân tích đa thức sau thành nhân tử: $3 x^2 + 33 x + 54$.
	\loigiai{ Ta có: $3 x^2 + 33 x + 54 = 3 \left(x + 2\right) \left(x + 9\right)$
	}
\end{bt}
%%%=========Bai_675=========%%%
\begin{bt}
	Phân tích đa thức sau thành nhân tử: $3 x^2 + 30 x + 27$.
	\loigiai{ Ta có: $3 x^2 + 30 x + 27 = 3 \left(x + 1\right) \left(x + 9\right)$
	}
\end{bt}
%%%=========Bai_676=========%%%
\begin{bt}
	Phân tích đa thức sau thành nhân tử: $3 x^2 + 24 x - 27$.
	\loigiai{ Ta có: $3 x^2 + 24 x - 27 = 3 \left(x - 1\right) \left(x + 9\right)$
	}
\end{bt}
%%%=========Bai_677=========%%%
\begin{bt}
	Phân tích đa thức sau thành nhân tử: $3 x^2 + 21 x - 54$.
	\loigiai{ Ta có: $3 x^2 + 21 x - 54 = 3 \left(x - 2\right) \left(x + 9\right)$
	}
\end{bt}
%%%=========Bai_678=========%%%
\begin{bt}
	Phân tích đa thức sau thành nhân tử: $3 x^2 + 18 x - 81$.
	\loigiai{ Ta có: $3 x^2 + 18 x - 81 = 3 \left(x - 3\right) \left(x + 9\right)$
	}
\end{bt}
%%%=========Bai_679=========%%%
\begin{bt}
	Phân tích đa thức sau thành nhân tử: $3 x^2 + 15 x - 108$.
	\loigiai{ Ta có: $3 x^2 + 15 x - 108 = 3 \left(x - 4\right) \left(x + 9\right)$
	}
\end{bt}
%%%=========Bai_680=========%%%
\begin{bt}
	Phân tích đa thức sau thành nhân tử: $3 x^2 + 12 x - 135$.
	\loigiai{ Ta có: $3 x^2 + 12 x - 135 = 3 \left(x - 5\right) \left(x + 9\right)$
	}
\end{bt}
%%%=========Bai_681=========%%%
\begin{bt}
	Phân tích đa thức sau thành nhân tử: $3 x^2 + 9 x - 162$.
	\loigiai{ Ta có: $3 x^2 + 9 x - 162 = 3 \left(x - 6\right) \left(x + 9\right)$
	}
\end{bt}
%%%=========Bai_682=========%%%
\begin{bt}
	Phân tích đa thức sau thành nhân tử: $3 x^2 + 6 x - 189$.
	\loigiai{ Ta có: $3 x^2 + 6 x - 189 = 3 \left(x - 7\right) \left(x + 9\right)$
	}
\end{bt}
%%%=========Bai_683=========%%%
\begin{bt}
	Phân tích đa thức sau thành nhân tử: $3 x^2 + 3 x - 216$.
	\loigiai{ Ta có: $3 x^2 + 3 x - 216 = 3 \left(x - 8\right) \left(x + 9\right)$
	}
\end{bt}
%%%=========Bai_684=========%%%
\begin{bt}
	Phân tích đa thức sau thành nhân tử: $3 x^2 + 54 x + 240$.
	\loigiai{ Ta có: $3 x^2 + 54 x + 240 = 3 \left(x + 8\right) \left(x + 10\right)$
	}
\end{bt}
%%%=========Bai_685=========%%%
\begin{bt}
	Phân tích đa thức sau thành nhân tử: $3 x^2 + 51 x + 216$.
	\loigiai{ Ta có: $3 x^2 + 51 x + 216 = 3 \left(x + 8\right) \left(x + 9\right)$
	}
\end{bt}
%%%=========Bai_686=========%%%
\begin{bt}
	Phân tích đa thức sau thành nhân tử: $3 x^2 + 45 x + 168$.
	\loigiai{ Ta có: $3 x^2 + 45 x + 168 = 3 \left(x + 7\right) \left(x + 8\right)$
	}
\end{bt}
%%%=========Bai_687=========%%%
\begin{bt}
	Phân tích đa thức sau thành nhân tử: $3 x^2 + 42 x + 144$.
	\loigiai{ Ta có: $3 x^2 + 42 x + 144 = 3 \left(x + 6\right) \left(x + 8\right)$
	}
\end{bt}
%%%=========Bai_688=========%%%
\begin{bt}
	Phân tích đa thức sau thành nhân tử: $3 x^2 + 39 x + 120$.
	\loigiai{ Ta có: $3 x^2 + 39 x + 120 = 3 \left(x + 5\right) \left(x + 8\right)$
	}
\end{bt}
%%%=========Bai_689=========%%%
\begin{bt}
	Phân tích đa thức sau thành nhân tử: $3 x^2 + 36 x + 96$.
	\loigiai{ Ta có: $3 x^2 + 36 x + 96 = 3 \left(x + 4\right) \left(x + 8\right)$
	}
\end{bt}
%%%=========Bai_690=========%%%
\begin{bt}
	Phân tích đa thức sau thành nhân tử: $3 x^2 + 33 x + 72$.
	\loigiai{ Ta có: $3 x^2 + 33 x + 72 = 3 \left(x + 3\right) \left(x + 8\right)$
	}
\end{bt}
%%%=========Bai_691=========%%%
\begin{bt}
	Phân tích đa thức sau thành nhân tử: $3 x^2 + 30 x + 48$.
	\loigiai{ Ta có: $3 x^2 + 30 x + 48 = 3 \left(x + 2\right) \left(x + 8\right)$
	}
\end{bt}
%%%=========Bai_692=========%%%
\begin{bt}
	Phân tích đa thức sau thành nhân tử: $3 x^2 + 27 x + 24$.
	\loigiai{ Ta có: $3 x^2 + 27 x + 24 = 3 \left(x + 1\right) \left(x + 8\right)$
	}
\end{bt}
%%%=========Bai_693=========%%%
\begin{bt}
	Phân tích đa thức sau thành nhân tử: $3 x^2 + 21 x - 24$.
	\loigiai{ Ta có: $3 x^2 + 21 x - 24 = 3 \left(x - 1\right) \left(x + 8\right)$
	}
\end{bt}
%%%=========Bai_694=========%%%
\begin{bt}
	Phân tích đa thức sau thành nhân tử: $3 x^2 + 18 x - 48$.
	\loigiai{ Ta có: $3 x^2 + 18 x - 48 = 3 \left(x - 2\right) \left(x + 8\right)$
	}
\end{bt}
%%%=========Bai_695=========%%%
\begin{bt}
	Phân tích đa thức sau thành nhân tử: $3 x^2 + 15 x - 72$.
	\loigiai{ Ta có: $3 x^2 + 15 x - 72 = 3 \left(x - 3\right) \left(x + 8\right)$
	}
\end{bt}
%%%=========Bai_696=========%%%
\begin{bt}
	Phân tích đa thức sau thành nhân tử: $3 x^2 + 12 x - 96$.
	\loigiai{ Ta có: $3 x^2 + 12 x - 96 = 3 \left(x - 4\right) \left(x + 8\right)$
	}
\end{bt}
%%%=========Bai_697=========%%%
\begin{bt}
	Phân tích đa thức sau thành nhân tử: $3 x^2 + 9 x - 120$.
	\loigiai{ Ta có: $3 x^2 + 9 x - 120 = 3 \left(x - 5\right) \left(x + 8\right)$
	}
\end{bt}
%%%=========Bai_698=========%%%
\begin{bt}
	Phân tích đa thức sau thành nhân tử: $3 x^2 + 6 x - 144$.
	\loigiai{ Ta có: $3 x^2 + 6 x - 144 = 3 \left(x - 6\right) \left(x + 8\right)$
	}
\end{bt}
%%%=========Bai_699=========%%%
\begin{bt}
	Phân tích đa thức sau thành nhân tử: $3 x^2 + 3 x - 168$.
	\loigiai{ Ta có: $3 x^2 + 3 x - 168 = 3 \left(x - 7\right) \left(x + 8\right)$
	}
\end{bt}
%%%=========Bai_700=========%%%
\begin{bt}
	Phân tích đa thức sau thành nhân tử: $3 x^2 - 3 x - 216$.
	\loigiai{ Ta có: $3 x^2 - 3 x - 216 = 3 \left(x - 9\right) \left(x + 8\right)$
	}
\end{bt}
%%%=========Bai_701=========%%%
\begin{bt}
	Phân tích đa thức sau thành nhân tử: $3 x^2 + 51 x + 210$.
	\loigiai{ Ta có: $3 x^2 + 51 x + 210 = 3 \left(x + 7\right) \left(x + 10\right)$
	}
\end{bt}
%%%=========Bai_702=========%%%
\begin{bt}
	Phân tích đa thức sau thành nhân tử: $3 x^2 + 48 x + 189$.
	\loigiai{ Ta có: $3 x^2 + 48 x + 189 = 3 \left(x + 7\right) \left(x + 9\right)$
	}
\end{bt}
%%%=========Bai_703=========%%%
\begin{bt}
	Phân tích đa thức sau thành nhân tử: $3 x^2 + 45 x + 168$.
	\loigiai{ Ta có: $3 x^2 + 45 x + 168 = 3 \left(x + 7\right) \left(x + 8\right)$
	}
\end{bt}
%%%=========Bai_704=========%%%
\begin{bt}
	Phân tích đa thức sau thành nhân tử: $3 x^2 + 39 x + 126$.
	\loigiai{ Ta có: $3 x^2 + 39 x + 126 = 3 \left(x + 6\right) \left(x + 7\right)$
	}
\end{bt}
%%%=========Bai_705=========%%%
\begin{bt}
	Phân tích đa thức sau thành nhân tử: $3 x^2 + 36 x + 105$.
	\loigiai{ Ta có: $3 x^2 + 36 x + 105 = 3 \left(x + 5\right) \left(x + 7\right)$
	}
\end{bt}
%%%=========Bai_706=========%%%
\begin{bt}
	Phân tích đa thức sau thành nhân tử: $3 x^2 + 33 x + 84$.
	\loigiai{ Ta có: $3 x^2 + 33 x + 84 = 3 \left(x + 4\right) \left(x + 7\right)$
	}
\end{bt}
%%%=========Bai_707=========%%%
\begin{bt}
	Phân tích đa thức sau thành nhân tử: $3 x^2 + 30 x + 63$.
	\loigiai{ Ta có: $3 x^2 + 30 x + 63 = 3 \left(x + 3\right) \left(x + 7\right)$
	}
\end{bt}
%%%=========Bai_708=========%%%
\begin{bt}
	Phân tích đa thức sau thành nhân tử: $3 x^2 + 27 x + 42$.
	\loigiai{ Ta có: $3 x^2 + 27 x + 42 = 3 \left(x + 2\right) \left(x + 7\right)$
	}
\end{bt}
%%%=========Bai_709=========%%%
\begin{bt}
	Phân tích đa thức sau thành nhân tử: $3 x^2 + 24 x + 21$.
	\loigiai{ Ta có: $3 x^2 + 24 x + 21 = 3 \left(x + 1\right) \left(x + 7\right)$
	}
\end{bt}
%%%=========Bai_710=========%%%
\begin{bt}
	Phân tích đa thức sau thành nhân tử: $3 x^2 + 18 x - 21$.
	\loigiai{ Ta có: $3 x^2 + 18 x - 21 = 3 \left(x - 1\right) \left(x + 7\right)$
	}
\end{bt}
%%%=========Bai_711=========%%%
\begin{bt}
	Phân tích đa thức sau thành nhân tử: $3 x^2 + 15 x - 42$.
	\loigiai{ Ta có: $3 x^2 + 15 x - 42 = 3 \left(x - 2\right) \left(x + 7\right)$
	}
\end{bt}
%%%=========Bai_712=========%%%
\begin{bt}
	Phân tích đa thức sau thành nhân tử: $3 x^2 + 12 x - 63$.
	\loigiai{ Ta có: $3 x^2 + 12 x - 63 = 3 \left(x - 3\right) \left(x + 7\right)$
	}
\end{bt}
%%%=========Bai_713=========%%%
\begin{bt}
	Phân tích đa thức sau thành nhân tử: $3 x^2 + 9 x - 84$.
	\loigiai{ Ta có: $3 x^2 + 9 x - 84 = 3 \left(x - 4\right) \left(x + 7\right)$
	}
\end{bt}
%%%=========Bai_714=========%%%
\begin{bt}
	Phân tích đa thức sau thành nhân tử: $3 x^2 + 6 x - 105$.
	\loigiai{ Ta có: $3 x^2 + 6 x - 105 = 3 \left(x - 5\right) \left(x + 7\right)$
	}
\end{bt}
%%%=========Bai_715=========%%%
\begin{bt}
	Phân tích đa thức sau thành nhân tử: $3 x^2 + 3 x - 126$.
	\loigiai{ Ta có: $3 x^2 + 3 x - 126 = 3 \left(x - 6\right) \left(x + 7\right)$
	}
\end{bt}
%%%=========Bai_716=========%%%
\begin{bt}
	Phân tích đa thức sau thành nhân tử: $3 x^2 - 3 x - 168$.
	\loigiai{ Ta có: $3 x^2 - 3 x - 168 = 3 \left(x - 8\right) \left(x + 7\right)$
	}
\end{bt}
%%%=========Bai_717=========%%%
\begin{bt}
	Phân tích đa thức sau thành nhân tử: $3 x^2 - 6 x - 189$.
	\loigiai{ Ta có: $3 x^2 - 6 x - 189 = 3 \left(x - 9\right) \left(x + 7\right)$
	}
\end{bt}
%%%=========Bai_718=========%%%
\begin{bt}
	Phân tích đa thức sau thành nhân tử: $3 x^2 + 48 x + 180$.
	\loigiai{ Ta có: $3 x^2 + 48 x + 180 = 3 \left(x + 6\right) \left(x + 10\right)$
	}
\end{bt}
%%%=========Bai_719=========%%%
\begin{bt}
	Phân tích đa thức sau thành nhân tử: $3 x^2 + 45 x + 162$.
	\loigiai{ Ta có: $3 x^2 + 45 x + 162 = 3 \left(x + 6\right) \left(x + 9\right)$
	}
\end{bt}
%%%=========Bai_720=========%%%
\begin{bt}
	Phân tích đa thức sau thành nhân tử: $3 x^2 + 42 x + 144$.
	\loigiai{ Ta có: $3 x^2 + 42 x + 144 = 3 \left(x + 6\right) \left(x + 8\right)$
	}
\end{bt}
%%%=========Bai_721=========%%%
\begin{bt}
	Phân tích đa thức sau thành nhân tử: $3 x^2 + 39 x + 126$.
	\loigiai{ Ta có: $3 x^2 + 39 x + 126 = 3 \left(x + 6\right) \left(x + 7\right)$
	}
\end{bt}
%%%=========Bai_722=========%%%
\begin{bt}
	Phân tích đa thức sau thành nhân tử: $3 x^2 + 33 x + 90$.
	\loigiai{ Ta có: $3 x^2 + 33 x + 90 = 3 \left(x + 5\right) \left(x + 6\right)$
	}
\end{bt}
%%%=========Bai_723=========%%%
\begin{bt}
	Phân tích đa thức sau thành nhân tử: $3 x^2 + 30 x + 72$.
	\loigiai{ Ta có: $3 x^2 + 30 x + 72 = 3 \left(x + 4\right) \left(x + 6\right)$
	}
\end{bt}
%%%=========Bai_724=========%%%
\begin{bt}
	Phân tích đa thức sau thành nhân tử: $3 x^2 + 27 x + 54$.
	\loigiai{ Ta có: $3 x^2 + 27 x + 54 = 3 \left(x + 3\right) \left(x + 6\right)$
	}
\end{bt}
%%%=========Bai_725=========%%%
\begin{bt}
	Phân tích đa thức sau thành nhân tử: $3 x^2 + 24 x + 36$.
	\loigiai{ Ta có: $3 x^2 + 24 x + 36 = 3 \left(x + 2\right) \left(x + 6\right)$
	}
\end{bt}
%%%=========Bai_726=========%%%
\begin{bt}
	Phân tích đa thức sau thành nhân tử: $3 x^2 + 21 x + 18$.
	\loigiai{ Ta có: $3 x^2 + 21 x + 18 = 3 \left(x + 1\right) \left(x + 6\right)$
	}
\end{bt}
%%%=========Bai_727=========%%%
\begin{bt}
	Phân tích đa thức sau thành nhân tử: $3 x^2 + 15 x - 18$.
	\loigiai{ Ta có: $3 x^2 + 15 x - 18 = 3 \left(x - 1\right) \left(x + 6\right)$
	}
\end{bt}
%%%=========Bai_728=========%%%
\begin{bt}
	Phân tích đa thức sau thành nhân tử: $3 x^2 + 12 x - 36$.
	\loigiai{ Ta có: $3 x^2 + 12 x - 36 = 3 \left(x - 2\right) \left(x + 6\right)$
	}
\end{bt}
%%%=========Bai_729=========%%%
\begin{bt}
	Phân tích đa thức sau thành nhân tử: $3 x^2 + 9 x - 54$.
	\loigiai{ Ta có: $3 x^2 + 9 x - 54 = 3 \left(x - 3\right) \left(x + 6\right)$
	}
\end{bt}
%%%=========Bai_730=========%%%
\begin{bt}
	Phân tích đa thức sau thành nhân tử: $3 x^2 + 6 x - 72$.
	\loigiai{ Ta có: $3 x^2 + 6 x - 72 = 3 \left(x - 4\right) \left(x + 6\right)$
	}
\end{bt}
%%%=========Bai_731=========%%%
\begin{bt}
	Phân tích đa thức sau thành nhân tử: $3 x^2 + 3 x - 90$.
	\loigiai{ Ta có: $3 x^2 + 3 x - 90 = 3 \left(x - 5\right) \left(x + 6\right)$
	}
\end{bt}
%%%=========Bai_732=========%%%
\begin{bt}
	Phân tích đa thức sau thành nhân tử: $3 x^2 - 3 x - 126$.
	\loigiai{ Ta có: $3 x^2 - 3 x - 126 = 3 \left(x - 7\right) \left(x + 6\right)$
	}
\end{bt}
%%%=========Bai_733=========%%%
\begin{bt}
	Phân tích đa thức sau thành nhân tử: $3 x^2 - 6 x - 144$.
	\loigiai{ Ta có: $3 x^2 - 6 x - 144 = 3 \left(x - 8\right) \left(x + 6\right)$
	}
\end{bt}
%%%=========Bai_734=========%%%
\begin{bt}
	Phân tích đa thức sau thành nhân tử: $3 x^2 - 9 x - 162$.
	\loigiai{ Ta có: $3 x^2 - 9 x - 162 = 3 \left(x - 9\right) \left(x + 6\right)$
	}
\end{bt}
%%%=========Bai_735=========%%%
\begin{bt}
	Phân tích đa thức sau thành nhân tử: $3 x^2 + 45 x + 150$.
	\loigiai{ Ta có: $3 x^2 + 45 x + 150 = 3 \left(x + 5\right) \left(x + 10\right)$
	}
\end{bt}
%%%=========Bai_736=========%%%
\begin{bt}
	Phân tích đa thức sau thành nhân tử: $3 x^2 + 42 x + 135$.
	\loigiai{ Ta có: $3 x^2 + 42 x + 135 = 3 \left(x + 5\right) \left(x + 9\right)$
	}
\end{bt}
%%%=========Bai_737=========%%%
\begin{bt}
	Phân tích đa thức sau thành nhân tử: $3 x^2 + 39 x + 120$.
	\loigiai{ Ta có: $3 x^2 + 39 x + 120 = 3 \left(x + 5\right) \left(x + 8\right)$
	}
\end{bt}
%%%=========Bai_738=========%%%
\begin{bt}
	Phân tích đa thức sau thành nhân tử: $3 x^2 + 36 x + 105$.
	\loigiai{ Ta có: $3 x^2 + 36 x + 105 = 3 \left(x + 5\right) \left(x + 7\right)$
	}
\end{bt}
%%%=========Bai_739=========%%%
\begin{bt}
	Phân tích đa thức sau thành nhân tử: $3 x^2 + 33 x + 90$.
	\loigiai{ Ta có: $3 x^2 + 33 x + 90 = 3 \left(x + 5\right) \left(x + 6\right)$
	}
\end{bt}
%%%=========Bai_740=========%%%
\begin{bt}
	Phân tích đa thức sau thành nhân tử: $3 x^2 + 27 x + 60$.
	\loigiai{ Ta có: $3 x^2 + 27 x + 60 = 3 \left(x + 4\right) \left(x + 5\right)$
	}
\end{bt}
%%%=========Bai_741=========%%%
\begin{bt}
	Phân tích đa thức sau thành nhân tử: $3 x^2 + 24 x + 45$.
	\loigiai{ Ta có: $3 x^2 + 24 x + 45 = 3 \left(x + 3\right) \left(x + 5\right)$
	}
\end{bt}
%%%=========Bai_742=========%%%
\begin{bt}
	Phân tích đa thức sau thành nhân tử: $3 x^2 + 21 x + 30$.
	\loigiai{ Ta có: $3 x^2 + 21 x + 30 = 3 \left(x + 2\right) \left(x + 5\right)$
	}
\end{bt}
%%%=========Bai_743=========%%%
\begin{bt}
	Phân tích đa thức sau thành nhân tử: $3 x^2 + 18 x + 15$.
	\loigiai{ Ta có: $3 x^2 + 18 x + 15 = 3 \left(x + 1\right) \left(x + 5\right)$
	}
\end{bt}
%%%=========Bai_744=========%%%
\begin{bt}
	Phân tích đa thức sau thành nhân tử: $3 x^2 + 12 x - 15$.
	\loigiai{ Ta có: $3 x^2 + 12 x - 15 = 3 \left(x - 1\right) \left(x + 5\right)$
	}
\end{bt}
%%%=========Bai_745=========%%%
\begin{bt}
	Phân tích đa thức sau thành nhân tử: $3 x^2 + 9 x - 30$.
	\loigiai{ Ta có: $3 x^2 + 9 x - 30 = 3 \left(x - 2\right) \left(x + 5\right)$
	}
\end{bt}
%%%=========Bai_746=========%%%
\begin{bt}
	Phân tích đa thức sau thành nhân tử: $3 x^2 + 6 x - 45$.
	\loigiai{ Ta có: $3 x^2 + 6 x - 45 = 3 \left(x - 3\right) \left(x + 5\right)$
	}
\end{bt}
%%%=========Bai_747=========%%%
\begin{bt}
	Phân tích đa thức sau thành nhân tử: $3 x^2 + 3 x - 60$.
	\loigiai{ Ta có: $3 x^2 + 3 x - 60 = 3 \left(x - 4\right) \left(x + 5\right)$
	}
\end{bt}
%%%=========Bai_748=========%%%
\begin{bt}
	Phân tích đa thức sau thành nhân tử: $3 x^2 - 3 x - 90$.
	\loigiai{ Ta có: $3 x^2 - 3 x - 90 = 3 \left(x - 6\right) \left(x + 5\right)$
	}
\end{bt}
%%%=========Bai_749=========%%%
\begin{bt}
	Phân tích đa thức sau thành nhân tử: $3 x^2 - 6 x - 105$.
	\loigiai{ Ta có: $3 x^2 - 6 x - 105 = 3 \left(x - 7\right) \left(x + 5\right)$
	}
\end{bt}
%%%=========Bai_750=========%%%
\begin{bt}
	Phân tích đa thức sau thành nhân tử: $3 x^2 - 9 x - 120$.
	\loigiai{ Ta có: $3 x^2 - 9 x - 120 = 3 \left(x - 8\right) \left(x + 5\right)$
	}
\end{bt}
%%%=========Bai_751=========%%%
\begin{bt}
	Phân tích đa thức sau thành nhân tử: $3 x^2 - 12 x - 135$.
	\loigiai{ Ta có: $3 x^2 - 12 x - 135 = 3 \left(x - 9\right) \left(x + 5\right)$
	}
\end{bt}
%%%=========Bai_752=========%%%
\begin{bt}
	Phân tích đa thức sau thành nhân tử: $3 x^2 + 42 x + 120$.
	\loigiai{ Ta có: $3 x^2 + 42 x + 120 = 3 \left(x + 4\right) \left(x + 10\right)$
	}
\end{bt}
%%%=========Bai_753=========%%%
\begin{bt}
	Phân tích đa thức sau thành nhân tử: $3 x^2 + 39 x + 108$.
	\loigiai{ Ta có: $3 x^2 + 39 x + 108 = 3 \left(x + 4\right) \left(x + 9\right)$
	}
\end{bt}
%%%=========Bai_754=========%%%
\begin{bt}
	Phân tích đa thức sau thành nhân tử: $3 x^2 + 36 x + 96$.
	\loigiai{ Ta có: $3 x^2 + 36 x + 96 = 3 \left(x + 4\right) \left(x + 8\right)$
	}
\end{bt}
%%%=========Bai_755=========%%%
\begin{bt}
	Phân tích đa thức sau thành nhân tử: $3 x^2 + 33 x + 84$.
	\loigiai{ Ta có: $3 x^2 + 33 x + 84 = 3 \left(x + 4\right) \left(x + 7\right)$
	}
\end{bt}
%%%=========Bai_756=========%%%
\begin{bt}
	Phân tích đa thức sau thành nhân tử: $3 x^2 + 30 x + 72$.
	\loigiai{ Ta có: $3 x^2 + 30 x + 72 = 3 \left(x + 4\right) \left(x + 6\right)$
	}
\end{bt}
%%%=========Bai_757=========%%%
\begin{bt}
	Phân tích đa thức sau thành nhân tử: $3 x^2 + 27 x + 60$.
	\loigiai{ Ta có: $3 x^2 + 27 x + 60 = 3 \left(x + 4\right) \left(x + 5\right)$
	}
\end{bt}
%%%=========Bai_758=========%%%
\begin{bt}
	Phân tích đa thức sau thành nhân tử: $3 x^2 + 21 x + 36$.
	\loigiai{ Ta có: $3 x^2 + 21 x + 36 = 3 \left(x + 3\right) \left(x + 4\right)$
	}
\end{bt}
%%%=========Bai_759=========%%%
\begin{bt}
	Phân tích đa thức sau thành nhân tử: $3 x^2 + 18 x + 24$.
	\loigiai{ Ta có: $3 x^2 + 18 x + 24 = 3 \left(x + 2\right) \left(x + 4\right)$
	}
\end{bt}
%%%=========Bai_760=========%%%
\begin{bt}
	Phân tích đa thức sau thành nhân tử: $3 x^2 + 15 x + 12$.
	\loigiai{ Ta có: $3 x^2 + 15 x + 12 = 3 \left(x + 1\right) \left(x + 4\right)$
	}
\end{bt}
%%%=========Bai_761=========%%%
\begin{bt}
	Phân tích đa thức sau thành nhân tử: $3 x^2 + 9 x - 12$.
	\loigiai{ Ta có: $3 x^2 + 9 x - 12 = 3 \left(x - 1\right) \left(x + 4\right)$
	}
\end{bt}
%%%=========Bai_762=========%%%
\begin{bt}
	Phân tích đa thức sau thành nhân tử: $3 x^2 + 6 x - 24$.
	\loigiai{ Ta có: $3 x^2 + 6 x - 24 = 3 \left(x - 2\right) \left(x + 4\right)$
	}
\end{bt}
%%%=========Bai_763=========%%%
\begin{bt}
	Phân tích đa thức sau thành nhân tử: $3 x^2 + 3 x - 36$.
	\loigiai{ Ta có: $3 x^2 + 3 x - 36 = 3 \left(x - 3\right) \left(x + 4\right)$
	}
\end{bt}
%%%=========Bai_764=========%%%
\begin{bt}
	Phân tích đa thức sau thành nhân tử: $3 x^2 - 3 x - 60$.
	\loigiai{ Ta có: $3 x^2 - 3 x - 60 = 3 \left(x - 5\right) \left(x + 4\right)$
	}
\end{bt}
%%%=========Bai_765=========%%%
\begin{bt}
	Phân tích đa thức sau thành nhân tử: $3 x^2 - 6 x - 72$.
	\loigiai{ Ta có: $3 x^2 - 6 x - 72 = 3 \left(x - 6\right) \left(x + 4\right)$
	}
\end{bt}
%%%=========Bai_766=========%%%
\begin{bt}
	Phân tích đa thức sau thành nhân tử: $3 x^2 - 9 x - 84$.
	\loigiai{ Ta có: $3 x^2 - 9 x - 84 = 3 \left(x - 7\right) \left(x + 4\right)$
	}
\end{bt}
%%%=========Bai_767=========%%%
\begin{bt}
	Phân tích đa thức sau thành nhân tử: $3 x^2 - 12 x - 96$.
	\loigiai{ Ta có: $3 x^2 - 12 x - 96 = 3 \left(x - 8\right) \left(x + 4\right)$
	}
\end{bt}
%%%=========Bai_768=========%%%
\begin{bt}
	Phân tích đa thức sau thành nhân tử: $3 x^2 - 15 x - 108$.
	\loigiai{ Ta có: $3 x^2 - 15 x - 108 = 3 \left(x - 9\right) \left(x + 4\right)$
	}
\end{bt}
%%%=========Bai_769=========%%%
\begin{bt}
	Phân tích đa thức sau thành nhân tử: $3 x^2 + 39 x + 90$.
	\loigiai{ Ta có: $3 x^2 + 39 x + 90 = 3 \left(x + 3\right) \left(x + 10\right)$
	}
\end{bt}
%%%=========Bai_770=========%%%
\begin{bt}
	Phân tích đa thức sau thành nhân tử: $3 x^2 + 36 x + 81$.
	\loigiai{ Ta có: $3 x^2 + 36 x + 81 = 3 \left(x + 3\right) \left(x + 9\right)$
	}
\end{bt}
%%%=========Bai_771=========%%%
\begin{bt}
	Phân tích đa thức sau thành nhân tử: $3 x^2 + 33 x + 72$.
	\loigiai{ Ta có: $3 x^2 + 33 x + 72 = 3 \left(x + 3\right) \left(x + 8\right)$
	}
\end{bt}
%%%=========Bai_772=========%%%
\begin{bt}
	Phân tích đa thức sau thành nhân tử: $3 x^2 + 30 x + 63$.
	\loigiai{ Ta có: $3 x^2 + 30 x + 63 = 3 \left(x + 3\right) \left(x + 7\right)$
	}
\end{bt}
%%%=========Bai_773=========%%%
\begin{bt}
	Phân tích đa thức sau thành nhân tử: $3 x^2 + 27 x + 54$.
	\loigiai{ Ta có: $3 x^2 + 27 x + 54 = 3 \left(x + 3\right) \left(x + 6\right)$
	}
\end{bt}
%%%=========Bai_774=========%%%
\begin{bt}
	Phân tích đa thức sau thành nhân tử: $3 x^2 + 24 x + 45$.
	\loigiai{ Ta có: $3 x^2 + 24 x + 45 = 3 \left(x + 3\right) \left(x + 5\right)$
	}
\end{bt}
%%%=========Bai_775=========%%%
\begin{bt}
	Phân tích đa thức sau thành nhân tử: $3 x^2 + 21 x + 36$.
	\loigiai{ Ta có: $3 x^2 + 21 x + 36 = 3 \left(x + 3\right) \left(x + 4\right)$
	}
\end{bt}
%%%=========Bai_776=========%%%
\begin{bt}
	Phân tích đa thức sau thành nhân tử: $3 x^2 + 15 x + 18$.
	\loigiai{ Ta có: $3 x^2 + 15 x + 18 = 3 \left(x + 2\right) \left(x + 3\right)$
	}
\end{bt}
%%%=========Bai_777=========%%%
\begin{bt}
	Phân tích đa thức sau thành nhân tử: $3 x^2 + 12 x + 9$.
	\loigiai{ Ta có: $3 x^2 + 12 x + 9 = 3 \left(x + 1\right) \left(x + 3\right)$
	}
\end{bt}
%%%=========Bai_778=========%%%
\begin{bt}
	Phân tích đa thức sau thành nhân tử: $3 x^2 + 6 x - 9$.
	\loigiai{ Ta có: $3 x^2 + 6 x - 9 = 3 \left(x - 1\right) \left(x + 3\right)$
	}
\end{bt}
%%%=========Bai_779=========%%%
\begin{bt}
	Phân tích đa thức sau thành nhân tử: $3 x^2 + 3 x - 18$.
	\loigiai{ Ta có: $3 x^2 + 3 x - 18 = 3 \left(x - 2\right) \left(x + 3\right)$
	}
\end{bt}
%%%=========Bai_780=========%%%
\begin{bt}
	Phân tích đa thức sau thành nhân tử: $3 x^2 - 3 x - 36$.
	\loigiai{ Ta có: $3 x^2 - 3 x - 36 = 3 \left(x - 4\right) \left(x + 3\right)$
	}
\end{bt}
%%%=========Bai_781=========%%%
\begin{bt}
	Phân tích đa thức sau thành nhân tử: $3 x^2 - 6 x - 45$.
	\loigiai{ Ta có: $3 x^2 - 6 x - 45 = 3 \left(x - 5\right) \left(x + 3\right)$
	}
\end{bt}
%%%=========Bai_782=========%%%
\begin{bt}
	Phân tích đa thức sau thành nhân tử: $3 x^2 - 9 x - 54$.
	\loigiai{ Ta có: $3 x^2 - 9 x - 54 = 3 \left(x - 6\right) \left(x + 3\right)$
	}
\end{bt}
%%%=========Bai_783=========%%%
\begin{bt}
	Phân tích đa thức sau thành nhân tử: $3 x^2 - 12 x - 63$.
	\loigiai{ Ta có: $3 x^2 - 12 x - 63 = 3 \left(x - 7\right) \left(x + 3\right)$
	}
\end{bt}
%%%=========Bai_784=========%%%
\begin{bt}
	Phân tích đa thức sau thành nhân tử: $3 x^2 - 15 x - 72$.
	\loigiai{ Ta có: $3 x^2 - 15 x - 72 = 3 \left(x - 8\right) \left(x + 3\right)$
	}
\end{bt}
%%%=========Bai_785=========%%%
\begin{bt}
	Phân tích đa thức sau thành nhân tử: $3 x^2 - 18 x - 81$.
	\loigiai{ Ta có: $3 x^2 - 18 x - 81 = 3 \left(x - 9\right) \left(x + 3\right)$
	}
\end{bt}
%%%=========Bai_786=========%%%
\begin{bt}
	Phân tích đa thức sau thành nhân tử: $3 x^2 + 36 x + 60$.
	\loigiai{ Ta có: $3 x^2 + 36 x + 60 = 3 \left(x + 2\right) \left(x + 10\right)$
	}
\end{bt}
%%%=========Bai_787=========%%%
\begin{bt}
	Phân tích đa thức sau thành nhân tử: $3 x^2 + 33 x + 54$.
	\loigiai{ Ta có: $3 x^2 + 33 x + 54 = 3 \left(x + 2\right) \left(x + 9\right)$
	}
\end{bt}
%%%=========Bai_788=========%%%
\begin{bt}
	Phân tích đa thức sau thành nhân tử: $3 x^2 + 30 x + 48$.
	\loigiai{ Ta có: $3 x^2 + 30 x + 48 = 3 \left(x + 2\right) \left(x + 8\right)$
	}
\end{bt}
%%%=========Bai_789=========%%%
\begin{bt}
	Phân tích đa thức sau thành nhân tử: $3 x^2 + 27 x + 42$.
	\loigiai{ Ta có: $3 x^2 + 27 x + 42 = 3 \left(x + 2\right) \left(x + 7\right)$
	}
\end{bt}
%%%=========Bai_790=========%%%
\begin{bt}
	Phân tích đa thức sau thành nhân tử: $3 x^2 + 24 x + 36$.
	\loigiai{ Ta có: $3 x^2 + 24 x + 36 = 3 \left(x + 2\right) \left(x + 6\right)$
	}
\end{bt}
%%%=========Bai_791=========%%%
\begin{bt}
	Phân tích đa thức sau thành nhân tử: $3 x^2 + 21 x + 30$.
	\loigiai{ Ta có: $3 x^2 + 21 x + 30 = 3 \left(x + 2\right) \left(x + 5\right)$
	}
\end{bt}
%%%=========Bai_792=========%%%
\begin{bt}
	Phân tích đa thức sau thành nhân tử: $3 x^2 + 18 x + 24$.
	\loigiai{ Ta có: $3 x^2 + 18 x + 24 = 3 \left(x + 2\right) \left(x + 4\right)$
	}
\end{bt}
%%%=========Bai_793=========%%%
\begin{bt}
	Phân tích đa thức sau thành nhân tử: $3 x^2 + 15 x + 18$.
	\loigiai{ Ta có: $3 x^2 + 15 x + 18 = 3 \left(x + 2\right) \left(x + 3\right)$
	}
\end{bt}
%%%=========Bai_794=========%%%
\begin{bt}
	Phân tích đa thức sau thành nhân tử: $3 x^2 + 9 x + 6$.
	\loigiai{ Ta có: $3 x^2 + 9 x + 6 = 3 \left(x + 1\right) \left(x + 2\right)$
	}
\end{bt}
%%%=========Bai_795=========%%%
\begin{bt}
	Phân tích đa thức sau thành nhân tử: $3 x^2 + 3 x - 6$.
	\loigiai{ Ta có: $3 x^2 + 3 x - 6 = 3 \left(x - 1\right) \left(x + 2\right)$
	}
\end{bt}
%%%=========Bai_796=========%%%
\begin{bt}
	Phân tích đa thức sau thành nhân tử: $3 x^2 - 3 x - 18$.
	\loigiai{ Ta có: $3 x^2 - 3 x - 18 = 3 \left(x - 3\right) \left(x + 2\right)$
	}
\end{bt}
%%%=========Bai_797=========%%%
\begin{bt}
	Phân tích đa thức sau thành nhân tử: $3 x^2 - 6 x - 24$.
	\loigiai{ Ta có: $3 x^2 - 6 x - 24 = 3 \left(x - 4\right) \left(x + 2\right)$
	}
\end{bt}
%%%=========Bai_798=========%%%
\begin{bt}
	Phân tích đa thức sau thành nhân tử: $3 x^2 - 9 x - 30$.
	\loigiai{ Ta có: $3 x^2 - 9 x - 30 = 3 \left(x - 5\right) \left(x + 2\right)$
	}
\end{bt}
%%%=========Bai_799=========%%%
\begin{bt}
	Phân tích đa thức sau thành nhân tử: $3 x^2 - 12 x - 36$.
	\loigiai{ Ta có: $3 x^2 - 12 x - 36 = 3 \left(x - 6\right) \left(x + 2\right)$
	}
\end{bt}
%%%=========Bai_800=========%%%
\begin{bt}
	Phân tích đa thức sau thành nhân tử: $3 x^2 - 15 x - 42$.
	\loigiai{ Ta có: $3 x^2 - 15 x - 42 = 3 \left(x - 7\right) \left(x + 2\right)$
	}
\end{bt}
%%%=========Bai_801=========%%%
\begin{bt}
	Phân tích đa thức sau thành nhân tử: $3 x^2 - 18 x - 48$.
	\loigiai{ Ta có: $3 x^2 - 18 x - 48 = 3 \left(x - 8\right) \left(x + 2\right)$
	}
\end{bt}
%%%=========Bai_802=========%%%
\begin{bt}
	Phân tích đa thức sau thành nhân tử: $3 x^2 - 21 x - 54$.
	\loigiai{ Ta có: $3 x^2 - 21 x - 54 = 3 \left(x - 9\right) \left(x + 2\right)$
	}
\end{bt}
%%%=========Bai_803=========%%%
\begin{bt}
	Phân tích đa thức sau thành nhân tử: $3 x^2 + 33 x + 30$.
	\loigiai{ Ta có: $3 x^2 + 33 x + 30 = 3 \left(x + 1\right) \left(x + 10\right)$
	}
\end{bt}
%%%=========Bai_804=========%%%
\begin{bt}
	Phân tích đa thức sau thành nhân tử: $3 x^2 + 30 x + 27$.
	\loigiai{ Ta có: $3 x^2 + 30 x + 27 = 3 \left(x + 1\right) \left(x + 9\right)$
	}
\end{bt}
%%%=========Bai_805=========%%%
\begin{bt}
	Phân tích đa thức sau thành nhân tử: $3 x^2 + 27 x + 24$.
	\loigiai{ Ta có: $3 x^2 + 27 x + 24 = 3 \left(x + 1\right) \left(x + 8\right)$
	}
\end{bt}
%%%=========Bai_806=========%%%
\begin{bt}
	Phân tích đa thức sau thành nhân tử: $3 x^2 + 24 x + 21$.
	\loigiai{ Ta có: $3 x^2 + 24 x + 21 = 3 \left(x + 1\right) \left(x + 7\right)$
	}
\end{bt}
%%%=========Bai_807=========%%%
\begin{bt}
	Phân tích đa thức sau thành nhân tử: $3 x^2 + 21 x + 18$.
	\loigiai{ Ta có: $3 x^2 + 21 x + 18 = 3 \left(x + 1\right) \left(x + 6\right)$
	}
\end{bt}
%%%=========Bai_808=========%%%
\begin{bt}
	Phân tích đa thức sau thành nhân tử: $3 x^2 + 18 x + 15$.
	\loigiai{ Ta có: $3 x^2 + 18 x + 15 = 3 \left(x + 1\right) \left(x + 5\right)$
	}
\end{bt}
%%%=========Bai_809=========%%%
\begin{bt}
	Phân tích đa thức sau thành nhân tử: $3 x^2 + 15 x + 12$.
	\loigiai{ Ta có: $3 x^2 + 15 x + 12 = 3 \left(x + 1\right) \left(x + 4\right)$
	}
\end{bt}
%%%=========Bai_810=========%%%
\begin{bt}
	Phân tích đa thức sau thành nhân tử: $3 x^2 + 12 x + 9$.
	\loigiai{ Ta có: $3 x^2 + 12 x + 9 = 3 \left(x + 1\right) \left(x + 3\right)$
	}
\end{bt}
%%%=========Bai_811=========%%%
\begin{bt}
	Phân tích đa thức sau thành nhân tử: $3 x^2 + 9 x + 6$.
	\loigiai{ Ta có: $3 x^2 + 9 x + 6 = 3 \left(x + 1\right) \left(x + 2\right)$
	}
\end{bt}
%%%=========Bai_812=========%%%
\begin{bt}
	Phân tích đa thức sau thành nhân tử: $3 x^2 - 3 x - 6$.
	\loigiai{ Ta có: $3 x^2 - 3 x - 6 = 3 \left(x - 2\right) \left(x + 1\right)$
	}
\end{bt}
%%%=========Bai_813=========%%%
\begin{bt}
	Phân tích đa thức sau thành nhân tử: $3 x^2 - 6 x - 9$.
	\loigiai{ Ta có: $3 x^2 - 6 x - 9 = 3 \left(x - 3\right) \left(x + 1\right)$
	}
\end{bt}
%%%=========Bai_814=========%%%
\begin{bt}
	Phân tích đa thức sau thành nhân tử: $3 x^2 - 9 x - 12$.
	\loigiai{ Ta có: $3 x^2 - 9 x - 12 = 3 \left(x - 4\right) \left(x + 1\right)$
	}
\end{bt}
%%%=========Bai_815=========%%%
\begin{bt}
	Phân tích đa thức sau thành nhân tử: $3 x^2 - 12 x - 15$.
	\loigiai{ Ta có: $3 x^2 - 12 x - 15 = 3 \left(x - 5\right) \left(x + 1\right)$
	}
\end{bt}
%%%=========Bai_816=========%%%
\begin{bt}
	Phân tích đa thức sau thành nhân tử: $3 x^2 - 15 x - 18$.
	\loigiai{ Ta có: $3 x^2 - 15 x - 18 = 3 \left(x - 6\right) \left(x + 1\right)$
	}
\end{bt}
%%%=========Bai_817=========%%%
\begin{bt}
	Phân tích đa thức sau thành nhân tử: $3 x^2 - 18 x - 21$.
	\loigiai{ Ta có: $3 x^2 - 18 x - 21 = 3 \left(x - 7\right) \left(x + 1\right)$
	}
\end{bt}
%%%=========Bai_818=========%%%
\begin{bt}
	Phân tích đa thức sau thành nhân tử: $3 x^2 - 21 x - 24$.
	\loigiai{ Ta có: $3 x^2 - 21 x - 24 = 3 \left(x - 8\right) \left(x + 1\right)$
	}
\end{bt}
%%%=========Bai_819=========%%%
\begin{bt}
	Phân tích đa thức sau thành nhân tử: $3 x^2 - 24 x - 27$.
	\loigiai{ Ta có: $3 x^2 - 24 x - 27 = 3 \left(x - 9\right) \left(x + 1\right)$
	}
\end{bt}
%%%=========Bai_820=========%%%
\begin{bt}
	Phân tích đa thức sau thành nhân tử: $3 x^2 + 27 x - 30$.
	\loigiai{ Ta có: $3 x^2 + 27 x - 30 = 3 \left(x - 1\right) \left(x + 10\right)$
	}
\end{bt}
%%%=========Bai_821=========%%%
\begin{bt}
	Phân tích đa thức sau thành nhân tử: $3 x^2 + 24 x - 27$.
	\loigiai{ Ta có: $3 x^2 + 24 x - 27 = 3 \left(x - 1\right) \left(x + 9\right)$
	}
\end{bt}
%%%=========Bai_822=========%%%
\begin{bt}
	Phân tích đa thức sau thành nhân tử: $3 x^2 + 21 x - 24$.
	\loigiai{ Ta có: $3 x^2 + 21 x - 24 = 3 \left(x - 1\right) \left(x + 8\right)$
	}
\end{bt}
%%%=========Bai_823=========%%%
\begin{bt}
	Phân tích đa thức sau thành nhân tử: $3 x^2 + 18 x - 21$.
	\loigiai{ Ta có: $3 x^2 + 18 x - 21 = 3 \left(x - 1\right) \left(x + 7\right)$
	}
\end{bt}
%%%=========Bai_824=========%%%
\begin{bt}
	Phân tích đa thức sau thành nhân tử: $3 x^2 + 15 x - 18$.
	\loigiai{ Ta có: $3 x^2 + 15 x - 18 = 3 \left(x - 1\right) \left(x + 6\right)$
	}
\end{bt}
%%%=========Bai_825=========%%%
\begin{bt}
	Phân tích đa thức sau thành nhân tử: $3 x^2 + 12 x - 15$.
	\loigiai{ Ta có: $3 x^2 + 12 x - 15 = 3 \left(x - 1\right) \left(x + 5\right)$
	}
\end{bt}
%%%=========Bai_826=========%%%
\begin{bt}
	Phân tích đa thức sau thành nhân tử: $3 x^2 + 9 x - 12$.
	\loigiai{ Ta có: $3 x^2 + 9 x - 12 = 3 \left(x - 1\right) \left(x + 4\right)$
	}
\end{bt}
%%%=========Bai_827=========%%%
\begin{bt}
	Phân tích đa thức sau thành nhân tử: $3 x^2 + 6 x - 9$.
	\loigiai{ Ta có: $3 x^2 + 6 x - 9 = 3 \left(x - 1\right) \left(x + 3\right)$
	}
\end{bt}
%%%=========Bai_828=========%%%
\begin{bt}
	Phân tích đa thức sau thành nhân tử: $3 x^2 + 3 x - 6$.
	\loigiai{ Ta có: $3 x^2 + 3 x - 6 = 3 \left(x - 1\right) \left(x + 2\right)$
	}
\end{bt}
%%%=========Bai_829=========%%%
\begin{bt}
	Phân tích đa thức sau thành nhân tử: $3 x^2 - 9 x + 6$.
	\loigiai{ Ta có: $3 x^2 - 9 x + 6 = 3 \left(x - 2\right) \left(x - 1\right)$
	}
\end{bt}
%%%=========Bai_830=========%%%
\begin{bt}
	Phân tích đa thức sau thành nhân tử: $3 x^2 - 12 x + 9$.
	\loigiai{ Ta có: $3 x^2 - 12 x + 9 = 3 \left(x - 3\right) \left(x - 1\right)$
	}
\end{bt}
%%%=========Bai_831=========%%%
\begin{bt}
	Phân tích đa thức sau thành nhân tử: $3 x^2 - 15 x + 12$.
	\loigiai{ Ta có: $3 x^2 - 15 x + 12 = 3 \left(x - 4\right) \left(x - 1\right)$
	}
\end{bt}
%%%=========Bai_832=========%%%
\begin{bt}
	Phân tích đa thức sau thành nhân tử: $3 x^2 - 18 x + 15$.
	\loigiai{ Ta có: $3 x^2 - 18 x + 15 = 3 \left(x - 5\right) \left(x - 1\right)$
	}
\end{bt}
%%%=========Bai_833=========%%%
\begin{bt}
	Phân tích đa thức sau thành nhân tử: $3 x^2 - 21 x + 18$.
	\loigiai{ Ta có: $3 x^2 - 21 x + 18 = 3 \left(x - 6\right) \left(x - 1\right)$
	}
\end{bt}
%%%=========Bai_834=========%%%
\begin{bt}
	Phân tích đa thức sau thành nhân tử: $3 x^2 - 24 x + 21$.
	\loigiai{ Ta có: $3 x^2 - 24 x + 21 = 3 \left(x - 7\right) \left(x - 1\right)$
	}
\end{bt}
%%%=========Bai_835=========%%%
\begin{bt}
	Phân tích đa thức sau thành nhân tử: $3 x^2 - 27 x + 24$.
	\loigiai{ Ta có: $3 x^2 - 27 x + 24 = 3 \left(x - 8\right) \left(x - 1\right)$
	}
\end{bt}
%%%=========Bai_836=========%%%
\begin{bt}
	Phân tích đa thức sau thành nhân tử: $3 x^2 - 30 x + 27$.
	\loigiai{ Ta có: $3 x^2 - 30 x + 27 = 3 \left(x - 9\right) \left(x - 1\right)$
	}
\end{bt}
%%%=========Bai_837=========%%%
\begin{bt}
	Phân tích đa thức sau thành nhân tử: $3 x^2 + 24 x - 60$.
	\loigiai{ Ta có: $3 x^2 + 24 x - 60 = 3 \left(x - 2\right) \left(x + 10\right)$
	}
\end{bt}
%%%=========Bai_838=========%%%
\begin{bt}
	Phân tích đa thức sau thành nhân tử: $3 x^2 + 21 x - 54$.
	\loigiai{ Ta có: $3 x^2 + 21 x - 54 = 3 \left(x - 2\right) \left(x + 9\right)$
	}
\end{bt}
%%%=========Bai_839=========%%%
\begin{bt}
	Phân tích đa thức sau thành nhân tử: $3 x^2 + 18 x - 48$.
	\loigiai{ Ta có: $3 x^2 + 18 x - 48 = 3 \left(x - 2\right) \left(x + 8\right)$
	}
\end{bt}
%%%=========Bai_840=========%%%
\begin{bt}
	Phân tích đa thức sau thành nhân tử: $3 x^2 + 15 x - 42$.
	\loigiai{ Ta có: $3 x^2 + 15 x - 42 = 3 \left(x - 2\right) \left(x + 7\right)$
	}
\end{bt}
%%%=========Bai_841=========%%%
\begin{bt}
	Phân tích đa thức sau thành nhân tử: $3 x^2 + 12 x - 36$.
	\loigiai{ Ta có: $3 x^2 + 12 x - 36 = 3 \left(x - 2\right) \left(x + 6\right)$
	}
\end{bt}
%%%=========Bai_842=========%%%
\begin{bt}
	Phân tích đa thức sau thành nhân tử: $3 x^2 + 9 x - 30$.
	\loigiai{ Ta có: $3 x^2 + 9 x - 30 = 3 \left(x - 2\right) \left(x + 5\right)$
	}
\end{bt}
%%%=========Bai_843=========%%%
\begin{bt}
	Phân tích đa thức sau thành nhân tử: $3 x^2 + 6 x - 24$.
	\loigiai{ Ta có: $3 x^2 + 6 x - 24 = 3 \left(x - 2\right) \left(x + 4\right)$
	}
\end{bt}
%%%=========Bai_844=========%%%
\begin{bt}
	Phân tích đa thức sau thành nhân tử: $3 x^2 + 3 x - 18$.
	\loigiai{ Ta có: $3 x^2 + 3 x - 18 = 3 \left(x - 2\right) \left(x + 3\right)$
	}
\end{bt}
%%%=========Bai_845=========%%%
\begin{bt}
	Phân tích đa thức sau thành nhân tử: $3 x^2 - 3 x - 6$.
	\loigiai{ Ta có: $3 x^2 - 3 x - 6 = 3 \left(x - 2\right) \left(x + 1\right)$
	}
\end{bt}
%%%=========Bai_846=========%%%
\begin{bt}
	Phân tích đa thức sau thành nhân tử: $3 x^2 - 9 x + 6$.
	\loigiai{ Ta có: $3 x^2 - 9 x + 6 = 3 \left(x - 2\right) \left(x - 1\right)$
	}
\end{bt}
%%%=========Bai_847=========%%%
\begin{bt}
	Phân tích đa thức sau thành nhân tử: $3 x^2 - 15 x + 18$.
	\loigiai{ Ta có: $3 x^2 - 15 x + 18 = 3 \left(x - 3\right) \left(x - 2\right)$
	}
\end{bt}
%%%=========Bai_848=========%%%
\begin{bt}
	Phân tích đa thức sau thành nhân tử: $3 x^2 - 18 x + 24$.
	\loigiai{ Ta có: $3 x^2 - 18 x + 24 = 3 \left(x - 4\right) \left(x - 2\right)$
	}
\end{bt}
%%%=========Bai_849=========%%%
\begin{bt}
	Phân tích đa thức sau thành nhân tử: $3 x^2 - 21 x + 30$.
	\loigiai{ Ta có: $3 x^2 - 21 x + 30 = 3 \left(x - 5\right) \left(x - 2\right)$
	}
\end{bt}
%%%=========Bai_850=========%%%
\begin{bt}
	Phân tích đa thức sau thành nhân tử: $3 x^2 - 24 x + 36$.
	\loigiai{ Ta có: $3 x^2 - 24 x + 36 = 3 \left(x - 6\right) \left(x - 2\right)$
	}
\end{bt}
%%%=========Bai_851=========%%%
\begin{bt}
	Phân tích đa thức sau thành nhân tử: $3 x^2 - 27 x + 42$.
	\loigiai{ Ta có: $3 x^2 - 27 x + 42 = 3 \left(x - 7\right) \left(x - 2\right)$
	}
\end{bt}
%%%=========Bai_852=========%%%
\begin{bt}
	Phân tích đa thức sau thành nhân tử: $3 x^2 - 30 x + 48$.
	\loigiai{ Ta có: $3 x^2 - 30 x + 48 = 3 \left(x - 8\right) \left(x - 2\right)$
	}
\end{bt}
%%%=========Bai_853=========%%%
\begin{bt}
	Phân tích đa thức sau thành nhân tử: $3 x^2 - 33 x + 54$.
	\loigiai{ Ta có: $3 x^2 - 33 x + 54 = 3 \left(x - 9\right) \left(x - 2\right)$
	}
\end{bt}
%%%=========Bai_854=========%%%
\begin{bt}
	Phân tích đa thức sau thành nhân tử: $3 x^2 + 21 x - 90$.
	\loigiai{ Ta có: $3 x^2 + 21 x - 90 = 3 \left(x - 3\right) \left(x + 10\right)$
	}
\end{bt}
%%%=========Bai_855=========%%%
\begin{bt}
	Phân tích đa thức sau thành nhân tử: $3 x^2 + 18 x - 81$.
	\loigiai{ Ta có: $3 x^2 + 18 x - 81 = 3 \left(x - 3\right) \left(x + 9\right)$
	}
\end{bt}
%%%=========Bai_856=========%%%
\begin{bt}
	Phân tích đa thức sau thành nhân tử: $3 x^2 + 15 x - 72$.
	\loigiai{ Ta có: $3 x^2 + 15 x - 72 = 3 \left(x - 3\right) \left(x + 8\right)$
	}
\end{bt}
%%%=========Bai_857=========%%%
\begin{bt}
	Phân tích đa thức sau thành nhân tử: $3 x^2 + 12 x - 63$.
	\loigiai{ Ta có: $3 x^2 + 12 x - 63 = 3 \left(x - 3\right) \left(x + 7\right)$
	}
\end{bt}
%%%=========Bai_858=========%%%
\begin{bt}
	Phân tích đa thức sau thành nhân tử: $3 x^2 + 9 x - 54$.
	\loigiai{ Ta có: $3 x^2 + 9 x - 54 = 3 \left(x - 3\right) \left(x + 6\right)$
	}
\end{bt}
%%%=========Bai_859=========%%%
\begin{bt}
	Phân tích đa thức sau thành nhân tử: $3 x^2 + 6 x - 45$.
	\loigiai{ Ta có: $3 x^2 + 6 x - 45 = 3 \left(x - 3\right) \left(x + 5\right)$
	}
\end{bt}
%%%=========Bai_860=========%%%
\begin{bt}
	Phân tích đa thức sau thành nhân tử: $3 x^2 + 3 x - 36$.
	\loigiai{ Ta có: $3 x^2 + 3 x - 36 = 3 \left(x - 3\right) \left(x + 4\right)$
	}
\end{bt}
%%%=========Bai_861=========%%%
\begin{bt}
	Phân tích đa thức sau thành nhân tử: $3 x^2 - 3 x - 18$.
	\loigiai{ Ta có: $3 x^2 - 3 x - 18 = 3 \left(x - 3\right) \left(x + 2\right)$
	}
\end{bt}
%%%=========Bai_862=========%%%
\begin{bt}
	Phân tích đa thức sau thành nhân tử: $3 x^2 - 6 x - 9$.
	\loigiai{ Ta có: $3 x^2 - 6 x - 9 = 3 \left(x - 3\right) \left(x + 1\right)$
	}
\end{bt}
%%%=========Bai_863=========%%%
\begin{bt}
	Phân tích đa thức sau thành nhân tử: $3 x^2 - 12 x + 9$.
	\loigiai{ Ta có: $3 x^2 - 12 x + 9 = 3 \left(x - 3\right) \left(x - 1\right)$
	}
\end{bt}
%%%=========Bai_864=========%%%
\begin{bt}
	Phân tích đa thức sau thành nhân tử: $3 x^2 - 15 x + 18$.
	\loigiai{ Ta có: $3 x^2 - 15 x + 18 = 3 \left(x - 3\right) \left(x - 2\right)$
	}
\end{bt}
%%%=========Bai_865=========%%%
\begin{bt}
	Phân tích đa thức sau thành nhân tử: $3 x^2 - 21 x + 36$.
	\loigiai{ Ta có: $3 x^2 - 21 x + 36 = 3 \left(x - 4\right) \left(x - 3\right)$
	}
\end{bt}
%%%=========Bai_866=========%%%
\begin{bt}
	Phân tích đa thức sau thành nhân tử: $3 x^2 - 24 x + 45$.
	\loigiai{ Ta có: $3 x^2 - 24 x + 45 = 3 \left(x - 5\right) \left(x - 3\right)$
	}
\end{bt}
%%%=========Bai_867=========%%%
\begin{bt}
	Phân tích đa thức sau thành nhân tử: $3 x^2 - 27 x + 54$.
	\loigiai{ Ta có: $3 x^2 - 27 x + 54 = 3 \left(x - 6\right) \left(x - 3\right)$
	}
\end{bt}
%%%=========Bai_868=========%%%
\begin{bt}
	Phân tích đa thức sau thành nhân tử: $3 x^2 - 30 x + 63$.
	\loigiai{ Ta có: $3 x^2 - 30 x + 63 = 3 \left(x - 7\right) \left(x - 3\right)$
	}
\end{bt}
%%%=========Bai_869=========%%%
\begin{bt}
	Phân tích đa thức sau thành nhân tử: $3 x^2 - 33 x + 72$.
	\loigiai{ Ta có: $3 x^2 - 33 x + 72 = 3 \left(x - 8\right) \left(x - 3\right)$
	}
\end{bt}
%%%=========Bai_870=========%%%
\begin{bt}
	Phân tích đa thức sau thành nhân tử: $3 x^2 - 36 x + 81$.
	\loigiai{ Ta có: $3 x^2 - 36 x + 81 = 3 \left(x - 9\right) \left(x - 3\right)$
	}
\end{bt}
%%%=========Bai_871=========%%%
\begin{bt}
	Phân tích đa thức sau thành nhân tử: $3 x^2 + 18 x - 120$.
	\loigiai{ Ta có: $3 x^2 + 18 x - 120 = 3 \left(x - 4\right) \left(x + 10\right)$
	}
\end{bt}
%%%=========Bai_872=========%%%
\begin{bt}
	Phân tích đa thức sau thành nhân tử: $3 x^2 + 15 x - 108$.
	\loigiai{ Ta có: $3 x^2 + 15 x - 108 = 3 \left(x - 4\right) \left(x + 9\right)$
	}
\end{bt}
%%%=========Bai_873=========%%%
\begin{bt}
	Phân tích đa thức sau thành nhân tử: $3 x^2 + 12 x - 96$.
	\loigiai{ Ta có: $3 x^2 + 12 x - 96 = 3 \left(x - 4\right) \left(x + 8\right)$
	}
\end{bt}
%%%=========Bai_874=========%%%
\begin{bt}
	Phân tích đa thức sau thành nhân tử: $3 x^2 + 9 x - 84$.
	\loigiai{ Ta có: $3 x^2 + 9 x - 84 = 3 \left(x - 4\right) \left(x + 7\right)$
	}
\end{bt}
%%%=========Bai_875=========%%%
\begin{bt}
	Phân tích đa thức sau thành nhân tử: $3 x^2 + 6 x - 72$.
	\loigiai{ Ta có: $3 x^2 + 6 x - 72 = 3 \left(x - 4\right) \left(x + 6\right)$
	}
\end{bt}
%%%=========Bai_876=========%%%
\begin{bt}
	Phân tích đa thức sau thành nhân tử: $3 x^2 + 3 x - 60$.
	\loigiai{ Ta có: $3 x^2 + 3 x - 60 = 3 \left(x - 4\right) \left(x + 5\right)$
	}
\end{bt}
%%%=========Bai_877=========%%%
\begin{bt}
	Phân tích đa thức sau thành nhân tử: $3 x^2 - 3 x - 36$.
	\loigiai{ Ta có: $3 x^2 - 3 x - 36 = 3 \left(x - 4\right) \left(x + 3\right)$
	}
\end{bt}
%%%=========Bai_878=========%%%
\begin{bt}
	Phân tích đa thức sau thành nhân tử: $3 x^2 - 6 x - 24$.
	\loigiai{ Ta có: $3 x^2 - 6 x - 24 = 3 \left(x - 4\right) \left(x + 2\right)$
	}
\end{bt}
%%%=========Bai_879=========%%%
\begin{bt}
	Phân tích đa thức sau thành nhân tử: $3 x^2 - 9 x - 12$.
	\loigiai{ Ta có: $3 x^2 - 9 x - 12 = 3 \left(x - 4\right) \left(x + 1\right)$
	}
\end{bt}
%%%=========Bai_880=========%%%
\begin{bt}
	Phân tích đa thức sau thành nhân tử: $3 x^2 - 15 x + 12$.
	\loigiai{ Ta có: $3 x^2 - 15 x + 12 = 3 \left(x - 4\right) \left(x - 1\right)$
	}
\end{bt}
%%%=========Bai_881=========%%%
\begin{bt}
	Phân tích đa thức sau thành nhân tử: $3 x^2 - 18 x + 24$.
	\loigiai{ Ta có: $3 x^2 - 18 x + 24 = 3 \left(x - 4\right) \left(x - 2\right)$
	}
\end{bt}
%%%=========Bai_882=========%%%
\begin{bt}
	Phân tích đa thức sau thành nhân tử: $3 x^2 - 21 x + 36$.
	\loigiai{ Ta có: $3 x^2 - 21 x + 36 = 3 \left(x - 4\right) \left(x - 3\right)$
	}
\end{bt}
%%%=========Bai_883=========%%%
\begin{bt}
	Phân tích đa thức sau thành nhân tử: $3 x^2 - 27 x + 60$.
	\loigiai{ Ta có: $3 x^2 - 27 x + 60 = 3 \left(x - 5\right) \left(x - 4\right)$
	}
\end{bt}
%%%=========Bai_884=========%%%
\begin{bt}
	Phân tích đa thức sau thành nhân tử: $3 x^2 - 30 x + 72$.
	\loigiai{ Ta có: $3 x^2 - 30 x + 72 = 3 \left(x - 6\right) \left(x - 4\right)$
	}
\end{bt}
%%%=========Bai_885=========%%%
\begin{bt}
	Phân tích đa thức sau thành nhân tử: $3 x^2 - 33 x + 84$.
	\loigiai{ Ta có: $3 x^2 - 33 x + 84 = 3 \left(x - 7\right) \left(x - 4\right)$
	}
\end{bt}
%%%=========Bai_886=========%%%
\begin{bt}
	Phân tích đa thức sau thành nhân tử: $3 x^2 - 36 x + 96$.
	\loigiai{ Ta có: $3 x^2 - 36 x + 96 = 3 \left(x - 8\right) \left(x - 4\right)$
	}
\end{bt}
%%%=========Bai_887=========%%%
\begin{bt}
	Phân tích đa thức sau thành nhân tử: $3 x^2 - 39 x + 108$.
	\loigiai{ Ta có: $3 x^2 - 39 x + 108 = 3 \left(x - 9\right) \left(x - 4\right)$
	}
\end{bt}
%%%=========Bai_888=========%%%
\begin{bt}
	Phân tích đa thức sau thành nhân tử: $3 x^2 + 15 x - 150$.
	\loigiai{ Ta có: $3 x^2 + 15 x - 150 = 3 \left(x - 5\right) \left(x + 10\right)$
	}
\end{bt}
%%%=========Bai_889=========%%%
\begin{bt}
	Phân tích đa thức sau thành nhân tử: $3 x^2 + 12 x - 135$.
	\loigiai{ Ta có: $3 x^2 + 12 x - 135 = 3 \left(x - 5\right) \left(x + 9\right)$
	}
\end{bt}
%%%=========Bai_890=========%%%
\begin{bt}
	Phân tích đa thức sau thành nhân tử: $3 x^2 + 9 x - 120$.
	\loigiai{ Ta có: $3 x^2 + 9 x - 120 = 3 \left(x - 5\right) \left(x + 8\right)$
	}
\end{bt}
%%%=========Bai_891=========%%%
\begin{bt}
	Phân tích đa thức sau thành nhân tử: $3 x^2 + 6 x - 105$.
	\loigiai{ Ta có: $3 x^2 + 6 x - 105 = 3 \left(x - 5\right) \left(x + 7\right)$
	}
\end{bt}
%%%=========Bai_892=========%%%
\begin{bt}
	Phân tích đa thức sau thành nhân tử: $3 x^2 + 3 x - 90$.
	\loigiai{ Ta có: $3 x^2 + 3 x - 90 = 3 \left(x - 5\right) \left(x + 6\right)$
	}
\end{bt}
%%%=========Bai_893=========%%%
\begin{bt}
	Phân tích đa thức sau thành nhân tử: $3 x^2 - 3 x - 60$.
	\loigiai{ Ta có: $3 x^2 - 3 x - 60 = 3 \left(x - 5\right) \left(x + 4\right)$
	}
\end{bt}
%%%=========Bai_894=========%%%
\begin{bt}
	Phân tích đa thức sau thành nhân tử: $3 x^2 - 6 x - 45$.
	\loigiai{ Ta có: $3 x^2 - 6 x - 45 = 3 \left(x - 5\right) \left(x + 3\right)$
	}
\end{bt}
%%%=========Bai_895=========%%%
\begin{bt}
	Phân tích đa thức sau thành nhân tử: $3 x^2 - 9 x - 30$.
	\loigiai{ Ta có: $3 x^2 - 9 x - 30 = 3 \left(x - 5\right) \left(x + 2\right)$
	}
\end{bt}
%%%=========Bai_896=========%%%
\begin{bt}
	Phân tích đa thức sau thành nhân tử: $3 x^2 - 12 x - 15$.
	\loigiai{ Ta có: $3 x^2 - 12 x - 15 = 3 \left(x - 5\right) \left(x + 1\right)$
	}
\end{bt}
%%%=========Bai_897=========%%%
\begin{bt}
	Phân tích đa thức sau thành nhân tử: $3 x^2 - 18 x + 15$.
	\loigiai{ Ta có: $3 x^2 - 18 x + 15 = 3 \left(x - 5\right) \left(x - 1\right)$
	}
\end{bt}
%%%=========Bai_898=========%%%
\begin{bt}
	Phân tích đa thức sau thành nhân tử: $3 x^2 - 21 x + 30$.
	\loigiai{ Ta có: $3 x^2 - 21 x + 30 = 3 \left(x - 5\right) \left(x - 2\right)$
	}
\end{bt}
%%%=========Bai_899=========%%%
\begin{bt}
	Phân tích đa thức sau thành nhân tử: $3 x^2 - 24 x + 45$.
	\loigiai{ Ta có: $3 x^2 - 24 x + 45 = 3 \left(x - 5\right) \left(x - 3\right)$
	}
\end{bt}
%%%=========Bai_900=========%%%
\begin{bt}
	Phân tích đa thức sau thành nhân tử: $3 x^2 - 27 x + 60$.
	\loigiai{ Ta có: $3 x^2 - 27 x + 60 = 3 \left(x - 5\right) \left(x - 4\right)$
	}
\end{bt}
%%%=========Bai_901=========%%%
\begin{bt}
	Phân tích đa thức sau thành nhân tử: $3 x^2 - 33 x + 90$.
	\loigiai{ Ta có: $3 x^2 - 33 x + 90 = 3 \left(x - 6\right) \left(x - 5\right)$
	}
\end{bt}
%%%=========Bai_902=========%%%
\begin{bt}
	Phân tích đa thức sau thành nhân tử: $3 x^2 - 36 x + 105$.
	\loigiai{ Ta có: $3 x^2 - 36 x + 105 = 3 \left(x - 7\right) \left(x - 5\right)$
	}
\end{bt}
%%%=========Bai_903=========%%%
\begin{bt}
	Phân tích đa thức sau thành nhân tử: $3 x^2 - 39 x + 120$.
	\loigiai{ Ta có: $3 x^2 - 39 x + 120 = 3 \left(x - 8\right) \left(x - 5\right)$
	}
\end{bt}
%%%=========Bai_904=========%%%
\begin{bt}
	Phân tích đa thức sau thành nhân tử: $3 x^2 - 42 x + 135$.
	\loigiai{ Ta có: $3 x^2 - 42 x + 135 = 3 \left(x - 9\right) \left(x - 5\right)$
	}
\end{bt}
%%%=========Bai_905=========%%%
\begin{bt}
	Phân tích đa thức sau thành nhân tử: $3 x^2 + 12 x - 180$.
	\loigiai{ Ta có: $3 x^2 + 12 x - 180 = 3 \left(x - 6\right) \left(x + 10\right)$
	}
\end{bt}
%%%=========Bai_906=========%%%
\begin{bt}
	Phân tích đa thức sau thành nhân tử: $3 x^2 + 9 x - 162$.
	\loigiai{ Ta có: $3 x^2 + 9 x - 162 = 3 \left(x - 6\right) \left(x + 9\right)$
	}
\end{bt}
%%%=========Bai_907=========%%%
\begin{bt}
	Phân tích đa thức sau thành nhân tử: $3 x^2 + 6 x - 144$.
	\loigiai{ Ta có: $3 x^2 + 6 x - 144 = 3 \left(x - 6\right) \left(x + 8\right)$
	}
\end{bt}
%%%=========Bai_908=========%%%
\begin{bt}
	Phân tích đa thức sau thành nhân tử: $3 x^2 + 3 x - 126$.
	\loigiai{ Ta có: $3 x^2 + 3 x - 126 = 3 \left(x - 6\right) \left(x + 7\right)$
	}
\end{bt}
%%%=========Bai_909=========%%%
\begin{bt}
	Phân tích đa thức sau thành nhân tử: $3 x^2 - 3 x - 90$.
	\loigiai{ Ta có: $3 x^2 - 3 x - 90 = 3 \left(x - 6\right) \left(x + 5\right)$
	}
\end{bt}
%%%=========Bai_910=========%%%
\begin{bt}
	Phân tích đa thức sau thành nhân tử: $3 x^2 - 6 x - 72$.
	\loigiai{ Ta có: $3 x^2 - 6 x - 72 = 3 \left(x - 6\right) \left(x + 4\right)$
	}
\end{bt}
%%%=========Bai_911=========%%%
\begin{bt}
	Phân tích đa thức sau thành nhân tử: $3 x^2 - 9 x - 54$.
	\loigiai{ Ta có: $3 x^2 - 9 x - 54 = 3 \left(x - 6\right) \left(x + 3\right)$
	}
\end{bt}
%%%=========Bai_912=========%%%
\begin{bt}
	Phân tích đa thức sau thành nhân tử: $3 x^2 - 12 x - 36$.
	\loigiai{ Ta có: $3 x^2 - 12 x - 36 = 3 \left(x - 6\right) \left(x + 2\right)$
	}
\end{bt}
%%%=========Bai_913=========%%%
\begin{bt}
	Phân tích đa thức sau thành nhân tử: $3 x^2 - 15 x - 18$.
	\loigiai{ Ta có: $3 x^2 - 15 x - 18 = 3 \left(x - 6\right) \left(x + 1\right)$
	}
\end{bt}
%%%=========Bai_914=========%%%
\begin{bt}
	Phân tích đa thức sau thành nhân tử: $3 x^2 - 21 x + 18$.
	\loigiai{ Ta có: $3 x^2 - 21 x + 18 = 3 \left(x - 6\right) \left(x - 1\right)$
	}
\end{bt}
%%%=========Bai_915=========%%%
\begin{bt}
	Phân tích đa thức sau thành nhân tử: $3 x^2 - 24 x + 36$.
	\loigiai{ Ta có: $3 x^2 - 24 x + 36 = 3 \left(x - 6\right) \left(x - 2\right)$
	}
\end{bt}
%%%=========Bai_916=========%%%
\begin{bt}
	Phân tích đa thức sau thành nhân tử: $3 x^2 - 27 x + 54$.
	\loigiai{ Ta có: $3 x^2 - 27 x + 54 = 3 \left(x - 6\right) \left(x - 3\right)$
	}
\end{bt}
%%%=========Bai_917=========%%%
\begin{bt}
	Phân tích đa thức sau thành nhân tử: $3 x^2 - 30 x + 72$.
	\loigiai{ Ta có: $3 x^2 - 30 x + 72 = 3 \left(x - 6\right) \left(x - 4\right)$
	}
\end{bt}
%%%=========Bai_918=========%%%
\begin{bt}
	Phân tích đa thức sau thành nhân tử: $3 x^2 - 33 x + 90$.
	\loigiai{ Ta có: $3 x^2 - 33 x + 90 = 3 \left(x - 6\right) \left(x - 5\right)$
	}
\end{bt}
%%%=========Bai_919=========%%%
\begin{bt}
	Phân tích đa thức sau thành nhân tử: $3 x^2 - 39 x + 126$.
	\loigiai{ Ta có: $3 x^2 - 39 x + 126 = 3 \left(x - 7\right) \left(x - 6\right)$
	}
\end{bt}
%%%=========Bai_920=========%%%
\begin{bt}
	Phân tích đa thức sau thành nhân tử: $3 x^2 - 42 x + 144$.
	\loigiai{ Ta có: $3 x^2 - 42 x + 144 = 3 \left(x - 8\right) \left(x - 6\right)$
	}
\end{bt}
%%%=========Bai_921=========%%%
\begin{bt}
	Phân tích đa thức sau thành nhân tử: $3 x^2 - 45 x + 162$.
	\loigiai{ Ta có: $3 x^2 - 45 x + 162 = 3 \left(x - 9\right) \left(x - 6\right)$
	}
\end{bt}
%%%=========Bai_922=========%%%
\begin{bt}
	Phân tích đa thức sau thành nhân tử: $3 x^2 + 9 x - 210$.
	\loigiai{ Ta có: $3 x^2 + 9 x - 210 = 3 \left(x - 7\right) \left(x + 10\right)$
	}
\end{bt}
%%%=========Bai_923=========%%%
\begin{bt}
	Phân tích đa thức sau thành nhân tử: $3 x^2 + 6 x - 189$.
	\loigiai{ Ta có: $3 x^2 + 6 x - 189 = 3 \left(x - 7\right) \left(x + 9\right)$
	}
\end{bt}
%%%=========Bai_924=========%%%
\begin{bt}
	Phân tích đa thức sau thành nhân tử: $3 x^2 + 3 x - 168$.
	\loigiai{ Ta có: $3 x^2 + 3 x - 168 = 3 \left(x - 7\right) \left(x + 8\right)$
	}
\end{bt}
%%%=========Bai_925=========%%%
\begin{bt}
	Phân tích đa thức sau thành nhân tử: $3 x^2 - 3 x - 126$.
	\loigiai{ Ta có: $3 x^2 - 3 x - 126 = 3 \left(x - 7\right) \left(x + 6\right)$
	}
\end{bt}
%%%=========Bai_926=========%%%
\begin{bt}
	Phân tích đa thức sau thành nhân tử: $3 x^2 - 6 x - 105$.
	\loigiai{ Ta có: $3 x^2 - 6 x - 105 = 3 \left(x - 7\right) \left(x + 5\right)$
	}
\end{bt}
%%%=========Bai_927=========%%%
\begin{bt}
	Phân tích đa thức sau thành nhân tử: $3 x^2 - 9 x - 84$.
	\loigiai{ Ta có: $3 x^2 - 9 x - 84 = 3 \left(x - 7\right) \left(x + 4\right)$
	}
\end{bt}
%%%=========Bai_928=========%%%
\begin{bt}
	Phân tích đa thức sau thành nhân tử: $3 x^2 - 12 x - 63$.
	\loigiai{ Ta có: $3 x^2 - 12 x - 63 = 3 \left(x - 7\right) \left(x + 3\right)$
	}
\end{bt}
%%%=========Bai_929=========%%%
\begin{bt}
	Phân tích đa thức sau thành nhân tử: $3 x^2 - 15 x - 42$.
	\loigiai{ Ta có: $3 x^2 - 15 x - 42 = 3 \left(x - 7\right) \left(x + 2\right)$
	}
\end{bt}
%%%=========Bai_930=========%%%
\begin{bt}
	Phân tích đa thức sau thành nhân tử: $3 x^2 - 18 x - 21$.
	\loigiai{ Ta có: $3 x^2 - 18 x - 21 = 3 \left(x - 7\right) \left(x + 1\right)$
	}
\end{bt}
%%%=========Bai_931=========%%%
\begin{bt}
	Phân tích đa thức sau thành nhân tử: $3 x^2 - 24 x + 21$.
	\loigiai{ Ta có: $3 x^2 - 24 x + 21 = 3 \left(x - 7\right) \left(x - 1\right)$
	}
\end{bt}
%%%=========Bai_932=========%%%
\begin{bt}
	Phân tích đa thức sau thành nhân tử: $3 x^2 - 27 x + 42$.
	\loigiai{ Ta có: $3 x^2 - 27 x + 42 = 3 \left(x - 7\right) \left(x - 2\right)$
	}
\end{bt}
%%%=========Bai_933=========%%%
\begin{bt}
	Phân tích đa thức sau thành nhân tử: $3 x^2 - 30 x + 63$.
	\loigiai{ Ta có: $3 x^2 - 30 x + 63 = 3 \left(x - 7\right) \left(x - 3\right)$
	}
\end{bt}
%%%=========Bai_934=========%%%
\begin{bt}
	Phân tích đa thức sau thành nhân tử: $3 x^2 - 33 x + 84$.
	\loigiai{ Ta có: $3 x^2 - 33 x + 84 = 3 \left(x - 7\right) \left(x - 4\right)$
	}
\end{bt}
%%%=========Bai_935=========%%%
\begin{bt}
	Phân tích đa thức sau thành nhân tử: $3 x^2 - 36 x + 105$.
	\loigiai{ Ta có: $3 x^2 - 36 x + 105 = 3 \left(x - 7\right) \left(x - 5\right)$
	}
\end{bt}
%%%=========Bai_936=========%%%
\begin{bt}
	Phân tích đa thức sau thành nhân tử: $3 x^2 - 39 x + 126$.
	\loigiai{ Ta có: $3 x^2 - 39 x + 126 = 3 \left(x - 7\right) \left(x - 6\right)$
	}
\end{bt}
%%%=========Bai_937=========%%%
\begin{bt}
	Phân tích đa thức sau thành nhân tử: $3 x^2 - 45 x + 168$.
	\loigiai{ Ta có: $3 x^2 - 45 x + 168 = 3 \left(x - 8\right) \left(x - 7\right)$
	}
\end{bt}
%%%=========Bai_938=========%%%
\begin{bt}
	Phân tích đa thức sau thành nhân tử: $3 x^2 - 48 x + 189$.
	\loigiai{ Ta có: $3 x^2 - 48 x + 189 = 3 \left(x - 9\right) \left(x - 7\right)$
	}
\end{bt}
%%%=========Bai_939=========%%%
\begin{bt}
	Phân tích đa thức sau thành nhân tử: $3 x^2 + 6 x - 240$.
	\loigiai{ Ta có: $3 x^2 + 6 x - 240 = 3 \left(x - 8\right) \left(x + 10\right)$
	}
\end{bt}
%%%=========Bai_940=========%%%
\begin{bt}
	Phân tích đa thức sau thành nhân tử: $3 x^2 + 3 x - 216$.
	\loigiai{ Ta có: $3 x^2 + 3 x - 216 = 3 \left(x - 8\right) \left(x + 9\right)$
	}
\end{bt}
%%%=========Bai_941=========%%%
\begin{bt}
	Phân tích đa thức sau thành nhân tử: $3 x^2 - 3 x - 168$.
	\loigiai{ Ta có: $3 x^2 - 3 x - 168 = 3 \left(x - 8\right) \left(x + 7\right)$
	}
\end{bt}
%%%=========Bai_942=========%%%
\begin{bt}
	Phân tích đa thức sau thành nhân tử: $3 x^2 - 6 x - 144$.
	\loigiai{ Ta có: $3 x^2 - 6 x - 144 = 3 \left(x - 8\right) \left(x + 6\right)$
	}
\end{bt}
%%%=========Bai_943=========%%%
\begin{bt}
	Phân tích đa thức sau thành nhân tử: $3 x^2 - 9 x - 120$.
	\loigiai{ Ta có: $3 x^2 - 9 x - 120 = 3 \left(x - 8\right) \left(x + 5\right)$
	}
\end{bt}
%%%=========Bai_944=========%%%
\begin{bt}
	Phân tích đa thức sau thành nhân tử: $3 x^2 - 12 x - 96$.
	\loigiai{ Ta có: $3 x^2 - 12 x - 96 = 3 \left(x - 8\right) \left(x + 4\right)$
	}
\end{bt}
%%%=========Bai_945=========%%%
\begin{bt}
	Phân tích đa thức sau thành nhân tử: $3 x^2 - 15 x - 72$.
	\loigiai{ Ta có: $3 x^2 - 15 x - 72 = 3 \left(x - 8\right) \left(x + 3\right)$
	}
\end{bt}
%%%=========Bai_946=========%%%
\begin{bt}
	Phân tích đa thức sau thành nhân tử: $3 x^2 - 18 x - 48$.
	\loigiai{ Ta có: $3 x^2 - 18 x - 48 = 3 \left(x - 8\right) \left(x + 2\right)$
	}
\end{bt}
%%%=========Bai_947=========%%%
\begin{bt}
	Phân tích đa thức sau thành nhân tử: $3 x^2 - 21 x - 24$.
	\loigiai{ Ta có: $3 x^2 - 21 x - 24 = 3 \left(x - 8\right) \left(x + 1\right)$
	}
\end{bt}
%%%=========Bai_948=========%%%
\begin{bt}
	Phân tích đa thức sau thành nhân tử: $3 x^2 - 27 x + 24$.
	\loigiai{ Ta có: $3 x^2 - 27 x + 24 = 3 \left(x - 8\right) \left(x - 1\right)$
	}
\end{bt}
%%%=========Bai_949=========%%%
\begin{bt}
	Phân tích đa thức sau thành nhân tử: $3 x^2 - 30 x + 48$.
	\loigiai{ Ta có: $3 x^2 - 30 x + 48 = 3 \left(x - 8\right) \left(x - 2\right)$
	}
\end{bt}
%%%=========Bai_950=========%%%
\begin{bt}
	Phân tích đa thức sau thành nhân tử: $3 x^2 - 33 x + 72$.
	\loigiai{ Ta có: $3 x^2 - 33 x + 72 = 3 \left(x - 8\right) \left(x - 3\right)$
	}
\end{bt}
%%%=========Bai_951=========%%%
\begin{bt}
	Phân tích đa thức sau thành nhân tử: $3 x^2 - 36 x + 96$.
	\loigiai{ Ta có: $3 x^2 - 36 x + 96 = 3 \left(x - 8\right) \left(x - 4\right)$
	}
\end{bt}
%%%=========Bai_952=========%%%
\begin{bt}
	Phân tích đa thức sau thành nhân tử: $3 x^2 - 39 x + 120$.
	\loigiai{ Ta có: $3 x^2 - 39 x + 120 = 3 \left(x - 8\right) \left(x - 5\right)$
	}
\end{bt}
%%%=========Bai_953=========%%%
\begin{bt}
	Phân tích đa thức sau thành nhân tử: $3 x^2 - 42 x + 144$.
	\loigiai{ Ta có: $3 x^2 - 42 x + 144 = 3 \left(x - 8\right) \left(x - 6\right)$
	}
\end{bt}
%%%=========Bai_954=========%%%
\begin{bt}
	Phân tích đa thức sau thành nhân tử: $3 x^2 - 45 x + 168$.
	\loigiai{ Ta có: $3 x^2 - 45 x + 168 = 3 \left(x - 8\right) \left(x - 7\right)$
	}
\end{bt}
%%%=========Bai_955=========%%%
\begin{bt}
	Phân tích đa thức sau thành nhân tử: $3 x^2 - 51 x + 216$.
	\loigiai{ Ta có: $3 x^2 - 51 x + 216 = 3 \left(x - 9\right) \left(x - 8\right)$
	}
\end{bt}
%%%=========Bai_956=========%%%
\begin{bt}
	Phân tích đa thức sau thành nhân tử: $3 x^2 + 3 x - 270$.
	\loigiai{ Ta có: $3 x^2 + 3 x - 270 = 3 \left(x - 9\right) \left(x + 10\right)$
	}
\end{bt}
%%%=========Bai_957=========%%%
\begin{bt}
	Phân tích đa thức sau thành nhân tử: $3 x^2 - 3 x - 216$.
	\loigiai{ Ta có: $3 x^2 - 3 x - 216 = 3 \left(x - 9\right) \left(x + 8\right)$
	}
\end{bt}
%%%=========Bai_958=========%%%
\begin{bt}
	Phân tích đa thức sau thành nhân tử: $3 x^2 - 6 x - 189$.
	\loigiai{ Ta có: $3 x^2 - 6 x - 189 = 3 \left(x - 9\right) \left(x + 7\right)$
	}
\end{bt}
%%%=========Bai_959=========%%%
\begin{bt}
	Phân tích đa thức sau thành nhân tử: $3 x^2 - 9 x - 162$.
	\loigiai{ Ta có: $3 x^2 - 9 x - 162 = 3 \left(x - 9\right) \left(x + 6\right)$
	}
\end{bt}
%%%=========Bai_960=========%%%
\begin{bt}
	Phân tích đa thức sau thành nhân tử: $3 x^2 - 12 x - 135$.
	\loigiai{ Ta có: $3 x^2 - 12 x - 135 = 3 \left(x - 9\right) \left(x + 5\right)$
	}
\end{bt}
%%%=========Bai_961=========%%%
\begin{bt}
	Phân tích đa thức sau thành nhân tử: $3 x^2 - 15 x - 108$.
	\loigiai{ Ta có: $3 x^2 - 15 x - 108 = 3 \left(x - 9\right) \left(x + 4\right)$
	}
\end{bt}
%%%=========Bai_962=========%%%
\begin{bt}
	Phân tích đa thức sau thành nhân tử: $3 x^2 - 18 x - 81$.
	\loigiai{ Ta có: $3 x^2 - 18 x - 81 = 3 \left(x - 9\right) \left(x + 3\right)$
	}
\end{bt}
%%%=========Bai_963=========%%%
\begin{bt}
	Phân tích đa thức sau thành nhân tử: $3 x^2 - 21 x - 54$.
	\loigiai{ Ta có: $3 x^2 - 21 x - 54 = 3 \left(x - 9\right) \left(x + 2\right)$
	}
\end{bt}
%%%=========Bai_964=========%%%
\begin{bt}
	Phân tích đa thức sau thành nhân tử: $3 x^2 - 24 x - 27$.
	\loigiai{ Ta có: $3 x^2 - 24 x - 27 = 3 \left(x - 9\right) \left(x + 1\right)$
	}
\end{bt}
%%%=========Bai_965=========%%%
\begin{bt}
	Phân tích đa thức sau thành nhân tử: $3 x^2 - 30 x + 27$.
	\loigiai{ Ta có: $3 x^2 - 30 x + 27 = 3 \left(x - 9\right) \left(x - 1\right)$
	}
\end{bt}
%%%=========Bai_966=========%%%
\begin{bt}
	Phân tích đa thức sau thành nhân tử: $3 x^2 - 33 x + 54$.
	\loigiai{ Ta có: $3 x^2 - 33 x + 54 = 3 \left(x - 9\right) \left(x - 2\right)$
	}
\end{bt}
%%%=========Bai_967=========%%%
\begin{bt}
	Phân tích đa thức sau thành nhân tử: $3 x^2 - 36 x + 81$.
	\loigiai{ Ta có: $3 x^2 - 36 x + 81 = 3 \left(x - 9\right) \left(x - 3\right)$
	}
\end{bt}
%%%=========Bai_968=========%%%
\begin{bt}
	Phân tích đa thức sau thành nhân tử: $3 x^2 - 39 x + 108$.
	\loigiai{ Ta có: $3 x^2 - 39 x + 108 = 3 \left(x - 9\right) \left(x - 4\right)$
	}
\end{bt}
%%%=========Bai_969=========%%%
\begin{bt}
	Phân tích đa thức sau thành nhân tử: $3 x^2 - 42 x + 135$.
	\loigiai{ Ta có: $3 x^2 - 42 x + 135 = 3 \left(x - 9\right) \left(x - 5\right)$
	}
\end{bt}
%%%=========Bai_970=========%%%
\begin{bt}
	Phân tích đa thức sau thành nhân tử: $3 x^2 - 45 x + 162$.
	\loigiai{ Ta có: $3 x^2 - 45 x + 162 = 3 \left(x - 9\right) \left(x - 6\right)$
	}
\end{bt}
%%%=========Bai_971=========%%%
\begin{bt}
	Phân tích đa thức sau thành nhân tử: $3 x^2 - 48 x + 189$.
	\loigiai{ Ta có: $3 x^2 - 48 x + 189 = 3 \left(x - 9\right) \left(x - 7\right)$
	}
\end{bt}
%%%=========Bai_972=========%%%
\begin{bt}
	Phân tích đa thức sau thành nhân tử: $3 x^2 - 51 x + 216$.
	\loigiai{ Ta có: $3 x^2 - 51 x + 216 = 3 \left(x - 9\right) \left(x - 8\right)$
	}
\end{bt}
%%%=========Bai_973=========%%%
\begin{bt}
	Phân tích đa thức sau thành nhân tử: $4 x^2 + 76 x + 360$.
	\loigiai{ Ta có: $4 x^2 + 76 x + 360 = 4 \left(x + 9\right) \left(x + 10\right)$
	}
\end{bt}
%%%=========Bai_974=========%%%
\begin{bt}
	Phân tích đa thức sau thành nhân tử: $4 x^2 + 72 x + 320$.
	\loigiai{ Ta có: $4 x^2 + 72 x + 320 = 4 \left(x + 8\right) \left(x + 10\right)$
	}
\end{bt}
%%%=========Bai_975=========%%%
\begin{bt}
	Phân tích đa thức sau thành nhân tử: $4 x^2 + 68 x + 280$.
	\loigiai{ Ta có: $4 x^2 + 68 x + 280 = 4 \left(x + 7\right) \left(x + 10\right)$
	}
\end{bt}
%%%=========Bai_976=========%%%
\begin{bt}
	Phân tích đa thức sau thành nhân tử: $4 x^2 + 64 x + 240$.
	\loigiai{ Ta có: $4 x^2 + 64 x + 240 = 4 \left(x + 6\right) \left(x + 10\right)$
	}
\end{bt}
%%%=========Bai_977=========%%%
\begin{bt}
	Phân tích đa thức sau thành nhân tử: $4 x^2 + 60 x + 200$.
	\loigiai{ Ta có: $4 x^2 + 60 x + 200 = 4 \left(x + 5\right) \left(x + 10\right)$
	}
\end{bt}
%%%=========Bai_978=========%%%
\begin{bt}
	Phân tích đa thức sau thành nhân tử: $4 x^2 + 56 x + 160$.
	\loigiai{ Ta có: $4 x^2 + 56 x + 160 = 4 \left(x + 4\right) \left(x + 10\right)$
	}
\end{bt}
%%%=========Bai_979=========%%%
\begin{bt}
	Phân tích đa thức sau thành nhân tử: $4 x^2 + 52 x + 120$.
	\loigiai{ Ta có: $4 x^2 + 52 x + 120 = 4 \left(x + 3\right) \left(x + 10\right)$
	}
\end{bt}
%%%=========Bai_980=========%%%
\begin{bt}
	Phân tích đa thức sau thành nhân tử: $4 x^2 + 48 x + 80$.
	\loigiai{ Ta có: $4 x^2 + 48 x + 80 = 4 \left(x + 2\right) \left(x + 10\right)$
	}
\end{bt}
%%%=========Bai_981=========%%%
\begin{bt}
	Phân tích đa thức sau thành nhân tử: $4 x^2 + 44 x + 40$.
	\loigiai{ Ta có: $4 x^2 + 44 x + 40 = 4 \left(x + 1\right) \left(x + 10\right)$
	}
\end{bt}
%%%=========Bai_982=========%%%
\begin{bt}
	Phân tích đa thức sau thành nhân tử: $4 x^2 + 36 x - 40$.
	\loigiai{ Ta có: $4 x^2 + 36 x - 40 = 4 \left(x - 1\right) \left(x + 10\right)$
	}
\end{bt}
%%%=========Bai_983=========%%%
\begin{bt}
	Phân tích đa thức sau thành nhân tử: $4 x^2 + 32 x - 80$.
	\loigiai{ Ta có: $4 x^2 + 32 x - 80 = 4 \left(x - 2\right) \left(x + 10\right)$
	}
\end{bt}
%%%=========Bai_984=========%%%
\begin{bt}
	Phân tích đa thức sau thành nhân tử: $4 x^2 + 28 x - 120$.
	\loigiai{ Ta có: $4 x^2 + 28 x - 120 = 4 \left(x - 3\right) \left(x + 10\right)$
	}
\end{bt}
%%%=========Bai_985=========%%%
\begin{bt}
	Phân tích đa thức sau thành nhân tử: $4 x^2 + 24 x - 160$.
	\loigiai{ Ta có: $4 x^2 + 24 x - 160 = 4 \left(x - 4\right) \left(x + 10\right)$
	}
\end{bt}
%%%=========Bai_986=========%%%
\begin{bt}
	Phân tích đa thức sau thành nhân tử: $4 x^2 + 20 x - 200$.
	\loigiai{ Ta có: $4 x^2 + 20 x - 200 = 4 \left(x - 5\right) \left(x + 10\right)$
	}
\end{bt}
%%%=========Bai_987=========%%%
\begin{bt}
	Phân tích đa thức sau thành nhân tử: $4 x^2 + 16 x - 240$.
	\loigiai{ Ta có: $4 x^2 + 16 x - 240 = 4 \left(x - 6\right) \left(x + 10\right)$
	}
\end{bt}
%%%=========Bai_988=========%%%
\begin{bt}
	Phân tích đa thức sau thành nhân tử: $4 x^2 + 12 x - 280$.
	\loigiai{ Ta có: $4 x^2 + 12 x - 280 = 4 \left(x - 7\right) \left(x + 10\right)$
	}
\end{bt}
%%%=========Bai_989=========%%%
\begin{bt}
	Phân tích đa thức sau thành nhân tử: $4 x^2 + 8 x - 320$.
	\loigiai{ Ta có: $4 x^2 + 8 x - 320 = 4 \left(x - 8\right) \left(x + 10\right)$
	}
\end{bt}
%%%=========Bai_990=========%%%
\begin{bt}
	Phân tích đa thức sau thành nhân tử: $4 x^2 + 4 x - 360$.
	\loigiai{ Ta có: $4 x^2 + 4 x - 360 = 4 \left(x - 9\right) \left(x + 10\right)$
	}
\end{bt}
%%%=========Bai_991=========%%%
\begin{bt}
	Phân tích đa thức sau thành nhân tử: $4 x^2 + 76 x + 360$.
	\loigiai{ Ta có: $4 x^2 + 76 x + 360 = 4 \left(x + 9\right) \left(x + 10\right)$
	}
\end{bt}
%%%=========Bai_992=========%%%
\begin{bt}
	Phân tích đa thức sau thành nhân tử: $4 x^2 + 68 x + 288$.
	\loigiai{ Ta có: $4 x^2 + 68 x + 288 = 4 \left(x + 8\right) \left(x + 9\right)$
	}
\end{bt}
%%%=========Bai_993=========%%%
\begin{bt}
	Phân tích đa thức sau thành nhân tử: $4 x^2 + 64 x + 252$.
	\loigiai{ Ta có: $4 x^2 + 64 x + 252 = 4 \left(x + 7\right) \left(x + 9\right)$
	}
\end{bt}
%%%=========Bai_994=========%%%
\begin{bt}
	Phân tích đa thức sau thành nhân tử: $4 x^2 + 60 x + 216$.
	\loigiai{ Ta có: $4 x^2 + 60 x + 216 = 4 \left(x + 6\right) \left(x + 9\right)$
	}
\end{bt}
%%%=========Bai_995=========%%%
\begin{bt}
	Phân tích đa thức sau thành nhân tử: $4 x^2 + 56 x + 180$.
	\loigiai{ Ta có: $4 x^2 + 56 x + 180 = 4 \left(x + 5\right) \left(x + 9\right)$
	}
\end{bt}
%%%=========Bai_996=========%%%
\begin{bt}
	Phân tích đa thức sau thành nhân tử: $4 x^2 + 52 x + 144$.
	\loigiai{ Ta có: $4 x^2 + 52 x + 144 = 4 \left(x + 4\right) \left(x + 9\right)$
	}
\end{bt}
%%%=========Bai_997=========%%%
\begin{bt}
	Phân tích đa thức sau thành nhân tử: $4 x^2 + 48 x + 108$.
	\loigiai{ Ta có: $4 x^2 + 48 x + 108 = 4 \left(x + 3\right) \left(x + 9\right)$
	}
\end{bt}
%%%=========Bai_998=========%%%
\begin{bt}
	Phân tích đa thức sau thành nhân tử: $4 x^2 + 44 x + 72$.
	\loigiai{ Ta có: $4 x^2 + 44 x + 72 = 4 \left(x + 2\right) \left(x + 9\right)$
	}
\end{bt}
%%%=========Bai_999=========%%%
\begin{bt}
	Phân tích đa thức sau thành nhân tử: $4 x^2 + 40 x + 36$.
	\loigiai{ Ta có: $4 x^2 + 40 x + 36 = 4 \left(x + 1\right) \left(x + 9\right)$
	}
\end{bt}
%%%=========Bai_1000=========%%%
\begin{bt}
	Phân tích đa thức sau thành nhân tử: $4 x^2 + 32 x - 36$.
	\loigiai{ Ta có: $4 x^2 + 32 x - 36 = 4 \left(x - 1\right) \left(x + 9\right)$
	}
\end{bt}
%%%=========Bai_1001=========%%%
\begin{bt}
	Phân tích đa thức sau thành nhân tử: $4 x^2 + 28 x - 72$.
	\loigiai{ Ta có: $4 x^2 + 28 x - 72 = 4 \left(x - 2\right) \left(x + 9\right)$
	}
\end{bt}
%%%=========Bai_1002=========%%%
\begin{bt}
	Phân tích đa thức sau thành nhân tử: $4 x^2 + 24 x - 108$.
	\loigiai{ Ta có: $4 x^2 + 24 x - 108 = 4 \left(x - 3\right) \left(x + 9\right)$
	}
\end{bt}
%%%=========Bai_1003=========%%%
\begin{bt}
	Phân tích đa thức sau thành nhân tử: $4 x^2 + 20 x - 144$.
	\loigiai{ Ta có: $4 x^2 + 20 x - 144 = 4 \left(x - 4\right) \left(x + 9\right)$
	}
\end{bt}
%%%=========Bai_1004=========%%%
\begin{bt}
	Phân tích đa thức sau thành nhân tử: $4 x^2 + 16 x - 180$.
	\loigiai{ Ta có: $4 x^2 + 16 x - 180 = 4 \left(x - 5\right) \left(x + 9\right)$
	}
\end{bt}
%%%=========Bai_1005=========%%%
\begin{bt}
	Phân tích đa thức sau thành nhân tử: $4 x^2 + 12 x - 216$.
	\loigiai{ Ta có: $4 x^2 + 12 x - 216 = 4 \left(x - 6\right) \left(x + 9\right)$
	}
\end{bt}
%%%=========Bai_1006=========%%%
\begin{bt}
	Phân tích đa thức sau thành nhân tử: $4 x^2 + 8 x - 252$.
	\loigiai{ Ta có: $4 x^2 + 8 x - 252 = 4 \left(x - 7\right) \left(x + 9\right)$
	}
\end{bt}
%%%=========Bai_1007=========%%%
\begin{bt}
	Phân tích đa thức sau thành nhân tử: $4 x^2 + 4 x - 288$.
	\loigiai{ Ta có: $4 x^2 + 4 x - 288 = 4 \left(x - 8\right) \left(x + 9\right)$
	}
\end{bt}
%%%=========Bai_1008=========%%%
\begin{bt}
	Phân tích đa thức sau thành nhân tử: $4 x^2 + 72 x + 320$.
	\loigiai{ Ta có: $4 x^2 + 72 x + 320 = 4 \left(x + 8\right) \left(x + 10\right)$
	}
\end{bt}
%%%=========Bai_1009=========%%%
\begin{bt}
	Phân tích đa thức sau thành nhân tử: $4 x^2 + 68 x + 288$.
	\loigiai{ Ta có: $4 x^2 + 68 x + 288 = 4 \left(x + 8\right) \left(x + 9\right)$
	}
\end{bt}
%%%=========Bai_1010=========%%%
\begin{bt}
	Phân tích đa thức sau thành nhân tử: $4 x^2 + 60 x + 224$.
	\loigiai{ Ta có: $4 x^2 + 60 x + 224 = 4 \left(x + 7\right) \left(x + 8\right)$
	}
\end{bt}
%%%=========Bai_1011=========%%%
\begin{bt}
	Phân tích đa thức sau thành nhân tử: $4 x^2 + 56 x + 192$.
	\loigiai{ Ta có: $4 x^2 + 56 x + 192 = 4 \left(x + 6\right) \left(x + 8\right)$
	}
\end{bt}
%%%=========Bai_1012=========%%%
\begin{bt}
	Phân tích đa thức sau thành nhân tử: $4 x^2 + 52 x + 160$.
	\loigiai{ Ta có: $4 x^2 + 52 x + 160 = 4 \left(x + 5\right) \left(x + 8\right)$
	}
\end{bt}
%%%=========Bai_1013=========%%%
\begin{bt}
	Phân tích đa thức sau thành nhân tử: $4 x^2 + 48 x + 128$.
	\loigiai{ Ta có: $4 x^2 + 48 x + 128 = 4 \left(x + 4\right) \left(x + 8\right)$
	}
\end{bt}
%%%=========Bai_1014=========%%%
\begin{bt}
	Phân tích đa thức sau thành nhân tử: $4 x^2 + 44 x + 96$.
	\loigiai{ Ta có: $4 x^2 + 44 x + 96 = 4 \left(x + 3\right) \left(x + 8\right)$
	}
\end{bt}
%%%=========Bai_1015=========%%%
\begin{bt}
	Phân tích đa thức sau thành nhân tử: $4 x^2 + 40 x + 64$.
	\loigiai{ Ta có: $4 x^2 + 40 x + 64 = 4 \left(x + 2\right) \left(x + 8\right)$
	}
\end{bt}
%%%=========Bai_1016=========%%%
\begin{bt}
	Phân tích đa thức sau thành nhân tử: $4 x^2 + 36 x + 32$.
	\loigiai{ Ta có: $4 x^2 + 36 x + 32 = 4 \left(x + 1\right) \left(x + 8\right)$
	}
\end{bt}
%%%=========Bai_1017=========%%%
\begin{bt}
	Phân tích đa thức sau thành nhân tử: $4 x^2 + 28 x - 32$.
	\loigiai{ Ta có: $4 x^2 + 28 x - 32 = 4 \left(x - 1\right) \left(x + 8\right)$
	}
\end{bt}
%%%=========Bai_1018=========%%%
\begin{bt}
	Phân tích đa thức sau thành nhân tử: $4 x^2 + 24 x - 64$.
	\loigiai{ Ta có: $4 x^2 + 24 x - 64 = 4 \left(x - 2\right) \left(x + 8\right)$
	}
\end{bt}
%%%=========Bai_1019=========%%%
\begin{bt}
	Phân tích đa thức sau thành nhân tử: $4 x^2 + 20 x - 96$.
	\loigiai{ Ta có: $4 x^2 + 20 x - 96 = 4 \left(x - 3\right) \left(x + 8\right)$
	}
\end{bt}
%%%=========Bai_1020=========%%%
\begin{bt}
	Phân tích đa thức sau thành nhân tử: $4 x^2 + 16 x - 128$.
	\loigiai{ Ta có: $4 x^2 + 16 x - 128 = 4 \left(x - 4\right) \left(x + 8\right)$
	}
\end{bt}
%%%=========Bai_1021=========%%%
\begin{bt}
	Phân tích đa thức sau thành nhân tử: $4 x^2 + 12 x - 160$.
	\loigiai{ Ta có: $4 x^2 + 12 x - 160 = 4 \left(x - 5\right) \left(x + 8\right)$
	}
\end{bt}
%%%=========Bai_1022=========%%%
\begin{bt}
	Phân tích đa thức sau thành nhân tử: $4 x^2 + 8 x - 192$.
	\loigiai{ Ta có: $4 x^2 + 8 x - 192 = 4 \left(x - 6\right) \left(x + 8\right)$
	}
\end{bt}
%%%=========Bai_1023=========%%%
\begin{bt}
	Phân tích đa thức sau thành nhân tử: $4 x^2 + 4 x - 224$.
	\loigiai{ Ta có: $4 x^2 + 4 x - 224 = 4 \left(x - 7\right) \left(x + 8\right)$
	}
\end{bt}
%%%=========Bai_1024=========%%%
\begin{bt}
	Phân tích đa thức sau thành nhân tử: $4 x^2 - 4 x - 288$.
	\loigiai{ Ta có: $4 x^2 - 4 x - 288 = 4 \left(x - 9\right) \left(x + 8\right)$
	}
\end{bt}
%%%=========Bai_1025=========%%%
\begin{bt}
	Phân tích đa thức sau thành nhân tử: $4 x^2 + 68 x + 280$.
	\loigiai{ Ta có: $4 x^2 + 68 x + 280 = 4 \left(x + 7\right) \left(x + 10\right)$
	}
\end{bt}
%%%=========Bai_1026=========%%%
\begin{bt}
	Phân tích đa thức sau thành nhân tử: $4 x^2 + 64 x + 252$.
	\loigiai{ Ta có: $4 x^2 + 64 x + 252 = 4 \left(x + 7\right) \left(x + 9\right)$
	}
\end{bt}
%%%=========Bai_1027=========%%%
\begin{bt}
	Phân tích đa thức sau thành nhân tử: $4 x^2 + 60 x + 224$.
	\loigiai{ Ta có: $4 x^2 + 60 x + 224 = 4 \left(x + 7\right) \left(x + 8\right)$
	}
\end{bt}
%%%=========Bai_1028=========%%%
\begin{bt}
	Phân tích đa thức sau thành nhân tử: $4 x^2 + 52 x + 168$.
	\loigiai{ Ta có: $4 x^2 + 52 x + 168 = 4 \left(x + 6\right) \left(x + 7\right)$
	}
\end{bt}
%%%=========Bai_1029=========%%%
\begin{bt}
	Phân tích đa thức sau thành nhân tử: $4 x^2 + 48 x + 140$.
	\loigiai{ Ta có: $4 x^2 + 48 x + 140 = 4 \left(x + 5\right) \left(x + 7\right)$
	}
\end{bt}
%%%=========Bai_1030=========%%%
\begin{bt}
	Phân tích đa thức sau thành nhân tử: $4 x^2 + 44 x + 112$.
	\loigiai{ Ta có: $4 x^2 + 44 x + 112 = 4 \left(x + 4\right) \left(x + 7\right)$
	}
\end{bt}
%%%=========Bai_1031=========%%%
\begin{bt}
	Phân tích đa thức sau thành nhân tử: $4 x^2 + 40 x + 84$.
	\loigiai{ Ta có: $4 x^2 + 40 x + 84 = 4 \left(x + 3\right) \left(x + 7\right)$
	}
\end{bt}
%%%=========Bai_1032=========%%%
\begin{bt}
	Phân tích đa thức sau thành nhân tử: $4 x^2 + 36 x + 56$.
	\loigiai{ Ta có: $4 x^2 + 36 x + 56 = 4 \left(x + 2\right) \left(x + 7\right)$
	}
\end{bt}
%%%=========Bai_1033=========%%%
\begin{bt}
	Phân tích đa thức sau thành nhân tử: $4 x^2 + 32 x + 28$.
	\loigiai{ Ta có: $4 x^2 + 32 x + 28 = 4 \left(x + 1\right) \left(x + 7\right)$
	}
\end{bt}
%%%=========Bai_1034=========%%%
\begin{bt}
	Phân tích đa thức sau thành nhân tử: $4 x^2 + 24 x - 28$.
	\loigiai{ Ta có: $4 x^2 + 24 x - 28 = 4 \left(x - 1\right) \left(x + 7\right)$
	}
\end{bt}
%%%=========Bai_1035=========%%%
\begin{bt}
	Phân tích đa thức sau thành nhân tử: $4 x^2 + 20 x - 56$.
	\loigiai{ Ta có: $4 x^2 + 20 x - 56 = 4 \left(x - 2\right) \left(x + 7\right)$
	}
\end{bt}
%%%=========Bai_1036=========%%%
\begin{bt}
	Phân tích đa thức sau thành nhân tử: $4 x^2 + 16 x - 84$.
	\loigiai{ Ta có: $4 x^2 + 16 x - 84 = 4 \left(x - 3\right) \left(x + 7\right)$
	}
\end{bt}
%%%=========Bai_1037=========%%%
\begin{bt}
	Phân tích đa thức sau thành nhân tử: $4 x^2 + 12 x - 112$.
	\loigiai{ Ta có: $4 x^2 + 12 x - 112 = 4 \left(x - 4\right) \left(x + 7\right)$
	}
\end{bt}
%%%=========Bai_1038=========%%%
\begin{bt}
	Phân tích đa thức sau thành nhân tử: $4 x^2 + 8 x - 140$.
	\loigiai{ Ta có: $4 x^2 + 8 x - 140 = 4 \left(x - 5\right) \left(x + 7\right)$
	}
\end{bt}
%%%=========Bai_1039=========%%%
\begin{bt}
	Phân tích đa thức sau thành nhân tử: $4 x^2 + 4 x - 168$.
	\loigiai{ Ta có: $4 x^2 + 4 x - 168 = 4 \left(x - 6\right) \left(x + 7\right)$
	}
\end{bt}
%%%=========Bai_1040=========%%%
\begin{bt}
	Phân tích đa thức sau thành nhân tử: $4 x^2 - 4 x - 224$.
	\loigiai{ Ta có: $4 x^2 - 4 x - 224 = 4 \left(x - 8\right) \left(x + 7\right)$
	}
\end{bt}
%%%=========Bai_1041=========%%%
\begin{bt}
	Phân tích đa thức sau thành nhân tử: $4 x^2 - 8 x - 252$.
	\loigiai{ Ta có: $4 x^2 - 8 x - 252 = 4 \left(x - 9\right) \left(x + 7\right)$
	}
\end{bt}
%%%=========Bai_1042=========%%%
\begin{bt}
	Phân tích đa thức sau thành nhân tử: $4 x^2 + 64 x + 240$.
	\loigiai{ Ta có: $4 x^2 + 64 x + 240 = 4 \left(x + 6\right) \left(x + 10\right)$
	}
\end{bt}
%%%=========Bai_1043=========%%%
\begin{bt}
	Phân tích đa thức sau thành nhân tử: $4 x^2 + 60 x + 216$.
	\loigiai{ Ta có: $4 x^2 + 60 x + 216 = 4 \left(x + 6\right) \left(x + 9\right)$
	}
\end{bt}
%%%=========Bai_1044=========%%%
\begin{bt}
	Phân tích đa thức sau thành nhân tử: $4 x^2 + 56 x + 192$.
	\loigiai{ Ta có: $4 x^2 + 56 x + 192 = 4 \left(x + 6\right) \left(x + 8\right)$
	}
\end{bt}
%%%=========Bai_1045=========%%%
\begin{bt}
	Phân tích đa thức sau thành nhân tử: $4 x^2 + 52 x + 168$.
	\loigiai{ Ta có: $4 x^2 + 52 x + 168 = 4 \left(x + 6\right) \left(x + 7\right)$
	}
\end{bt}
%%%=========Bai_1046=========%%%
\begin{bt}
	Phân tích đa thức sau thành nhân tử: $4 x^2 + 44 x + 120$.
	\loigiai{ Ta có: $4 x^2 + 44 x + 120 = 4 \left(x + 5\right) \left(x + 6\right)$
	}
\end{bt}
%%%=========Bai_1047=========%%%
\begin{bt}
	Phân tích đa thức sau thành nhân tử: $4 x^2 + 40 x + 96$.
	\loigiai{ Ta có: $4 x^2 + 40 x + 96 = 4 \left(x + 4\right) \left(x + 6\right)$
	}
\end{bt}
%%%=========Bai_1048=========%%%
\begin{bt}
	Phân tích đa thức sau thành nhân tử: $4 x^2 + 36 x + 72$.
	\loigiai{ Ta có: $4 x^2 + 36 x + 72 = 4 \left(x + 3\right) \left(x + 6\right)$
	}
\end{bt}
%%%=========Bai_1049=========%%%
\begin{bt}
	Phân tích đa thức sau thành nhân tử: $4 x^2 + 32 x + 48$.
	\loigiai{ Ta có: $4 x^2 + 32 x + 48 = 4 \left(x + 2\right) \left(x + 6\right)$
	}
\end{bt}
%%%=========Bai_1050=========%%%
\begin{bt}
	Phân tích đa thức sau thành nhân tử: $4 x^2 + 28 x + 24$.
	\loigiai{ Ta có: $4 x^2 + 28 x + 24 = 4 \left(x + 1\right) \left(x + 6\right)$
	}
\end{bt}
%%%=========Bai_1051=========%%%
\begin{bt}
	Phân tích đa thức sau thành nhân tử: $4 x^2 + 20 x - 24$.
	\loigiai{ Ta có: $4 x^2 + 20 x - 24 = 4 \left(x - 1\right) \left(x + 6\right)$
	}
\end{bt}
%%%=========Bai_1052=========%%%
\begin{bt}
	Phân tích đa thức sau thành nhân tử: $4 x^2 + 16 x - 48$.
	\loigiai{ Ta có: $4 x^2 + 16 x - 48 = 4 \left(x - 2\right) \left(x + 6\right)$
	}
\end{bt}
%%%=========Bai_1053=========%%%
\begin{bt}
	Phân tích đa thức sau thành nhân tử: $4 x^2 + 12 x - 72$.
	\loigiai{ Ta có: $4 x^2 + 12 x - 72 = 4 \left(x - 3\right) \left(x + 6\right)$
	}
\end{bt}
%%%=========Bai_1054=========%%%
\begin{bt}
	Phân tích đa thức sau thành nhân tử: $4 x^2 + 8 x - 96$.
	\loigiai{ Ta có: $4 x^2 + 8 x - 96 = 4 \left(x - 4\right) \left(x + 6\right)$
	}
\end{bt}
%%%=========Bai_1055=========%%%
\begin{bt}
	Phân tích đa thức sau thành nhân tử: $4 x^2 + 4 x - 120$.
	\loigiai{ Ta có: $4 x^2 + 4 x - 120 = 4 \left(x - 5\right) \left(x + 6\right)$
	}
\end{bt}
%%%=========Bai_1056=========%%%
\begin{bt}
	Phân tích đa thức sau thành nhân tử: $4 x^2 - 4 x - 168$.
	\loigiai{ Ta có: $4 x^2 - 4 x - 168 = 4 \left(x - 7\right) \left(x + 6\right)$
	}
\end{bt}
%%%=========Bai_1057=========%%%
\begin{bt}
	Phân tích đa thức sau thành nhân tử: $4 x^2 - 8 x - 192$.
	\loigiai{ Ta có: $4 x^2 - 8 x - 192 = 4 \left(x - 8\right) \left(x + 6\right)$
	}
\end{bt}
%%%=========Bai_1058=========%%%
\begin{bt}
	Phân tích đa thức sau thành nhân tử: $4 x^2 - 12 x - 216$.
	\loigiai{ Ta có: $4 x^2 - 12 x - 216 = 4 \left(x - 9\right) \left(x + 6\right)$
	}
\end{bt}
%%%=========Bai_1059=========%%%
\begin{bt}
	Phân tích đa thức sau thành nhân tử: $4 x^2 + 60 x + 200$.
	\loigiai{ Ta có: $4 x^2 + 60 x + 200 = 4 \left(x + 5\right) \left(x + 10\right)$
	}
\end{bt}
%%%=========Bai_1060=========%%%
\begin{bt}
	Phân tích đa thức sau thành nhân tử: $4 x^2 + 56 x + 180$.
	\loigiai{ Ta có: $4 x^2 + 56 x + 180 = 4 \left(x + 5\right) \left(x + 9\right)$
	}
\end{bt}
%%%=========Bai_1061=========%%%
\begin{bt}
	Phân tích đa thức sau thành nhân tử: $4 x^2 + 52 x + 160$.
	\loigiai{ Ta có: $4 x^2 + 52 x + 160 = 4 \left(x + 5\right) \left(x + 8\right)$
	}
\end{bt}
%%%=========Bai_1062=========%%%
\begin{bt}
	Phân tích đa thức sau thành nhân tử: $4 x^2 + 48 x + 140$.
	\loigiai{ Ta có: $4 x^2 + 48 x + 140 = 4 \left(x + 5\right) \left(x + 7\right)$
	}
\end{bt}
%%%=========Bai_1063=========%%%
\begin{bt}
	Phân tích đa thức sau thành nhân tử: $4 x^2 + 44 x + 120$.
	\loigiai{ Ta có: $4 x^2 + 44 x + 120 = 4 \left(x + 5\right) \left(x + 6\right)$
	}
\end{bt}
%%%=========Bai_1064=========%%%
\begin{bt}
	Phân tích đa thức sau thành nhân tử: $4 x^2 + 36 x + 80$.
	\loigiai{ Ta có: $4 x^2 + 36 x + 80 = 4 \left(x + 4\right) \left(x + 5\right)$
	}
\end{bt}
%%%=========Bai_1065=========%%%
\begin{bt}
	Phân tích đa thức sau thành nhân tử: $4 x^2 + 32 x + 60$.
	\loigiai{ Ta có: $4 x^2 + 32 x + 60 = 4 \left(x + 3\right) \left(x + 5\right)$
	}
\end{bt}
%%%=========Bai_1066=========%%%
\begin{bt}
	Phân tích đa thức sau thành nhân tử: $4 x^2 + 28 x + 40$.
	\loigiai{ Ta có: $4 x^2 + 28 x + 40 = 4 \left(x + 2\right) \left(x + 5\right)$
	}
\end{bt}
%%%=========Bai_1067=========%%%
\begin{bt}
	Phân tích đa thức sau thành nhân tử: $4 x^2 + 24 x + 20$.
	\loigiai{ Ta có: $4 x^2 + 24 x + 20 = 4 \left(x + 1\right) \left(x + 5\right)$
	}
\end{bt}
%%%=========Bai_1068=========%%%
\begin{bt}
	Phân tích đa thức sau thành nhân tử: $4 x^2 + 16 x - 20$.
	\loigiai{ Ta có: $4 x^2 + 16 x - 20 = 4 \left(x - 1\right) \left(x + 5\right)$
	}
\end{bt}
%%%=========Bai_1069=========%%%
\begin{bt}
	Phân tích đa thức sau thành nhân tử: $4 x^2 + 12 x - 40$.
	\loigiai{ Ta có: $4 x^2 + 12 x - 40 = 4 \left(x - 2\right) \left(x + 5\right)$
	}
\end{bt}
%%%=========Bai_1070=========%%%
\begin{bt}
	Phân tích đa thức sau thành nhân tử: $4 x^2 + 8 x - 60$.
	\loigiai{ Ta có: $4 x^2 + 8 x - 60 = 4 \left(x - 3\right) \left(x + 5\right)$
	}
\end{bt}
%%%=========Bai_1071=========%%%
\begin{bt}
	Phân tích đa thức sau thành nhân tử: $4 x^2 + 4 x - 80$.
	\loigiai{ Ta có: $4 x^2 + 4 x - 80 = 4 \left(x - 4\right) \left(x + 5\right)$
	}
\end{bt}
%%%=========Bai_1072=========%%%
\begin{bt}
	Phân tích đa thức sau thành nhân tử: $4 x^2 - 4 x - 120$.
	\loigiai{ Ta có: $4 x^2 - 4 x - 120 = 4 \left(x - 6\right) \left(x + 5\right)$
	}
\end{bt}
%%%=========Bai_1073=========%%%
\begin{bt}
	Phân tích đa thức sau thành nhân tử: $4 x^2 - 8 x - 140$.
	\loigiai{ Ta có: $4 x^2 - 8 x - 140 = 4 \left(x - 7\right) \left(x + 5\right)$
	}
\end{bt}
%%%=========Bai_1074=========%%%
\begin{bt}
	Phân tích đa thức sau thành nhân tử: $4 x^2 - 12 x - 160$.
	\loigiai{ Ta có: $4 x^2 - 12 x - 160 = 4 \left(x - 8\right) \left(x + 5\right)$
	}
\end{bt}
%%%=========Bai_1075=========%%%
\begin{bt}
	Phân tích đa thức sau thành nhân tử: $4 x^2 - 16 x - 180$.
	\loigiai{ Ta có: $4 x^2 - 16 x - 180 = 4 \left(x - 9\right) \left(x + 5\right)$
	}
\end{bt}
%%%=========Bai_1076=========%%%
\begin{bt}
	Phân tích đa thức sau thành nhân tử: $4 x^2 + 56 x + 160$.
	\loigiai{ Ta có: $4 x^2 + 56 x + 160 = 4 \left(x + 4\right) \left(x + 10\right)$
	}
\end{bt}
%%%=========Bai_1077=========%%%
\begin{bt}
	Phân tích đa thức sau thành nhân tử: $4 x^2 + 52 x + 144$.
	\loigiai{ Ta có: $4 x^2 + 52 x + 144 = 4 \left(x + 4\right) \left(x + 9\right)$
	}
\end{bt}
%%%=========Bai_1078=========%%%
\begin{bt}
	Phân tích đa thức sau thành nhân tử: $4 x^2 + 48 x + 128$.
	\loigiai{ Ta có: $4 x^2 + 48 x + 128 = 4 \left(x + 4\right) \left(x + 8\right)$
	}
\end{bt}
%%%=========Bai_1079=========%%%
\begin{bt}
	Phân tích đa thức sau thành nhân tử: $4 x^2 + 44 x + 112$.
	\loigiai{ Ta có: $4 x^2 + 44 x + 112 = 4 \left(x + 4\right) \left(x + 7\right)$
	}
\end{bt}
%%%=========Bai_1080=========%%%
\begin{bt}
	Phân tích đa thức sau thành nhân tử: $4 x^2 + 40 x + 96$.
	\loigiai{ Ta có: $4 x^2 + 40 x + 96 = 4 \left(x + 4\right) \left(x + 6\right)$
	}
\end{bt}
%%%=========Bai_1081=========%%%
\begin{bt}
	Phân tích đa thức sau thành nhân tử: $4 x^2 + 36 x + 80$.
	\loigiai{ Ta có: $4 x^2 + 36 x + 80 = 4 \left(x + 4\right) \left(x + 5\right)$
	}
\end{bt}
%%%=========Bai_1082=========%%%
\begin{bt}
	Phân tích đa thức sau thành nhân tử: $4 x^2 + 28 x + 48$.
	\loigiai{ Ta có: $4 x^2 + 28 x + 48 = 4 \left(x + 3\right) \left(x + 4\right)$
	}
\end{bt}
%%%=========Bai_1083=========%%%
\begin{bt}
	Phân tích đa thức sau thành nhân tử: $4 x^2 + 24 x + 32$.
	\loigiai{ Ta có: $4 x^2 + 24 x + 32 = 4 \left(x + 2\right) \left(x + 4\right)$
	}
\end{bt}
%%%=========Bai_1084=========%%%
\begin{bt}
	Phân tích đa thức sau thành nhân tử: $4 x^2 + 20 x + 16$.
	\loigiai{ Ta có: $4 x^2 + 20 x + 16 = 4 \left(x + 1\right) \left(x + 4\right)$
	}
\end{bt}
%%%=========Bai_1085=========%%%
\begin{bt}
	Phân tích đa thức sau thành nhân tử: $4 x^2 + 12 x - 16$.
	\loigiai{ Ta có: $4 x^2 + 12 x - 16 = 4 \left(x - 1\right) \left(x + 4\right)$
	}
\end{bt}
%%%=========Bai_1086=========%%%
\begin{bt}
	Phân tích đa thức sau thành nhân tử: $4 x^2 + 8 x - 32$.
	\loigiai{ Ta có: $4 x^2 + 8 x - 32 = 4 \left(x - 2\right) \left(x + 4\right)$
	}
\end{bt}
%%%=========Bai_1087=========%%%
\begin{bt}
	Phân tích đa thức sau thành nhân tử: $4 x^2 + 4 x - 48$.
	\loigiai{ Ta có: $4 x^2 + 4 x - 48 = 4 \left(x - 3\right) \left(x + 4\right)$
	}
\end{bt}
%%%=========Bai_1088=========%%%
\begin{bt}
	Phân tích đa thức sau thành nhân tử: $4 x^2 - 4 x - 80$.
	\loigiai{ Ta có: $4 x^2 - 4 x - 80 = 4 \left(x - 5\right) \left(x + 4\right)$
	}
\end{bt}
%%%=========Bai_1089=========%%%
\begin{bt}
	Phân tích đa thức sau thành nhân tử: $4 x^2 - 8 x - 96$.
	\loigiai{ Ta có: $4 x^2 - 8 x - 96 = 4 \left(x - 6\right) \left(x + 4\right)$
	}
\end{bt}
%%%=========Bai_1090=========%%%
\begin{bt}
	Phân tích đa thức sau thành nhân tử: $4 x^2 - 12 x - 112$.
	\loigiai{ Ta có: $4 x^2 - 12 x - 112 = 4 \left(x - 7\right) \left(x + 4\right)$
	}
\end{bt}
%%%=========Bai_1091=========%%%
\begin{bt}
	Phân tích đa thức sau thành nhân tử: $4 x^2 - 16 x - 128$.
	\loigiai{ Ta có: $4 x^2 - 16 x - 128 = 4 \left(x - 8\right) \left(x + 4\right)$
	}
\end{bt}
%%%=========Bai_1092=========%%%
\begin{bt}
	Phân tích đa thức sau thành nhân tử: $4 x^2 - 20 x - 144$.
	\loigiai{ Ta có: $4 x^2 - 20 x - 144 = 4 \left(x - 9\right) \left(x + 4\right)$
	}
\end{bt}
%%%=========Bai_1093=========%%%
\begin{bt}
	Phân tích đa thức sau thành nhân tử: $4 x^2 + 52 x + 120$.
	\loigiai{ Ta có: $4 x^2 + 52 x + 120 = 4 \left(x + 3\right) \left(x + 10\right)$
	}
\end{bt}
%%%=========Bai_1094=========%%%
\begin{bt}
	Phân tích đa thức sau thành nhân tử: $4 x^2 + 48 x + 108$.
	\loigiai{ Ta có: $4 x^2 + 48 x + 108 = 4 \left(x + 3\right) \left(x + 9\right)$
	}
\end{bt}
%%%=========Bai_1095=========%%%
\begin{bt}
	Phân tích đa thức sau thành nhân tử: $4 x^2 + 44 x + 96$.
	\loigiai{ Ta có: $4 x^2 + 44 x + 96 = 4 \left(x + 3\right) \left(x + 8\right)$
	}
\end{bt}
%%%=========Bai_1096=========%%%
\begin{bt}
	Phân tích đa thức sau thành nhân tử: $4 x^2 + 40 x + 84$.
	\loigiai{ Ta có: $4 x^2 + 40 x + 84 = 4 \left(x + 3\right) \left(x + 7\right)$
	}
\end{bt}
%%%=========Bai_1097=========%%%
\begin{bt}
	Phân tích đa thức sau thành nhân tử: $4 x^2 + 36 x + 72$.
	\loigiai{ Ta có: $4 x^2 + 36 x + 72 = 4 \left(x + 3\right) \left(x + 6\right)$
	}
\end{bt}
%%%=========Bai_1098=========%%%
\begin{bt}
	Phân tích đa thức sau thành nhân tử: $4 x^2 + 32 x + 60$.
	\loigiai{ Ta có: $4 x^2 + 32 x + 60 = 4 \left(x + 3\right) \left(x + 5\right)$
	}
\end{bt}
%%%=========Bai_1099=========%%%
\begin{bt}
	Phân tích đa thức sau thành nhân tử: $4 x^2 + 28 x + 48$.
	\loigiai{ Ta có: $4 x^2 + 28 x + 48 = 4 \left(x + 3\right) \left(x + 4\right)$
	}
\end{bt}
%%%=========Bai_1100=========%%%
\begin{bt}
	Phân tích đa thức sau thành nhân tử: $4 x^2 + 20 x + 24$.
	\loigiai{ Ta có: $4 x^2 + 20 x + 24 = 4 \left(x + 2\right) \left(x + 3\right)$
	}
\end{bt}
%%%=========Bai_1101=========%%%
\begin{bt}
	Phân tích đa thức sau thành nhân tử: $4 x^2 + 16 x + 12$.
	\loigiai{ Ta có: $4 x^2 + 16 x + 12 = 4 \left(x + 1\right) \left(x + 3\right)$
	}
\end{bt}
%%%=========Bai_1102=========%%%
\begin{bt}
	Phân tích đa thức sau thành nhân tử: $4 x^2 + 8 x - 12$.
	\loigiai{ Ta có: $4 x^2 + 8 x - 12 = 4 \left(x - 1\right) \left(x + 3\right)$
	}
\end{bt}
%%%=========Bai_1103=========%%%
\begin{bt}
	Phân tích đa thức sau thành nhân tử: $4 x^2 + 4 x - 24$.
	\loigiai{ Ta có: $4 x^2 + 4 x - 24 = 4 \left(x - 2\right) \left(x + 3\right)$
	}
\end{bt}
%%%=========Bai_1104=========%%%
\begin{bt}
	Phân tích đa thức sau thành nhân tử: $4 x^2 - 4 x - 48$.
	\loigiai{ Ta có: $4 x^2 - 4 x - 48 = 4 \left(x - 4\right) \left(x + 3\right)$
	}
\end{bt}
%%%=========Bai_1105=========%%%
\begin{bt}
	Phân tích đa thức sau thành nhân tử: $4 x^2 - 8 x - 60$.
	\loigiai{ Ta có: $4 x^2 - 8 x - 60 = 4 \left(x - 5\right) \left(x + 3\right)$
	}
\end{bt}
%%%=========Bai_1106=========%%%
\begin{bt}
	Phân tích đa thức sau thành nhân tử: $4 x^2 - 12 x - 72$.
	\loigiai{ Ta có: $4 x^2 - 12 x - 72 = 4 \left(x - 6\right) \left(x + 3\right)$
	}
\end{bt}
%%%=========Bai_1107=========%%%
\begin{bt}
	Phân tích đa thức sau thành nhân tử: $4 x^2 - 16 x - 84$.
	\loigiai{ Ta có: $4 x^2 - 16 x - 84 = 4 \left(x - 7\right) \left(x + 3\right)$
	}
\end{bt}
%%%=========Bai_1108=========%%%
\begin{bt}
	Phân tích đa thức sau thành nhân tử: $4 x^2 - 20 x - 96$.
	\loigiai{ Ta có: $4 x^2 - 20 x - 96 = 4 \left(x - 8\right) \left(x + 3\right)$
	}
\end{bt}
%%%=========Bai_1109=========%%%
\begin{bt}
	Phân tích đa thức sau thành nhân tử: $4 x^2 - 24 x - 108$.
	\loigiai{ Ta có: $4 x^2 - 24 x - 108 = 4 \left(x - 9\right) \left(x + 3\right)$
	}
\end{bt}
%%%=========Bai_1110=========%%%
\begin{bt}
	Phân tích đa thức sau thành nhân tử: $4 x^2 + 48 x + 80$.
	\loigiai{ Ta có: $4 x^2 + 48 x + 80 = 4 \left(x + 2\right) \left(x + 10\right)$
	}
\end{bt}
%%%=========Bai_1111=========%%%
\begin{bt}
	Phân tích đa thức sau thành nhân tử: $4 x^2 + 44 x + 72$.
	\loigiai{ Ta có: $4 x^2 + 44 x + 72 = 4 \left(x + 2\right) \left(x + 9\right)$
	}
\end{bt}
%%%=========Bai_1112=========%%%
\begin{bt}
	Phân tích đa thức sau thành nhân tử: $4 x^2 + 40 x + 64$.
	\loigiai{ Ta có: $4 x^2 + 40 x + 64 = 4 \left(x + 2\right) \left(x + 8\right)$
	}
\end{bt}
%%%=========Bai_1113=========%%%
\begin{bt}
	Phân tích đa thức sau thành nhân tử: $4 x^2 + 36 x + 56$.
	\loigiai{ Ta có: $4 x^2 + 36 x + 56 = 4 \left(x + 2\right) \left(x + 7\right)$
	}
\end{bt}
%%%=========Bai_1114=========%%%
\begin{bt}
	Phân tích đa thức sau thành nhân tử: $4 x^2 + 32 x + 48$.
	\loigiai{ Ta có: $4 x^2 + 32 x + 48 = 4 \left(x + 2\right) \left(x + 6\right)$
	}
\end{bt}
%%%=========Bai_1115=========%%%
\begin{bt}
	Phân tích đa thức sau thành nhân tử: $4 x^2 + 28 x + 40$.
	\loigiai{ Ta có: $4 x^2 + 28 x + 40 = 4 \left(x + 2\right) \left(x + 5\right)$
	}
\end{bt}
%%%=========Bai_1116=========%%%
\begin{bt}
	Phân tích đa thức sau thành nhân tử: $4 x^2 + 24 x + 32$.
	\loigiai{ Ta có: $4 x^2 + 24 x + 32 = 4 \left(x + 2\right) \left(x + 4\right)$
	}
\end{bt}
%%%=========Bai_1117=========%%%
\begin{bt}
	Phân tích đa thức sau thành nhân tử: $4 x^2 + 20 x + 24$.
	\loigiai{ Ta có: $4 x^2 + 20 x + 24 = 4 \left(x + 2\right) \left(x + 3\right)$
	}
\end{bt}
%%%=========Bai_1118=========%%%
\begin{bt}
	Phân tích đa thức sau thành nhân tử: $4 x^2 + 12 x + 8$.
	\loigiai{ Ta có: $4 x^2 + 12 x + 8 = 4 \left(x + 1\right) \left(x + 2\right)$
	}
\end{bt}
%%%=========Bai_1119=========%%%
\begin{bt}
	Phân tích đa thức sau thành nhân tử: $4 x^2 + 4 x - 8$.
	\loigiai{ Ta có: $4 x^2 + 4 x - 8 = 4 \left(x - 1\right) \left(x + 2\right)$
	}
\end{bt}
%%%=========Bai_1120=========%%%
\begin{bt}
	Phân tích đa thức sau thành nhân tử: $4 x^2 - 4 x - 24$.
	\loigiai{ Ta có: $4 x^2 - 4 x - 24 = 4 \left(x - 3\right) \left(x + 2\right)$
	}
\end{bt}
%%%=========Bai_1121=========%%%
\begin{bt}
	Phân tích đa thức sau thành nhân tử: $4 x^2 - 8 x - 32$.
	\loigiai{ Ta có: $4 x^2 - 8 x - 32 = 4 \left(x - 4\right) \left(x + 2\right)$
	}
\end{bt}
%%%=========Bai_1122=========%%%
\begin{bt}
	Phân tích đa thức sau thành nhân tử: $4 x^2 - 12 x - 40$.
	\loigiai{ Ta có: $4 x^2 - 12 x - 40 = 4 \left(x - 5\right) \left(x + 2\right)$
	}
\end{bt}
%%%=========Bai_1123=========%%%
\begin{bt}
	Phân tích đa thức sau thành nhân tử: $4 x^2 - 16 x - 48$.
	\loigiai{ Ta có: $4 x^2 - 16 x - 48 = 4 \left(x - 6\right) \left(x + 2\right)$
	}
\end{bt}
%%%=========Bai_1124=========%%%
\begin{bt}
	Phân tích đa thức sau thành nhân tử: $4 x^2 - 20 x - 56$.
	\loigiai{ Ta có: $4 x^2 - 20 x - 56 = 4 \left(x - 7\right) \left(x + 2\right)$
	}
\end{bt}
%%%=========Bai_1125=========%%%
\begin{bt}
	Phân tích đa thức sau thành nhân tử: $4 x^2 - 24 x - 64$.
	\loigiai{ Ta có: $4 x^2 - 24 x - 64 = 4 \left(x - 8\right) \left(x + 2\right)$
	}
\end{bt}
%%%=========Bai_1126=========%%%
\begin{bt}
	Phân tích đa thức sau thành nhân tử: $4 x^2 - 28 x - 72$.
	\loigiai{ Ta có: $4 x^2 - 28 x - 72 = 4 \left(x - 9\right) \left(x + 2\right)$
	}
\end{bt}
%%%=========Bai_1127=========%%%
\begin{bt}
	Phân tích đa thức sau thành nhân tử: $4 x^2 + 44 x + 40$.
	\loigiai{ Ta có: $4 x^2 + 44 x + 40 = 4 \left(x + 1\right) \left(x + 10\right)$
	}
\end{bt}
%%%=========Bai_1128=========%%%
\begin{bt}
	Phân tích đa thức sau thành nhân tử: $4 x^2 + 40 x + 36$.
	\loigiai{ Ta có: $4 x^2 + 40 x + 36 = 4 \left(x + 1\right) \left(x + 9\right)$
	}
\end{bt}
%%%=========Bai_1129=========%%%
\begin{bt}
	Phân tích đa thức sau thành nhân tử: $4 x^2 + 36 x + 32$.
	\loigiai{ Ta có: $4 x^2 + 36 x + 32 = 4 \left(x + 1\right) \left(x + 8\right)$
	}
\end{bt}
%%%=========Bai_1130=========%%%
\begin{bt}
	Phân tích đa thức sau thành nhân tử: $4 x^2 + 32 x + 28$.
	\loigiai{ Ta có: $4 x^2 + 32 x + 28 = 4 \left(x + 1\right) \left(x + 7\right)$
	}
\end{bt}
%%%=========Bai_1131=========%%%
\begin{bt}
	Phân tích đa thức sau thành nhân tử: $4 x^2 + 28 x + 24$.
	\loigiai{ Ta có: $4 x^2 + 28 x + 24 = 4 \left(x + 1\right) \left(x + 6\right)$
	}
\end{bt}
%%%=========Bai_1132=========%%%
\begin{bt}
	Phân tích đa thức sau thành nhân tử: $4 x^2 + 24 x + 20$.
	\loigiai{ Ta có: $4 x^2 + 24 x + 20 = 4 \left(x + 1\right) \left(x + 5\right)$
	}
\end{bt}
%%%=========Bai_1133=========%%%
\begin{bt}
	Phân tích đa thức sau thành nhân tử: $4 x^2 + 20 x + 16$.
	\loigiai{ Ta có: $4 x^2 + 20 x + 16 = 4 \left(x + 1\right) \left(x + 4\right)$
	}
\end{bt}
%%%=========Bai_1134=========%%%
\begin{bt}
	Phân tích đa thức sau thành nhân tử: $4 x^2 + 16 x + 12$.
	\loigiai{ Ta có: $4 x^2 + 16 x + 12 = 4 \left(x + 1\right) \left(x + 3\right)$
	}
\end{bt}
%%%=========Bai_1135=========%%%
\begin{bt}
	Phân tích đa thức sau thành nhân tử: $4 x^2 + 12 x + 8$.
	\loigiai{ Ta có: $4 x^2 + 12 x + 8 = 4 \left(x + 1\right) \left(x + 2\right)$
	}
\end{bt}
%%%=========Bai_1136=========%%%
\begin{bt}
	Phân tích đa thức sau thành nhân tử: $4 x^2 - 4 x - 8$.
	\loigiai{ Ta có: $4 x^2 - 4 x - 8 = 4 \left(x - 2\right) \left(x + 1\right)$
	}
\end{bt}
%%%=========Bai_1137=========%%%
\begin{bt}
	Phân tích đa thức sau thành nhân tử: $4 x^2 - 8 x - 12$.
	\loigiai{ Ta có: $4 x^2 - 8 x - 12 = 4 \left(x - 3\right) \left(x + 1\right)$
	}
\end{bt}
%%%=========Bai_1138=========%%%
\begin{bt}
	Phân tích đa thức sau thành nhân tử: $4 x^2 - 12 x - 16$.
	\loigiai{ Ta có: $4 x^2 - 12 x - 16 = 4 \left(x - 4\right) \left(x + 1\right)$
	}
\end{bt}
%%%=========Bai_1139=========%%%
\begin{bt}
	Phân tích đa thức sau thành nhân tử: $4 x^2 - 16 x - 20$.
	\loigiai{ Ta có: $4 x^2 - 16 x - 20 = 4 \left(x - 5\right) \left(x + 1\right)$
	}
\end{bt}
%%%=========Bai_1140=========%%%
\begin{bt}
	Phân tích đa thức sau thành nhân tử: $4 x^2 - 20 x - 24$.
	\loigiai{ Ta có: $4 x^2 - 20 x - 24 = 4 \left(x - 6\right) \left(x + 1\right)$
	}
\end{bt}
%%%=========Bai_1141=========%%%
\begin{bt}
	Phân tích đa thức sau thành nhân tử: $4 x^2 - 24 x - 28$.
	\loigiai{ Ta có: $4 x^2 - 24 x - 28 = 4 \left(x - 7\right) \left(x + 1\right)$
	}
\end{bt}
%%%=========Bai_1142=========%%%
\begin{bt}
	Phân tích đa thức sau thành nhân tử: $4 x^2 - 28 x - 32$.
	\loigiai{ Ta có: $4 x^2 - 28 x - 32 = 4 \left(x - 8\right) \left(x + 1\right)$
	}
\end{bt}
%%%=========Bai_1143=========%%%
\begin{bt}
	Phân tích đa thức sau thành nhân tử: $4 x^2 - 32 x - 36$.
	\loigiai{ Ta có: $4 x^2 - 32 x - 36 = 4 \left(x - 9\right) \left(x + 1\right)$
	}
\end{bt}
%%%=========Bai_1144=========%%%
\begin{bt}
	Phân tích đa thức sau thành nhân tử: $4 x^2 + 36 x - 40$.
	\loigiai{ Ta có: $4 x^2 + 36 x - 40 = 4 \left(x - 1\right) \left(x + 10\right)$
	}
\end{bt}
%%%=========Bai_1145=========%%%
\begin{bt}
	Phân tích đa thức sau thành nhân tử: $4 x^2 + 32 x - 36$.
	\loigiai{ Ta có: $4 x^2 + 32 x - 36 = 4 \left(x - 1\right) \left(x + 9\right)$
	}
\end{bt}
%%%=========Bai_1146=========%%%
\begin{bt}
	Phân tích đa thức sau thành nhân tử: $4 x^2 + 28 x - 32$.
	\loigiai{ Ta có: $4 x^2 + 28 x - 32 = 4 \left(x - 1\right) \left(x + 8\right)$
	}
\end{bt}
%%%=========Bai_1147=========%%%
\begin{bt}
	Phân tích đa thức sau thành nhân tử: $4 x^2 + 24 x - 28$.
	\loigiai{ Ta có: $4 x^2 + 24 x - 28 = 4 \left(x - 1\right) \left(x + 7\right)$
	}
\end{bt}
%%%=========Bai_1148=========%%%
\begin{bt}
	Phân tích đa thức sau thành nhân tử: $4 x^2 + 20 x - 24$.
	\loigiai{ Ta có: $4 x^2 + 20 x - 24 = 4 \left(x - 1\right) \left(x + 6\right)$
	}
\end{bt}
%%%=========Bai_1149=========%%%
\begin{bt}
	Phân tích đa thức sau thành nhân tử: $4 x^2 + 16 x - 20$.
	\loigiai{ Ta có: $4 x^2 + 16 x - 20 = 4 \left(x - 1\right) \left(x + 5\right)$
	}
\end{bt}
%%%=========Bai_1150=========%%%
\begin{bt}
	Phân tích đa thức sau thành nhân tử: $4 x^2 + 12 x - 16$.
	\loigiai{ Ta có: $4 x^2 + 12 x - 16 = 4 \left(x - 1\right) \left(x + 4\right)$
	}
\end{bt}
%%%=========Bai_1151=========%%%
\begin{bt}
	Phân tích đa thức sau thành nhân tử: $4 x^2 + 8 x - 12$.
	\loigiai{ Ta có: $4 x^2 + 8 x - 12 = 4 \left(x - 1\right) \left(x + 3\right)$
	}
\end{bt}
%%%=========Bai_1152=========%%%
\begin{bt}
	Phân tích đa thức sau thành nhân tử: $4 x^2 + 4 x - 8$.
	\loigiai{ Ta có: $4 x^2 + 4 x - 8 = 4 \left(x - 1\right) \left(x + 2\right)$
	}
\end{bt}
%%%=========Bai_1153=========%%%
\begin{bt}
	Phân tích đa thức sau thành nhân tử: $4 x^2 - 12 x + 8$.
	\loigiai{ Ta có: $4 x^2 - 12 x + 8 = 4 \left(x - 2\right) \left(x - 1\right)$
	}
\end{bt}
%%%=========Bai_1154=========%%%
\begin{bt}
	Phân tích đa thức sau thành nhân tử: $4 x^2 - 16 x + 12$.
	\loigiai{ Ta có: $4 x^2 - 16 x + 12 = 4 \left(x - 3\right) \left(x - 1\right)$
	}
\end{bt}
%%%=========Bai_1155=========%%%
\begin{bt}
	Phân tích đa thức sau thành nhân tử: $4 x^2 - 20 x + 16$.
	\loigiai{ Ta có: $4 x^2 - 20 x + 16 = 4 \left(x - 4\right) \left(x - 1\right)$
	}
\end{bt}
%%%=========Bai_1156=========%%%
\begin{bt}
	Phân tích đa thức sau thành nhân tử: $4 x^2 - 24 x + 20$.
	\loigiai{ Ta có: $4 x^2 - 24 x + 20 = 4 \left(x - 5\right) \left(x - 1\right)$
	}
\end{bt}
%%%=========Bai_1157=========%%%
\begin{bt}
	Phân tích đa thức sau thành nhân tử: $4 x^2 - 28 x + 24$.
	\loigiai{ Ta có: $4 x^2 - 28 x + 24 = 4 \left(x - 6\right) \left(x - 1\right)$
	}
\end{bt}
%%%=========Bai_1158=========%%%
\begin{bt}
	Phân tích đa thức sau thành nhân tử: $4 x^2 - 32 x + 28$.
	\loigiai{ Ta có: $4 x^2 - 32 x + 28 = 4 \left(x - 7\right) \left(x - 1\right)$
	}
\end{bt}
%%%=========Bai_1159=========%%%
\begin{bt}
	Phân tích đa thức sau thành nhân tử: $4 x^2 - 36 x + 32$.
	\loigiai{ Ta có: $4 x^2 - 36 x + 32 = 4 \left(x - 8\right) \left(x - 1\right)$
	}
\end{bt}
%%%=========Bai_1160=========%%%
\begin{bt}
	Phân tích đa thức sau thành nhân tử: $4 x^2 - 40 x + 36$.
	\loigiai{ Ta có: $4 x^2 - 40 x + 36 = 4 \left(x - 9\right) \left(x - 1\right)$
	}
\end{bt}
%%%=========Bai_1161=========%%%
\begin{bt}
	Phân tích đa thức sau thành nhân tử: $4 x^2 + 32 x - 80$.
	\loigiai{ Ta có: $4 x^2 + 32 x - 80 = 4 \left(x - 2\right) \left(x + 10\right)$
	}
\end{bt}
%%%=========Bai_1162=========%%%
\begin{bt}
	Phân tích đa thức sau thành nhân tử: $4 x^2 + 28 x - 72$.
	\loigiai{ Ta có: $4 x^2 + 28 x - 72 = 4 \left(x - 2\right) \left(x + 9\right)$
	}
\end{bt}
%%%=========Bai_1163=========%%%
\begin{bt}
	Phân tích đa thức sau thành nhân tử: $4 x^2 + 24 x - 64$.
	\loigiai{ Ta có: $4 x^2 + 24 x - 64 = 4 \left(x - 2\right) \left(x + 8\right)$
	}
\end{bt}
%%%=========Bai_1164=========%%%
\begin{bt}
	Phân tích đa thức sau thành nhân tử: $4 x^2 + 20 x - 56$.
	\loigiai{ Ta có: $4 x^2 + 20 x - 56 = 4 \left(x - 2\right) \left(x + 7\right)$
	}
\end{bt}
%%%=========Bai_1165=========%%%
\begin{bt}
	Phân tích đa thức sau thành nhân tử: $4 x^2 + 16 x - 48$.
	\loigiai{ Ta có: $4 x^2 + 16 x - 48 = 4 \left(x - 2\right) \left(x + 6\right)$
	}
\end{bt}
%%%=========Bai_1166=========%%%
\begin{bt}
	Phân tích đa thức sau thành nhân tử: $4 x^2 + 12 x - 40$.
	\loigiai{ Ta có: $4 x^2 + 12 x - 40 = 4 \left(x - 2\right) \left(x + 5\right)$
	}
\end{bt}
%%%=========Bai_1167=========%%%
\begin{bt}
	Phân tích đa thức sau thành nhân tử: $4 x^2 + 8 x - 32$.
	\loigiai{ Ta có: $4 x^2 + 8 x - 32 = 4 \left(x - 2\right) \left(x + 4\right)$
	}
\end{bt}
%%%=========Bai_1168=========%%%
\begin{bt}
	Phân tích đa thức sau thành nhân tử: $4 x^2 + 4 x - 24$.
	\loigiai{ Ta có: $4 x^2 + 4 x - 24 = 4 \left(x - 2\right) \left(x + 3\right)$
	}
\end{bt}
%%%=========Bai_1169=========%%%
\begin{bt}
	Phân tích đa thức sau thành nhân tử: $4 x^2 - 4 x - 8$.
	\loigiai{ Ta có: $4 x^2 - 4 x - 8 = 4 \left(x - 2\right) \left(x + 1\right)$
	}
\end{bt}
%%%=========Bai_1170=========%%%
\begin{bt}
	Phân tích đa thức sau thành nhân tử: $4 x^2 - 12 x + 8$.
	\loigiai{ Ta có: $4 x^2 - 12 x + 8 = 4 \left(x - 2\right) \left(x - 1\right)$
	}
\end{bt}
%%%=========Bai_1171=========%%%
\begin{bt}
	Phân tích đa thức sau thành nhân tử: $4 x^2 - 20 x + 24$.
	\loigiai{ Ta có: $4 x^2 - 20 x + 24 = 4 \left(x - 3\right) \left(x - 2\right)$
	}
\end{bt}
%%%=========Bai_1172=========%%%
\begin{bt}
	Phân tích đa thức sau thành nhân tử: $4 x^2 - 24 x + 32$.
	\loigiai{ Ta có: $4 x^2 - 24 x + 32 = 4 \left(x - 4\right) \left(x - 2\right)$
	}
\end{bt}
%%%=========Bai_1173=========%%%
\begin{bt}
	Phân tích đa thức sau thành nhân tử: $4 x^2 - 28 x + 40$.
	\loigiai{ Ta có: $4 x^2 - 28 x + 40 = 4 \left(x - 5\right) \left(x - 2\right)$
	}
\end{bt}
%%%=========Bai_1174=========%%%
\begin{bt}
	Phân tích đa thức sau thành nhân tử: $4 x^2 - 32 x + 48$.
	\loigiai{ Ta có: $4 x^2 - 32 x + 48 = 4 \left(x - 6\right) \left(x - 2\right)$
	}
\end{bt}
%%%=========Bai_1175=========%%%
\begin{bt}
	Phân tích đa thức sau thành nhân tử: $4 x^2 - 36 x + 56$.
	\loigiai{ Ta có: $4 x^2 - 36 x + 56 = 4 \left(x - 7\right) \left(x - 2\right)$
	}
\end{bt}
%%%=========Bai_1176=========%%%
\begin{bt}
	Phân tích đa thức sau thành nhân tử: $4 x^2 - 40 x + 64$.
	\loigiai{ Ta có: $4 x^2 - 40 x + 64 = 4 \left(x - 8\right) \left(x - 2\right)$
	}
\end{bt}
%%%=========Bai_1177=========%%%
\begin{bt}
	Phân tích đa thức sau thành nhân tử: $4 x^2 - 44 x + 72$.
	\loigiai{ Ta có: $4 x^2 - 44 x + 72 = 4 \left(x - 9\right) \left(x - 2\right)$
	}
\end{bt}
%%%=========Bai_1178=========%%%
\begin{bt}
	Phân tích đa thức sau thành nhân tử: $4 x^2 + 28 x - 120$.
	\loigiai{ Ta có: $4 x^2 + 28 x - 120 = 4 \left(x - 3\right) \left(x + 10\right)$
	}
\end{bt}
%%%=========Bai_1179=========%%%
\begin{bt}
	Phân tích đa thức sau thành nhân tử: $4 x^2 + 24 x - 108$.
	\loigiai{ Ta có: $4 x^2 + 24 x - 108 = 4 \left(x - 3\right) \left(x + 9\right)$
	}
\end{bt}
%%%=========Bai_1180=========%%%
\begin{bt}
	Phân tích đa thức sau thành nhân tử: $4 x^2 + 20 x - 96$.
	\loigiai{ Ta có: $4 x^2 + 20 x - 96 = 4 \left(x - 3\right) \left(x + 8\right)$
	}
\end{bt}
%%%=========Bai_1181=========%%%
\begin{bt}
	Phân tích đa thức sau thành nhân tử: $4 x^2 + 16 x - 84$.
	\loigiai{ Ta có: $4 x^2 + 16 x - 84 = 4 \left(x - 3\right) \left(x + 7\right)$
	}
\end{bt}
%%%=========Bai_1182=========%%%
\begin{bt}
	Phân tích đa thức sau thành nhân tử: $4 x^2 + 12 x - 72$.
	\loigiai{ Ta có: $4 x^2 + 12 x - 72 = 4 \left(x - 3\right) \left(x + 6\right)$
	}
\end{bt}
%%%=========Bai_1183=========%%%
\begin{bt}
	Phân tích đa thức sau thành nhân tử: $4 x^2 + 8 x - 60$.
	\loigiai{ Ta có: $4 x^2 + 8 x - 60 = 4 \left(x - 3\right) \left(x + 5\right)$
	}
\end{bt}
%%%=========Bai_1184=========%%%
\begin{bt}
	Phân tích đa thức sau thành nhân tử: $4 x^2 + 4 x - 48$.
	\loigiai{ Ta có: $4 x^2 + 4 x - 48 = 4 \left(x - 3\right) \left(x + 4\right)$
	}
\end{bt}
%%%=========Bai_1185=========%%%
\begin{bt}
	Phân tích đa thức sau thành nhân tử: $4 x^2 - 4 x - 24$.
	\loigiai{ Ta có: $4 x^2 - 4 x - 24 = 4 \left(x - 3\right) \left(x + 2\right)$
	}
\end{bt}
%%%=========Bai_1186=========%%%
\begin{bt}
	Phân tích đa thức sau thành nhân tử: $4 x^2 - 8 x - 12$.
	\loigiai{ Ta có: $4 x^2 - 8 x - 12 = 4 \left(x - 3\right) \left(x + 1\right)$
	}
\end{bt}
%%%=========Bai_1187=========%%%
\begin{bt}
	Phân tích đa thức sau thành nhân tử: $4 x^2 - 16 x + 12$.
	\loigiai{ Ta có: $4 x^2 - 16 x + 12 = 4 \left(x - 3\right) \left(x - 1\right)$
	}
\end{bt}
%%%=========Bai_1188=========%%%
\begin{bt}
	Phân tích đa thức sau thành nhân tử: $4 x^2 - 20 x + 24$.
	\loigiai{ Ta có: $4 x^2 - 20 x + 24 = 4 \left(x - 3\right) \left(x - 2\right)$
	}
\end{bt}
%%%=========Bai_1189=========%%%
\begin{bt}
	Phân tích đa thức sau thành nhân tử: $4 x^2 - 28 x + 48$.
	\loigiai{ Ta có: $4 x^2 - 28 x + 48 = 4 \left(x - 4\right) \left(x - 3\right)$
	}
\end{bt}
%%%=========Bai_1190=========%%%
\begin{bt}
	Phân tích đa thức sau thành nhân tử: $4 x^2 - 32 x + 60$.
	\loigiai{ Ta có: $4 x^2 - 32 x + 60 = 4 \left(x - 5\right) \left(x - 3\right)$
	}
\end{bt}
%%%=========Bai_1191=========%%%
\begin{bt}
	Phân tích đa thức sau thành nhân tử: $4 x^2 - 36 x + 72$.
	\loigiai{ Ta có: $4 x^2 - 36 x + 72 = 4 \left(x - 6\right) \left(x - 3\right)$
	}
\end{bt}
%%%=========Bai_1192=========%%%
\begin{bt}
	Phân tích đa thức sau thành nhân tử: $4 x^2 - 40 x + 84$.
	\loigiai{ Ta có: $4 x^2 - 40 x + 84 = 4 \left(x - 7\right) \left(x - 3\right)$
	}
\end{bt}
%%%=========Bai_1193=========%%%
\begin{bt}
	Phân tích đa thức sau thành nhân tử: $4 x^2 - 44 x + 96$.
	\loigiai{ Ta có: $4 x^2 - 44 x + 96 = 4 \left(x - 8\right) \left(x - 3\right)$
	}
\end{bt}
%%%=========Bai_1194=========%%%
\begin{bt}
	Phân tích đa thức sau thành nhân tử: $4 x^2 - 48 x + 108$.
	\loigiai{ Ta có: $4 x^2 - 48 x + 108 = 4 \left(x - 9\right) \left(x - 3\right)$
	}
\end{bt}
%%%=========Bai_1195=========%%%
\begin{bt}
	Phân tích đa thức sau thành nhân tử: $4 x^2 + 24 x - 160$.
	\loigiai{ Ta có: $4 x^2 + 24 x - 160 = 4 \left(x - 4\right) \left(x + 10\right)$
	}
\end{bt}
%%%=========Bai_1196=========%%%
\begin{bt}
	Phân tích đa thức sau thành nhân tử: $4 x^2 + 20 x - 144$.
	\loigiai{ Ta có: $4 x^2 + 20 x - 144 = 4 \left(x - 4\right) \left(x + 9\right)$
	}
\end{bt}
%%%=========Bai_1197=========%%%
\begin{bt}
	Phân tích đa thức sau thành nhân tử: $4 x^2 + 16 x - 128$.
	\loigiai{ Ta có: $4 x^2 + 16 x - 128 = 4 \left(x - 4\right) \left(x + 8\right)$
	}
\end{bt}
%%%=========Bai_1198=========%%%
\begin{bt}
	Phân tích đa thức sau thành nhân tử: $4 x^2 + 12 x - 112$.
	\loigiai{ Ta có: $4 x^2 + 12 x - 112 = 4 \left(x - 4\right) \left(x + 7\right)$
	}
\end{bt}
%%%=========Bai_1199=========%%%
\begin{bt}
	Phân tích đa thức sau thành nhân tử: $4 x^2 + 8 x - 96$.
	\loigiai{ Ta có: $4 x^2 + 8 x - 96 = 4 \left(x - 4\right) \left(x + 6\right)$
	}
\end{bt}
%%%=========Bai_1200=========%%%
\begin{bt}
	Phân tích đa thức sau thành nhân tử: $4 x^2 + 4 x - 80$.
	\loigiai{ Ta có: $4 x^2 + 4 x - 80 = 4 \left(x - 4\right) \left(x + 5\right)$
	}
\end{bt}
%%%=========Bai_1201=========%%%
\begin{bt}
	Phân tích đa thức sau thành nhân tử: $4 x^2 - 4 x - 48$.
	\loigiai{ Ta có: $4 x^2 - 4 x - 48 = 4 \left(x - 4\right) \left(x + 3\right)$
	}
\end{bt}
%%%=========Bai_1202=========%%%
\begin{bt}
	Phân tích đa thức sau thành nhân tử: $4 x^2 - 8 x - 32$.
	\loigiai{ Ta có: $4 x^2 - 8 x - 32 = 4 \left(x - 4\right) \left(x + 2\right)$
	}
\end{bt}
%%%=========Bai_1203=========%%%
\begin{bt}
	Phân tích đa thức sau thành nhân tử: $4 x^2 - 12 x - 16$.
	\loigiai{ Ta có: $4 x^2 - 12 x - 16 = 4 \left(x - 4\right) \left(x + 1\right)$
	}
\end{bt}
%%%=========Bai_1204=========%%%
\begin{bt}
	Phân tích đa thức sau thành nhân tử: $4 x^2 - 20 x + 16$.
	\loigiai{ Ta có: $4 x^2 - 20 x + 16 = 4 \left(x - 4\right) \left(x - 1\right)$
	}
\end{bt}
%%%=========Bai_1205=========%%%
\begin{bt}
	Phân tích đa thức sau thành nhân tử: $4 x^2 - 24 x + 32$.
	\loigiai{ Ta có: $4 x^2 - 24 x + 32 = 4 \left(x - 4\right) \left(x - 2\right)$
	}
\end{bt}
%%%=========Bai_1206=========%%%
\begin{bt}
	Phân tích đa thức sau thành nhân tử: $4 x^2 - 28 x + 48$.
	\loigiai{ Ta có: $4 x^2 - 28 x + 48 = 4 \left(x - 4\right) \left(x - 3\right)$
	}
\end{bt}
%%%=========Bai_1207=========%%%
\begin{bt}
	Phân tích đa thức sau thành nhân tử: $4 x^2 - 36 x + 80$.
	\loigiai{ Ta có: $4 x^2 - 36 x + 80 = 4 \left(x - 5\right) \left(x - 4\right)$
	}
\end{bt}
%%%=========Bai_1208=========%%%
\begin{bt}
	Phân tích đa thức sau thành nhân tử: $4 x^2 - 40 x + 96$.
	\loigiai{ Ta có: $4 x^2 - 40 x + 96 = 4 \left(x - 6\right) \left(x - 4\right)$
	}
\end{bt}
%%%=========Bai_1209=========%%%
\begin{bt}
	Phân tích đa thức sau thành nhân tử: $4 x^2 - 44 x + 112$.
	\loigiai{ Ta có: $4 x^2 - 44 x + 112 = 4 \left(x - 7\right) \left(x - 4\right)$
	}
\end{bt}
%%%=========Bai_1210=========%%%
\begin{bt}
	Phân tích đa thức sau thành nhân tử: $4 x^2 - 48 x + 128$.
	\loigiai{ Ta có: $4 x^2 - 48 x + 128 = 4 \left(x - 8\right) \left(x - 4\right)$
	}
\end{bt}
%%%=========Bai_1211=========%%%
\begin{bt}
	Phân tích đa thức sau thành nhân tử: $4 x^2 - 52 x + 144$.
	\loigiai{ Ta có: $4 x^2 - 52 x + 144 = 4 \left(x - 9\right) \left(x - 4\right)$
	}
\end{bt}
%%%=========Bai_1212=========%%%
\begin{bt}
	Phân tích đa thức sau thành nhân tử: $4 x^2 + 20 x - 200$.
	\loigiai{ Ta có: $4 x^2 + 20 x - 200 = 4 \left(x - 5\right) \left(x + 10\right)$
	}
\end{bt}
%%%=========Bai_1213=========%%%
\begin{bt}
	Phân tích đa thức sau thành nhân tử: $4 x^2 + 16 x - 180$.
	\loigiai{ Ta có: $4 x^2 + 16 x - 180 = 4 \left(x - 5\right) \left(x + 9\right)$
	}
\end{bt}
%%%=========Bai_1214=========%%%
\begin{bt}
	Phân tích đa thức sau thành nhân tử: $4 x^2 + 12 x - 160$.
	\loigiai{ Ta có: $4 x^2 + 12 x - 160 = 4 \left(x - 5\right) \left(x + 8\right)$
	}
\end{bt}
%%%=========Bai_1215=========%%%
\begin{bt}
	Phân tích đa thức sau thành nhân tử: $4 x^2 + 8 x - 140$.
	\loigiai{ Ta có: $4 x^2 + 8 x - 140 = 4 \left(x - 5\right) \left(x + 7\right)$
	}
\end{bt}
%%%=========Bai_1216=========%%%
\begin{bt}
	Phân tích đa thức sau thành nhân tử: $4 x^2 + 4 x - 120$.
	\loigiai{ Ta có: $4 x^2 + 4 x - 120 = 4 \left(x - 5\right) \left(x + 6\right)$
	}
\end{bt}
%%%=========Bai_1217=========%%%
\begin{bt}
	Phân tích đa thức sau thành nhân tử: $4 x^2 - 4 x - 80$.
	\loigiai{ Ta có: $4 x^2 - 4 x - 80 = 4 \left(x - 5\right) \left(x + 4\right)$
	}
\end{bt}
%%%=========Bai_1218=========%%%
\begin{bt}
	Phân tích đa thức sau thành nhân tử: $4 x^2 - 8 x - 60$.
	\loigiai{ Ta có: $4 x^2 - 8 x - 60 = 4 \left(x - 5\right) \left(x + 3\right)$
	}
\end{bt}
%%%=========Bai_1219=========%%%
\begin{bt}
	Phân tích đa thức sau thành nhân tử: $4 x^2 - 12 x - 40$.
	\loigiai{ Ta có: $4 x^2 - 12 x - 40 = 4 \left(x - 5\right) \left(x + 2\right)$
	}
\end{bt}
%%%=========Bai_1220=========%%%
\begin{bt}
	Phân tích đa thức sau thành nhân tử: $4 x^2 - 16 x - 20$.
	\loigiai{ Ta có: $4 x^2 - 16 x - 20 = 4 \left(x - 5\right) \left(x + 1\right)$
	}
\end{bt}
%%%=========Bai_1221=========%%%
\begin{bt}
	Phân tích đa thức sau thành nhân tử: $4 x^2 - 24 x + 20$.
	\loigiai{ Ta có: $4 x^2 - 24 x + 20 = 4 \left(x - 5\right) \left(x - 1\right)$
	}
\end{bt}
%%%=========Bai_1222=========%%%
\begin{bt}
	Phân tích đa thức sau thành nhân tử: $4 x^2 - 28 x + 40$.
	\loigiai{ Ta có: $4 x^2 - 28 x + 40 = 4 \left(x - 5\right) \left(x - 2\right)$
	}
\end{bt}
%%%=========Bai_1223=========%%%
\begin{bt}
	Phân tích đa thức sau thành nhân tử: $4 x^2 - 32 x + 60$.
	\loigiai{ Ta có: $4 x^2 - 32 x + 60 = 4 \left(x - 5\right) \left(x - 3\right)$
	}
\end{bt}
%%%=========Bai_1224=========%%%
\begin{bt}
	Phân tích đa thức sau thành nhân tử: $4 x^2 - 36 x + 80$.
	\loigiai{ Ta có: $4 x^2 - 36 x + 80 = 4 \left(x - 5\right) \left(x - 4\right)$
	}
\end{bt}
%%%=========Bai_1225=========%%%
\begin{bt}
	Phân tích đa thức sau thành nhân tử: $4 x^2 - 44 x + 120$.
	\loigiai{ Ta có: $4 x^2 - 44 x + 120 = 4 \left(x - 6\right) \left(x - 5\right)$
	}
\end{bt}
%%%=========Bai_1226=========%%%
\begin{bt}
	Phân tích đa thức sau thành nhân tử: $4 x^2 - 48 x + 140$.
	\loigiai{ Ta có: $4 x^2 - 48 x + 140 = 4 \left(x - 7\right) \left(x - 5\right)$
	}
\end{bt}
%%%=========Bai_1227=========%%%
\begin{bt}
	Phân tích đa thức sau thành nhân tử: $4 x^2 - 52 x + 160$.
	\loigiai{ Ta có: $4 x^2 - 52 x + 160 = 4 \left(x - 8\right) \left(x - 5\right)$
	}
\end{bt}
%%%=========Bai_1228=========%%%
\begin{bt}
	Phân tích đa thức sau thành nhân tử: $4 x^2 - 56 x + 180$.
	\loigiai{ Ta có: $4 x^2 - 56 x + 180 = 4 \left(x - 9\right) \left(x - 5\right)$
	}
\end{bt}
%%%=========Bai_1229=========%%%
\begin{bt}
	Phân tích đa thức sau thành nhân tử: $4 x^2 + 16 x - 240$.
	\loigiai{ Ta có: $4 x^2 + 16 x - 240 = 4 \left(x - 6\right) \left(x + 10\right)$
	}
\end{bt}
%%%=========Bai_1230=========%%%
\begin{bt}
	Phân tích đa thức sau thành nhân tử: $4 x^2 + 12 x - 216$.
	\loigiai{ Ta có: $4 x^2 + 12 x - 216 = 4 \left(x - 6\right) \left(x + 9\right)$
	}
\end{bt}
%%%=========Bai_1231=========%%%
\begin{bt}
	Phân tích đa thức sau thành nhân tử: $4 x^2 + 8 x - 192$.
	\loigiai{ Ta có: $4 x^2 + 8 x - 192 = 4 \left(x - 6\right) \left(x + 8\right)$
	}
\end{bt}
%%%=========Bai_1232=========%%%
\begin{bt}
	Phân tích đa thức sau thành nhân tử: $4 x^2 + 4 x - 168$.
	\loigiai{ Ta có: $4 x^2 + 4 x - 168 = 4 \left(x - 6\right) \left(x + 7\right)$
	}
\end{bt}
%%%=========Bai_1233=========%%%
\begin{bt}
	Phân tích đa thức sau thành nhân tử: $4 x^2 - 4 x - 120$.
	\loigiai{ Ta có: $4 x^2 - 4 x - 120 = 4 \left(x - 6\right) \left(x + 5\right)$
	}
\end{bt}
%%%=========Bai_1234=========%%%
\begin{bt}
	Phân tích đa thức sau thành nhân tử: $4 x^2 - 8 x - 96$.
	\loigiai{ Ta có: $4 x^2 - 8 x - 96 = 4 \left(x - 6\right) \left(x + 4\right)$
	}
\end{bt}
%%%=========Bai_1235=========%%%
\begin{bt}
	Phân tích đa thức sau thành nhân tử: $4 x^2 - 12 x - 72$.
	\loigiai{ Ta có: $4 x^2 - 12 x - 72 = 4 \left(x - 6\right) \left(x + 3\right)$
	}
\end{bt}
%%%=========Bai_1236=========%%%
\begin{bt}
	Phân tích đa thức sau thành nhân tử: $4 x^2 - 16 x - 48$.
	\loigiai{ Ta có: $4 x^2 - 16 x - 48 = 4 \left(x - 6\right) \left(x + 2\right)$
	}
\end{bt}
%%%=========Bai_1237=========%%%
\begin{bt}
	Phân tích đa thức sau thành nhân tử: $4 x^2 - 20 x - 24$.
	\loigiai{ Ta có: $4 x^2 - 20 x - 24 = 4 \left(x - 6\right) \left(x + 1\right)$
	}
\end{bt}
%%%=========Bai_1238=========%%%
\begin{bt}
	Phân tích đa thức sau thành nhân tử: $4 x^2 - 28 x + 24$.
	\loigiai{ Ta có: $4 x^2 - 28 x + 24 = 4 \left(x - 6\right) \left(x - 1\right)$
	}
\end{bt}
%%%=========Bai_1239=========%%%
\begin{bt}
	Phân tích đa thức sau thành nhân tử: $4 x^2 - 32 x + 48$.
	\loigiai{ Ta có: $4 x^2 - 32 x + 48 = 4 \left(x - 6\right) \left(x - 2\right)$
	}
\end{bt}
%%%=========Bai_1240=========%%%
\begin{bt}
	Phân tích đa thức sau thành nhân tử: $4 x^2 - 36 x + 72$.
	\loigiai{ Ta có: $4 x^2 - 36 x + 72 = 4 \left(x - 6\right) \left(x - 3\right)$
	}
\end{bt}
%%%=========Bai_1241=========%%%
\begin{bt}
	Phân tích đa thức sau thành nhân tử: $4 x^2 - 40 x + 96$.
	\loigiai{ Ta có: $4 x^2 - 40 x + 96 = 4 \left(x - 6\right) \left(x - 4\right)$
	}
\end{bt}
%%%=========Bai_1242=========%%%
\begin{bt}
	Phân tích đa thức sau thành nhân tử: $4 x^2 - 44 x + 120$.
	\loigiai{ Ta có: $4 x^2 - 44 x + 120 = 4 \left(x - 6\right) \left(x - 5\right)$
	}
\end{bt}
%%%=========Bai_1243=========%%%
\begin{bt}
	Phân tích đa thức sau thành nhân tử: $4 x^2 - 52 x + 168$.
	\loigiai{ Ta có: $4 x^2 - 52 x + 168 = 4 \left(x - 7\right) \left(x - 6\right)$
	}
\end{bt}
%%%=========Bai_1244=========%%%
\begin{bt}
	Phân tích đa thức sau thành nhân tử: $4 x^2 - 56 x + 192$.
	\loigiai{ Ta có: $4 x^2 - 56 x + 192 = 4 \left(x - 8\right) \left(x - 6\right)$
	}
\end{bt}
%%%=========Bai_1245=========%%%
\begin{bt}
	Phân tích đa thức sau thành nhân tử: $4 x^2 - 60 x + 216$.
	\loigiai{ Ta có: $4 x^2 - 60 x + 216 = 4 \left(x - 9\right) \left(x - 6\right)$
	}
\end{bt}
%%%=========Bai_1246=========%%%
\begin{bt}
	Phân tích đa thức sau thành nhân tử: $4 x^2 + 12 x - 280$.
	\loigiai{ Ta có: $4 x^2 + 12 x - 280 = 4 \left(x - 7\right) \left(x + 10\right)$
	}
\end{bt}
%%%=========Bai_1247=========%%%
\begin{bt}
	Phân tích đa thức sau thành nhân tử: $4 x^2 + 8 x - 252$.
	\loigiai{ Ta có: $4 x^2 + 8 x - 252 = 4 \left(x - 7\right) \left(x + 9\right)$
	}
\end{bt}
%%%=========Bai_1248=========%%%
\begin{bt}
	Phân tích đa thức sau thành nhân tử: $4 x^2 + 4 x - 224$.
	\loigiai{ Ta có: $4 x^2 + 4 x - 224 = 4 \left(x - 7\right) \left(x + 8\right)$
	}
\end{bt}
%%%=========Bai_1249=========%%%
\begin{bt}
	Phân tích đa thức sau thành nhân tử: $4 x^2 - 4 x - 168$.
	\loigiai{ Ta có: $4 x^2 - 4 x - 168 = 4 \left(x - 7\right) \left(x + 6\right)$
	}
\end{bt}
%%%=========Bai_1250=========%%%
\begin{bt}
	Phân tích đa thức sau thành nhân tử: $4 x^2 - 8 x - 140$.
	\loigiai{ Ta có: $4 x^2 - 8 x - 140 = 4 \left(x - 7\right) \left(x + 5\right)$
	}
\end{bt}
%%%=========Bai_1251=========%%%
\begin{bt}
	Phân tích đa thức sau thành nhân tử: $4 x^2 - 12 x - 112$.
	\loigiai{ Ta có: $4 x^2 - 12 x - 112 = 4 \left(x - 7\right) \left(x + 4\right)$
	}
\end{bt}
%%%=========Bai_1252=========%%%
\begin{bt}
	Phân tích đa thức sau thành nhân tử: $4 x^2 - 16 x - 84$.
	\loigiai{ Ta có: $4 x^2 - 16 x - 84 = 4 \left(x - 7\right) \left(x + 3\right)$
	}
\end{bt}
%%%=========Bai_1253=========%%%
\begin{bt}
	Phân tích đa thức sau thành nhân tử: $4 x^2 - 20 x - 56$.
	\loigiai{ Ta có: $4 x^2 - 20 x - 56 = 4 \left(x - 7\right) \left(x + 2\right)$
	}
\end{bt}
%%%=========Bai_1254=========%%%
\begin{bt}
	Phân tích đa thức sau thành nhân tử: $4 x^2 - 24 x - 28$.
	\loigiai{ Ta có: $4 x^2 - 24 x - 28 = 4 \left(x - 7\right) \left(x + 1\right)$
	}
\end{bt}
%%%=========Bai_1255=========%%%
\begin{bt}
	Phân tích đa thức sau thành nhân tử: $4 x^2 - 32 x + 28$.
	\loigiai{ Ta có: $4 x^2 - 32 x + 28 = 4 \left(x - 7\right) \left(x - 1\right)$
	}
\end{bt}
%%%=========Bai_1256=========%%%
\begin{bt}
	Phân tích đa thức sau thành nhân tử: $4 x^2 - 36 x + 56$.
	\loigiai{ Ta có: $4 x^2 - 36 x + 56 = 4 \left(x - 7\right) \left(x - 2\right)$
	}
\end{bt}
%%%=========Bai_1257=========%%%
\begin{bt}
	Phân tích đa thức sau thành nhân tử: $4 x^2 - 40 x + 84$.
	\loigiai{ Ta có: $4 x^2 - 40 x + 84 = 4 \left(x - 7\right) \left(x - 3\right)$
	}
\end{bt}
%%%=========Bai_1258=========%%%
\begin{bt}
	Phân tích đa thức sau thành nhân tử: $4 x^2 - 44 x + 112$.
	\loigiai{ Ta có: $4 x^2 - 44 x + 112 = 4 \left(x - 7\right) \left(x - 4\right)$
	}
\end{bt}
%%%=========Bai_1259=========%%%
\begin{bt}
	Phân tích đa thức sau thành nhân tử: $4 x^2 - 48 x + 140$.
	\loigiai{ Ta có: $4 x^2 - 48 x + 140 = 4 \left(x - 7\right) \left(x - 5\right)$
	}
\end{bt}
%%%=========Bai_1260=========%%%
\begin{bt}
	Phân tích đa thức sau thành nhân tử: $4 x^2 - 52 x + 168$.
	\loigiai{ Ta có: $4 x^2 - 52 x + 168 = 4 \left(x - 7\right) \left(x - 6\right)$
	}
\end{bt}
%%%=========Bai_1261=========%%%
\begin{bt}
	Phân tích đa thức sau thành nhân tử: $4 x^2 - 60 x + 224$.
	\loigiai{ Ta có: $4 x^2 - 60 x + 224 = 4 \left(x - 8\right) \left(x - 7\right)$
	}
\end{bt}
%%%=========Bai_1262=========%%%
\begin{bt}
	Phân tích đa thức sau thành nhân tử: $4 x^2 - 64 x + 252$.
	\loigiai{ Ta có: $4 x^2 - 64 x + 252 = 4 \left(x - 9\right) \left(x - 7\right)$
	}
\end{bt}
%%%=========Bai_1263=========%%%
\begin{bt}
	Phân tích đa thức sau thành nhân tử: $4 x^2 + 8 x - 320$.
	\loigiai{ Ta có: $4 x^2 + 8 x - 320 = 4 \left(x - 8\right) \left(x + 10\right)$
	}
\end{bt}
%%%=========Bai_1264=========%%%
\begin{bt}
	Phân tích đa thức sau thành nhân tử: $4 x^2 + 4 x - 288$.
	\loigiai{ Ta có: $4 x^2 + 4 x - 288 = 4 \left(x - 8\right) \left(x + 9\right)$
	}
\end{bt}
%%%=========Bai_1265=========%%%
\begin{bt}
	Phân tích đa thức sau thành nhân tử: $4 x^2 - 4 x - 224$.
	\loigiai{ Ta có: $4 x^2 - 4 x - 224 = 4 \left(x - 8\right) \left(x + 7\right)$
	}
\end{bt}
%%%=========Bai_1266=========%%%
\begin{bt}
	Phân tích đa thức sau thành nhân tử: $4 x^2 - 8 x - 192$.
	\loigiai{ Ta có: $4 x^2 - 8 x - 192 = 4 \left(x - 8\right) \left(x + 6\right)$
	}
\end{bt}
%%%=========Bai_1267=========%%%
\begin{bt}
	Phân tích đa thức sau thành nhân tử: $4 x^2 - 12 x - 160$.
	\loigiai{ Ta có: $4 x^2 - 12 x - 160 = 4 \left(x - 8\right) \left(x + 5\right)$
	}
\end{bt}
%%%=========Bai_1268=========%%%
\begin{bt}
	Phân tích đa thức sau thành nhân tử: $4 x^2 - 16 x - 128$.
	\loigiai{ Ta có: $4 x^2 - 16 x - 128 = 4 \left(x - 8\right) \left(x + 4\right)$
	}
\end{bt}
%%%=========Bai_1269=========%%%
\begin{bt}
	Phân tích đa thức sau thành nhân tử: $4 x^2 - 20 x - 96$.
	\loigiai{ Ta có: $4 x^2 - 20 x - 96 = 4 \left(x - 8\right) \left(x + 3\right)$
	}
\end{bt}
%%%=========Bai_1270=========%%%
\begin{bt}
	Phân tích đa thức sau thành nhân tử: $4 x^2 - 24 x - 64$.
	\loigiai{ Ta có: $4 x^2 - 24 x - 64 = 4 \left(x - 8\right) \left(x + 2\right)$
	}
\end{bt}
%%%=========Bai_1271=========%%%
\begin{bt}
	Phân tích đa thức sau thành nhân tử: $4 x^2 - 28 x - 32$.
	\loigiai{ Ta có: $4 x^2 - 28 x - 32 = 4 \left(x - 8\right) \left(x + 1\right)$
	}
\end{bt}
%%%=========Bai_1272=========%%%
\begin{bt}
	Phân tích đa thức sau thành nhân tử: $4 x^2 - 36 x + 32$.
	\loigiai{ Ta có: $4 x^2 - 36 x + 32 = 4 \left(x - 8\right) \left(x - 1\right)$
	}
\end{bt}
%%%=========Bai_1273=========%%%
\begin{bt}
	Phân tích đa thức sau thành nhân tử: $4 x^2 - 40 x + 64$.
	\loigiai{ Ta có: $4 x^2 - 40 x + 64 = 4 \left(x - 8\right) \left(x - 2\right)$
	}
\end{bt}
%%%=========Bai_1274=========%%%
\begin{bt}
	Phân tích đa thức sau thành nhân tử: $4 x^2 - 44 x + 96$.
	\loigiai{ Ta có: $4 x^2 - 44 x + 96 = 4 \left(x - 8\right) \left(x - 3\right)$
	}
\end{bt}
%%%=========Bai_1275=========%%%
\begin{bt}
	Phân tích đa thức sau thành nhân tử: $4 x^2 - 48 x + 128$.
	\loigiai{ Ta có: $4 x^2 - 48 x + 128 = 4 \left(x - 8\right) \left(x - 4\right)$
	}
\end{bt}
%%%=========Bai_1276=========%%%
\begin{bt}
	Phân tích đa thức sau thành nhân tử: $4 x^2 - 52 x + 160$.
	\loigiai{ Ta có: $4 x^2 - 52 x + 160 = 4 \left(x - 8\right) \left(x - 5\right)$
	}
\end{bt}
%%%=========Bai_1277=========%%%
\begin{bt}
	Phân tích đa thức sau thành nhân tử: $4 x^2 - 56 x + 192$.
	\loigiai{ Ta có: $4 x^2 - 56 x + 192 = 4 \left(x - 8\right) \left(x - 6\right)$
	}
\end{bt}
%%%=========Bai_1278=========%%%
\begin{bt}
	Phân tích đa thức sau thành nhân tử: $4 x^2 - 60 x + 224$.
	\loigiai{ Ta có: $4 x^2 - 60 x + 224 = 4 \left(x - 8\right) \left(x - 7\right)$
	}
\end{bt}
%%%=========Bai_1279=========%%%
\begin{bt}
	Phân tích đa thức sau thành nhân tử: $4 x^2 - 68 x + 288$.
	\loigiai{ Ta có: $4 x^2 - 68 x + 288 = 4 \left(x - 9\right) \left(x - 8\right)$
	}
\end{bt}
%%%=========Bai_1280=========%%%
\begin{bt}
	Phân tích đa thức sau thành nhân tử: $4 x^2 + 4 x - 360$.
	\loigiai{ Ta có: $4 x^2 + 4 x - 360 = 4 \left(x - 9\right) \left(x + 10\right)$
	}
\end{bt}
%%%=========Bai_1281=========%%%
\begin{bt}
	Phân tích đa thức sau thành nhân tử: $4 x^2 - 4 x - 288$.
	\loigiai{ Ta có: $4 x^2 - 4 x - 288 = 4 \left(x - 9\right) \left(x + 8\right)$
	}
\end{bt}
%%%=========Bai_1282=========%%%
\begin{bt}
	Phân tích đa thức sau thành nhân tử: $4 x^2 - 8 x - 252$.
	\loigiai{ Ta có: $4 x^2 - 8 x - 252 = 4 \left(x - 9\right) \left(x + 7\right)$
	}
\end{bt}
%%%=========Bai_1283=========%%%
\begin{bt}
	Phân tích đa thức sau thành nhân tử: $4 x^2 - 12 x - 216$.
	\loigiai{ Ta có: $4 x^2 - 12 x - 216 = 4 \left(x - 9\right) \left(x + 6\right)$
	}
\end{bt}
%%%=========Bai_1284=========%%%
\begin{bt}
	Phân tích đa thức sau thành nhân tử: $4 x^2 - 16 x - 180$.
	\loigiai{ Ta có: $4 x^2 - 16 x - 180 = 4 \left(x - 9\right) \left(x + 5\right)$
	}
\end{bt}
%%%=========Bai_1285=========%%%
\begin{bt}
	Phân tích đa thức sau thành nhân tử: $4 x^2 - 20 x - 144$.
	\loigiai{ Ta có: $4 x^2 - 20 x - 144 = 4 \left(x - 9\right) \left(x + 4\right)$
	}
\end{bt}
%%%=========Bai_1286=========%%%
\begin{bt}
	Phân tích đa thức sau thành nhân tử: $4 x^2 - 24 x - 108$.
	\loigiai{ Ta có: $4 x^2 - 24 x - 108 = 4 \left(x - 9\right) \left(x + 3\right)$
	}
\end{bt}
%%%=========Bai_1287=========%%%
\begin{bt}
	Phân tích đa thức sau thành nhân tử: $4 x^2 - 28 x - 72$.
	\loigiai{ Ta có: $4 x^2 - 28 x - 72 = 4 \left(x - 9\right) \left(x + 2\right)$
	}
\end{bt}
%%%=========Bai_1288=========%%%
\begin{bt}
	Phân tích đa thức sau thành nhân tử: $4 x^2 - 32 x - 36$.
	\loigiai{ Ta có: $4 x^2 - 32 x - 36 = 4 \left(x - 9\right) \left(x + 1\right)$
	}
\end{bt}
%%%=========Bai_1289=========%%%
\begin{bt}
	Phân tích đa thức sau thành nhân tử: $4 x^2 - 40 x + 36$.
	\loigiai{ Ta có: $4 x^2 - 40 x + 36 = 4 \left(x - 9\right) \left(x - 1\right)$
	}
\end{bt}
%%%=========Bai_1290=========%%%
\begin{bt}
	Phân tích đa thức sau thành nhân tử: $4 x^2 - 44 x + 72$.
	\loigiai{ Ta có: $4 x^2 - 44 x + 72 = 4 \left(x - 9\right) \left(x - 2\right)$
	}
\end{bt}
%%%=========Bai_1291=========%%%
\begin{bt}
	Phân tích đa thức sau thành nhân tử: $4 x^2 - 48 x + 108$.
	\loigiai{ Ta có: $4 x^2 - 48 x + 108 = 4 \left(x - 9\right) \left(x - 3\right)$
	}
\end{bt}
%%%=========Bai_1292=========%%%
\begin{bt}
	Phân tích đa thức sau thành nhân tử: $4 x^2 - 52 x + 144$.
	\loigiai{ Ta có: $4 x^2 - 52 x + 144 = 4 \left(x - 9\right) \left(x - 4\right)$
	}
\end{bt}
%%%=========Bai_1293=========%%%
\begin{bt}
	Phân tích đa thức sau thành nhân tử: $4 x^2 - 56 x + 180$.
	\loigiai{ Ta có: $4 x^2 - 56 x + 180 = 4 \left(x - 9\right) \left(x - 5\right)$
	}
\end{bt}
%%%=========Bai_1294=========%%%
\begin{bt}
	Phân tích đa thức sau thành nhân tử: $4 x^2 - 60 x + 216$.
	\loigiai{ Ta có: $4 x^2 - 60 x + 216 = 4 \left(x - 9\right) \left(x - 6\right)$
	}
\end{bt}
%%%=========Bai_1295=========%%%
\begin{bt}
	Phân tích đa thức sau thành nhân tử: $4 x^2 - 64 x + 252$.
	\loigiai{ Ta có: $4 x^2 - 64 x + 252 = 4 \left(x - 9\right) \left(x - 7\right)$
	}
\end{bt}
%%%=========Bai_1296=========%%%
\begin{bt}
	Phân tích đa thức sau thành nhân tử: $4 x^2 - 68 x + 288$.
	\loigiai{ Ta có: $4 x^2 - 68 x + 288 = 4 \left(x - 9\right) \left(x - 8\right)$
	}
\end{bt}