\begin{name}[Kiểm tra cuối kì II][Hóa][9][Sở Giáo dục và Đào tạo]{Trường THCS}{2023 - 2024}
\end{name}
\tieumuc{Trắc nghiệm (5 điểm)}
\Opensolutionfile{ans}[Ans/DATN-9CKII-De-03]
%\luuloigiaiex
%\hienthiloigiaiex
%\taoNdongke[5]{ex}
%\tatloigiaiex
\Opensolutionfile{ansex}[LOIGIAITN/LGTN-9CKII-De-03]
%%%============EX_1==============%%%
\begin{ex}
	Đun nóng tinh bột trong dung dịch axit vô cơ loãng sẽ thu được
	\choice
	{xenlulozơ}
	{\True glucozơ}
	{glixerol}
	{etyl axetat}
	\loigiai{Tinh bột khi đun nóng trong dung dịch axit vô cơ loãng bị thủy phân tạo ra glucozo}
\end{ex}
%%%============EX_2==============%%%
\begin{ex}
	Quả chuối xanh có chứa chất $X$ làm iot chuyển thành màu xanh tím. Chất $X$ là:
	\choice
	{\True Tinh bột}
	{Xenlulozơ}
	{Fructozơ}
	{Glucozơ}
	\loigiai{Tinh bột phản ứng màu với dung dịch iot tạo dung dịch xanh tím}
\end{ex}
%%%============EX_3==============%%%
\begin{ex}
	Chỉ ra điều \textbf{sai} khi nói về polime:
	\choice
	{Có phân tử khối lớn}
	{Phân tử do nhiều mắt xích tạo nên}
	{\True Có nhiêt đô nóng chảy và nhiêt đô sôi xác định}
	{Không tan trong nước và các dung môi thông thường}
	\loigiai{polime có nhiệt độ sôi và nhiệt độ nóng chảy không xác định}
\end{ex}
%%%============EX_4==============%%%
\begin{ex}
	Trứng là loại thực phẩm chứa nhiều
	\choice
	{chất béo}
	{chất đường}
	{chất bột}
	{\True protein}
	\loigiai{trứng là một trong những loại thực phẩm chứa nhiều protein}
\end{ex}
%%%============EX_5==============%%%
\begin{ex}
	Chất nào không thủy phân?
	\choice
	{Tinh bột}
	{Protein}
	{Saccarozơ}
	{\True Glucozơ}
	\loigiai{Trong các hợp chất cacbohidrat thì monosaccarit không tham gia phản ứng thủy phân}
\end{ex}
%%%============EX_6==============%%%
\begin{ex}
	Trong các phát biểu dưới đây, phát biểu đúng là:
	\choice
	{polime là chất dễ bay hơi}
	{polime là những chất dễ tan trong nước}
	{polime chỉ được tạo ra bởi con người và không có trong tự nhiên}
	{\True polime là những chất rắn, không bay hơi, thường không tan trong nước}
	\loigiai{polime là những chất rắn, không bay hơi, thường không tan trong nước}
\end{ex}
%%%============EX_7==============%%%
\begin{ex}
	Cho 10 ml ancol etylic 46 $^o$ phản ứng hết với kim loại Na (dư), thu được V lít khí H $_2$ (đktc). Biết khối lượng riêng của ancol etylic nguyên chất bằng 0,8 g/ml. Giá trị của V là
	\choice
	{\True 4,256}
	{0,896}
	{3,360}
	{2,128}
	\loigiai{}
\end{ex}
%%%============EX_8==============%%%
\begin{ex}
	Cho dung dịch axit axetic có nồng độ x\% tác dụng vừa đủ với dung dịch NaOH 10\%, thu được dung dịch muối có nồng độ 10,25\%. Giá trị của x là
	\choice
	{$20$}
	{$16$}
	{\True $15$}
	{$13$}
	\loigiai{}
\end{ex}
%%%============EX_9==============%%%
\begin{ex} Xà phòng hóa hoàn toàn 17,8 gam chất béo X cần vừa đủ dung dịch chứa 0,06 mol NaOH. Cô cạn dung dịch sau phản ứng thu được m gam muối khan. Giá trị của m là
	\choice
	{19,12}
	{\True 18,36}
	{19,04}
	{14,68}
	\loigiai{}
\end{ex}
%%%============EX_10==============%%%
\begin{ex}
	Sử dụng 1 tấn khoai (chứa 20\% tinh bột) để điều chế glucozơ. Tính khối lượng glucozơ thu được, biết hiệu suất phản ứng đạt 70\%.
	\choice
	{162 kg}
	{\True 155,56 kg}
	{143,33 kg}
	{133,33 kg}
	\loigiai{}
\end{ex}
%%%============EX_11==============%%%
\begin{ex}
	Trong số các cách chữa cháy sau, có mấy cách chữa cháy do xăng dầu gây ra?
	(1) Phun nước vào ngọn lửa;
	(2) Dùng chăn ướt trùm lên ngọn lửa;
	(3) Phủ cát vào ngọn lửa;
	(4) Dùng bình chữa cháy.
	\choice
	{$1$}
	{$2$}
	{\True $3$}
	{$4$}
	\loigiai{}
\end{ex}
%%%============EX_12==============%%%
\begin{ex}
	Phương pháp nào sau đây nhằm thu được khí metan tinh khiết từ hỗn hợp khí metan và khí cacbonic?
	\choice
	{\True Dẫn hỗn hợp qua dung dịch nước vôi trong dư}
	{Đốt cháy hỗn hợp rồi dẫn qua nước vôi trong}
	{Dẫn hỗn hợp qua bình đựng dung dịch $H_2SO_4$}
	{Dẫn hỗn hợp qua bình đựng nước brom dư}
	\loigiai{}
\end{ex}
%%%============EX_13==============%%%
\begin{ex}
	Trong điều kiện nhiệt độ, áp suất không đổi thì khí etilen phản ứng với khí oxi theo tỉ lệ tích là
	\choice
	{\True 1 lít khí $C_2H_4$ phản ứng với 3 lít khí $O_2$}
	{1 lít khí $C_2 H_4$ phản ứng với 2 lít khí $O_2$}
	{2 lít khí $C_2H_4$ phản ứng với 2 lít khí $O_2$}
	{2 lít khí $C_2H_4$ phản ứng với 3 lít khí $O_2$}
	\loigiai{}
\end{ex}
%%%============EX_14==============%%%
\begin{ex}
	Đốt cháy hoàn toàn một hiđrocacbon, thu được số mol  $H_2O$  gấp đôi số mol  $CO_2$. Công thức phân tử hiđrocacbon đó là
	\choice
	{ $C_2H_4$}
	{ $C_2H_6$}
	{\True  $CH_4$}
	{ $C_2H_2$}
	\loigiai{}
\end{ex}
%%%============EX_15==============%%%
\begin{ex}
	Đốt cháy chất nào sau đây cho số mol CO $_2$ bằng số mol H $_2$ O?
	\choice
	{$CH_4$}
	{\True $C_2H_4$}
	{$C_2H_2$}
	{$C_6H_6$}
	\loigiai{}
\end{ex}
%%%============EX_16==============%%%
\begin{ex}
	Hợp chất hữu cơ là chất khí ít tan trong nước, làm mất màu dung dịch brom, đốt cháy hoàn toàn 1 mol khí này sinh ra khí cacbonic và 1 mol hơi nước. Hợp chất đó là
	\choice
	{metan}
	{etilen}
	{\True axetilen}
	{benzen}
	\loigiai{}
\end{ex}
%%%============EX_17==============%%%
\begin{ex}
	Nguyên tố X có số hiệu nguyên tử là 11, chu kỳ 3, nhóm I trong bảng tuần hoàn các nguyên tố hóa học. Phát biểu nào sau đây đúng?
	\choice
	{\True Điện tích hạt nhân 11+, 3 lớp electron, lớp ngoài cùng có 1 electron}
	{Điện tích hạt nhân 11+, 1 lớp electron, lớp ngoài cùng có 3 electron}
	{Điện tích hạt nhân 11+, 3 lớp electron, lớp ngoài cùng có 3 electron}
	{Điện tích hạt nhân 11+, 1 lớp electron, lớp ngoài cùng có 1 electron}
	\loigiai{}
\end{ex}
%%%============EX_18==============%%%
\begin{ex}
	Số thứ tự nhóm A trong bảng hệ thống tuần hoàn cho biết:
	\choice
	{\True Số electron lớp ngoài cùng}
	{Số lớp electron}
	{Số hiệu nguyên tử}
	{Số thứ tự của nguyên tố}
	\loigiai{}
\end{ex}
%%%============EX_19==============%%%
\begin{ex}
	Đun nóng một hỗn hợp gồm 5,6 gam sắt và 6,4 gam bột lưu huỳnh trong ống kín. Sau khi phản ứng xảy ra hoàn toàn thu được m gam chất rắn. Giá trị của m là
	\choice
	{8,8}
	{6,0}
	{\True 12,0}
	{17,6}
	\loigiai{}
\end{ex}
%%%============EX_20==============%%%
\begin{ex}
	Cho $2{,}24$ lít khí CO (đktc) phản ứng vừa đủ với $10$ gam hỗn hợp X gồm CuO và MgO. Phần trăm khối lượng của MgO trong X là
	\choice
	{\True $20\%$}
	{$40\%$}
	{$60\%$}
	{$80\%$}
	\loigiai{}
\end{ex}
\Closesolutionfile{ansex}
\Closesolutionfile{ans}

%%%==========Phần trắc nghiệm đúng sai============%%%

%\tieumuc{Trắc Nghiệm Đúng Sai}-- \textit{Trong mỗi câu có 4 ý tương ứng A, B, C, D; Học sinh chọn đúng hoặc sai.}
%\Opensolutionfile{ans}[Ans/DATAM1]
%\Opensolutionfile{ansbook}[Ans/DATNTF-1]
%\luulgEXTF
%\Opensolutionfile{ansex}[LOIGIAITN/LGTNTF-1]
%Soạn ở đây
%\Closesolutionfile{ansex}
%\Closesolutionfile{ansbook}
%\Closesolutionfile{ans}	

%%%==========Phần tự luận============%%%
\tieumuc{Tự Luận (5 điểm)}
\Opensolutionfile{ansbt}[LOIGIAITL/LGTL-9CKII-De-03]
%\luuloigiaibt
%\hienthiloigiaibt
%\taoNdongke[15]{bt}
%%%==============BT_1==============%%%
\begin{bt}[$1{,}5$ điểm]
	Nêu hiện tượng và viết PTHH xảy ra (nếu có) trong các thí nghiệm sau:
	\begin{enumerate}
		\item Bỏ viên đá vôi vào dung dịch axit axetic.
		\item Cho glucozơ vào dung dịch $\mathrm{AgNO}_3$ trong $NH_3$ dư, đun nóng.
		\item Nhỏ vài giọt iot lên mặt cắt củ khoai lang.
	\end{enumerate}
	\loigiai{}
\end{bt}
%%%==============BT_2==============%%%
\begin{bt}[$1{,}5$ điểm]
	Viết phương trình thực hiện chuỗi chuyển hóa sau:
	\begin{center}
		$CaC_2$ \MuiTen[(1)] $C_2H_2$ \MuiTen[(2)] $C_2H_4$ \MuiTen[(3)] $C_2H_5OH$
	\end{center}
	\loigiai{}
\end{bt}
%\taoNdongke[12]{bt}
%%%==============BT_3==============%%%
\begin{bt}[2,0 điểm]
	\begin{enumerate}
		\item Đốt cháy hoàn toàn 6,4 gam chất hữu cơ $X(C, H, O)$, thu được 4,48 lít khí(đktc) $CO_2$ và 7,2 gam $H_2O$. Biết $X$ có phân tử khối là 32. Xác định công thức phân tử của $X$.
		\item Trên nhãn một bình cồn có ghi thông tin 1 lít dung dịch cồn $70^\circ$. Xác định thể tich và khối lượng rượu etylic (ethanol) nguyên chất có trong bình cồn đó. Biết khối lượng riêng của rượu etylic nguyên chất là $0,8\mathrm{~g} / \mathrm{ml}$.
	\end{enumerate}
	\loigiai{}
\end{bt}

\Closesolutionfile{ansbt}

\fileend

\begin{center}
	\rule[4pt]{2cm}{1pt}\large \textbf{HẾT}\rule[4pt]{2cm}{1pt}
\end{center}