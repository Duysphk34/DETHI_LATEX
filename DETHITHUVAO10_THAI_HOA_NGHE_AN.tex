\documentclass[FileMain.tex]{subfiles}
\gdef\sophong{Sở Giáo dục và đào tạo Nghệ An}
\gdef\truong{Trường THCS Thái Hòa}
\gdef\thoigian{90 phút}
\gdef\lop{9}
\gdef\mon{Toán học}
\gdef\made{208}
\setcounter{demsode}{6}
\begin{document}
%\begin{name}[Đề thi thử lần 1 tuyển sinh vào 10][Toán][9][UBND TP NAM ĐỊNH]{Phòng giáo dục và đào tạo}{2024 - 2025}\end{name}
\section[Thi thử lần 1 Thái Hòa  - Nghệ AN  - Mã đề \made]{Đề thi thử tuyển sinh vào 10 lần 1}
%	\subsection{Trắc nghiệm khách quan ( 2,0 điểm)}
%	{\itshape Chọn chữ cái đứng trước câu trả lời đúng và ghi vào tờ giấy làm bài.}
%	\Opensolutionfile{ans}[Ans/Ans-Thithuvao10-Nam_Dinh-Lan-1]
%	\luuloigiaiex
%	\Opensolutionfile{ansex}[LOIGIAITN/LGTN-Thithuvao10-Nam_Dinh-Lan-1]
%	\Writetofile{ansex}{\protect\thongtin{Lời giải chiết phần trắc nghiệm}}
%	%%%==============Cau_1==============%%%
%	\begin{ex}
%		Điều kiện xác định của biểu thức $\sqrt{\dfrac{-1}{x-2}}$ là
%		\choice
%		{$x > 2$}
%		{$x \neq 2$}
%		{\True $x < 2$}
%		{$x \leq 2$}
%		\loigiai{
%			Điều kiện xác định của $\sqrt{\dfrac{-1}{x-2}}$ là $x-2<0$ $\Leftrightarrow$ $x<2$
%		}
%	\end{ex}
%	%%%==============HetCau_1==============%%%
%	
%	%%%==============Cau_2==============%%%
%	\begin{ex}
%		Rút gọn biểu thức $\sqrt{(a-1)^2}-a$ với $a < 0$ được kết quả bằng
%		\choice
%		{\True $1-2a$}
%		{$1$}
%		{$1-a$}
%		{$2a-1$}
%		\loigiai{
%			Ta có $\sqrt{(a-1)^2}-a = |a-1|-a = -(a-1) -a =1-2a$  
%		}
%	\end{ex}
%	%%%==============HetCau_2==============%%%
%	
%	%%%==============Cau_3==============%%%
%	\begin{ex}
%		Khi đồ thị hàm số $y=x+1-m$ cắt trục hoành tạo điểm có hoành độ $x=2$ thì giá trị của tham số $m$ bằng
%		\choice
%		{$m=1$}
%		{\True $m=3$}
%		{$m=2$}
%		{$m=-1$}
%		\loigiai{
%			Vì đồ thị hàm số $y=x+1-m$ cắt trục hoành tạo điểm có hoành độ $x=2$ nên ta có $2+1-m =0 $ $\Leftrightarrow$ $m=3$
%		}
%	\end{ex}
%	%%%==============HetCau_3==============%%%
%	
%	%%%==============Cau_4==============%%%
%	\begin{ex}
%		Biết phương trình $x^2-2x-1=0$ có hai nghiệm $x_1; x_2$. Giá trị của biểu thức $x_1^2+x_2^2$ bằng
%		\choice
%		{$4$}
%		{$-2$}
%		{\True $6$}
%		{$2$}
%		\loigiai{
%			Phương trình $x^2-2x-1=0$ có hai nghiệm $x_1; x_2$ nên theo hệ thức Vi-et ta có $\heva{x_1+x_2 &=2\\x_1\cdot x_2&=-1}$\\
%			Ta có $x_1^2 +x_2^2 =\left( x_1+x_2\right)^2-2x_1\cdot x_2 =2^2-2\cdot (-1) =6$
%		}
%	\end{ex}
%	%%%==============HetCau_4==============%%%
%	
%	%%%==============Cau_5==============%%%
%	\begin{ex}
%		Một hình chữ nhật có chiều dài gấp đôi chiều rộng. Nếu giảm chiều dài $5m$ và tăng chiều rộng $5m$ thì được một hình vuông. Chu vi của hình chữ nhật ban đầu là
%		\choice
%		{$30m$}
%		{$45m$}
%		{$50m$}
%		{\True $60m$}
%		\loigiai{
%			Gọi $x$ (m) là chiều rộng hình chữ nhật lúc đầu $\left(x>0\right)$.
%			\\
%			Chiều dài hình chữ nhật lúc đầu là $2x$ (m).
%			\\
%			Theo đề bài ta có phương trình $2x-5=x+5$ $\Leftrightarrow$ $x=10$ (m).\\Vậy chu vi hình chữ nhật là $\left(x+2x\right)\cdot 2 = \left(10+20\right)\cdot 2 =60$ (m)
%		}
%	\end{ex}
%	%%%==============HetCau_5==============%%%
%	
%	%%%==============Cau_6==============%%%
%	\begin{ex}
%		Cho hai đường tròn $(O; 3\mathrm{~cm})$ và $\left(O^{\prime}; 5\mathrm{~cm}\right)$ có đoạn nối tâm $OO^{\prime}=7\mathrm{~cm}$. Vị trí tương đối của hai đường tròn là
%		\choice
%		{\True cắt nhau}
%		{tiếp xúc trong}
%		{không giao nhau}
%		{tiếp xúc ngoài}
%		\loigiai{
%			Ta có $|5-3|=2<7<8=5+3$ hay $|R-r|<OO^\prime<R+r$.
%			\\
%			Vậy hai đường tròn cắt nhau.
%		}
%	\end{ex}
%	%%%==============HetCau_6==============%%%
%	
%	%%%==============Cau_7==============%%%
%	\begin{ex}
%		Cho đường tròn $(O; 1\mathrm{~cm})$. Từ điểm $A$ nằm ngoài đường tròn $(O; 1\mathrm{~cm})$ sao cho $OA=2\mathrm{~cm}$ kẻ hai tiếp tuyến $AB, AC$ đến đường tròn $(B, C\in(O; 1\mathrm{~cm}))$. Độ dài cung $BC$ lớn bằng
%		\choice
%		{$\dfrac{2}{3} \mathrm{~cm}$}
%		{$\dfrac{2\pi}{3} \mathrm{~cm}$}
%		{\True $\dfrac{4\pi}{3} \mathrm{~cm}$}
%		{$\dfrac{4}{3} \mathrm{~cm}$}
%		\loigiai{
%			\immini{
%				Vì $AB$ là tiếp tuyến của $(O)$ nên $\triangle ABO $ vuông tại $O$.
%				$\Rightarrow$ $\cos AOB =\dfrac{BO}{AO}= \dfrac{1}{2}$ $\Rightarrow AOB = 60^\circ $
%				\\
%				Theo tính chất hai tiếp tuyến cắt nhau ta có $OA$ là tia phân giác $BOC$ 
%				\\
%				$\Rightarrow$ $BOC=2AOB=120^\circ$ $\Rightarrow \text{sđ}\overarc{BC}=120^\circ $ 
%				\\
%				Độ dài cung $BC$ $l_{BC} =\dfrac{120}{360}\cdot2\pi\cdot1 = $
%			}{}
%		}
%	\end{ex}
%	%%%==============HetCau_7==============%%%
%	
%	%%%==============Cau_8==============%%%
%	\begin{ex}
%		Quay tam giác $ABC$ vuông tại $A$ có $BC=10\mathrm{~cm}; AC=6\mathrm{~cm}$ quanh cạnh $AB$ cố định được hình nón. Thể tích của hình nón đó bằng
%		\choice
%		{\True $96\pi \mathrm{cm}^3$}
%		{$128\pi \mathrm{cm}^3$}
%		{$200\pi \mathrm{cm}^3$}
%		{$218\pi \mathrm{cm}^3$}
%		\loigiai{}
%	\end{ex}
%	%%%==============HetCau_8==============%%%
%	\Closesolutionfile{ansex}
%	\Closesolutionfile{ans}
%	\subsection{Tự luận (8,0 điểm)}{\itshape \vphantom{Học sinh làm bài vào giấy thi}}
\setcounter{bt}{0}
\luuloigiaibt
\Opensolutionfile{ansbth}[Ans/AnsBT-Thithuvao10-Thai-Hoa-Nghe-An-Lan-1]
%%%==============Bai_BT1==============%%%
\begin{bt}[2,5 điểm]
	\begin{enumerate}
		\item Tính giá trị của biểu thức $A=\sqrt{32}+\sqrt{(3-2\sqrt{2})^2}-\sqrt{8}$.
		\item Rút gọn biểu thức $B=\left(\dfrac{\sqrt{x}}{x-16}+\dfrac{1}{\sqrt{x}-4}\right): \dfrac{\sqrt{x}+2}{\sqrt{x}-4}$ với $x \geq 0$ và $x \neq 16$.
		\item Tìm các giá trị của $m$ để hai đường thẳng $y=2x+m^2$ và $y=(m-1) x+2m+3(m \neq 1)$ cắt nhau tại một điểm trên trục tung.
	\end{enumerate}
	\loigiai{}
\end{bt}
%%%==============HetBai_BT1==============%%%

%%%==============Bai_BT2==============%%%
\begin{bt}[2,0 điểm]
	\begin{enumerate}
		\item Giải phương trình $x^2-4x-1=0$.
		\item Cho phương trình $x^2-4x+2=0$ có hai nghiệm dương $x_1, x_2$ thoả mãn $x_1 > x_2$. Không giải phương trình, hãy tính giá trị của biểu thức $P=\dfrac{1}{x_1^2}-\dfrac{1}{x_2^2}+2024$.
	\end{enumerate}
	\loigiai{}
\end{bt}
%%%==============HetBai_BT2==============%%%

%%%==============Bai_BT3==============%%%
\begin{bt}[2,0 điểm]
	\begin{enumerate}
		\item Một phòng họp có 320 ghế ngồi (loại ghế một chỗ ngồi) được xếp thành nhiều hàng ghế và số lượng ghế ở mỗi hàng là như nhau. Người ta tổ chức một buổi hội thảo dành cho 429 người tại phòng họp đó nên phải xếp thêm 1 hàng ghế và mỗi hàng ghế phải xếp nhiều hơn số lượng ban đầu 3 ghế. Hỏi lúc đầu phòng họp đó có bao nhiêu hàng ghế.
		\immini{\item Người ta muốn làm một cái xô đựng nước có dạng hình nón cụt, có các kích thước cho ở hình vẽ bên, hãy tính diện tích tôn cần dùng để làm cái xô đó (cho biết phần mép nối không đáng kể và lấy $\pi \approx 3,14$).}{\begin{tikzpicture}[declare function={rm=0.45;rh=2;R=3.5;},font=\scriptsize\sffamily,line join=round,line cap=round]
			\path (0,0) coordinate (O)
			(-rh,0) coordinate (A)
			(rh,0) coordinate (B)
			($(O)+(-90:0.9*R)$) coordinate (O')
			($(O')+(-rh*0.5,0)$) coordinate (A')
			($(O')+(rh*0.5,0)$) coordinate (B')
			;
			\path(B')--(B) node[pos=0.5,right,font=\color{red}\scriptsize] {$30$ cm}
			;
			\draw (O) ellipse ({rh} and {rm});
			\draw [dashed](O') ellipse ({(rh*0.5)} and {(rm*0.5)});
			\draw (A') arc (180:360: {rh*0.5} and {(rm*0.5)});
			\draw[ultra thick] (A)..controls ++(93:R) and ++(87:R)..(B);
			\draw (A)--(A') (B)--(B');
			\draw[red] (O) circle (1pt)--(B) circle (1pt) node [pos=0.5,above=0pt]{$20$ cm};
			\draw[red] (O') circle (1pt)--(B') circle (1pt) node [pos=0.5,above=-2pt]{$10$ cm};
		\end{tikzpicture}}
	\end{enumerate}
	\loigiai{}
\end{bt}
%%%==============HetBai_BT3==============%%%

%%%==============Bai_BT4==============%%%
\begin{bt}[3,0 điểm]
	Cho đường tròn $(O; R)$ có hai đường kính $AB$ và $CD$ vuông góc với nhau. Trên cung nhỏ $AC$ lấy điểm $M$ bất kì ($M$ khác $A$ và $C), BM$ cắt $OC$ tại điểm $E$ và $DM$ cắt $OA$ tại điểm $F$.
	\begin{enumerate}
		\item Chứng minh $OFMC$ là tứ giác nội tiếp.
		\item Gọi $K$ là giao điểm của hai đường thẳng $CM$ và $AB$. Chứng minh $\triangle CMF\sim \triangle EMA$ và $KF. OA=AF. KB$.
		\item Xác định vị trí của điểm $M$ trên cung nhỏ $AC$ để tổng $\dfrac{OA}{AF}+\dfrac{OC}{CE}$ đạt giá trị nhỏ nhất.
	\end{enumerate}
	\loigiai{}
\end{bt}
%%%==============HetBai_BT4==============%%%

%%%==============Bai_BT5==============%%%
\begin{bt}[0,5 điểm]
	Giải phương trình $(x+1) \sqrt{x+2}+(x+6) \sqrt{x+7}=x^2+7x+12$.
	\loigiai{}
\end{bt}
%%%==============HetBai_BT5==============%%%
\Closesolutionfile{ansbth}
%\input{Ans/AnsBT-Thithuvao10-MyLoi-De-7}
\fileend
\end{document}