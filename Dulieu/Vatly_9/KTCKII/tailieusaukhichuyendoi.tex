
%%%===========Bài tập dạng 1 Điện từ học=============%%%
%%%=============EX_1=============%%%

%%%=============EX_2=============%%%

%%%=============EX_3=============%%%
\begin{ex}
	Trường hợp nào dưới đây thì trong cuộn dây dẫn kín xuất hiện dòng điện cảm ứng xoay chiều?
	\choice
	{\True Cho nam châm chuyển động lại gần cuộn dây}
	{Cho cuộn dây quay trong từ trường của nam châm và cắt các đường sức từ}
	{Đặt thanh nam châm vào trong lòng cuộn dây rồi cho cả hai quay đều quanh một trục với cùng tốc độ}
	{Đặt một thanh nam châm hình trụ trước một cuộn dây, vuông góc với tiết diện cuộn dây rồi cho thanh nam châm quay quanh trục của nó}
	\loigiai{}
\end{ex}
%%%=============EX_4=============%%%

%%%=============EX_5=============%%%

%%%=============EX_6=============%%%
\begin{ex}
	Trường hợp nào sau đây, trong cuộn dây dẫn kín xuất hiện dòng điện cảm ứng?
	\choice
	{Số đường sức từ qua tiết diện S của cuộn dây dẫn kín lớn}
	{Số đường sức từ qua tiết diện S của cuộn dây dẫn kín được giữ không thay đổi}
	{\True Số đường sức từ qua tiết diện S của cuộn dây dẫn kín thay đổi}
	{Từ trường xuyên qua tiết diện S của cuộn dây dẫn kín mạnh}
	\loigiai{}
\end{ex}
%%%=============EX_7=============%%%
\begin{ex}
	Cách làm nào dưới đây không tạo ra được dòng điện cảm ứng trong một cuộn dây kín?
	\choice
	{\True Cho cuộn dây dẫn chuyển động theo phương song song với các đường sức từ ở giữa hai nhánh của nam châm chữ U}
	{Cho cuộn dây quay cắt các đường sức từ của nam châm chữ U}
	{Cho một đầu của nam châm điện chuyển động lại gần một đầu của cuộn dây dẫn}
	{Đặt nam châm điện ở trước đầu cuộn dây rồi ngắt mạch điện của nam châm}
	\loigiai{}
\end{ex}
%%%=============EX_8=============%%%
\begin{ex}
	Trong cuộn dây dẫn kín xuất hiện dòng điện cảm ứng xoay chiều khi số đường sức từ xuyên qua tiết diện S của cuộn dây
	\choice
	{luôn luôn tăng}
	{luôn luôn giảm}
	{\True luân phiên tăng, giảm}
	{luôn luôn không đổi}
	\loigiai{}
\end{ex}
%%%=============EX_9=============%%%
\begin{ex}
	Máy phát điện xoay chiều bắt buộc phải gồm các bộ phận chính nào để có thể tạo ra dòng điện?
	\choice
	{Nam châm vĩnh cửu và sợi dây dẫn nối hai cực nam châm}
	{Nam châm điện và sợi dây dẫn nối nam châm với đèn}
	{\True Cuộn dây dẫn và nam châm}
	{Cuộn dây dẫn và lõi sắt}
	\loigiai{}
\end{ex}
%%%=============EX_10=============%%%
\begin{ex}
	Dòng điện xoay chiều có cường độ và hiệu điện thế luôn thay đổi theo thời gian. Vậy ampe kế xoay chiều chỉ giá trị nào của cường độ dòng điện xoay chiều?
	\choice
	{Giá trị cực đại}
	{Giá trị cực tiểu}
	{Giá trị trung bình}
	{\True Giá trị hiệu dụng}
	\loigiai{}
\end{ex}
%%%=============EX_11=============%%%

%%%=============EX_12=============%%%

%%%=============EX_13=============%%%
\begin{ex}
	Trên hình vẽ, thanh nam châm chuyển động như thế nào thì không tạo ra dòng điện cảm ứng trong cuộn dây?
	\choice
	{Chuyển động từ ngoài vào trong cuộn dây}
	{Quay quanh trục AB}
	{Quay quanh trục C}
	{\True D. Quay quanh trục PQ}
	\loigiai{}
\end{ex}

%%%==================Bài tập về máy biến thế =======================%%%

%%%=============EX_1=============%%%

%%%=============EX_2=============%%%

%%%=============EX_3=============%%%

%%%=============EX_4=============%%%

%%%=============EX_5=============%%%


%%%==============Bài tập phần truyền tải điện năng=======================%%%
%%%=============EX_1=============%%%

%%%=============EX_2=============%%%

%%%=============EX_3=============%%%

%%%=============EX_4=============%%%

%%%=============EX_5=============%%%



