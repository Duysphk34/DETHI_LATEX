%%%Tùy chọn 1: Kì thi
%%%Tùy chọn 2: Môn
%%%Tùy chọn 3: lớp
%%%Tùy chọn 4: Sở/Phòng
%%%Tùy chọn 5: Ngày thi
\begin{name}[Kiểm tra cuối kì II][Vật lý][9][Sở Giáo dục và Đào tạo]{Trường THCS}{2023 - 2024}
\end{name}
\tieumuc{Trắc nghiệm (4 điểm)}
\Opensolutionfile{ans}[Ans/DATN-VL9CKII-De-04]
%\luuloigiaiex
\hienthiloigiaiex
%\taoNdongke[4]{ex}
%\tatloigiaiex
\Opensolutionfile{ansex}[LOIGIAITN/LGTN-VL9CKII-De-04]

%%%=============EX_1=============%%%
\begin{ex}
	Chiếu ánh sáng trắng qua một tấm kính màu xanh thì ta được ánh sáng
	\choice
	{đỏ}
	{\True xanh}
	{đen}
	{trắng}
	\loigiai{}
\end{ex}

%%%=============EX_2=============%%%
\begin{ex}
	Từ nhà máy thủy điện người ta truyền đi một công suất 10 MW dưới hiệu điện thế 500 kV. Công suất hao phí trên đường dây bằng 200 kW. Điện trở của đường dây truyền tải bằng
	\choice
	{\True 500 $\Omega$}
	{100 $\Omega$}
	{250 $\Omega$}
	{200 $\Omega$}
	\loigiai{}
\end{ex}
%%%=============EX_3=============%%%
\begin{ex}
	Chỉ ra câu sai?
	Có thể thu được ánh sáng đỏ nếu
	\choice
	{thắp sáng một đèn LED đỏ}
	{chiếu một chùm sáng trắng qua một tấm lọc màu đỏ}
	{chiếu một chùm sáng đỏ qua một tấm lọc màu đỏ}
	{\True chiếu một chùm sáng đỏ qua một tấm lọc màu tím}
	\loigiai{}
\end{ex}

%%%=============EX_4=============%%%
\begin{ex}
	Chỉ ra câu sai.
	Máy ảnh cho phép ta làm được những gì?
	\choice
	{Tạo ảnh thật của vật, nhỏ hơn vật}
	{Ghi lại ảnh thật đó trên phim hoặc bộ phận ghi ảnh}
	{Tháo phim hoặc bộ phận ghi ảnh ra khỏi máy}
	{\True Phóng to và in ảnh trong phim hoặc bộ phận ghi ảnh trên giấy ảnh}
	\loigiai{}
\end{ex}

%%%=============EX_5=============%%%
\begin{ex}
	Một máy biến áp cuộn sơ cấp có 200 vòng, cuộn thứ cấp có 1000 vòng. Nếu đặt vào hai đầu cuộn sơ cấp một hiệu điện thế 50 V thì hiệu điện thế ở hai đầu cuộn thứ cấp bằng
	\choice
	{10 V}
	{100 V}
	{\True 250 V}
	{500 V}
	\loigiai{}
\end{ex}

%%%=============EX_6=============%%%
\begin{ex}
	Trong hiện tượng khúc xạ ánh sáng, góc tới là:
	\choice
	{\True góc tạo bởi tia tới và pháp tuyến qua điểm tới}
	{góc tạo bởi tia tới và mặt phân cách giữa hai môi trường}
	{góc tạo bởi tia tới và tia khúc xạ}
	{góc tạo bởi tia khúc xạ và pháp tuyến}
	\loigiai{}
\end{ex}

%%%=============EX_7=============%%%
\begin{ex}
	Tác dụng từ của dòng điện thay đổi như thế nào khi dòng điện đổi chiều?
	\choice
	{Không còn tác dụng từ}
	{Tác dụng từ mạnh lên gấp đôi}
	{Tác dụng từ giảm đi}
	{\True Lực từ đổi chiều}
	\loigiai{}
\end{ex}

%%%=============EX_8=============%%%
\begin{ex}
	Di chuyển một ngọn nến dọc theo trục chính của một thấu kính phân kì rồi tìm ảnh của nó, ta sẽ thấy gì?
	\choice
	{Có lúc ta thu được ảnh thật, có lúc ta thu được ảnh ảo}
	{Nếu đặt ngọn nến ngoài khoảng tiêu cự của thấu kính ta sẽ thu được ảnh thật}
	{Ta chỉ thu được ảnh ảo nếu đặt ngọn nến trong khoảng tiêu cự của thấu kính}
	{\True Ta luôn luôn thu được ảnh ảo dù đặt ngọn nến ở bất kì vị trí nào}
	\loigiai{}
\end{ex}

%%%=============EX_9=============%%%
\begin{ex}
	Một người cận thị phải đeo kính có tiêu cự 60 cm. Hỏi khi không đeo kính thì người ấy nhìn rõ được vật xa nhất cách mắt bao nhiêu?
	\choice
	{30 cm}
	{40 cm}
	{\True 60 cm}
	{120 cm}
	\loigiai{}
\end{ex}

%%%=============EX_10=============%%%
\begin{ex}
	Chỉ ra câu sai?
	Đặt một cây nến trước một thấu kính hội tụ.
	\choice
	{Ta có thể thu được ảnh của cây nến trên màn ảnh}
	{Ảnh của cây nến trên màn có thể lớn hoặc nhỏ hơn cây nến}
	{\True Ảnh của cây nến trên màn ảnh có thể là ảnh thật hoặc ảnh ảo}
	{Ảnh ảo của cây nến luôn lớn hơn cây nến}
	\loigiai{}
\end{ex}

%%%=============EX_11=============%%%
\begin{ex}\immini{
	Cho điểm sáng S có vị trí đối với một thấu kính như hình vẽ. Ảnh của S qua thấu kính là
	\choice
	{ảnh thật, nằm cùng phía đối với S so với trục chính}
	{ảnh ảo, nằm cùng phía đối với S so với trục chính}
	{\True ảnh thật, nằm khác phía đối với S so với trục chính}
	{ảnh ảo, nằm khác phía đối với S so với trục chính}
	}{
	\begin{tikzpicture}[declare function={h=0.7*d;d=3.5;f=0.5*d;},font=\scriptsize\sffamily\bfseries]
		\path[draw=\mycolor,Stealth-Stealth,line width =1.5pt] (0,-h) coordinate (M)--(0,h) coordinate (N);
		\path[draw=\mycolor](-d,0) coordinate (Xp)--(d,0) coordinate (X);
		\path (0,0) coordinate (O) 
		(-f,0) coordinate (F) 
		(f,0) coordinate (F')
		({-1.6*f},{h-1.5}) coordinate (S);
		\foreach \t/\g in {F'/-90,F/90,O/-135,S/135}{
			\path[fill=red,draw=black] (\t) circle (1pt) node[shift={(\g:7pt)}]{\t};}
	\end{tikzpicture}
	}
	\loigiai{}
\end{ex}

%%%=============EX_12=============%%%
\begin{ex}
	Khi nhìn một vật ở điểm cực viễn thì
	\choice
	{mắt phải điều tiết mạnh nhất}
	{\True mắt không phải điều tiết}
	{mắt điều tiết ở mức độ vừa phải}
	{màng lưới di chuyển lại gần thể thủy tinh}
	\loigiai{}
\end{ex}

%%%=============EX_13=============%%%
\begin{ex}
	Chiếu một tia sáng qua quang tâm của một thấu kính phân kì, theo phương không song song với trục chính. Tia sáng ló ra khỏi thấu kính sẽ đi theo phương nào?
	\choice
	{Phương bất kì}
	{Phương lệch ra xa trục chính so với tia tới}
	{Phương lệch lại gần trục chính so với tia tới}
	{\True Phương cũ}
	\loigiai{}
\end{ex}

%%%=============EX_14=============%%%
\begin{ex}
	Thấu kính phân kì có đặc điểm và tác dụng nào dưới đây?
	\choice
	{Có phần giữa mỏng hơn phần rìa và cho phép thu được ảnh của Mặt Trời}
	{\True Có phần giữa mỏng hơn phần rìa và không cho phép thu được ảnh của Mặt Trời}
	{Có phần giữa dày hơn phần rìa và cho phép thu được ảnh của Mặt Trời}
	{Có phần giữa dày hơn phần rìa và không cho phép thu được ảnh của Mặt Trời}
	\loigiai{}
\end{ex}

%%%=============EX_15=============%%%
\begin{ex}
	Trong trường hợp nào dưới đây, ánh sáng trắng sẽ không bị phân tích?
	\choice
	{Chiếu tia sáng trắng qua một lăng kính}
	{\True Chiếu tia sáng trắng nghiêng góc vào một gương phẳng}
	{Chiếu tia sáng trắng nghiêng góc vào mặt ghi âm của một đĩa C}
	{Chiếu chùm sáng trắng vào một bong bóng xà phòng}
	\loigiai{}
\end{ex}

%%%=============EX_16=============%%%
\begin{ex}
	Trong các công việc nào dưới đây, ta đã sử dụng tác dụng nhiệt của ánh sáng?
	\choice
	{Đưa một chậu cây ra ngoài sân phơi cho đỡ cớm}
	{Kê bàn học cạnh của sổ cho sáng}
	{\True Phơi thóc ngoài sân khi trời nắng}
	{Cho ánh sáng chiếu vào bộ pin Mặt Trời của máy tính để nó hoạt động}
	\loigiai{}
\end{ex}



\Closesolutionfile{ansex}
\Closesolutionfile{ans}

%%%==========Phần trắc nghiệm đúng sai============%%%

\tieumuc{Trắc Nghiệm Đúng Sai (1 điểm)}(\textit{Trong mỗi câu có 4 ý tương ứng A, B, C, D; Học sinh chọn đúng hoặc sai.})
\Opensolutionfile{ans}[Ans/DATAM-VL9CKII-De-04]
\Opensolutionfile{ansbook}[Ans/DATNTF-VL9CKII-De-04]
%\luulgEXTF
\LGexTF
\Opensolutionfile{ansex}[LOIGIAITN/LGTNTF-VL9CKII-De-04]
%%%=============EX_17=============%%%
\begin{ex}
	Trong các phát biểu dưới đây , phát biểu nào sai, phát biểu nào đúng?
	\choiceTF
	{Thủy tinh thể ở mắt đóng vai trò như buồng tối ở máy ảnh}
	{Ta nhìn thấy một vật màu đỏ là do ánh sáng màu đỏ từ vật đó truyền đến mắt ta}
	{Ảnh của một vật tạo bởi thấu kính phân kì luôn là ảnh ảo, cùng chiều và lớn hơn vật}
	{\True Chiều dòng điện xoay chiều xuất hiện trong cuộn dây dẫn kín phụ thuộc vào chiều của đường sức từ xuyên qua cuộn dây}
	\loigiai{}
\end{ex}

\Closesolutionfile{ansex}
\Closesolutionfile{ansbook}
\Closesolutionfile{ans}	

%%%==========Phần tự luận============%%%
\tieumuc{Tự Luận (5 điểm)}
\Opensolutionfile{ansbt}[LOIGIAITL/LGTL-VL9CKII-De-04]
%\luuloigiaibt
%\hienthiloigiaibt
%\taoNdongke[15]{bt}
%%%=============BT_1=============%%%
\begin{bt}[$1{,}0$ điểm]Trình bày nguyên tắc hoạt đông của máy phát điện xoay chiều có rôto là nam châm?
	\loigiai{}
\end{bt}
%%%=============BT_2=============%%%
\begin{bt}[$1{,}5$ điểm]Một máy biến thế gồm cuộn sơ cấp có $1000$ vòng, cuộn thứ cấp có $5000$ vòng đặt ở đầu một đường dây tải điện để truyền đi một công suất điện là $10\,000\mathrm{~kW}$. Biết hiệu điện thế hai đầu cuộn thứ cấp là $100\mathrm{~kV}$.
	\begin{enumerate}[a)]
		\item Tính hiệu điện thế đặt vào hai đầu cuộn sơ cấp?
		\item Cho biết điện trở của toàn bộ đường dây là $100~\Omega$.Tính công suất hao phí tỏa nhiệt trên đường dây?
	\end{enumerate}
	\loigiai{}
\end{bt}
%%%=============BT_3=============%%%
\begin{bt}[$2{,}5$ điểm] Một vật sáng $AB$ cao $2$ cm đặt vuông góc với trục chính của một thấu kính hội tụ có tiêu cự bằng 10 cm, A nằm trên trục chính và cách quang tâm của thấu kính $15\mathrm{~cm}$.
	\begin{enumerate}
		\item Vẽ ảnh $A'B'$ của $AB$ qua thấu kính.Ảnh $A'B'$ là ảnh thật hay ảnh ảo?
		\item Xác định chiều cao của ảnh?
		\item Di chuyển vật ra ra thấu kính $5$ cm. Thì ảnh dịch chuyển lại gần hay ra xa thấu kính? Tính độ dịch chuyển đó.
	\end{enumerate}
	\loigiai{}
\end{bt}
\Closesolutionfile{ansbt}
\fileend
\begin{center}
	\rule[4pt]{2cm}{1pt}\large \textbf{HẾT}\rule[4pt]{2cm}{1pt}
\end{center}











