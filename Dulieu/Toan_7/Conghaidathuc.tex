
%%%=====Bắt đầu Bài_1=====%%%
\begin{bt}
Cho hai đa thức $A(x) =x^3 - 3 x^2 + x - 2$ và $B(x)=5 x^3 + 2 x^2 + 8 x + 3$
 Hãy tính:$A(x) +B(x)$.
\loigiai{
\begin{eqnarray*}
	\text{Ta có}\ A(x)+B(x)
	&=&(x^3 - 3 x^2 + x - 2) + (5 x^3 + 2 x^2 + 8 x + 3)\\
	&=&(1+5)x^3+(-3+2)x^2+(1+8)x+(-2+3)\\
	&=&6 x^3- x^2+9 x+1
\end{eqnarray*}
}
\end{bt}
%%%====Kết thúc Bài_1=====%%%
%%%=====Bắt đầu Bài_2=====%%%
\begin{bt}
Cho hai đa thức $A(x) =- x^3 + 3 x^2 + x - 5$ và $B(x)=- 2 x^3 + 2 x^2 + 8 x$
 Hãy tính:$A(x) +B(x)$.
\loigiai{
\begin{eqnarray*}
	\text{Ta có}\ A(x)+B(x)
	&=&(- x^3 + 3 x^2 + x - 5) + (- 2 x^3 + 2 x^2 + 8 x)\\
	&=&(-1-2)x^3+(3+2)x^2+(1+8)x-5\\
	&=&- 3 x^3+5 x^2+9 x-5
\end{eqnarray*}
}
\end{bt}
%%%====Kết thúc Bài_2=====%%%
%%%=====Bắt đầu Bài_3=====%%%
\begin{bt}
Cho hai đa thức $A(x) =- 5 x^3 + 3 x^2 + x - 5$ và $B(x)=- 3 x^3 + 2 x^2 + 2 x + 2$
 Hãy tính:$A(x) +B(x)$.
\loigiai{
\begin{eqnarray*}
	\text{Ta có}\ A(x)+B(x)
	&=&(- 5 x^3 + 3 x^2 + x - 5) + (- 3 x^3 + 2 x^2 + 2 x + 2)\\
	&=&(-5-3)x^3+(3+2)x^2+(1+2)x+(-5+2)\\
	&=&- 8 x^3+5 x^2+3 x-3
\end{eqnarray*}
}
\end{bt}
%%%====Kết thúc Bài_3=====%%%
%%%=====Bắt đầu Bài_4=====%%%
\begin{bt}
Cho hai đa thức $A(x) =- 5 x^3 - 4 x^2 + x + 4$ và $B(x)=6 x^3 + 11 x^2 + 8 x - 7$
 Hãy tính:$A(x) +B(x)$.
\loigiai{
\begin{eqnarray*}
	\text{Ta có}\ A(x)+B(x)
	&=&(- 5 x^3 - 4 x^2 + x + 4) + (6 x^3 + 11 x^2 + 8 x - 7)\\
	&=&(-5+6)x^3+(-4+11)x^2+(1+8)x+(4-7)\\
	&=&x^3+7 x^2+9 x-3
\end{eqnarray*}
}
\end{bt}
%%%====Kết thúc Bài_4=====%%%
%%%=====Bắt đầu Bài_5=====%%%
\begin{bt}
Cho hai đa thức $A(x) =- 4 x^3 - 3 x^2 + x + 5$ và $B(x)=5 x^3 - 7 x^2 + 8 x + 1$
 Hãy tính:$A(x) +B(x)$.
\loigiai{
\begin{eqnarray*}
	\text{Ta có}\ A(x)+B(x)
	&=&(- 4 x^3 - 3 x^2 + x + 5) + (5 x^3 - 7 x^2 + 8 x + 1)\\
	&=&(-4+5)x^3+(-3-7)x^2+(1+8)x+(5+1)\\
	&=&x^3- 10 x^2+9 x+6
\end{eqnarray*}
}
\end{bt}
%%%====Kết thúc Bài_5=====%%%
%%%=====Bắt đầu Bài_6=====%%%
\begin{bt}
Cho hai đa thức $A(x) =3 x^3 + x^2 - x - 5$ và $B(x)=- 4 x^3 - 3 x^2 - 2 x - 2$
 Hãy tính:$A(x) +B(x)$.
\loigiai{
\begin{eqnarray*}
	\text{Ta có}\ A(x)+B(x)
	&=&(3 x^3 + x^2 - x - 5) + (- 4 x^3 - 3 x^2 - 2 x - 2)\\
	&=&(3-4)x^3+(1-3)x^2+(-1-2)x+(-5-2)\\
	&=&- x^3- 2 x^2- 3 x-7
\end{eqnarray*}
}
\end{bt}
%%%====Kết thúc Bài_6=====%%%
%%%=====Bắt đầu Bài_7=====%%%
\begin{bt}
Cho hai đa thức $A(x) =- 4 x^3 + x^2 + 2 x - 4$ và $B(x)=- 4 x^3 + 2 x^2 - 8 x - 2$
 Hãy tính:$A(x) +B(x)$.
\loigiai{
\begin{eqnarray*}
	\text{Ta có}\ A(x)+B(x)
	&=&(- 4 x^3 + x^2 + 2 x - 4) + (- 4 x^3 + 2 x^2 - 8 x - 2)\\
	&=&(-4-4)x^3+(1+2)x^2+(2-8)x+(-4-2)\\
	&=&- 8 x^3+3 x^2- 6 x-6
\end{eqnarray*}
}
\end{bt}
%%%====Kết thúc Bài_7=====%%%
%%%=====Bắt đầu Bài_8=====%%%
\begin{bt}
Cho hai đa thức $A(x) =2 x^3 - x^2 - x + 4$ và $B(x)=3 x^3 - 5 x^2 - 2 x + 9$
 Hãy tính:$A(x) +B(x)$.
\loigiai{
\begin{eqnarray*}
	\text{Ta có}\ A(x)+B(x)
	&=&(2 x^3 - x^2 - x + 4) + (3 x^3 - 5 x^2 - 2 x + 9)\\
	&=&(2+3)x^3+(-1-5)x^2+(-1-2)x+(4+9)\\
	&=&5 x^3- 6 x^2- 3 x+13
\end{eqnarray*}
}
\end{bt}
%%%====Kết thúc Bài_8=====%%%
%%%=====Bắt đầu Bài_9=====%%%
\begin{bt}
Cho hai đa thức $A(x) =3 x^3 + x^2 - 2 x + 2$ và $B(x)=4 x^3 + 9 x^2 + 5 x + 8$
 Hãy tính:$A(x) +B(x)$.
\loigiai{
\begin{eqnarray*}
	\text{Ta có}\ A(x)+B(x)
	&=&(3 x^3 + x^2 - 2 x + 2) + (4 x^3 + 9 x^2 + 5 x + 8)\\
	&=&(3+4)x^3+(1+9)x^2+(-2+5)x+(2+8)\\
	&=&7 x^3+10 x^2+3 x+10
\end{eqnarray*}
}
\end{bt}
%%%====Kết thúc Bài_9=====%%%
%%%====Bắt đầu Bài_10=====%%%
\begin{bt}
Cho hai đa thức $A(x) =- 5 x^3 + 2 x^2 + x + 6$ và $B(x)=- 2 x^3 + 13 x^2 + 9 x - 1$
 Hãy tính:$A(x) +B(x)$.
\loigiai{
\begin{eqnarray*}
	\text{Ta có}\ A(x)+B(x)
	&=&(- 5 x^3 + 2 x^2 + x + 6) + (- 2 x^3 + 13 x^2 + 9 x - 1)\\
	&=&(-5-2)x^3+(2+13)x^2+(1+9)x+(6-1)\\
	&=&- 7 x^3+15 x^2+10 x+5
\end{eqnarray*}
}
\end{bt}
%%%====Kết thúc Bài_10====%%%
%%%====Bắt đầu Bài_11=====%%%
\begin{bt}
Cho hai đa thức $A(x) =- 5 x^3 - x^2 - 5$ và $B(x)=- 3 x^3 + x^2 + 7 x$
 Hãy tính:$A(x) +B(x)$.
\loigiai{
\begin{eqnarray*}
	\text{Ta có}\ A(x)+B(x)
	&=&(- 5 x^3 - x^2 - 5) + (- 3 x^3 + x^2 + 7 x)\\
	&=&(-5-3)x^3+(-1+1)x^2+7x-5\\
	&=&- 8 x^3+7 x-5
\end{eqnarray*}
}
\end{bt}
%%%====Kết thúc Bài_11====%%%
%%%====Bắt đầu Bài_12=====%%%
\begin{bt}
Cho hai đa thức $A(x) =- x^3 - 3 x^2 + x + 1$ và $B(x)=5 x^3 + 5 x^2 - 5 x + 2$
 Hãy tính:$A(x) +B(x)$.
\loigiai{
\begin{eqnarray*}
	\text{Ta có}\ A(x)+B(x)
	&=&(- x^3 - 3 x^2 + x + 1) + (5 x^3 + 5 x^2 - 5 x + 2)\\
	&=&(-1+5)x^3+(-3+5)x^2+(1-5)x+(1+2)\\
	&=&4 x^3+2 x^2- 4 x+3
\end{eqnarray*}
}
\end{bt}
%%%====Kết thúc Bài_12====%%%
%%%====Bắt đầu Bài_13=====%%%
\begin{bt}
Cho hai đa thức $A(x) =3 x^3 + x^2 - 5$ và $B(x)=- 4 x^3 + 2 x^2 - 2 x + 8$
 Hãy tính:$A(x) +B(x)$.
\loigiai{
\begin{eqnarray*}
	\text{Ta có}\ A(x)+B(x)
	&=&(3 x^3 + x^2 - 5) + (- 4 x^3 + 2 x^2 - 2 x + 8)\\
	&=&(3-4)x^3+(1+2)x^2-2x+(-5+8)\\
	&=&- x^3+3 x^2- 2 x+3
\end{eqnarray*}
}
\end{bt}
%%%====Kết thúc Bài_13====%%%
%%%====Bắt đầu Bài_14=====%%%
\begin{bt}
Cho hai đa thức $A(x) =3 x^3 + x^2 - 2 x + 1$ và $B(x)=- 4 x^3 - x^2 + 5 x + 6$
 Hãy tính:$A(x) +B(x)$.
\loigiai{
\begin{eqnarray*}
	\text{Ta có}\ A(x)+B(x)
	&=&(3 x^3 + x^2 - 2 x + 1) + (- 4 x^3 - x^2 + 5 x + 6)\\
	&=&(3-4)x^3+(1-1)x^2+(-2+5)x+(1+6)\\
	&=&- x^3+3 x+7
\end{eqnarray*}
}
\end{bt}
%%%====Kết thúc Bài_14====%%%
%%%====Bắt đầu Bài_15=====%%%
\begin{bt}
Cho hai đa thức $A(x) =- 2 x^3 + 3 x^2 + x$ và $B(x)=5 x^3 - x^2 - 6 x + 2$
 Hãy tính:$A(x) +B(x)$.
\loigiai{
\begin{eqnarray*}
	\text{Ta có}\ A(x)+B(x)
	&=&(- 2 x^3 + 3 x^2 + x) + (5 x^3 - x^2 - 6 x + 2)\\
	&=&(-2+5)x^3+(3-1)x^2+(1-6)x+2\\
	&=&3 x^3+2 x^2- 5 x+2
\end{eqnarray*}
}
\end{bt}
%%%====Kết thúc Bài_15====%%%
%%%====Bắt đầu Bài_16=====%%%
\begin{bt}
Cho hai đa thức $A(x) =- 3 x^3 + 2 x^2 - 6$ và $B(x)=4 x^3 + x^2 - 3 x - 1$
 Hãy tính:$A(x) +B(x)$.
\loigiai{
\begin{eqnarray*}
	\text{Ta có}\ A(x)+B(x)
	&=&(- 3 x^3 + 2 x^2 - 6) + (4 x^3 + x^2 - 3 x - 1)\\
	&=&(-3+4)x^3+(2+1)x^2-3x+(-6-1)\\
	&=&x^3+3 x^2- 3 x-7
\end{eqnarray*}
}
\end{bt}
%%%====Kết thúc Bài_16====%%%
%%%====Bắt đầu Bài_17=====%%%
\begin{bt}
Cho hai đa thức $A(x) =3 x^3 - 4 x^2 + 2 x + 2$ và $B(x)=4 x^3 + 3 x^2 - x + 8$
 Hãy tính:$A(x) +B(x)$.
\loigiai{
\begin{eqnarray*}
	\text{Ta có}\ A(x)+B(x)
	&=&(3 x^3 - 4 x^2 + 2 x + 2) + (4 x^3 + 3 x^2 - x + 8)\\
	&=&(3+4)x^3+(-4+3)x^2+(2-1)x+(2+8)\\
	&=&7 x^3- x^2+x+10
\end{eqnarray*}
}
\end{bt}
%%%====Kết thúc Bài_17====%%%
%%%====Bắt đầu Bài_18=====%%%
\begin{bt}
Cho hai đa thức $A(x) =- 3 x^3 - 2 x^2 - 2 x + 1$ và $B(x)=- 7 x^3 + 3 x^2 - 7 x + 2$
 Hãy tính:$A(x) +B(x)$.
\loigiai{
\begin{eqnarray*}
	\text{Ta có}\ A(x)+B(x)
	&=&(- 3 x^3 - 2 x^2 - 2 x + 1) + (- 7 x^3 + 3 x^2 - 7 x + 2)\\
	&=&(-3-7)x^3+(-2+3)x^2+(-2-7)x+(1+2)\\
	&=&- 10 x^3+x^2- 9 x+3
\end{eqnarray*}
}
\end{bt}
%%%====Kết thúc Bài_18====%%%
%%%====Bắt đầu Bài_19=====%%%
\begin{bt}
Cho hai đa thức $A(x) =2 x^3 - x^2 + 2 x - 1$ và $B(x)=- 6 x^3 + 6 x^2 - 2 x - 4$
 Hãy tính:$A(x) +B(x)$.
\loigiai{
\begin{eqnarray*}
	\text{Ta có}\ A(x)+B(x)
	&=&(2 x^3 - x^2 + 2 x - 1) + (- 6 x^3 + 6 x^2 - 2 x - 4)\\
	&=&(2-6)x^3+(-1+6)x^2+(2-2)x+(-1-4)\\
	&=&- 4 x^3+5 x^2-5
\end{eqnarray*}
}
\end{bt}
%%%====Kết thúc Bài_19====%%%
%%%====Bắt đầu Bài_20=====%%%
\begin{bt}
Cho hai đa thức $A(x) =2 x^3 - 2 x^2 - x + 3$ và $B(x)=5 x^3 - 6 x^2 + 6 x - 5$
 Hãy tính:$A(x) +B(x)$.
\loigiai{
\begin{eqnarray*}
	\text{Ta có}\ A(x)+B(x)
	&=&(2 x^3 - 2 x^2 - x + 3) + (5 x^3 - 6 x^2 + 6 x - 5)\\
	&=&(2+5)x^3+(-2-6)x^2+(-1+6)x+(3-5)\\
	&=&7 x^3- 8 x^2+5 x-2
\end{eqnarray*}
}
\end{bt}
%%%====Kết thúc Bài_20====%%%
%%%====Bắt đầu Bài_21=====%%%
\begin{bt}
Cho hai đa thức $A(x) =- 5 x^3 + 3 x^2 - 2 x - 6$ và $B(x)=- 2 x^3 + x^2 + 2 x$
 Hãy tính:$A(x) +B(x)$.
\loigiai{
\begin{eqnarray*}
	\text{Ta có}\ A(x)+B(x)
	&=&(- 5 x^3 + 3 x^2 - 2 x - 6) + (- 2 x^3 + x^2 + 2 x)\\
	&=&(-5-2)x^3+(3+1)x^2+(-2+2)x-6\\
	&=&- 7 x^3+4 x^2-6
\end{eqnarray*}
}
\end{bt}
%%%====Kết thúc Bài_21====%%%
%%%====Bắt đầu Bài_22=====%%%
\begin{bt}
Cho hai đa thức $A(x) =x^3 - 2 x^2 + x - 1$ và $B(x)=- 7 x^3 + 6 x^2 + 2 x + 2$
 Hãy tính:$A(x) +B(x)$.
\loigiai{
\begin{eqnarray*}
	\text{Ta có}\ A(x)+B(x)
	&=&(x^3 - 2 x^2 + x - 1) + (- 7 x^3 + 6 x^2 + 2 x + 2)\\
	&=&(1-7)x^3+(-2+6)x^2+(1+2)x+(-1+2)\\
	&=&- 6 x^3+4 x^2+3 x+1
\end{eqnarray*}
}
\end{bt}
%%%====Kết thúc Bài_22====%%%
%%%====Bắt đầu Bài_23=====%%%
\begin{bt}
Cho hai đa thức $A(x) =- 2 x^3 - 3 x^2 - 2 x + 2$ và $B(x)=x^3 - x^2 + 5 x + 7$
 Hãy tính:$A(x) +B(x)$.
\loigiai{
\begin{eqnarray*}
	\text{Ta có}\ A(x)+B(x)
	&=&(- 2 x^3 - 3 x^2 - 2 x + 2) + (x^3 - x^2 + 5 x + 7)\\
	&=&(-2+1)x^3+(-3-1)x^2+(-2+5)x+(2+7)\\
	&=&- x^3- 4 x^2+3 x+9
\end{eqnarray*}
}
\end{bt}
%%%====Kết thúc Bài_23====%%%
%%%====Bắt đầu Bài_24=====%%%
\begin{bt}
Cho hai đa thức $A(x) =- 4 x^3 - 4 x^2 - 5$ và $B(x)=- 3 x^3 + 2 x^2 - 8 x - 7$
 Hãy tính:$A(x) +B(x)$.
\loigiai{
\begin{eqnarray*}
	\text{Ta có}\ A(x)+B(x)
	&=&(- 4 x^3 - 4 x^2 - 5) + (- 3 x^3 + 2 x^2 - 8 x - 7)\\
	&=&(-4-3)x^3+(-4+2)x^2-8x+(-5-7)\\
	&=&- 7 x^3- 2 x^2- 8 x-12
\end{eqnarray*}
}
\end{bt}
%%%====Kết thúc Bài_24====%%%
%%%====Bắt đầu Bài_25=====%%%
\begin{bt}
Cho hai đa thức $A(x) =- 4 x^3 + 2 x + 3$ và $B(x)=- x^3 + 2 x^2 - x + 8$
 Hãy tính:$A(x) +B(x)$.
\loigiai{
\begin{eqnarray*}
	\text{Ta có}\ A(x)+B(x)
	&=&(- 4 x^3 + 2 x + 3) + (- x^3 + 2 x^2 - x + 8)\\
	&=&(-4-1)x^3+2x^2+(2-1)x+(3+8)\\
	&=&- 5 x^3+2 x^2+x+11
\end{eqnarray*}
}
\end{bt}
%%%====Kết thúc Bài_25====%%%
%%%====Bắt đầu Bài_26=====%%%
\begin{bt}
Cho hai đa thức $A(x) =- 2 x^3 + x^2 + 2 x + 5$ và $B(x)=7 x^3 - 3 x^2 - 6 x + 10$
 Hãy tính:$A(x) +B(x)$.
\loigiai{
\begin{eqnarray*}
	\text{Ta có}\ A(x)+B(x)
	&=&(- 2 x^3 + x^2 + 2 x + 5) + (7 x^3 - 3 x^2 - 6 x + 10)\\
	&=&(-2+7)x^3+(1-3)x^2+(2-6)x+(5+10)\\
	&=&5 x^3- 2 x^2- 4 x+15
\end{eqnarray*}
}
\end{bt}
%%%====Kết thúc Bài_26====%%%
%%%====Bắt đầu Bài_27=====%%%
\begin{bt}
Cho hai đa thức $A(x) =2 - 2 x^3$ và $B(x)=4 x^3 + x^2 + 5 x + 7$
 Hãy tính:$A(x) +B(x)$.
\loigiai{
\begin{eqnarray*}
	\text{Ta có}\ A(x)+B(x)
	&=&(2 - 2 x^3) + (4 x^3 + x^2 + 5 x + 7)\\
	&=&(-2+4)x^3+x^2+5x+(2+7)\\
	&=&2 x^3+x^2+5 x+9
\end{eqnarray*}
}
\end{bt}
%%%====Kết thúc Bài_27====%%%
%%%====Bắt đầu Bài_28=====%%%
\begin{bt}
Cho hai đa thức $A(x) =- 5 x^3 - 2 x^2 - 2 x + 6$ và $B(x)=- 2 x^3 - x^2 - 3 x$
 Hãy tính:$A(x) +B(x)$.
\loigiai{
\begin{eqnarray*}
	\text{Ta có}\ A(x)+B(x)
	&=&(- 5 x^3 - 2 x^2 - 2 x + 6) + (- 2 x^3 - x^2 - 3 x)\\
	&=&(-5-2)x^3+(-2-1)x^2+(-2-3)x+6\\
	&=&- 7 x^3- 3 x^2- 5 x+6
\end{eqnarray*}
}
\end{bt}
%%%====Kết thúc Bài_28====%%%
%%%====Bắt đầu Bài_29=====%%%
\begin{bt}
Cho hai đa thức $A(x) =- 3 x^3 + 3 x^2 + 2 x - 2$ và $B(x)=6 x^3 + 5 x^2 + x + 2$
 Hãy tính:$A(x) +B(x)$.
\loigiai{
\begin{eqnarray*}
	\text{Ta có}\ A(x)+B(x)
	&=&(- 3 x^3 + 3 x^2 + 2 x - 2) + (6 x^3 + 5 x^2 + x + 2)\\
	&=&(-3+6)x^3+(3+5)x^2+(2+1)x+(-2+2)\\
	&=&3 x^3+8 x^2+3 x
\end{eqnarray*}
}
\end{bt}
%%%====Kết thúc Bài_29====%%%
%%%====Bắt đầu Bài_30=====%%%
\begin{bt}
Cho hai đa thức $A(x) =x^3 - 4 x^2 - x + 6$ và $B(x)=4 x^3 + 7 x^2 + 9 x + 11$
 Hãy tính:$A(x) +B(x)$.
\loigiai{
\begin{eqnarray*}
	\text{Ta có}\ A(x)+B(x)
	&=&(x^3 - 4 x^2 - x + 6) + (4 x^3 + 7 x^2 + 9 x + 11)\\
	&=&(1+4)x^3+(-4+7)x^2+(-1+9)x+(6+11)\\
	&=&5 x^3+3 x^2+8 x+17
\end{eqnarray*}
}
\end{bt}
%%%====Kết thúc Bài_30====%%%
%%%====Bắt đầu Bài_31=====%%%
\begin{bt}
Cho hai đa thức $A(x) =- 5 x^3 + 2 x + 1$ và $B(x)=3 x^3 + 8 x^2 - x + 3$
 Hãy tính:$A(x) +B(x)$.
\loigiai{
\begin{eqnarray*}
	\text{Ta có}\ A(x)+B(x)
	&=&(- 5 x^3 + 2 x + 1) + (3 x^3 + 8 x^2 - x + 3)\\
	&=&(-5+3)x^3+8x^2+(2-1)x+(1+3)\\
	&=&- 2 x^3+8 x^2+x+4
\end{eqnarray*}
}
\end{bt}
%%%====Kết thúc Bài_31====%%%
%%%====Bắt đầu Bài_32=====%%%
\begin{bt}
Cho hai đa thức $A(x) =- 4 x^3 + x^2$ và $B(x)=- 4 x^3 - 3 x^2 - 8 x - 2$
 Hãy tính:$A(x) +B(x)$.
\loigiai{
\begin{eqnarray*}
	\text{Ta có}\ A(x)+B(x)
	&=&(- 4 x^3 + x^2) + (- 4 x^3 - 3 x^2 - 8 x - 2)\\
	&=&(-4-4)x^3+(1-3)x^2-8x-2\\
	&=&- 8 x^3- 2 x^2- 8 x-2
\end{eqnarray*}
}
\end{bt}
%%%====Kết thúc Bài_32====%%%
%%%====Bắt đầu Bài_33=====%%%
\begin{bt}
Cho hai đa thức $A(x) =- 5 x^3 - 2 x^2 + 2 x + 6$ và $B(x)=8 x^3 + 3 x^2 - 9 x + 3$
 Hãy tính:$A(x) +B(x)$.
\loigiai{
\begin{eqnarray*}
	\text{Ta có}\ A(x)+B(x)
	&=&(- 5 x^3 - 2 x^2 + 2 x + 6) + (8 x^3 + 3 x^2 - 9 x + 3)\\
	&=&(-5+8)x^3+(-2+3)x^2+(2-9)x+(6+3)\\
	&=&3 x^3+x^2- 7 x+9
\end{eqnarray*}
}
\end{bt}
%%%====Kết thúc Bài_33====%%%
%%%====Bắt đầu Bài_34=====%%%
\begin{bt}
Cho hai đa thức $A(x) =2 x^3 - 3 x^2 + 3$ và $B(x)=5 x^3 + 8 x^2 - 4 x + 7$
 Hãy tính:$A(x) +B(x)$.
\loigiai{
\begin{eqnarray*}
	\text{Ta có}\ A(x)+B(x)
	&=&(2 x^3 - 3 x^2 + 3) + (5 x^3 + 8 x^2 - 4 x + 7)\\
	&=&(2+5)x^3+(-3+8)x^2-4x+(3+7)\\
	&=&7 x^3+5 x^2- 4 x+10
\end{eqnarray*}
}
\end{bt}
%%%====Kết thúc Bài_34====%%%
%%%====Bắt đầu Bài_35=====%%%
\begin{bt}
Cho hai đa thức $A(x) =- 5 x^3 - 3 x^2 + 2$ và $B(x)=- 2 x^3 + 9 x^2 - 9 x + 7$
 Hãy tính:$A(x) +B(x)$.
\loigiai{
\begin{eqnarray*}
	\text{Ta có}\ A(x)+B(x)
	&=&(- 5 x^3 - 3 x^2 + 2) + (- 2 x^3 + 9 x^2 - 9 x + 7)\\
	&=&(-5-2)x^3+(-3+9)x^2-9x+(2+7)\\
	&=&- 7 x^3+6 x^2- 9 x+9
\end{eqnarray*}
}
\end{bt}
%%%====Kết thúc Bài_35====%%%
%%%====Bắt đầu Bài_36=====%%%
\begin{bt}
Cho hai đa thức $A(x) =- 5 x^3 - 4 x^2 + 2 x - 6$ và $B(x)=6 x^3 + 3 x^2 - 3 x - 7$
 Hãy tính:$A(x) +B(x)$.
\loigiai{
\begin{eqnarray*}
	\text{Ta có}\ A(x)+B(x)
	&=&(- 5 x^3 - 4 x^2 + 2 x - 6) + (6 x^3 + 3 x^2 - 3 x - 7)\\
	&=&(-5+6)x^3+(-4+3)x^2+(2-3)x+(-6-7)\\
	&=&x^3- x^2- x-13
\end{eqnarray*}
}
\end{bt}
%%%====Kết thúc Bài_36====%%%
%%%====Bắt đầu Bài_37=====%%%
\begin{bt}
Cho hai đa thức $A(x) =- 2 x^3 - x$ và $B(x)=- 5 x^3 + 7 x^2 - 6 x + 3$
 Hãy tính:$A(x) +B(x)$.
\loigiai{
\begin{eqnarray*}
	\text{Ta có}\ A(x)+B(x)
	&=&(- 2 x^3 - x) + (- 5 x^3 + 7 x^2 - 6 x + 3)\\
	&=&(-2-5)x^3+7x^2+(-1-6)x+3\\
	&=&- 7 x^3+7 x^2- 7 x+3
\end{eqnarray*}
}
\end{bt}
%%%====Kết thúc Bài_37====%%%
%%%====Bắt đầu Bài_38=====%%%
\begin{bt}
Cho hai đa thức $A(x) =2 x^3 - 4 x^2 - 2 x + 2$ và $B(x)=- 9 x^3 + 9 x^2 - 5 x - 7$
 Hãy tính:$A(x) +B(x)$.
\loigiai{
\begin{eqnarray*}
	\text{Ta có}\ A(x)+B(x)
	&=&(2 x^3 - 4 x^2 - 2 x + 2) + (- 9 x^3 + 9 x^2 - 5 x - 7)\\
	&=&(2-9)x^3+(-4+9)x^2+(-2-5)x+(2-7)\\
	&=&- 7 x^3+5 x^2- 7 x-5
\end{eqnarray*}
}
\end{bt}
%%%====Kết thúc Bài_38====%%%
%%%====Bắt đầu Bài_39=====%%%
\begin{bt}
Cho hai đa thức $A(x) =- 2 x^3 + x^2 - 3 x - 3$ và $B(x)=- 4 x^3 - 3 x^2 + 4 x - 2$
 Hãy tính:$A(x) +B(x)$.
\loigiai{
\begin{eqnarray*}
	\text{Ta có}\ A(x)+B(x)
	&=&(- 2 x^3 + x^2 - 3 x - 3) + (- 4 x^3 - 3 x^2 + 4 x - 2)\\
	&=&(-2-4)x^3+(1-3)x^2+(-3+4)x+(-3-2)\\
	&=&- 6 x^3- 2 x^2+x-5
\end{eqnarray*}
}
\end{bt}
%%%====Kết thúc Bài_39====%%%
%%%====Bắt đầu Bài_40=====%%%
\begin{bt}
Cho hai đa thức $A(x) =- 3 x^3 + 2 x^2 - 3 x + 5$ và $B(x)=x^3 + 3 x^2 + x + 2$
 Hãy tính:$A(x) +B(x)$.
\loigiai{
\begin{eqnarray*}
	\text{Ta có}\ A(x)+B(x)
	&=&(- 3 x^3 + 2 x^2 - 3 x + 5) + (x^3 + 3 x^2 + x + 2)\\
	&=&(-3+1)x^3+(2+3)x^2+(-3+1)x+(5+2)\\
	&=&- 2 x^3+5 x^2- 2 x+7
\end{eqnarray*}
}
\end{bt}
%%%====Kết thúc Bài_40====%%%
%%%====Bắt đầu Bài_41=====%%%
\begin{bt}
Cho hai đa thức $A(x) =- x^3 - 3 x$ và $B(x)=x^3 + 7 x^2 + 3 x - 3$
 Hãy tính:$A(x) +B(x)$.
\loigiai{
\begin{eqnarray*}
	\text{Ta có}\ A(x)+B(x)
	&=&(- x^3 - 3 x) + (x^3 + 7 x^2 + 3 x - 3)\\
	&=&(-1+1)x^3+7x^2+(-3+3)x-3\\
	&=&7 x^2-3
\end{eqnarray*}
}
\end{bt}
%%%====Kết thúc Bài_41====%%%
%%%====Bắt đầu Bài_42=====%%%
\begin{bt}
Cho hai đa thức $A(x) =- 4 x^3 - 3 x - 5$ và $B(x)=- 3 x^3 + 2 x^2 - 2 x - 2$
 Hãy tính:$A(x) +B(x)$.
\loigiai{
\begin{eqnarray*}
	\text{Ta có}\ A(x)+B(x)
	&=&(- 4 x^3 - 3 x - 5) + (- 3 x^3 + 2 x^2 - 2 x - 2)\\
	&=&(-4-3)x^3+2x^2+(-3-2)x+(-5-2)\\
	&=&- 7 x^3+2 x^2- 5 x-7
\end{eqnarray*}
}
\end{bt}
%%%====Kết thúc Bài_42====%%%
%%%====Bắt đầu Bài_43=====%%%
\begin{bt}
Cho hai đa thức $A(x) =- 5 x^3 - x^2 + x - 5$ và $B(x)=5 x^3 + 2 x^2 - 2 x$
 Hãy tính:$A(x) +B(x)$.
\loigiai{
\begin{eqnarray*}
	\text{Ta có}\ A(x)+B(x)
	&=&(- 5 x^3 - x^2 + x - 5) + (5 x^3 + 2 x^2 - 2 x)\\
	&=&(-5+5)x^3+(-1+2)x^2+(1-2)x-5\\
	&=&x^2- x-5
\end{eqnarray*}
}
\end{bt}
%%%====Kết thúc Bài_43====%%%
%%%====Bắt đầu Bài_44=====%%%
\begin{bt}
Cho hai đa thức $A(x) =- 3 x^3 + 2 x^2 - 3 x - 4$ và $B(x)=- 3 x^3 + 3 x^2 + 4 x - 2$
 Hãy tính:$A(x) +B(x)$.
\loigiai{
\begin{eqnarray*}
	\text{Ta có}\ A(x)+B(x)
	&=&(- 3 x^3 + 2 x^2 - 3 x - 4) + (- 3 x^3 + 3 x^2 + 4 x - 2)\\
	&=&(-3-3)x^3+(2+3)x^2+(-3+4)x+(-4-2)\\
	&=&- 6 x^3+5 x^2+x-6
\end{eqnarray*}
}
\end{bt}
%%%====Kết thúc Bài_44====%%%
%%%====Bắt đầu Bài_45=====%%%
\begin{bt}
Cho hai đa thức $A(x) =3 x^3 + 3 x^2 + 2 x + 3$ và $B(x)=6 x^3 + 9 x^2 + 6 x + 8$
 Hãy tính:$A(x) +B(x)$.
\loigiai{
\begin{eqnarray*}
	\text{Ta có}\ A(x)+B(x)
	&=&(3 x^3 + 3 x^2 + 2 x + 3) + (6 x^3 + 9 x^2 + 6 x + 8)\\
	&=&(3+6)x^3+(3+9)x^2+(2+6)x+(3+8)\\
	&=&9 x^3+12 x^2+8 x+11
\end{eqnarray*}
}
\end{bt}
%%%====Kết thúc Bài_45====%%%
%%%====Bắt đầu Bài_46=====%%%
\begin{bt}
Cho hai đa thức $A(x) =x^3 - 2 x^2 + 4$ và $B(x)=- 7 x^3 + x^2 - 3 x + 1$
 Hãy tính:$A(x) +B(x)$.
\loigiai{
\begin{eqnarray*}
	\text{Ta có}\ A(x)+B(x)
	&=&(x^3 - 2 x^2 + 4) + (- 7 x^3 + x^2 - 3 x + 1)\\
	&=&(1-7)x^3+(-2+1)x^2-3x+(4+1)\\
	&=&- 6 x^3- x^2- 3 x+5
\end{eqnarray*}
}
\end{bt}
%%%====Kết thúc Bài_46====%%%
%%%====Bắt đầu Bài_47=====%%%
\begin{bt}
Cho hai đa thức $A(x) =- 3 x^3 + 2 x$ và $B(x)=2 x^3 - 4 x^2 - 7 x + 2$
 Hãy tính:$A(x) +B(x)$.
\loigiai{
\begin{eqnarray*}
	\text{Ta có}\ A(x)+B(x)
	&=&(- 3 x^3 + 2 x) + (2 x^3 - 4 x^2 - 7 x + 2)\\
	&=&(-3+2)x^3-4x^2+(2-7)x+2\\
	&=&- x^3- 4 x^2- 5 x+2
\end{eqnarray*}
}
\end{bt}
%%%====Kết thúc Bài_47====%%%
%%%====Bắt đầu Bài_48=====%%%
\begin{bt}
Cho hai đa thức $A(x) =- 4 x^3 + x^2 - x$ và $B(x)=- 4 x^3 - 3 x^2 - 8 x - 2$
 Hãy tính:$A(x) +B(x)$.
\loigiai{
\begin{eqnarray*}
	\text{Ta có}\ A(x)+B(x)
	&=&(- 4 x^3 + x^2 - x) + (- 4 x^3 - 3 x^2 - 8 x - 2)\\
	&=&(-4-4)x^3+(1-3)x^2+(-1-8)x-2\\
	&=&- 8 x^3- 2 x^2- 9 x-2
\end{eqnarray*}
}
\end{bt}
%%%====Kết thúc Bài_48====%%%
%%%====Bắt đầu Bài_49=====%%%
\begin{bt}
Cho hai đa thức $A(x) =- 3 x^3 - 2 x^2 - 3$ và $B(x)=- 7 x^3 - 6 x^2 - 7 x + 1$
 Hãy tính:$A(x) +B(x)$.
\loigiai{
\begin{eqnarray*}
	\text{Ta có}\ A(x)+B(x)
	&=&(- 3 x^3 - 2 x^2 - 3) + (- 7 x^3 - 6 x^2 - 7 x + 1)\\
	&=&(-3-7)x^3+(-2-6)x^2-7x+(-3+1)\\
	&=&- 10 x^3- 8 x^2- 7 x-2
\end{eqnarray*}
}
\end{bt}
%%%====Kết thúc Bài_49====%%%
%%%====Bắt đầu Bài_50=====%%%
\begin{bt}
Cho hai đa thức $A(x) =4 x^3 + 3 x^2 + 1$ và $B(x)=4 x^3 + x^2 + 2 x + 6$
 Hãy tính:$A(x) +B(x)$.
\loigiai{
\begin{eqnarray*}
	\text{Ta có}\ A(x)+B(x)
	&=&(4 x^3 + 3 x^2 + 1) + (4 x^3 + x^2 + 2 x + 6)\\
	&=&(4+4)x^3+(3+1)x^2+2x+(1+6)\\
	&=&8 x^3+4 x^2+2 x+7
\end{eqnarray*}
}
\end{bt}
%%%====Kết thúc Bài_50====%%%
%%%====Bắt đầu Bài_51=====%%%
\begin{bt}
Cho hai đa thức $A(x) =4 x^3 + 2 x^2 - 2 x$ và $B(x)=7 x^3 + 7 x^2 + x + 5$
 Hãy tính:$A(x) +B(x)$.
\loigiai{
\begin{eqnarray*}
	\text{Ta có}\ A(x)+B(x)
	&=&(4 x^3 + 2 x^2 - 2 x) + (7 x^3 + 7 x^2 + x + 5)\\
	&=&(4+7)x^3+(2+7)x^2+(-2+1)x+5\\
	&=&11 x^3+9 x^2- x+5
\end{eqnarray*}
}
\end{bt}
%%%====Kết thúc Bài_51====%%%
%%%====Bắt đầu Bài_52=====%%%
\begin{bt}
Cho hai đa thức $A(x) =2 x^3 - 4 x^2$ và $B(x)=- 9 x^3 + 7 x^2 + 3 x + 5$
 Hãy tính:$A(x) +B(x)$.
\loigiai{
\begin{eqnarray*}
	\text{Ta có}\ A(x)+B(x)
	&=&(2 x^3 - 4 x^2) + (- 9 x^3 + 7 x^2 + 3 x + 5)\\
	&=&(2-9)x^3+(-4+7)x^2+3x+5\\
	&=&- 7 x^3+3 x^2+3 x+5
\end{eqnarray*}
}
\end{bt}
%%%====Kết thúc Bài_52====%%%
%%%====Bắt đầu Bài_53=====%%%
\begin{bt}
Cho hai đa thức $A(x) =2 x^3 - 2 x + 4$ và $B(x)=5 x^3 + 11 x^2 + 5 x - 3$
 Hãy tính:$A(x) +B(x)$.
\loigiai{
\begin{eqnarray*}
	\text{Ta có}\ A(x)+B(x)
	&=&(2 x^3 - 2 x + 4) + (5 x^3 + 11 x^2 + 5 x - 3)\\
	&=&(2+5)x^3+11x^2+(-2+5)x+(4-3)\\
	&=&7 x^3+11 x^2+3 x+1
\end{eqnarray*}
}
\end{bt}
%%%====Kết thúc Bài_53====%%%
%%%====Bắt đầu Bài_54=====%%%
\begin{bt}
Cho hai đa thức $A(x) =- x^3 + 3 x^2 + x - 4$ và $B(x)=- 2 x^3 + 2 x^2 - 5 x + 2$
 Hãy tính:$A(x) +B(x)$.
\loigiai{
\begin{eqnarray*}
	\text{Ta có}\ A(x)+B(x)
	&=&(- x^3 + 3 x^2 + x - 4) + (- 2 x^3 + 2 x^2 - 5 x + 2)\\
	&=&(-1-2)x^3+(3+2)x^2+(1-5)x+(-4+2)\\
	&=&- 3 x^3+5 x^2- 4 x-2
\end{eqnarray*}
}
\end{bt}
%%%====Kết thúc Bài_54====%%%
%%%====Bắt đầu Bài_55=====%%%
\begin{bt}
Cho hai đa thức $A(x) =3 x^3 + 4$ và $B(x)=- 5 x^3 + 9 x^2 - x + 1$
 Hãy tính:$A(x) +B(x)$.
\loigiai{
\begin{eqnarray*}
	\text{Ta có}\ A(x)+B(x)
	&=&(3 x^3 + 4) + (- 5 x^3 + 9 x^2 - x + 1)\\
	&=&(3-5)x^3+9x^2-x+(4+1)\\
	&=&- 2 x^3+9 x^2- x+5
\end{eqnarray*}
}
\end{bt}
%%%====Kết thúc Bài_55====%%%
%%%====Bắt đầu Bài_56=====%%%
\begin{bt}
Cho hai đa thức $A(x) =4 x^3 + x^2 + x + 3$ và $B(x)=5 x^3 - 3 x^2 + 9$
 Hãy tính:$A(x) +B(x)$.
\loigiai{
\begin{eqnarray*}
	\text{Ta có}\ A(x)+B(x)
	&=&(4 x^3 + x^2 + x + 3) + (5 x^3 - 3 x^2 + 9)\\
	&=&(4+5)x^3+(1-3)x^2+x+(3+9)\\
	&=&9 x^3- 2 x^2+x+12
\end{eqnarray*}
}
\end{bt}
%%%====Kết thúc Bài_56====%%%
%%%====Bắt đầu Bài_57=====%%%
\begin{bt}
Cho hai đa thức $A(x) =4 x^3 - x^2 - 3 x - 2$ và $B(x)=7 x^3 - 5 x^2 - 2 x + 3$
 Hãy tính:$A(x) +B(x)$.
\loigiai{
\begin{eqnarray*}
	\text{Ta có}\ A(x)+B(x)
	&=&(4 x^3 - x^2 - 3 x - 2) + (7 x^3 - 5 x^2 - 2 x + 3)\\
	&=&(4+7)x^3+(-1-5)x^2+(-3-2)x+(-2+3)\\
	&=&11 x^3- 6 x^2- 5 x+1
\end{eqnarray*}
}
\end{bt}
%%%====Kết thúc Bài_57====%%%
%%%====Bắt đầu Bài_58=====%%%
\begin{bt}
Cho hai đa thức $A(x) =- 2 x^3 + 2 x^2 + 2 x + 3$ và $B(x)=6 x^3 + 4 x^2 + 6 x + 8$
 Hãy tính:$A(x) +B(x)$.
\loigiai{
\begin{eqnarray*}
	\text{Ta có}\ A(x)+B(x)
	&=&(- 2 x^3 + 2 x^2 + 2 x + 3) + (6 x^3 + 4 x^2 + 6 x + 8)\\
	&=&(-2+6)x^3+(2+4)x^2+(2+6)x+(3+8)\\
	&=&4 x^3+6 x^2+8 x+11
\end{eqnarray*}
}
\end{bt}
%%%====Kết thúc Bài_58====%%%
%%%====Bắt đầu Bài_59=====%%%
\begin{bt}
Cho hai đa thức $A(x) =- 5 x^3 - 4 x^2 - x - 3$ và $B(x)=- 9 x^3 + 4 x^2 - 5 x$
 Hãy tính:$A(x) +B(x)$.
\loigiai{
\begin{eqnarray*}
	\text{Ta có}\ A(x)+B(x)
	&=&(- 5 x^3 - 4 x^2 - x - 3) + (- 9 x^3 + 4 x^2 - 5 x)\\
	&=&(-5-9)x^3+(-4+4)x^2+(-1-5)x-3\\
	&=&- 14 x^3- 6 x-3
\end{eqnarray*}
}
\end{bt}
%%%====Kết thúc Bài_59====%%%
%%%====Bắt đầu Bài_60=====%%%
\begin{bt}
Cho hai đa thức $A(x) =- 3 x^3 - 3 x^2 - x - 2$ và $B(x)=3 x^3 + 5 x^2 + 6 x$
 Hãy tính:$A(x) +B(x)$.
\loigiai{
\begin{eqnarray*}
	\text{Ta có}\ A(x)+B(x)
	&=&(- 3 x^3 - 3 x^2 - x - 2) + (3 x^3 + 5 x^2 + 6 x)\\
	&=&(-3+3)x^3+(-3+5)x^2+(-1+6)x-2\\
	&=&2 x^2+5 x-2
\end{eqnarray*}
}
\end{bt}
%%%====Kết thúc Bài_60====%%%
%%%====Bắt đầu Bài_61=====%%%
\begin{bt}
Cho hai đa thức $A(x) =- 4 x^3 + 2 x - 3$ và $B(x)=- x^3 + 4 x^2 + 9 x + 2$
 Hãy tính:$A(x) +B(x)$.
\loigiai{
\begin{eqnarray*}
	\text{Ta có}\ A(x)+B(x)
	&=&(- 4 x^3 + 2 x - 3) + (- x^3 + 4 x^2 + 9 x + 2)\\
	&=&(-4-1)x^3+4x^2+(2+9)x+(-3+2)\\
	&=&- 5 x^3+4 x^2+11 x-1
\end{eqnarray*}
}
\end{bt}
%%%====Kết thúc Bài_61====%%%
%%%====Bắt đầu Bài_62=====%%%
\begin{bt}
Cho hai đa thức $A(x) =- 5 x^3 + x^2 - x - 6$ và $B(x)=- 4 x^3 - 3 x^2 - 2$
 Hãy tính:$A(x) +B(x)$.
\loigiai{
\begin{eqnarray*}
	\text{Ta có}\ A(x)+B(x)
	&=&(- 5 x^3 + x^2 - x - 6) + (- 4 x^3 - 3 x^2 - 2)\\
	&=&(-5-4)x^3+(1-3)x^2-x+(-6-2)\\
	&=&- 9 x^3- 2 x^2- x-8
\end{eqnarray*}
}
\end{bt}
%%%====Kết thúc Bài_62====%%%
%%%====Bắt đầu Bài_63=====%%%
\begin{bt}
Cho hai đa thức $A(x) =- 5 x^3 + 2 x - 2$ và $B(x)=6 x^3 - 4 x^2 + 9 x$
 Hãy tính:$A(x) +B(x)$.
\loigiai{
\begin{eqnarray*}
	\text{Ta có}\ A(x)+B(x)
	&=&(- 5 x^3 + 2 x - 2) + (6 x^3 - 4 x^2 + 9 x)\\
	&=&(-5+6)x^3-4x^2+(2+9)x-2\\
	&=&x^3- 4 x^2+11 x-2
\end{eqnarray*}
}
\end{bt}
%%%====Kết thúc Bài_63====%%%
%%%====Bắt đầu Bài_64=====%%%
\begin{bt}
Cho hai đa thức $A(x) =x^3 - 2 x^2 - 3 x - 2$ và $B(x)=4 x^3 + 5 x^2 + 4 x + 6$
 Hãy tính:$A(x) +B(x)$.
\loigiai{
\begin{eqnarray*}
	\text{Ta có}\ A(x)+B(x)
	&=&(x^3 - 2 x^2 - 3 x - 2) + (4 x^3 + 5 x^2 + 4 x + 6)\\
	&=&(1+4)x^3+(-2+5)x^2+(-3+4)x+(-2+6)\\
	&=&5 x^3+3 x^2+x+4
\end{eqnarray*}
}
\end{bt}
%%%====Kết thúc Bài_64====%%%
%%%====Bắt đầu Bài_65=====%%%
\begin{bt}
Cho hai đa thức $A(x) =- 3 x^3 - 2 x^2 - 2 x - 2$ và $B(x)=- 7 x^3 - 6 x^2 + x + 3$
 Hãy tính:$A(x) +B(x)$.
\loigiai{
\begin{eqnarray*}
	\text{Ta có}\ A(x)+B(x)
	&=&(- 3 x^3 - 2 x^2 - 2 x - 2) + (- 7 x^3 - 6 x^2 + x + 3)\\
	&=&(-3-7)x^3+(-2-6)x^2+(-2+1)x+(-2+3)\\
	&=&- 10 x^3- 8 x^2- x+1
\end{eqnarray*}
}
\end{bt}
%%%====Kết thúc Bài_65====%%%
%%%====Bắt đầu Bài_66=====%%%
\begin{bt}
Cho hai đa thức $A(x) =- x^3 - 3 x + 5$ và $B(x)=- 5 x^3 - 2 x^2 + 4 x + 10$
 Hãy tính:$A(x) +B(x)$.
\loigiai{
\begin{eqnarray*}
	\text{Ta có}\ A(x)+B(x)
	&=&(- x^3 - 3 x + 5) + (- 5 x^3 - 2 x^2 + 4 x + 10)\\
	&=&(-1-5)x^3-2x^2+(-3+4)x+(5+10)\\
	&=&- 6 x^3- 2 x^2+x+15
\end{eqnarray*}
}
\end{bt}
%%%====Kết thúc Bài_66====%%%
%%%====Bắt đầu Bài_67=====%%%
\begin{bt}
Cho hai đa thức $A(x) =- x^3 - 3 x^2 + 4$ và $B(x)=6 x^3 + 11 x^2 + 7 x + 1$
 Hãy tính:$A(x) +B(x)$.
\loigiai{
\begin{eqnarray*}
	\text{Ta có}\ A(x)+B(x)
	&=&(- x^3 - 3 x^2 + 4) + (6 x^3 + 11 x^2 + 7 x + 1)\\
	&=&(-1+6)x^3+(-3+11)x^2+7x+(4+1)\\
	&=&5 x^3+8 x^2+7 x+5
\end{eqnarray*}
}
\end{bt}
%%%====Kết thúc Bài_67====%%%
%%%====Bắt đầu Bài_68=====%%%
\begin{bt}
Cho hai đa thức $A(x) =4 x^3 - 3 x^2 - 2 x - 2$ và $B(x)=- 8 x^3 + 5 x^2 - 6$
 Hãy tính:$A(x) +B(x)$.
\loigiai{
\begin{eqnarray*}
	\text{Ta có}\ A(x)+B(x)
	&=&(4 x^3 - 3 x^2 - 2 x - 2) + (- 8 x^3 + 5 x^2 - 6)\\
	&=&(4-8)x^3+(-3+5)x^2-2x+(-2-6)\\
	&=&- 4 x^3+2 x^2- 2 x-8
\end{eqnarray*}
}
\end{bt}
%%%====Kết thúc Bài_68====%%%
%%%====Bắt đầu Bài_69=====%%%
\begin{bt}
Cho hai đa thức $A(x) =- 3 x^3 - 4 x^2 - 2 x - 6$ và $B(x)=2 x^3 + x^2 + 5 x + 2$
 Hãy tính:$A(x) +B(x)$.
\loigiai{
\begin{eqnarray*}
	\text{Ta có}\ A(x)+B(x)
	&=&(- 3 x^3 - 4 x^2 - 2 x - 6) + (2 x^3 + x^2 + 5 x + 2)\\
	&=&(-3+2)x^3+(-4+1)x^2+(-2+5)x+(-6+2)\\
	&=&- x^3- 3 x^2+3 x-4
\end{eqnarray*}
}
\end{bt}
%%%====Kết thúc Bài_69====%%%
%%%====Bắt đầu Bài_70=====%%%
\begin{bt}
Cho hai đa thức $A(x) =- 4 x^3 + 2 x^2 + x - 3$ và $B(x)=5 x^3 + 4 x^2 + 8 x - 1$
 Hãy tính:$A(x) +B(x)$.
\loigiai{
\begin{eqnarray*}
	\text{Ta có}\ A(x)+B(x)
	&=&(- 4 x^3 + 2 x^2 + x - 3) + (5 x^3 + 4 x^2 + 8 x - 1)\\
	&=&(-4+5)x^3+(2+4)x^2+(1+8)x+(-3-1)\\
	&=&x^3+6 x^2+9 x-4
\end{eqnarray*}
}
\end{bt}
%%%====Kết thúc Bài_70====%%%
%%%====Bắt đầu Bài_71=====%%%
\begin{bt}
Cho hai đa thức $A(x) =3 x^3 - 4 x^2 + 2 x - 1$ và $B(x)=- 9 x^3 + 6 x^2 - 5 x + 3$
 Hãy tính:$A(x) +B(x)$.
\loigiai{
\begin{eqnarray*}
	\text{Ta có}\ A(x)+B(x)
	&=&(3 x^3 - 4 x^2 + 2 x - 1) + (- 9 x^3 + 6 x^2 - 5 x + 3)\\
	&=&(3-9)x^3+(-4+6)x^2+(2-5)x+(-1+3)\\
	&=&- 6 x^3+2 x^2- 3 x+2
\end{eqnarray*}
}
\end{bt}
%%%====Kết thúc Bài_71====%%%
%%%====Bắt đầu Bài_72=====%%%
\begin{bt}
Cho hai đa thức $A(x) =x^3 + x^2 - x - 2$ và $B(x)=- 4 x^3 + 7 x^2 + x$
 Hãy tính:$A(x) +B(x)$.
\loigiai{
\begin{eqnarray*}
	\text{Ta có}\ A(x)+B(x)
	&=&(x^3 + x^2 - x - 2) + (- 4 x^3 + 7 x^2 + x)\\
	&=&(1-4)x^3+(1+7)x^2+(-1+1)x-2\\
	&=&- 3 x^3+8 x^2-2
\end{eqnarray*}
}
\end{bt}
%%%====Kết thúc Bài_72====%%%
%%%====Bắt đầu Bài_73=====%%%
\begin{bt}
Cho hai đa thức $A(x) =- 2 x^3 - 3 x^2 + 2 x + 4$ và $B(x)=x^3 + 11 x^2 + 7 x + 3$
 Hãy tính:$A(x) +B(x)$.
\loigiai{
\begin{eqnarray*}
	\text{Ta có}\ A(x)+B(x)
	&=&(- 2 x^3 - 3 x^2 + 2 x + 4) + (x^3 + 11 x^2 + 7 x + 3)\\
	&=&(-2+1)x^3+(-3+11)x^2+(2+7)x+(4+3)\\
	&=&- x^3+8 x^2+9 x+7
\end{eqnarray*}
}
\end{bt}
%%%====Kết thúc Bài_73====%%%
%%%====Bắt đầu Bài_74=====%%%
\begin{bt}
Cho hai đa thức $A(x) =x^3 - x^2 - x + 1$ và $B(x)=3 x^3 + 8 x^2 + 4 x$
 Hãy tính:$A(x) +B(x)$.
\loigiai{
\begin{eqnarray*}
	\text{Ta có}\ A(x)+B(x)
	&=&(x^3 - x^2 - x + 1) + (3 x^3 + 8 x^2 + 4 x)\\
	&=&(1+3)x^3+(-1+8)x^2+(-1+4)x+1\\
	&=&4 x^3+7 x^2+3 x+1
\end{eqnarray*}
}
\end{bt}
%%%====Kết thúc Bài_74====%%%
%%%====Bắt đầu Bài_75=====%%%
\begin{bt}
Cho hai đa thức $A(x) =2 x^3 + 2 x^2 - 2$ và $B(x)=4 x^3 - 2 x^2 + x - 1$
 Hãy tính:$A(x) +B(x)$.
\loigiai{
\begin{eqnarray*}
	\text{Ta có}\ A(x)+B(x)
	&=&(2 x^3 + 2 x^2 - 2) + (4 x^3 - 2 x^2 + x - 1)\\
	&=&(2+4)x^3+(2-2)x^2+x+(-2-1)\\
	&=&6 x^3+x-3
\end{eqnarray*}
}
\end{bt}
%%%====Kết thúc Bài_75====%%%
%%%====Bắt đầu Bài_76=====%%%
\begin{bt}
Cho hai đa thức $A(x) =- 3 x^3 - 3 x^2 + x + 5$ và $B(x)=7 x^3 + 12 x^2 + 8 x + 2$
 Hãy tính:$A(x) +B(x)$.
\loigiai{
\begin{eqnarray*}
	\text{Ta có}\ A(x)+B(x)
	&=&(- 3 x^3 - 3 x^2 + x + 5) + (7 x^3 + 12 x^2 + 8 x + 2)\\
	&=&(-3+7)x^3+(-3+12)x^2+(1+8)x+(5+2)\\
	&=&4 x^3+9 x^2+9 x+7
\end{eqnarray*}
}
\end{bt}
%%%====Kết thúc Bài_76====%%%
%%%====Bắt đầu Bài_77=====%%%
\begin{bt}
Cho hai đa thức $A(x) =2 x^3 + 3 x^2$ và $B(x)=5 x^3 - x^2 + 2 x + 1$
 Hãy tính:$A(x) +B(x)$.
\loigiai{
\begin{eqnarray*}
	\text{Ta có}\ A(x)+B(x)
	&=&(2 x^3 + 3 x^2) + (5 x^3 - x^2 + 2 x + 1)\\
	&=&(2+5)x^3+(3-1)x^2+2x+1\\
	&=&7 x^3+2 x^2+2 x+1
\end{eqnarray*}
}
\end{bt}
%%%====Kết thúc Bài_77====%%%
%%%====Bắt đầu Bài_78=====%%%
\begin{bt}
Cho hai đa thức $A(x) =4 x^3 - x - 5$ và $B(x)=3 x^3 + 10 x^2 - 3$
 Hãy tính:$A(x) +B(x)$.
\loigiai{
\begin{eqnarray*}
	\text{Ta có}\ A(x)+B(x)
	&=&(4 x^3 - x - 5) + (3 x^3 + 10 x^2 - 3)\\
	&=&(4+3)x^3+10x^2-x+(-5-3)\\
	&=&7 x^3+10 x^2- x-8
\end{eqnarray*}
}
\end{bt}
%%%====Kết thúc Bài_78====%%%
%%%====Bắt đầu Bài_79=====%%%
\begin{bt}
Cho hai đa thức $A(x) =4 x^3 + 2 x^2 + 2 x + 4$ và $B(x)=- 3 x^3 - 2 x^2 + 9$
 Hãy tính:$A(x) +B(x)$.
\loigiai{
\begin{eqnarray*}
	\text{Ta có}\ A(x)+B(x)
	&=&(4 x^3 + 2 x^2 + 2 x + 4) + (- 3 x^3 - 2 x^2 + 9)\\
	&=&(4-3)x^3+(2-2)x^2+2x+(4+9)\\
	&=&x^3+2 x+13
\end{eqnarray*}
}
\end{bt}
%%%====Kết thúc Bài_79====%%%
%%%====Bắt đầu Bài_80=====%%%
\begin{bt}
Cho hai đa thức $A(x) =- 2 x^3 - 2 x + 6$ và $B(x)=2 x^3 - 4 x^2 + 5 x + 11$
 Hãy tính:$A(x) +B(x)$.
\loigiai{
\begin{eqnarray*}
	\text{Ta có}\ A(x)+B(x)
	&=&(- 2 x^3 - 2 x + 6) + (2 x^3 - 4 x^2 + 5 x + 11)\\
	&=&(-2+2)x^3-4x^2+(-2+5)x+(6+11)\\
	&=&- 4 x^2+3 x+17
\end{eqnarray*}
}
\end{bt}
%%%====Kết thúc Bài_80====%%%
%%%====Bắt đầu Bài_81=====%%%
\begin{bt}
Cho hai đa thức $A(x) =2 x^3 + 2 x^2 + x + 4$ và $B(x)=- 3 x^3 + 11 x^2 + 7 x + 7$
 Hãy tính:$A(x) +B(x)$.
\loigiai{
\begin{eqnarray*}
	\text{Ta có}\ A(x)+B(x)
	&=&(2 x^3 + 2 x^2 + x + 4) + (- 3 x^3 + 11 x^2 + 7 x + 7)\\
	&=&(2-3)x^3+(2+11)x^2+(1+7)x+(4+7)\\
	&=&- x^3+13 x^2+8 x+11
\end{eqnarray*}
}
\end{bt}
%%%====Kết thúc Bài_81====%%%
%%%====Bắt đầu Bài_82=====%%%
\begin{bt}
Cho hai đa thức $A(x) =4 - 2 x^3$ và $B(x)=x^3 + 4 x^2 + 7 x + 9$
 Hãy tính:$A(x) +B(x)$.
\loigiai{
\begin{eqnarray*}
	\text{Ta có}\ A(x)+B(x)
	&=&(4 - 2 x^3) + (x^3 + 4 x^2 + 7 x + 9)\\
	&=&(-2+1)x^3+4x^2+7x+(4+9)\\
	&=&- x^3+4 x^2+7 x+13
\end{eqnarray*}
}
\end{bt}
%%%====Kết thúc Bài_82====%%%
%%%====Bắt đầu Bài_83=====%%%
\begin{bt}
Cho hai đa thức $A(x) =3 x^3 - 3 x^2 - 3 x - 2$ và $B(x)=6 x^3 - 2 x^2 - x + 3$
 Hãy tính:$A(x) +B(x)$.
\loigiai{
\begin{eqnarray*}
	\text{Ta có}\ A(x)+B(x)
	&=&(3 x^3 - 3 x^2 - 3 x - 2) + (6 x^3 - 2 x^2 - x + 3)\\
	&=&(3+6)x^3+(-3-2)x^2+(-3-1)x+(-2+3)\\
	&=&9 x^3- 5 x^2- 4 x+1
\end{eqnarray*}
}
\end{bt}
%%%====Kết thúc Bài_83====%%%
%%%====Bắt đầu Bài_84=====%%%
\begin{bt}
Cho hai đa thức $A(x) =3 x^3 + 3 x^2 + 2 x + 1$ và $B(x)=6 x^3 + 9 x^2 - x$
 Hãy tính:$A(x) +B(x)$.
\loigiai{
\begin{eqnarray*}
	\text{Ta có}\ A(x)+B(x)
	&=&(3 x^3 + 3 x^2 + 2 x + 1) + (6 x^3 + 9 x^2 - x)\\
	&=&(3+6)x^3+(3+9)x^2+(2-1)x+1\\
	&=&9 x^3+12 x^2+x+1
\end{eqnarray*}
}
\end{bt}
%%%====Kết thúc Bài_84====%%%
%%%====Bắt đầu Bài_85=====%%%
\begin{bt}
Cho hai đa thức $A(x) =4 x^3 - 2 x^2 - x + 5$ và $B(x)=7 x^3 + 10 x^2 + 8 x + 9$
 Hãy tính:$A(x) +B(x)$.
\loigiai{
\begin{eqnarray*}
	\text{Ta có}\ A(x)+B(x)
	&=&(4 x^3 - 2 x^2 - x + 5) + (7 x^3 + 10 x^2 + 8 x + 9)\\
	&=&(4+7)x^3+(-2+10)x^2+(-1+8)x+(5+9)\\
	&=&11 x^3+8 x^2+7 x+14
\end{eqnarray*}
}
\end{bt}
%%%====Kết thúc Bài_85====%%%
%%%====Bắt đầu Bài_86=====%%%
\begin{bt}
Cho hai đa thức $A(x) =- 2 x^3 + 3 x^2 - 3 x + 5$ và $B(x)=7 x^3 - 2 x^2 + 8 x + 10$
 Hãy tính:$A(x) +B(x)$.
\loigiai{
\begin{eqnarray*}
	\text{Ta có}\ A(x)+B(x)
	&=&(- 2 x^3 + 3 x^2 - 3 x + 5) + (7 x^3 - 2 x^2 + 8 x + 10)\\
	&=&(-2+7)x^3+(3-2)x^2+(-3+8)x+(5+10)\\
	&=&5 x^3+x^2+5 x+15
\end{eqnarray*}
}
\end{bt}
%%%====Kết thúc Bài_86====%%%
%%%====Bắt đầu Bài_87=====%%%
\begin{bt}
Cho hai đa thức $A(x) =- x^3 + 2 x^2 - 2 x + 3$ và $B(x)=- 3 x^3 - 2 x^2 + x + 8$
 Hãy tính:$A(x) +B(x)$.
\loigiai{
\begin{eqnarray*}
	\text{Ta có}\ A(x)+B(x)
	&=&(- x^3 + 2 x^2 - 2 x + 3) + (- 3 x^3 - 2 x^2 + x + 8)\\
	&=&(-1-3)x^3+(2-2)x^2+(-2+1)x+(3+8)\\
	&=&- 4 x^3- x+11
\end{eqnarray*}
}
\end{bt}
%%%====Kết thúc Bài_87====%%%
%%%====Bắt đầu Bài_88=====%%%
\begin{bt}
Cho hai đa thức $A(x) =- 3 x^3 - 4 x^2 - 3 x + 1$ và $B(x)=3 x^3 - 2 x^2 - 7 x - 7$
 Hãy tính:$A(x) +B(x)$.
\loigiai{
\begin{eqnarray*}
	\text{Ta có}\ A(x)+B(x)
	&=&(- 3 x^3 - 4 x^2 - 3 x + 1) + (3 x^3 - 2 x^2 - 7 x - 7)\\
	&=&(-3+3)x^3+(-4-2)x^2+(-3-7)x+(1-7)\\
	&=&- 6 x^2- 10 x-6
\end{eqnarray*}
}
\end{bt}
%%%====Kết thúc Bài_88====%%%
%%%====Bắt đầu Bài_89=====%%%
\begin{bt}
Cho hai đa thức $A(x) =2 x^3 - 3 x - 2$ và $B(x)=x^3 - 4 x^2 + 4 x + 7$
 Hãy tính:$A(x) +B(x)$.
\loigiai{
\begin{eqnarray*}
	\text{Ta có}\ A(x)+B(x)
	&=&(2 x^3 - 3 x - 2) + (x^3 - 4 x^2 + 4 x + 7)\\
	&=&(2+1)x^3-4x^2+(-3+4)x+(-2+7)\\
	&=&3 x^3- 4 x^2+x+5
\end{eqnarray*}
}
\end{bt}
%%%====Kết thúc Bài_89====%%%
%%%====Bắt đầu Bài_90=====%%%
\begin{bt}
Cho hai đa thức $A(x) =- 3 x^3 - 4 x^2 + 2 x + 5$ và $B(x)=- 9 x^3 - 8 x^2 + 8 x + 10$
 Hãy tính:$A(x) +B(x)$.
\loigiai{
\begin{eqnarray*}
	\text{Ta có}\ A(x)+B(x)
	&=&(- 3 x^3 - 4 x^2 + 2 x + 5) + (- 9 x^3 - 8 x^2 + 8 x + 10)\\
	&=&(-3-9)x^3+(-4-8)x^2+(2+8)x+(5+10)\\
	&=&- 12 x^3- 12 x^2+10 x+15
\end{eqnarray*}
}
\end{bt}
%%%====Kết thúc Bài_90====%%%
%%%====Bắt đầu Bài_91=====%%%
\begin{bt}
Cho hai đa thức $A(x) =- 4 x^3 - x^2 + 2 x + 4$ và $B(x)=- 6 x^3 + 3 x^2 - 2 x + 3$
 Hãy tính:$A(x) +B(x)$.
\loigiai{
\begin{eqnarray*}
	\text{Ta có}\ A(x)+B(x)
	&=&(- 4 x^3 - x^2 + 2 x + 4) + (- 6 x^3 + 3 x^2 - 2 x + 3)\\
	&=&(-4-6)x^3+(-1+3)x^2+(2-2)x+(4+3)\\
	&=&- 10 x^3+2 x^2+7
\end{eqnarray*}
}
\end{bt}
%%%====Kết thúc Bài_91====%%%
%%%====Bắt đầu Bài_92=====%%%
\begin{bt}
Cho hai đa thức $A(x) =2 x^3 + 2 x^2 - 3 x + 1$ và $B(x)=3 x^3 - 2 x^2 + x - 2$
 Hãy tính:$A(x) +B(x)$.
\loigiai{
\begin{eqnarray*}
	\text{Ta có}\ A(x)+B(x)
	&=&(2 x^3 + 2 x^2 - 3 x + 1) + (3 x^3 - 2 x^2 + x - 2)\\
	&=&(2+3)x^3+(2-2)x^2+(-3+1)x+(1-2)\\
	&=&5 x^3- 2 x-1
\end{eqnarray*}
}
\end{bt}
%%%====Kết thúc Bài_92====%%%
%%%====Bắt đầu Bài_93=====%%%
\begin{bt}
Cho hai đa thức $A(x) =- x^3 - 3 x^2 + x + 4$ và $B(x)=- 8 x^3 + 2 x^2 - 4 x + 2$
 Hãy tính:$A(x) +B(x)$.
\loigiai{
\begin{eqnarray*}
	\text{Ta có}\ A(x)+B(x)
	&=&(- x^3 - 3 x^2 + x + 4) + (- 8 x^3 + 2 x^2 - 4 x + 2)\\
	&=&(-1-8)x^3+(-3+2)x^2+(1-4)x+(4+2)\\
	&=&- 9 x^3- x^2- 3 x+6
\end{eqnarray*}
}
\end{bt}
%%%====Kết thúc Bài_93====%%%
%%%====Bắt đầu Bài_94=====%%%
\begin{bt}
Cho hai đa thức $A(x) =- 2 x^3 - 3 x^2 + x + 2$ và $B(x)=4 x^3 + 4 x^2 + 5 x + 3$
 Hãy tính:$A(x) +B(x)$.
\loigiai{
\begin{eqnarray*}
	\text{Ta có}\ A(x)+B(x)
	&=&(- 2 x^3 - 3 x^2 + x + 2) + (4 x^3 + 4 x^2 + 5 x + 3)\\
	&=&(-2+4)x^3+(-3+4)x^2+(1+5)x+(2+3)\\
	&=&2 x^3+x^2+6 x+5
\end{eqnarray*}
}
\end{bt}
%%%====Kết thúc Bài_94====%%%
%%%====Bắt đầu Bài_95=====%%%
\begin{bt}
Cho hai đa thức $A(x) =- x^3 + 2 x^2 - 3 x - 6$ và $B(x)=- 3 x^3 + 5 x^2 + 4 x - 2$
 Hãy tính:$A(x) +B(x)$.
\loigiai{
\begin{eqnarray*}
	\text{Ta có}\ A(x)+B(x)
	&=&(- x^3 + 2 x^2 - 3 x - 6) + (- 3 x^3 + 5 x^2 + 4 x - 2)\\
	&=&(-1-3)x^3+(2+5)x^2+(-3+4)x+(-6-2)\\
	&=&- 4 x^3+7 x^2+x-8
\end{eqnarray*}
}
\end{bt}
%%%====Kết thúc Bài_95====%%%
%%%====Bắt đầu Bài_96=====%%%
\begin{bt}
Cho hai đa thức $A(x) =- 5 x^3 + 3 x^2 - 2 x - 5$ và $B(x)=- 3 x^3 - x^2 + 2 x$
 Hãy tính:$A(x) +B(x)$.
\loigiai{
\begin{eqnarray*}
	\text{Ta có}\ A(x)+B(x)
	&=&(- 5 x^3 + 3 x^2 - 2 x - 5) + (- 3 x^3 - x^2 + 2 x)\\
	&=&(-5-3)x^3+(3-1)x^2+(-2+2)x-5\\
	&=&- 8 x^3+2 x^2-5
\end{eqnarray*}
}
\end{bt}
%%%====Kết thúc Bài_96====%%%
%%%====Bắt đầu Bài_97=====%%%
\begin{bt}
Cho hai đa thức $A(x) =- 3 x^3 - x^2 - 2 x - 5$ và $B(x)=2 x^3 - 5 x^2 - 2 x$
 Hãy tính:$A(x) +B(x)$.
\loigiai{
\begin{eqnarray*}
	\text{Ta có}\ A(x)+B(x)
	&=&(- 3 x^3 - x^2 - 2 x - 5) + (2 x^3 - 5 x^2 - 2 x)\\
	&=&(-3+2)x^3+(-1-5)x^2+(-2-2)x-5\\
	&=&- x^3- 6 x^2- 4 x-5
\end{eqnarray*}
}
\end{bt}
%%%====Kết thúc Bài_97====%%%
%%%====Bắt đầu Bài_98=====%%%
\begin{bt}
Cho hai đa thức $A(x) =- 5 x^3 - 4 x^2 + 2 x + 4$ và $B(x)=- 2 x^3 - 8 x^2 - 9 x$
 Hãy tính:$A(x) +B(x)$.
\loigiai{
\begin{eqnarray*}
	\text{Ta có}\ A(x)+B(x)
	&=&(- 5 x^3 - 4 x^2 + 2 x + 4) + (- 2 x^3 - 8 x^2 - 9 x)\\
	&=&(-5-2)x^3+(-4-8)x^2+(2-9)x+4\\
	&=&- 7 x^3- 12 x^2- 7 x+4
\end{eqnarray*}
}
\end{bt}
%%%====Kết thúc Bài_98====%%%
%%%====Bắt đầu Bài_99=====%%%
\begin{bt}
Cho hai đa thức $A(x) =- 2 x^3 - 4 x^2 + 2 x - 3$ và $B(x)=- x^3 + 3 x^2 + 3$
 Hãy tính:$A(x) +B(x)$.
\loigiai{
\begin{eqnarray*}
	\text{Ta có}\ A(x)+B(x)
	&=&(- 2 x^3 - 4 x^2 + 2 x - 3) + (- x^3 + 3 x^2 + 3)\\
	&=&(-2-1)x^3+(-4+3)x^2+2x+(-3+3)\\
	&=&- 3 x^3- x^2+2 x
\end{eqnarray*}
}
\end{bt}
%%%====Kết thúc Bài_99====%%%
%%%====Bắt đầu Bài_100====%%%
\begin{bt}
Cho hai đa thức $A(x) =3 x^3 - 4 x^2 - 3 x + 6$ và $B(x)=x^3 - 8 x^2 - x + 11$
 Hãy tính:$A(x) +B(x)$.
\loigiai{
\begin{eqnarray*}
	\text{Ta có}\ A(x)+B(x)
	&=&(3 x^3 - 4 x^2 - 3 x + 6) + (x^3 - 8 x^2 - x + 11)\\
	&=&(3+1)x^3+(-4-8)x^2+(-3-1)x+(6+11)\\
	&=&4 x^3- 12 x^2- 4 x+17
\end{eqnarray*}
}
\end{bt}
%%%===Kết thúc Bài_100====%%%