\begin{name}[Kiểm tra cuối kì II][Hóa][9][Sở Giáo dục và Đào tạo]{Trường THCS}{2023 - 2024}
\end{name}
\tieumuc{Trắc nghiệm (5 điểm)}
\Opensolutionfile{ans}[Ans/DATN-9CKII-De-01]
%\luuloigiaiex
%\hienthiloigiaiex
%\taoNdongke[5]{ex}
%\tatloigiaiex
\Opensolutionfile{ansex}[LOIGIAITN/LGTN-9CKII-De-01]
%%%============EX_1==============%%%
\begin{ex}
	Hợp chất $X$ là chất rắn màu trắng, không tan trong nước ngay cà khi đun nóng. Chất $X$ là thình phản chù yếu trong sợ bông, tre, nứa, gỗ,$\ldots$ Chất X là
	\choice
	{glucozo}
	{saccarozo}
	{tinh bôt}
	{\True xenlulozo}
	\loigiai{Xen lulozo là chất rắn , màu trắng, không tan trong nước ngay cả khi đun nóng và có nhiều trong sợi bông, tre, nứa, gỗ.}
\end{ex}
%%%============EX_2==============%%%
\begin{ex}
	Khi tiến hành thủy phân tinh bột hoặc xenlulozơ thì cần có chất xúc tác nào sau đây?
	\choice
	{Dung dịch nước vôi}
	{Dung dịch muối ăn}
	{Dung dịch bazơ}
	{\True Dung dịch axit loãng}
	\loigiai{Tinh bột và xenlulozơ bị thủy phân trong môi trường axit}
\end{ex}
%%%============EX_3==============%%%
\begin{ex}Cho sơ đồ chuyển hóa sau (mỗi mũi tên là một phương trình phản ứng):\\ 
	Tinh bột $\rightarrow$ X $\rightarrow$ Y $\rightarrow$ Z $\rightarrow$ etyl axetat.
	Các chất $Y, Z$ trong sơ đồ trên lần lượt là:
	\choice
	{\True $C_2H_5OH$, $CH_3COOH$}
	{$CH_3COOH, CH_3OH$}
	{$CH_3COOH, C_2H_5OH$}
	{$C_2H_4$, $CH_3COOH$}
	\loigiai{
	\begin{enumerate}[a)]
		\item \puhh[$H^+$][enzim][][][-Stealth]{$(C_6H_{10}O_5)_n$ \+ $nH_2O$}{->}{$nC_6H_{12}O_6$}
		\item \puhh[men rượu ][$30-35^\circ$][0.9][][-Stealth]{$C_6H_{12}O_6$}{->}{$2C_2H_5OH$\+$2CO_2$}
		\item \puhh[men giấm ][][0.9][][-Stealth]{$C_2H_5OH$\+ $O_2$}{->}{$CH_3COOH$\+$H_2O$}
		\item \puhh[${H_2SO_4}_{\text{đặc}}$][$t^\circ$][0.9][]{$CH_3COOH$\+ $C_2H_5OH$}{<=>}{$CH_3COOC_2H_5$\+$H_2O$}
	\end{enumerate} 
	Vậy Y, Z lần lượt là $2C_2H_5OH$ và $CH_3COOH$.
	}
\end{ex}
%%%============EX_4==============%%%
\begin{ex}
	Trong thành phần cấu tạo phân tử của protein ngoài các nguyên tố C, H, O thì nhất thiết phải có nguyên tố
	\choice
	{lưu huỳnh}
	{sắt}
	{clo}
	{\True nitơ}
	\loigiai{Trong phân tử protein có nhóm amino ($NH_2$) và nhóm axit ($COOH$)}
\end{ex}
%%%============EX_5==============%%%
\begin{ex}
	Tơ nilon được gọi là
	\choice
	{Tơ thiên nhiên}
	{\True Tơ tổng hợp}
	{Tơ nhân tạo}
	{Vừa là tơ nhân tạo vừa là tơ thiên nhiên}
	\loigiai{Tơ nilon thuộc loại tơ tổng hợp}
\end{ex}
%%%============EX_6==============%%%
\begin{ex}
	Để phân biệt vải dệt bằng tơ tằm và vải dệt bằng sợi bông. Chúng ta có thể
	\choice
	{gia nhiệt để thực hiện phàn ứng đông tụ}
	{\True đốt và ngửi nếu có mùi khét là vải bằng tơ tằm}
	{dùng quỳ tím}
	{dùng phản ứng thủy phân}
	\loigiai{vải làm từ tơ tằm có thành phần protein khi đốt có mùi khét còn vải làm từ sợi bông có thành phần xenlulozo khi đốt không có mùi khét}
\end{ex}
%%%============EX_7==============%%%
\begin{ex}
	Khi cho 9,2 gam ancol etylic tác dụng với Na vừa đủ, thu được V lít H $_2$ (đktc). Giá trị của V là
	\choice
	{\True 2,24}
	{6,72}
	{4,48}
	{3,36}
	\loigiai{
		 $\begin{array}{ccccccccc}
			\text{\textbf{PTHH:} }&C_2H_5OH & \explus & Na & \xrightarrow[]{\makebox[1.2cm]{}} & C_2H_5ONa & \explus & \dfrac{1}{2}H_2&\\
			&0{,}2& & & & &\xrightarrow[]{\makebox[0.65cm]{}} &0{,}1 &(mol)
		\end{array}$\\
	Ta có $n_{C_2H_5OH}=\dfrac{9.2}{46}=0{,}2$ (mol). Theo phương trình hóa học $n_{H_2}=\dfrac{1}{2}n_{C_2H_5OH}=\dfrac{1}{2}\cdot 0.2 =0{,}1 $ (mol).\\
	Vậy thể tích $H_2$ thu được sau phản ứng là: $V_{H_2} = 0{,}1\cdot22{,}4=2{,}24$ (lít).
	}
\end{ex}
%%%============EX_8==============%%%
\begin{ex}
	Thể tích dung dịch NaOH 1M cần để trung hòa vừa đủ 200 gam dung dịch axit axetic 6\% là
	\choice
	{$100$ ml}
	{\True 200 ml}
	{$300$ ml}
	{$400$ ml}
	\loigiai{}
\end{ex}
%%%============EX_9==============%%%
\begin{ex}Thủy phân hoàn toàn m gam chất béo bằng dung dịch NaOH, đun nóng thu được $9{,}2$ gam glixerol và $91{,}8$ gam muối. Giá trị của m là
	\choice
	{\True $89$}
	{$101$}
	{$85$}
	{$93$}
	\loigiai{}
\end{ex}
%%%============EX_10==============%%%
\begin{ex}
	Cho 90 gam glucozơ lên men rượu với hiệu suất $80\%$, thu được $m$ gam  $C_2H_5OH$ . Giá trị của m là
	\choice
	{\True $36{,}8$}
	{$18{,}4$}
	{$23{,}0$}
	{$46{,}0$}
	\loigiai{}
\end{ex}
%%%============EX_11==============%%%
\begin{ex}
	Dầu mỏ không có nhiệt độ sôi nhất định vì:
	\choice
	{Dầu mỏ không tan trong nước}
	{\True Dầu mỏ là hỗn hợp phức tạp nhiều hiđrocacbon}
	{Dầu mỏ nổi lên trên mặt nước}
	{Dầu mỏ là chất lỏng sánh}
	\loigiai{}
\end{ex}
%%%============EX_12==============%%%
\begin{ex}
	Khi đốt cháy khí metan bằng khí oxi thì tỉ lệ thể tích của khí metan và khí oxi nào dưới đây để được hỗn hợp nổ?
	\choice
	{1 thể tích khí metan và 3 thể tích khí oxi}
	{2 thể tích khí metan và 1 thể tích khí oxi}
	{3 thể tích khí metan và 2 thể tích oxi}
	{\True 1 thể tích khí metan và 2 thể tích khí oxi}
	\loigiai{}
\end{ex}
%%%%============EX_13==============%%%
%\begin{ex}
%	Khí  $CH_4$ và  $C_2H_4$ có tính chất hóa học giống nhau là
%	\choice
%	{tham gia phản ứng cộng với dung dịch brom}
%	{tham gia phản ứng cộng với khí hiđro}
%	{tham gia phản ứng trùng hợp}
%	{\True tham gia phản ứng cháy với khí oxi sinh ra khí cacbonic và nước}
%	\loigiai{}
%\end{ex}
%%%============EX_13==============%%%
\begin{ex}
	Phương trình đốt cháy hiđrocacbon X như sau: $X+3O_2 \stackrel{t^o}{\to}2CO_2+2H_2 O$
	Hiđrocacbon X là
	\choice
	{\True $C_2H_4$}
	{$C_2H_6$}
	{$CH_4$}
	{$C_2H_2$}
	\loigiai{}
\end{ex}
%%%============EX_14==============%%%
\begin{ex}
	Cho các phát biểu sau:
	\begin{enumerate}[(a)]
		\item  Metan, etilen, axetilen lần lượt có công thức phân tử là $CH_4$, $C_2H_2$, $C_2H_4$.
		\item  Metan, etilen, axetilen đều là các khí không màu, không mùi, nhẹ hơn nước, ít tan trong nước.
		\item  Tính chất hóa học đặc trưng của metan là phản ứng thế.
		\item  Để nhận biết metan và etilen ta có thể dùng dung dịch brom.
		\item  Khi đốt cháy metan ta thu được số mol $CO_2$ bằng số mol  $H_2O$ .
	\end{enumerate}
	Số phát biểu đúng là
	\choice
	{$1$}
	{$2$}
	{\True $3$}
	{$4$}
	\loigiai{}
\end{ex}
%%%============EX_15==============%%%
\begin{ex}
	Thuốc thử dùng để nhận biết metan và etilen là
	\choice
	{quì tím}
	{HCl}
	{NaOH}
	{\True dung dịch $Br_2$}
	\loigiai{}
\end{ex}
%%%============EX_16==============%%%
\begin{ex}
	Cho phương trình hóa học: 2X+7O $_2$ $\stackrel{t^o}{\to}$ 4CO $_2$+6H $_2$ O. X là
	\choice
	{$C_2H_2$}
	{$C_2H_4$}
	{\True $C_2H_6$}
	{$C_6H_6$}
	\loigiai{}
\end{ex}
%%%============EX_17==============%%%
\begin{ex}
	Các chất nào trong dãy tác dụng được với SiO $_2$?
	\choice
	{$CO_2$, $H_2O$ ,$H_2SO_4$, $NaOH$}
	{$CO_2$, $CaO$, $NaOH$}
	{$H_2SO_4$, $NaOH$, $CaO$,  $H_2$ O}
	{\True $NaOH$,  $K_2O$ , $CaO$}
	\loigiai{}
\end{ex}
%%%============EX_18==============%%%
\begin{ex}
	Nguyên liệu để sản xuất đồ gốm là:
	\choice
	{\True Đất sét, thạch anh, fenpat}
	{Đất sét, đá vôi, cát}
	{cát thạch anh, đá vôi, sođa}
	{Đất sét, thạch anh, đá vôi}
	\loigiai{}
\end{ex}
%%%============EX_19==============%%%
\begin{ex}
	Oxi hoá m gam hỗn hợp $X$ gồm $Al, Mg$ và kim loại $M$ có tỉ lệ số mol $Al: Mg: M=1: 2: 1$ cần $10{,}08$ lít  $Cl_2$ (đktc) thu được $45{,}95$ gam hỗn hợp Y gồm các muối clorua. Kim loại M là.
	\choice
	{Ca}
	{Ba}
	{\True Zn}
	{Fe}
	\loigiai{}
\end{ex}
%%%============EX_20==============%%%
\begin{ex}
	Nhiệt phân hoàn toàn $40$ gam một loại quặng đôlômit có lẫn tạp chất trơ sinh ra 8,96 lít khí $CO_2$ (ở đktc). Thành phần phần trăm về khối lượng của  $CaCO_3.MgCO_3$ trong loại quặng nêu trên là
	\choice
	{$40\%$}
	{$50\%$}
	{$84\%$}
	{\True $92\%$}
	\loigiai{}
\end{ex}
\Closesolutionfile{ansex}
\Closesolutionfile{ans}

%%%==========Phần trắc nghiệm đúng sai============%%%

%\tieumuc{Trắc Nghiệm Đúng Sai}-- \textit{Trong mỗi câu có 4 ý tương ứng A, B, C, D; Học sinh chọn đúng hoặc sai.}
%\Opensolutionfile{ans}[Ans/DATAM1]
%\Opensolutionfile{ansbook}[Ans/DATNTF-1]
%\luulgEXTF
%\Opensolutionfile{ansex}[LOIGIAITN/LGTNTF-1]
%Soạn ở đây
%\Closesolutionfile{ansex}
%\Closesolutionfile{ansbook}
%\Closesolutionfile{ans}	

%%%==========Phần tự luận============%%%
\tieumuc{Tự Luận (5 điểm)}
\Opensolutionfile{ansbt}[LOIGIAITL/LGTL-9CKII-De-01]
%\luuloigiaibt
%\hienthiloigiaibt
%\tatloigiaibt
%\taoNdongke[6]{bt}
%%%============BT_01==============%%%
\begin{bt}[1,5 điểm]Viết các PTHH thực hiện chuyển đổi hóa học sau (ghi rõ điều kiện phản ứng nếu có):
	\begin{tikzpicture}[declare function={d=2cm;}, font =\bfseries,>=stealth,baseline=(char.base)]
		\foreach \x/\p/\t in {0/A/$C_2H_4$,1.2/B/$C_2H_5OH$,2.8/C/$CH_3COOH$,4.8/D/$CH_3COOC_2H_5$,7/E/$CH_3COONa$}{
			\path (\x*d,0) node  [anchor=west] (\p) {\t};
		}
		
		\foreach \x/\y/\n in {A/B/1,B/C/2,C/D/3,D/E/4}{
			\draw[->] (\x)--(\y)node [pos =0.5,sloped,above,font=\scriptsize] {(\n)}; 
		}
		\end{tikzpicture}
\loigiai{}
\end{bt}
%\newpage
%\taoNdongke[10]{bt}
%%%============BT_02==============%%%
\begin{bt}[1,0 điểm]\begin{enumerate}[a)]
	\item Bằng phương pháp hóa học hãy phân biệt dầu lạc, dầu hỏa và giấm ăn.
	\item Viết công thức của chất béo là este của axit béo stearic $C_{17}H_{35}COOH$ và glixerol $C_3H_5(OH)_3$
	\end{enumerate}
	\loigiai{}
\end{bt}
%%%============BT_3==============%%%
%\taoNdongke[20]{bt}
\begin{bt}[2.5 điểm]
	Cho $16{,}6\mathrm{~g}$ hỗn hợp Y gồm rượu etylic và axit axetic tác dụng với natri dư, khi phản úng kết thuc thu được $3{,}36$ lít khí ở đktc.
	\begin{enumerate}[a)]
		\item  Viết PTHH xảy ra
		\item  Tính thành phần phần trăm theo khối lượng mỗi chất trong hỗn hợp ban đầu.
		\item  Đun nóng hỗn hợp Y với dung dịch ${H_2SO_4}_{\text{đặc}} $ nóng (làm chất xúc tác), thu được $m$ gam este. Biết hiệu suất phản ứng là $ 60\%$. Tính giá trị của $m$.
	\end{enumerate}
	Cho: $H=1; C=12; O=16; \mathrm{Br}=80, \mathrm{Ca}=40, \mathrm{Na}=23$
	\loigiai{}
\end{bt}
\Closesolutionfile{ansbt}

\fileend

\begin{center}
	\rule[4pt]{2cm}{1pt}\large \textbf{HẾT}\rule[4pt]{2cm}{1pt}
\end{center}