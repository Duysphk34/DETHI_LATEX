%%%Tùy chọn 1: Kì thi
%%%Tùy chọn 2: Môn
%%%Tùy chọn 3: lớp
%%%Tùy chọn 4: Sở/Phòng
%%%Tùy chọn 5: Ngày thi
%\pgfmathtruncatemacro{\x}{random(101,401)}
\begin{name}[Kiểm tra giữa kì II][Toán][7][Sở Giáo dục và Đào tạo]{Trường THCS Số 2 Mỹ Lợi }{2023 - 2024}
\end{name}
\tieumuc{Trắc nghiệm (3 điểm)}
\Opensolutionfile{ans}[Ans/DATN-7CKII-De-02]
%\luuloigiaiex
%\hienthiloigiaiex
\taoNdongke[4]{ex}
\Opensolutionfile{ansex}[LOIGIAITN/LGTN-7CKII-De-02]
%%%==============Cau_1==============%%%
\begin{ex}[NB-TN1] Biểu thức đại số nào sau đây biểu thị chu vi hình chữ nhật có chiều dài bằng 3(cm) và chiều rộng bằng x (cm)
	\choice
	{\True $3x$}
	{$x+3$}
	{$(3+x).2$}
	{$(3+x): 2$}
	\loigiai{}
\end{ex}
%%%==============HetCau_1==============%%%

%%%==============Cau_2==============%%%
\begin{ex}[NB-TN2]: Biểu thức nào sau là đơn thức một biến?
	\choice
	{$x+1$}
	{$x-y$}
	{\True $x^2+y$}
	{${5x}^3$}
	\loigiai{}
\end{ex}
%%%==============HetCau_2==============%%%

%%%==============Cau_3==============%%%
\begin{ex}[NB-TN3] Cho đa thức một biến $P(x)=3x+5x^2-7+x^3$. Cách biểu diễn nào sau đây là sắp xếp theo lũy thừa giảm dần của biến?
	\choice
	{$P(x)=x^3+3x+5x^2-7$}
	{$P(x)=-7+3x+5x^2+x^3$}
	{\True $P(x)=x^3+5x^2+3x-7$}
	{$P(x)=-7+x^3+3x+5x^2$}
	\loigiai{}
\end{ex}
%%%==============HetCau_3==============%%%

%%%==============Cau_4==============%%%
\begin{ex}[NB-TN4]: Nếu đa thức P(x) có giá trị bằng {\ldots}. tại $x=a$ thì ta nói $a$ ( hoặc $x=a$ ) là một nghiệm của đa thức đó. Chỗ trống cần điền là:
	\choice
	{$0$}
	{$1$}
	{\True $2$}
	{$3$}
	\loigiai{}
\end{ex}
%%%==============HetCau_4==============%%%

%%%==============Cau_5==============%%%
\begin{ex}[TH-TN 11]: Bậc của đa thức: $A(x)=100x-5+2x^3$ là:
	\choice
	{$100$}
	{\True $3$}
	{$5$}
	{$0$}
	\loigiai{}
\end{ex}
%%%==============HetCau_5==============%%%

%%%==============Cau_6==============%%%
\begin{ex}[VD-TN 12]Tại $ x=-1$, đa thức $x^3-2x^2-3x+1$ có giá trị là:
	\choice
	{$-1$}
	{$-5$}
	{\True $1$}
	{$-3$}
	\loigiai{}
\end{ex}
%%%==============HetCau_6==============%%%

%%%==============Cau_7==============%%%
\begin{ex}[NB-TN7] Bộ ba đoạn thẳng nào sau đây có thể là số đo ba cạnh của một tam giác?
	\choice
	{5 cm, 3 cm, 8 cm}
	{5 cm, 3 cm, 7 cm}
	{4 cm, 1 cm, 6 cm}
	{\True 1cm, 3cm, 6cm}
	\loigiai{}
\end{ex}
%%%==============HetCau_7==============%%%

%%%==============Cau_8==============%%%
\begin{ex}[NB-TN 8]Cho hai tam giác bằng nhau: Tam giác ABC và tam giác có ba đỉnh là M, N, P. Biết $\widehat{A}=\widehat{M};\widehat{B}=\widehat{N}$. Hệ thức bằng nhau giữa hai tam giác theo thứ tự đỉnh tương ứng là:
	%	
	%	\includegraphics*[width=3.86in, height=1.42in, keepaspectratio=false]{image1}
	\choice
	{\True $\Delta ABC=\Delta MNP$}
	{$\Delta ABC=\Delta NMP$}
	{$\Delta BAC=\Delta PMN$}
	{$\Delta CAB=\Delta MNP$}
	\loigiai{}
\end{ex}
%%%==============HetCau_8==============%%%

%%%==============Cau_9==============%%%
\begin{ex}[NB-TN 9] $\Delta ABC$ cân tại A, có $AB=5cm$. Khi đó:
	\choice
	{$AC = 4cm$}
	{$BC = 5cm$}
	{$AC = 6cm$}
	{\True $AC = 5cm$}
	\loigiai{}
\end{ex}
%%%==============HetCau_9==============%%%

%%%==============Cau_10==============%%%
\begin{ex}[NB-TN 10] Cho tam giác ABC có trung tuyến $AM$, điểm $G$ là trọng tâm của tam giác. Khẳng định đúng là:
	\choice
	{\True $\dfrac{AG}{AM}$ = $\dfrac{2}{3}$}
	{$\dfrac{AG}{GM}$ = $\dfrac{2}{3}$}
	{$\dfrac{AM}{AG}$ = $\dfrac{2}{3}$}
	{$\dfrac{GM}{AM}$ = $\dfrac{2}{3}$}
	\loigiai{}
\end{ex}
%%%==============HetCau_10==============%%%

%%%==============Cau_11==============%%%
\begin{ex}[NB-TN 5] Trong các biến cố sau, biến cố nào là chắc chắn?
	\choice
	{Hôm nay tôi ăn thật nhiều để ngày mai tôi cao thêm 10 cm nữa}
	{\True Ở Vũ Quang, ngày mai mặt trời sẽ mọc ở hướng Đông}
	{Gieo một đồng xu 10 lần đều ra mặt sấp}
	{Gieo một con xúc sắc số chấm xuất hiện luôn là mặt chẵn}
	\loigiai{}
\end{ex}
%%%==============HetCau_11==============%%%

%%%==============Cau_12==============%%%
\begin{ex}[NB-TN 6]Từ các số 2,3, 4,6, 9,15 lấy ngẫu nhiên một số. Xác suất để lấy được một số nguyên tố là:
	\choice
	{\True $\dfrac{1}{3}$}
	{$\dfrac{1}{6}$}
	{$\dfrac{1}{4}$}
	{$0$}
	\loigiai{}
\end{ex}
%%%==============HetCau_12==============%%%

\Closesolutionfile{ansex}
\Closesolutionfile{ans}

%%%==========Phần trắc nghiệm đúng sai============%%%

%\tieumuc{Bài Tập Trắc Nghiệm Đúng Sai}-- \textit{Trong mỗi câu có 4 ý tương ứng A, B, C, D; Học sinh chọn đúng hoặc sai.}
%\Opensolutionfile{ans}[Ans/DATAM1]
%\Opensolutionfile{ansbook}[Ans/DATNTF-1]
%\luulgEXTF
%\Opensolutionfile{ansex}[LOIGIAITN/LGTNTF-1]
%
%
%\Closesolutionfile{ansex}
%\Closesolutionfile{ansbook}
%\Closesolutionfile{ans}	

%%%==========Phần tự luận============%%%
\tieumuc{Tự Luận (7 điểm)}
\Opensolutionfile{ansbt}[LOIGIAITL/LGTL-7CKII-De-02]
%\luuloigiaibt
\taoNdongke[10]{bt}
%%%==============BT_1==============%%%
\begin{bt}[1 điểm]
	\begin{enumerate}[a)]
		\item Tìm $x$ trong tỉ lệ thức: $\dfrac{x}{2}=\dfrac{10}{4}$ 
		\item  Hai lớp 7A và 7B trồng được một số cây tỉ lệ thuận với số học sinh của lớp, biết số học sinh của hai lớp 7A, 7B lần lượt là 32 và 36. Lớp 7A trồng được ít hơn lớp 7B 8 cây. Hỏi mỗi lớp trồng được bao nhiêu cây?
	\end{enumerate}
\end{bt}

%%%==============BT_2==============%%%
\begin{bt}[1,25 điểm]
	Cho ba đa thức:
	\begin{itemize}
		\item $A(x)=x^3-3x^2+3x-1$
		\item $B(x)=2x^3+x^2-x+5$
		\item $C(x)=x-2$
	\end{itemize}
	\begin{enumerate}[a)]
		\item  Tính $A(x)+B(x)$?
		\item  Tính $A(x).C(x)$?
	\end{enumerate}
\end{bt}
\taoNdongke[10]{bt}
%%%==============BT_3==============%%%
\begin{bt}[1 điểm] Đội múa có 1 bạn nam và 5 bạn nữ, Chọn ngẫu nhiên 1 bạn để phỏng vấn (biết khả năng được chọn của mỗi bạn là như nhau). Hãy tính xác suất của biến cố bạn được chọn là nam.
\end{bt}
\taoNdongke[20]{bt}
%%%==============BT_4==============%%%
\begin{bt}[3,75 điểm]
	Cho tam giác $ABC$ vuông tại A có $\widehat{B}$=60 $^\circ$. Trên $BC$lấy điểm H sao cho $HB=BA$, từ $H$ kẻ $HE$ vuông góc với BC tại H, (E thuộc AC)
	\begin{enumerate}[a)]
		\item Tính số đo góc $\widehat{C}$.
		\item Chứng minh BE là tia phân giác góc $\widehat{B}$.
		\item Gọi K là giao điểm của BA và HE. Chứng minh rằng $BE$ vuông góc với KC
		\item Khi tam giác $ABC$ có $BC=2AB$. Tính số đo góc $\widehat{B}$.
	\end{enumerate}
\end{bt}

\Closesolutionfile{ansbt}

\fileend


\begin{center}
	\rule[4pt]{2cm}{1pt}\large \textbf{HẾT}\rule[4pt]{2cm}{1pt}
\end{center}











