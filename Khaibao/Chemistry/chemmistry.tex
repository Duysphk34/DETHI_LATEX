\usepackage{pgfplots}
\usepackage{filecontents}
\pgfplotsset{compat=1.14}
\usepackage{chemfig}
\usepackage{chemarrow}
\usetikzlibrary{decorations.markings, positioning}
\everymath{\rm}
%%%================Tạo lệnh mới trong chemfig==================%%%
%\renewcommand*\printatom[1]{\ensuremath{\mathbf{#1}}}%%lệnh viết font không chân/in dậm\mathbf
%%% =======Lệnh tô màu nguyên tử ==========%%%
\def\colorchem{\mycolor!80!black}
\newcommand*\maucthh[1]{\color{\colorchem}{\printatom{#1}}}
%%% =======Lệnh khoanh tròn nguyên tử ==========%%%
\newcommand*\circleatom[1]{\tikz\node[circle,draw,color=\mycolor]{\printatom{#1}};}
%%%=============Tạo style nhiều tùy chọn===================%%%
\tikzset{cfhbond/.style={% 
		/utils/exec=\tikzset{cfh/.cd,#1},
		draw=none,
		postaction={decorate},
		decoration={
			markings,
			mark = at position \pgfkeysvalueof{/tikz/cfh/position} with {\fill[\pgfkeysvalueof{/tikz/cfh/color}] circle[radius=\pgfkeysvalueof{/tikz/cfh/radius}*1pt];},
			mark = at position 0.5 with {\fill[\pgfkeysvalueof{/tikz/cfh/color}] circle[radius=\pgfkeysvalueof{/tikz/cfh/radius}*1pt];},
			mark = at position 1-\pgfkeysvalueof{/tikz/cfh/position} with {\fill[\pgfkeysvalueof{/tikz/cfh/color}] circle[radius=\pgfkeysvalueof{/tikz/cfh/radius}*1pt];}
		}%
	},cfh/.cd,color/.initial=red,radius/.initial=1,position/.initial=0.22} 
%%%===========================Vòng có nhánh 5 tùy chọn bỏ trống=======================%%%
\NewDocumentCommand{\cyclohexan}{O{}O{}O{}O{}O{}m}{%
	\ifblank{#1}{\definesubmol\ortho{}}{\definesubmol\ortho{(-[,.7]#1)}}
	\ifblank{#2}{\definesubmol\meta{}}{\definesubmol\meta{(-[,.7]#2)}}
	\ifblank{#3}{\definesubmol\para{}}{\definesubmol\para{(-[,.7]#3)}}
	\ifblank{#4}{\definesubmol\metax{}}{\definesubmol\metax{(-[,.7]#4)}}
	\ifblank{#5}{\definesubmol\orthox{}}{\definesubmol\orthox{(-[,.7,]#5)}}
	{\chemfig{**6(!\metax-!\para-!\meta-!\ortho-(-[,.7]#6)-!\orthox-)}}
}

%\newcommand{\arrowS}{-{Stealth[width=3.65pt,length=4pt]}}
%\newcommand{\arrowL}{-{Latex[width=3.65pt,length=4pt]}}
%\newcommand{\arrowTN}{-{Stealth[left]}} %%% tùy chọn khi dùng phản ứng thuận nghịch
%\newcommand{\explus}{\;+\;}


%%%Tùy chỉnh mũi tên\rightxarrow[]{}%%%%%%%%%%
\NewDocumentCommand{\rightxarrow}{O{}O{}}{%
	\ifblank{#1}{\def\tuychonone{1}}{\def\tuychonone{#1}}
	\ifblank{#2}{\def\tuychontwo{}}{\def\tuychontwo{#2}}
		\vcenter{\hbox{%
				\begin{tikzpicture}[join=round,cap=round]
					\node[minimum width=\tuychonone cm,minimum height=1ex,align=center,anchor=center,inner sep=0pt,outer sep =0pt] (a){\text{\vphantom{hg}}\\[0.0 ex]\text{\vphantom{hg}} };
					\draw[->,\tuychontwo] ([yshift=0.2ex]a.west) -- ([yshift=0.2ex]a.east);
					\path(a) node[pos=0.5,midway,yshift=0.2 ex]{\tikz[join=round,cap=round)]
					\draw[\tuychontwo] (-2pt,-2pt) --(2pt,2pt)(2pt,-2pt) --(-2pt,2pt);
					};
				\end{tikzpicture}%thêm % xóa khoảng trắng phia sua lệnh
		}}%
	}%



%%%Tùy chỉnh mũi tên\xrightarrow[]{}%%%%%%%%%%
\RenewDocumentCommand{\xrightarrow}{O{}O{}O{}O{}O{black}O{line width=0.65pt}}{%
	\ifblank{#1}{\def\tuychonone{}}{\def\tuychonone{#1}}
	\ifblank{#2}{\def\tuychontwo{}}{\def\tuychontwo{#2}}
	\ifblank{#3}{\def\tuychonthree{1}}{\def\tuychonthree{#3}}
	\ifblank{#4}{\def\tuychonfour{0.2}}{\def\tuychonfour{#4}}
	\mathrel{%
		\vcenter{\hbox{%
				\begin{tikzpicture}[join=round,cap=round,#6]
					\node[minimum width=\tuychonthree cm,minimum height=1ex,align=center,anchor=center,inner sep=0pt,outer sep =0pt] (a){\text{\vphantom{hg}\small\tuychonone}\\[\tuychonfour ex] \text{\vphantom{hg}\small\tuychontwo}};
					\draw[->,#5] ([yshift=0.2ex]a.west) -- ([yshift=0.2ex]a.east);
				\end{tikzpicture}%thêm % xóa khoảng trắng phia sua lệnh
		}}%
	}%
}
%%%Tùy chỉnh mũi tên\xleftarrow[]%%%%%%%%%%
\RenewDocumentCommand{\xleftarrow}{O{}O{}O{}O{}O{black}O{line width=0.65pt}}{%
	\ifblank{#1}{\def\tuychonone{}}{\def\tuychonone{#1}}
	\ifblank{#2}{\def\tuychontwo{}}{\def\tuychontwo{#2}}
	\ifblank{#3}{\def\tuychonthree{1}}{\def\tuychonthree{#3}}
	\ifblank{#4}{\def\tuychonfour{0.2}}{\def\tuychonfour{#4}}
	\mathrel{%
		\vcenter{\hbox{%
				\begin{tikzpicture}[join=round,cap=round,#6]
					\node[minimum width=\tuychonthree cm,minimum height=1ex,align=center,anchor=center,inner sep=0pt,outer sep =0pt] (a){\text{\vphantom{hg}\small\tuychonone}\\[\tuychonfour ex] \text{\vphantom{hg}\small\tuychontwo}};
					\draw[<-,#5] ([yshift=0.2ex]a.west) -- ([yshift=0.2ex]a.east);
				\end{tikzpicture}%
		}}%
	}%
}

%%%Tùy chỉnh mũi tên\uparrow[]{}%%%%%%%%%%
\RenewDocumentCommand{\uparrow}{O{}O{}}{%
	\ifblank{#1}{\def\tuychonone{1.5}}{\def\tuychonone{#1}}
	\ifblank{#2}{\def\tuychontwo{}}{\def\tuychontwo{#2}}
	\mathrel{%
		\vcenter{\hbox{%
				\begin{tikzpicture}[join=round,cap=round]
					\node[minimum width=1pt,minimum height=\tuychonone ex,align=center,anchor=center,inner sep=0pt,outer sep =0pt] (a){\text{}};
					\draw[->,\tuychontwo] ([xshift=-0.5ex]a.south) -- ([xshift=-0.5ex]a.north);
				\end{tikzpicture}%thêm % xóa khoảng trắng phia sua lệnh
		}}%
	}%
}

%%%Tùy chỉnh mũi tên\downarrow[]{}%%%%%%%%%%
\RenewDocumentCommand{\downarrow}{O{}O{}}{%
	\ifblank{#1}{\def\tuychonone{1.5}}{\def\tuychonone{#1}}
	\ifblank{#2}{\def\tuychontwo{}}{\def\tuychontwo{#2}}
	\mathrel{%
		\vcenter{\hbox{%
				\begin{tikzpicture}[join=round,cap=round]
					\node[minimum width=1pt,minimum height=\tuychonone ex,align=center,anchor=center,inner sep=0pt,outer sep =0pt] (a){\text{}};
					\draw[->,\tuychontwo] ([xshift=-0.5ex]a.north) -- ([xshift=-0.5ex]a.south);
				\end{tikzpicture}%thêm % xóa khoảng trắng phia sua lệnh
		}}%
	}%
}

%%%=================Tạo mũi tên hai chiều=================%%%
\NewDocumentCommand{\xleftrightarrow}{O{}O{}O{}O{}O{black}O{line width=0.65pt}}{%
	\ifblank{#1}{\def\tuychonone{}}{\def\tuychonone{#1}}
	\ifblank{#2}{\def\tuychontwo{}}{\def\tuychontwo{#2}}
	\ifblank{#3}{\def\tuychonthree{1.65}}{\def\tuychonthree{#3}}
	\ifblank{#4}{\def\tuychonfour{-0.9}}{\def\tuychonfour{#4}}
	\mathrel{%
		\vcenter{\hbox{%
				\begin{tikzpicture}[join=round,cap=round,#6]
					\node[minimum width=\tuychonthree cm,minimum height=1ex,anchor=south,align=center,inner sep=0pt,outer sep =0pt] (a){\text{\vphantom{hg}\scriptsize\tuychonone}\\[\tuychonfour ex] \text{\vphantom{hg}\scriptsize\tuychontwo}};
					\draw[->,#5] ([yshift=0.45ex]a.west) -- ([yshift=0.45ex]a.east);
					\draw[<-,#5] ([yshift=-0.10ex]a.west) -- ([yshift=-0.10ex]a.east);
				\end{tikzpicture}%thêm % xóa khoảng trắng phia sua lệnh
		}}%
	}%
}


%%%=================Tạo mũi tên hai chiều=================%%%
\NewDocumentCommand{\xharpoonarrow}{O{}O{}O{}O{}O{black}O{line width=0.65pt}}{%
	\ifblank{#1}{\def\tuychonone{}}{\def\tuychonone{#1}}
	\ifblank{#2}{\def\tuychontwo{}}{\def\tuychontwo{#2}}
	\ifblank{#3}{\def\tuychonthree{1.65}}{\def\tuychonthree{#3}}
	\ifblank{#4}{\def\tuychonfour{-0.9}}{\def\tuychonfour{#4}}
	\mathrel{%
		\vcenter{\hbox{%
				\begin{tikzpicture}[join=round,cap=round,#6]
					\node[minimum width=\tuychonthree cm,minimum height=1ex,anchor=south,align=center,inner sep=0pt,outer sep =0pt] (a){\text{\vphantom{hg}\scriptsize\tuychonone}\\[\tuychonfour ex] \text{\vphantom{hg}\scriptsize\tuychontwo}};
					\draw[arrows = {-Stealth[harpoon]},#5] ([yshift=0.35ex]a.west) -- ([yshift=0.35ex]a.east);
					\draw[arrows = {-Stealth[harpoon]},#5] ([yshift=-0.10ex]a.east) -- ([yshift=-0.10ex]a.west);
				\end{tikzpicture}%thêm % xóa khoảng trắng phia sua lệnh
		}}%
	}%
}


%%%============Tùy chỉnh toàn cục=======================%%%

%\setchemfig{%
%	bond join=true,
%	arrow label sep=3pt,
%	arrow offset=1.5pt,
%	arrow coeff=2.5pt,
%	atom sep=3em,
%	arrow style={\mycolor!40!black,>=stealth,thick},
%	atom style={color = \mycolor!40!black},
%	bond style={\mycolor!40!black,thick},
%	bond offset=1pt
%}
%%%==================Lệnh tạo Electron điền vào các orbitan===================
%%%Tùy chọn 1:màu sắc [có thể bỏ trống]
%%%Tùy chọn 2:Kích thước [có thể bỏ trống]
%%%Tùy chọn 3:Độ đậm nhạt mũi tên
\NewDocumentCommand{\Eghepdoi}{O{}O{}O{1pt}}{%
	\ifblank{#1}{\def\optonce{black}}{\def\optonce{#1}}
	\ifblank{#2}{\def\opttwo{0.6cm}}{\def\opttwo{#2}}
	\begin{tikzpicture}
		\tikzset{%	
			ntdvuong/.style={%
				regular polygon, 
				regular polygon sides= 4,
				draw=\optonce,line width =1.2pt, 
				inner sep=0pt,
				minimum size=.6cm ,
				anchor=center
			},
			AOghepdoi/.pic={%
				\path (0,0) node[ntdvuong,name=AO] {%
					\begin{tikzpicture}[declare function={d=\opttwo;r={d/2.4};}]
						\draw[arrows = {-Stealth},line width=#3,\optonce]
						(0,0)--(0,d);
						\draw[arrows = {-Stealth},line width=#3,\optonce]
						(r,d)--(r,0);
					\end{tikzpicture}
				};
			},
		}
		\draw 
		(0,0) pic{AOghepdoi};
	\end{tikzpicture}
}
%%%Tùy chọn 1:màu sắc [có thể bỏ trống]
%%%Tùy chọn 2:Kích thước [có thể bỏ trống]
%%%Tùy chọn 3:Độ đậm nhạt mũi tên
\NewDocumentCommand{\Edocthan}{O{}O{}O{1pt}}{%
	\ifblank{#1}{\def\optonce{black}}{\def\optonce{#1}}
	\ifblank{#2}{\def\opttwo{0.6cm}}{\def\opttwo{#2}}
	\begin{tikzpicture}
		\tikzset{%	
			ntdvuong/.style={%
				regular polygon, 
				regular polygon sides= 4,
				draw=\optonce,line width =1.2pt, 
				inner sep=0pt,
				minimum size=.5cm ,
				anchor=center
			},
			AOghepdoi/.pic={%
				\path (0,0) node[ntdvuong,name=AO] {%
					\begin{tikzpicture}[declare function={d=\opttwo;r={d/2.4};}]
						\draw[arrows = {-Stealth},line width=#3,\optonce]
						(0,0)--(0,d);
					\end{tikzpicture}
				};
			},
		}
		\draw 
		(0,0) pic{AOghepdoi};
	\end{tikzpicture}
}

%%%Tùy chọn 1:màu sắc [có thể bỏ trống]
%%%Tùy chọn 2:Kích thước [có thể bỏ trống]
%%%Tùy chọn 3:hướng e [0] hướng lên [180] hướng xuống
%%%Tùy chọn 4:Độ đậm nhạt mũi tên
\NewDocumentCommand{\Ecbghepdoi}{O{}O{}O{0}O{1pt}}{%
	\ifblank{#1}{\def\optonce{black}}{\def\optonce{#1}}
	\ifblank{#2}{\def\opttwo{0.6cm}}{\def\opttwo{#2}}
	\begin{tikzpicture}
		\tikzset{%	
			ntdvuong/.style={%
				regular polygon, 
				regular polygon sides= 4,
				draw=\optonce,line width =1.2pt, 
				inner sep=0pt,
				minimum size=.5cm ,
				anchor=center
			},
			AOghepdoi/.pic={%
				\path (0,0) node[ntdvuong,name=AO] {%
					\begin{tikzpicture}[declare function={d=\opttwo;r={d/2.4};}]
						\begin{scope}[rotate=#3]
							\draw[arrows = {-Stealth},line width=#4,\optonce]
							(0,0)--(0,d);
							\path[arrows = {-Stealth},line width=#4,\optonce]
							(r,d)--(r,0);
						\end{scope}
					\end{tikzpicture}
				};
			},
		}
		\draw 
		(0,0) pic{AOghepdoi};
	\end{tikzpicture}
}

\NewDocumentCommand{\AOtrong}{O{}O{}O{1pt}}{%
	\ifblank{#1}{\def\optonce{black}}{\def\optonce{#1}}
	\ifblank{#2}{\def\opttwo{0.636cm}}{\def\opttwo{#2}}
	\begin{tikzpicture}
		\tikzset{%	
			ntdvuong/.style={%
				regular polygon, 
				regular polygon sides= 4,
				draw=\optonce,line width =1.0pt, 
				inner sep=0pt,
				minimum size=.8cm ,
				anchor=center
			},
			AOghepdoi/.pic={%
				\path (0,0) node[ntdvuong,name=AO] {%
					\begin{tikzpicture}[declare function={d=\opttwo;r={d/2.4};}]
						\path[arrows = {-Stealth},line width=#3,\optonce]
						(0,0)--(0,d);
						\path[arrows = {-Stealth},line width=#3,\optonce]
						(r,d)--(r,0);
					\end{tikzpicture}
				};
			},
		}
		\draw 
		(0,0) pic{AOghepdoi};
	\end{tikzpicture}
}

%%% Liên kết Pi giải tỏa đều %%%
\catcode`\_=11 % manual p. 28
\tikzset{
	ddbond/.style args={#1}{
		draw=none,
		decoration={%
			markings,
			mark=at position 0 with {
				\coordinate (CF@startdeloc) at (0,\dimexpr#1\CF_doublesep/2)
				coordinate (CF@startaxis) at (0,\dimexpr-#1\CF_doublesep/2);
			},
			mark=at position 1 with {
				\coordinate (CF@enddeloc) at (0,\dimexpr#1\CF_doublesep/2)
				coordinate (CF@endaxis) at (0,\dimexpr-#1\CF_doublesep/2);
				\draw[dash pattern=on 2pt off 1.5pt] (CF@startdeloc)--(CF@enddeloc);
				\draw (CF@startaxis)--(CF@endaxis);
			}
		},
		postaction={decorate}
	}
}
\catcode`\_=8

%%%========================Lệnh tạo tùy chọn liên kết hidro================%%%
\catcode`\_=11
\tikzset{
	hidrobond/.style args={#1/#2/#3}{
		draw=none,
		decoration={%
			markings,
			mark=at position 0 with {
				\coordinate (CF@startdeloc) at (0,\dimexpr#1\CF_doublesep/50);
				\coordinate (CF@startaxis) at (0,\dimexpr#1\CF_doublesep/50);
			},
			mark=at position 1 with {
				\coordinate (CF@enddeloc) at (0,\dimexpr#1\CF_doublesep/50);
				\coordinate (CF@endaxis) at (0,\dimexpr#1\CF_doublesep/50);
				\path (CF@startdeloc)--(CF@enddeloc) node[pos =.5]{
					\begin{tikzpicture}[declare function={d=#2;k=.30*d;}]
						\foreach \x in {1*k,2*k,3*k}{\fill[#3] (\x,0) circle (0.037*d);}
					\end{tikzpicture}
				};
			}
		},
		postaction={decorate}
	}
}
\catcode`\_=8

%%%================Tạo lệnh chữ in đậm trong chemfig=================%%%
\renewcommand*\printatom[1]{\ensuremath{\mathbf{#1}}}
%%==========Phân tử nước đầu chuỗi============%%
\definesubmol\ptnuocdau{%
	\charge{135:2pt[red]=$\delta^{+} $}{H}-[:60]
	\charge{90:2pt[red,anchor=-90]=$\delta^{-} $}{O}-\charge{-90:2pt[red,anchor=90]=$\delta^{+} $}{H}}%
%%==========Phân tử nước giữa============%%
\definesubmol\ptnuoca{%
	\charge{-90:2pt[red,anchor=90]=$\delta^{-} $}{O}(-[:120]\charge{60:2pt[red,anchor=-120]=$\delta^{+} $}{H})-\charge{60:2pt[red,anchor=-90]=$\delta^{+} $}{H}}%
%%==========Phân tử nước cuối============%%
\definesubmol\ptnuocb{%
	\charge{90:2pt[red,anchor=-90]=$\delta^{-} $}{O}(-[:-120]\charge{135:2pt[red,anchor=-120]=$\delta^{+} $}{H})-\charge{60:2pt[red,anchor=-90]=$\delta^{+} $}{H}}%
%%=============== Lệnh tạo liên kết H ==================%%
\definesubmol\hbond{%
	-[,1.25,,,hidrobond={+}/{2.5em}/{red}]
}%

\newcommand{\lkhlpt}[1]{%
	\ifcase#1\relax%
	\or
	\chemfig[chemfig style={line width =1pt}]{%
		!\ptnuocdau!\hbond!\ptnuoca}
	\or
	\chemfig[chemfig style={line width =1pt}]{%
		!\ptnuocdau!\hbond!\ptnuoca!\hbond!\ptnuocb
	}%
	\or
	\chemfig[chemfig style={line width =1pt}]{%
		!\ptnuocdau!\hbond!\ptnuoca!\hbond!\ptnuocb!\hbond!\ptnuoca
	}%
	\or
	\chemfig[chemfig style={line width =1pt}]{%
		!\ptnuocdau!\hbond!\ptnuoca!\hbond!\ptnuocb!\hbond!\ptnuoca!\hbond!\ptnuocb
	}%
	\else
	\chemfig[chemfig style={line width =1pt}]{%
		!\ptnuocdau!\hbond!\ptnuoca!\hbond!\ptnuocb!\hbond!\ptnuoca!\hbond!\ptnuocb!\hbond!\ptnuoca
	}%
	\fi
}































%%% Tùy chỉnh mũi tên %%%
%%%% Tùy chỉnh mũi tên không phản ứng %%%
\catcode`\_=11
\definearrow4{-x>}{%
	\edef\leng{\ifx\empty#4\empty 8pt\else #4\fi}% dot radius
	\CF_arrowshiftnodes{#3}%
	\expandafter\draw\expandafter[\CF_arrowcurrentstyle](\CF_arrowstartnode)--(\CF_arrowendnode)
	coordinate[midway](mid@point);
	\path (\CF_arrowstartnode)--(\CF_arrowendnode) node[midway,sloped,pos =.5]{\tikz{
			\expandafter\draw\expandafter[\CF_arrowcurrentstyle,-](0,0)--++(35:\leng)--([turn]180:{2*\leng});
			\expandafter\draw\expandafter[\CF_arrowcurrentstyle,-](0,0)--++(-35:\leng)--([turn]180:{2*\leng});
	}};
	\CF_arrowdisplaylabel{#1}{0.5}{+}{\CF_arrowstartnode}{#2}{0.5}{-}{\CF_arrowendnode}
}
\catcode`\_=8


\makeatletter
\definearrow{9}{<X>}{%
	\CF@arrow@shift@nodes{#7}%
	%\expandafter\draw\expandafter[\CF@arrow@current@style,-CF](\CF@arrow@start@node)--(\CF@arrow@end@node)node[midway](Xarrow@arctangent){};%
	\path[allow upside down](\CF@arrow@start@node)--(\CF@arrow@end@node)%
	node[pos=0,sloped,yshift=1pt](\CF@arrow@start@node @u0){}%
	node[pos=0,sloped,yshift=-1pt](\CF@arrow@start@node @d0){}%
	node[pos=1,sloped,yshift=1pt](\CF@arrow@start@node @u1){}%
	node[pos=1,sloped,yshift=-1pt](\CF@arrow@start@node @d1){}%
	node[midway,yshift=1pt](Xarrow@arctangent@u){}%
	node[midway,yshift=-1pt](Xarrow@arctangent@d){};%
	\begingroup
	\pgfarrowharpoontrue
	\expandafter\draw\expandafter[\CF@arrow@current@style](\CF@arrow@start@node @u0)--(\CF@arrow@start@node @u1);%
	\expandafter\draw\expandafter[\CF@arrow@current@style](\CF@arrow@start@node @d1)--(\CF@arrow@start@node @d0);%
	\endgroup
	\edef\CF@tmp@str{\ifx\@empty#1\@empty[draw=none]\fi}%
	\expandafter\draw\CF@tmp@str (Xarrow@arctangent@u)%
	arc[radius=\CF@compound@sep*\CF@current@arrow@length*\ifx\@empty#8\@empty0.333\else#8\fi,start angle=\CF@arrow@current@angle-90,%
	delta angle=-\ifx\@empty#9\@empty60\else#9\fi]node(Xarrow1@start){};
	\edef\CF@tmp@str{[\ifx\@empty#2\@empty draw=none,\fi-CF]}%
	\expandafter\draw\CF@tmp@str (Xarrow@arctangent@u)%
	arc[radius=\CF@compound@sep*\CF@current@arrow@length*\ifx\@empty#8\@empty0.333\else#8\fi,start angle=\CF@arrow@current@angle-90,%
	delta angle=\ifx\@empty#9\@empty60\else#9\fi]node(Xarrow1@end){};
	\edef\CF@tmp@str{[\ifx\@empty#4\@empty draw=none,\fi-CF]}%
	\expandafter\draw\CF@tmp@str (Xarrow@arctangent@d)%
	arc[radius=\CF@compound@sep*\CF@current@arrow@length*\ifx\@empty#8\@empty0.333\else#8\fi,start angle=\CF@arrow@current@angle+90,%
	delta angle=\ifx\@empty#9\@empty60\else#9\fi]node(Xarrow2@start){};
	\edef\CF@tmp@str{[\ifx\@empty#5\@empty draw=none,\fi-CF]}%
	\expandafter\draw\CF@tmp@str (Xarrow@arctangent@d)%
	arc[radius=\CF@compound@sep*\CF@current@arrow@length*\ifx\@empty#8\@empty0.333\else#8\fi,start angle=\CF@arrow@current@angle+90,%
	delta angle=-\ifx\@empty#9\@empty60\else#9\fi]node(Xarrow2@end){};
	\edef\CF@tmp@str{\if\string-\expandafter\@car\detokenize{#7.}\@nil-\else+\fi}%
	\CF@arrow@display@label{#1}{0}\CF@tmp@str{Xarrow1@start}{#2}{1}\CF@tmp@str{Xarrow1@end}%
	\CF@arrow@display@label{#3}{0.5}\CF@tmp@str\CF@arrow@start@node{}{}{}\CF@arrow@end@node%
	\edef\CF@tmp@str{\if\string-\expandafter\@car\detokenize{#7.}\@nil+\else-\fi}%
	\CF@arrow@display@label{#4}{0}\CF@tmp@str{Xarrow2@start}{#5}{1}\CF@tmp@str{Xarrow2@end}%
	\CF@arrow@display@label{#6}{0.5}\CF@tmp@str\CF@arrow@start@node{}{}{}\CF@arrow@end@node%
}
\makeatother
